\section{Heal}

\label{f0}
\subsection{(Wis)}

				
You are skilled at tending to wounds and ailments.
				
\textbf{Check}: The DC and effect of a Heal check depend on the task you attempt.
% <thead href="../gettingStarted.html#wisdom">
\begin{table}
 \begin{tabular}{ll}
\textbf{Task} & \textbf{DC}\\
First aid & 15\\
Long-term care & 15\\
Treat wounds from caltrops, \textit{spike growth,} or \textit{spike stones} & 15\\
Treat deadly wounds & 20\\
Treat poison & Poison's save DC\\
Treat disease & Disease's save DC\\
 \end{tabular}

\end{table}

				
\textit{First Aid}: You usually use first aid to save a dying character. If a character has negative hit points and is losing hit points (at the rate of 1 per round, 1 per hour, or 1 per day), you can make him stable. A stable character regains no hit points but stops losing them. First aid also stops a character from losing hit points due to effects that cause bleed (see Conditions for rules on bleed damage).
				
\textit{Long-Term Care}: Providing long-term care means treating a wounded person for a day or more. If your Heal check is successful, the patient recovers hit points or ability score points lost to ability damage at twice the normal rate: 2 hit points per level for a full 8 hours of rest in a day, or 4 hit points per level for each full day of complete rest; 2 ability score points for a full 8 hours of rest in a day, or 4 ability score points for each full day of complete rest.
				
You can tend to as many as six patients at a time. You need a few items and supplies (bandages, salves, and so on) that are easy to come by in settled lands. Giving long-term care counts as light activity for the healer. You cannot give long-term care to yourself.
				
\textit{Treat Wounds from Caltrops, Spike Growth, or Spike Stones}: A creature wounded by stepping on a caltrop moves at half normal speed. A successful Heal check removes this movement penalty.
				
A creature wounded by a \textit{spike growth} or \textit{spike stones} spell must succeed on a Reflex save or take injuries that reduce his speed by one-third. Another character can remove this penalty by taking 10 minutes to dress the victim's injuries and succeeding on a Heal check against the spell's save DC.
				
\textit{Treat Deadly Wounds}: When treating deadly wounds, you can restore hit points to a damaged creature. Treating deadly wounds restores 1 hit point per level of the creature. If you exceed the DC by 5 or more, add your Wisdom modifier (if positive) to this amount. A creature can only benefit from its deadly wounds being treated within 24 hours of being injured and never more than once per day. You must expend two uses from a healer's kit to perform this task. You take a --2 penalty on your Heal skill check for each use from the healer's kit that you lack.
				
\textit{Treat Poison}: To treat poison means to tend to a single character who has been poisoned and who is going to take more damage from the poison (or suffer some other effect). Every time the poisoned character makes a saving throw against the poison, you make a Heal check. If your Heal check exceeds the DC of the poison, the character receives a +4 competence bonus on his saving throw against the poison.
				
\textit{Treat Disease}: To treat a disease means to tend to a single diseased character. Every time the diseased character makes a saving throw against disease effects, you make a Heal check. If your Heal check exceeds the DC of the disease, the character receives a +4 competence bonus on his saving throw against the disease.
				
\textbf{Action}: Providing first aid, treating a wound, or treating poison is a standard action. Treating a disease or tending a creature wounded by a \textit{spike growth} or \textit{spike stones} spell takes 10 minutes of work. Treating deadly wounds takes 1 hour of work. Providing long-term care requires 8 hours of light activity.
				
\textbf{Try Again}: Varies. Generally speaking, you can't try a Heal check again without witnessing proof of the original check's failure. You can always retry a check to provide first aid, assuming the target of the previous attempt is still alive.
				
\textbf{Special}: A character with the Self-Sufficient feat gets a bonus on Heal checks (see Feats).
				
A healer's kit gives you a +2 circumstance bonus on Heal checks.
        	
