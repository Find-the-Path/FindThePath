\spellentry{Trap the Soul}
\textbf{School} conjuration (summoning); \textbf{Level} sorcerer/wizard 8\\
\textbf{Casting Time} 1 standard action or see text\\
\textbf{Components} V, S, M (gem worth 1,000 gp per HD of the trapped creature)\\
\textbf{Range} close (25 ft. + 5 ft./2 levels)\\
\textbf{Target} one creature\\
\textbf{Duration} permanent; see text\\
\textbf{Saving Throw} see text; \textbf{Spell Resistance} yes; see text\\
\textit{Trap the soul }forces a creature's life force (and its material body) into a gem. The gem holds the trapped entity indefinitely or until the gem is broken and the life force is released, which allows the material body to reform. If the trapped creature is a powerful creature from another plane, it can be required to perform a service immediately upon being freed. Otherwise, the creature can go free once the gem imprisoning it is broken.\\
Depending on the version selected, the spell can be triggered in one of two ways.\\
\textit{Spell Completion}: First, the spell can be completed by speaking its final word as a standard action as if you were casting a regular spell at the subject. This allows spell resistance (if any) and a Will save to avoid the effect. If the creature's name is spoken as well, any spell resistance is ignored and the save DC increases by 2. If the save or spell resistance is successful, the gem shatters.\\
\textit{Trigger Object}: The second method is far more insidious, for it tricks the subject into accepting a trigger object inscribed with the final spell word, automatically placing the creature's soul in the trap. To use this method, both the creature's name and the trigger word must be inscribed on the trigger object when the gem is enspelled. A \textit{sympathy }spell can also be placed on the trigger object. As soon as the subject picks up or accepts the trigger object, its life force is automatically transferred to the gem without the benefit of spell resistance or a save.\\
