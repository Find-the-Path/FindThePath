\spellentry{Dispel Magic}
\textbf{School} abjuration; \textbf{Level} bard 3, cleric 3, druid 4, paladin 3, sorcerer/wizard 3\\
\textbf{Casting Time} 1 standard action\\
\textbf{Components} V, S\\
\textbf{Range }medium (100 ft. + 10 ft./level)\\
\textbf{Target or Area} one spellcaster, creature, or object\\
\textbf{Duration} instantaneous\\
\textbf{Saving Throw} none; \textbf{Spell Resistance} no\\
You can use \textit{dispel magic }to end one ongoing spell that has been cast on a creature or object, to temporarily suppress the magical abilities of a magic item, or to counter another spellcaster's spell. A dispelled spell ends as if its duration had expired. Some spells, as detailed in their descriptions, can't be defeated by \textit{dispel magic}. \textit{Dispel magic }can dispel (but not counter) spell-like effects just as it does spells. The effect of a spell with an instantaneous duration can't be dispelled, because the magical effect is already over before the \textit{dispel magic }can take effect. \\
You choose to use \textit{dispel magic }in one of two ways: a targeted dispel or a counterspell.\\
\textit{Targeted Dispel}: One object, creature, or spell is the target of the \textit{dispel magic }spell. You make one dispel check (1d20 + your caster level) and compare that to the spell with highest caster level (DC = 11 + the spell's caster level). If successful, that spell ends. If not, compare the same result to the spell with the next highest caster level. Repeat this process until you have dispelled one spell affecting the target, or you have failed to dispel every spell.\\
 For example, a 7th-level caster casts \textit{dispel magic}, targeting a creature affected by \textit{stoneskin }(caster level 12th) and \textit{fly} (caster level 6th). The caster level check results in a 19. This check is not high enough to end the \textit{stoneskin} (which would have required a 23 or higher), but it is high enough to end the \textit{fly} (which only required a 17). Had the dispel check resulted in a 23 or higher, the \textit{stoneskin} would have been dispelled, leaving the \textit{fly }intact. Had the dispel check been a 16 or less, no spells would have been affected.\\
You can also use a targeted dispel to specifically end one spell affecting the target or one spell affecting an area (such as a \textit{wall of fire}). You must name the specific spell effect to be targeted in this way. If your caster level check is equal to or higher than the DC of that spell, it ends. No other spells or effects on the target are dispelled if your check is not high enough to end the targeted effect.\\
If you target an object or creature that is the effect of an ongoing spell (such as a monster summoned by \textit{summon monster}), you make a dispel check to end the spell that conjured the object or creature.\\
If the object that you target is a magic item, you make a dispel check against the item's caster level (DC = 11 + the item's caster level). If you succeed, all the item's magical properties are suppressed for 1d4 rounds, after which the item recovers its magical properties. A suppressed item becomes nonmagical for the duration of the effect. An interdimensional opening (such as a \textit{bag of holding}) is temporarily closed. A magic item's physical properties are unchanged: A suppressed magic sword is still a sword (a masterwork sword, in fact). Artifacts and deities are unaffected by mortal magic such as this.\\
You automatically succeed on your dispel check against any spell that you cast yourself.\\
\textit{Counterspell}: When \textit{dispel magic }is used in this way, the spell targets a spellcaster and is cast as a counterspell. Unlike a true counterspell, however, \textit{dispel magic }may not work; you must make a dispel check to counter the other spellcaster's spell.\\
\textbf{Dispel Magic, Greater}\\
\textbf{School} abjuration; \textbf{Level} bard 5, cleric 6, druid 6, sorcerer/wizard 6\\
\textbf{Target or Area} one spellcaster, creature, or object; or a 20-ft.-radius burst\\
This spell functions like \textit{dispel magic}, except that it can end more than one spell on a target and it can be used to target multiple creatures. \\
You choose to use \textit{greater dispel magic }in one of three ways: a targeted dispel, area dispel, or a counterspell:\\
\textit{Targeted Dispel}: This functions as a targeted \textit{dispel magic}, but it can dispel one spell for every four caster levels you possess, starting with the highest level spells and proceeding to lower level spells.\\
Additionally, \textit{greater dispel magic }has a chance to dispel any effect that \textit{remove curse }can remove, even if \textit{dispel magic }can't dispel that effect. The DC of this check is equal to the curse's DC.\\
\textit{Area Dispel}: When \textit{greater dispel magic }is used in this way, the spell affects everything within a 20-foot-radius burst. Roll one dispel check and apply that check to each creature in the area, as if targeted by \textit{dispel magic}. For each object within the area that is the target of one or more spells, apply the dispel check as with creatures. Magic items are not affected by an area dispel.\\
For each ongoing area or effect spell whose point of origin is within the area of the \textit{greater dispel magic }spell, apply the dispel check to dispel the spell. For each ongoing spell whose area overlaps that of the \textit{greater dispel magic }spell, apply the dispel check to end the effect, but only within the overlapping area.\\
If an object or creature that is the effect of an ongoing spell (such as a monster summoned by \textit{summon monster}) is in the area, apply the dispel check to end the spell that conjured that object or creature (returning it whence it came) in addition to attempting to dispel one spell targeting the creature or object.\\
You may choose to automatically succeed on dispel checks against any spell that you have cast.\\
\textit{Counterspell}: This functions as \textit{dispel magic, }but you receive a +4 bonus on your dispel check to counter the other spellcaster's spell.\\
