\spellentry{Grease}
\textbf{School }conjuration (creation); \textbf{Level }bard 1, sorcerer/wizard 1\\
\textbf{Casting Time }1 standard action\\
\textbf{Components }V, S, M (butter)\\
\textbf{Range }close (25 ft. + 5 ft./2 levels)\\
\textbf{Target }one object or 10-ft. square\\
\textbf{Duration }1 min./level (D)\\
\textbf{Save }see text; \textbf{SR }no\\
A \textit{grease }spell covers a solid surface with a layer of slippery grease. Any creature in the area when the spell is cast must make a successful Reflex save or fall. A creature can walk within or through the area of grease at half normal speed with a DC 10 Acrobatics check. Failure means it can't move that round (and must then make a Reflex save or fall), while failure by 5 or more means it falls (see the Acrobatics skill for details). Creatures that do not move on their turn do not need to make this check and are not considered flat-footed.\\
The spell can also be used to create a greasy coating on an item. Material objects not in use are always affected by this spell, while an object wielded or employed by a creature requires its bearer to make a Reflex saving throw to avoid the effect. If the initial saving throw fails, the creature immediately drops the item. A saving throw must be made in each round that the creature attempts to pick up or use the \textit{greased }item. A creature wearing \textit{greased }armor or clothing gains a +10 circumstance bonus on Escape Artist checks and combat maneuver checks made to escape a grapple, and to their CMD to avoid being grappled.\\
