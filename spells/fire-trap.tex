\spellentry{Fire Trap}
\textbf{School} abjuration [fire]; \textbf{Level} druid 2, sorcerer/wizard 4\\
\textbf{Casting Time} 10 minutes\\
\textbf{Components} V, S, M (gold dust worth 25 gp)\\
\textbf{Range} touch\\
\textbf{Target} object touched\\
\textbf{Duration} permanent until discharged (D)\\
\textbf{Saving Throw} Reflex half; see text; \textbf{Spell Resistance} yes\\
\textit{Fire trap }creates a fiery explosion when an intruder opens the item that the trap protects. A \textit{fire trap }spell can ward any object that can be opened and closed.\\
When casting \textit{fire trap, }you select a point on the object as the spell's center. When someone other than you opens the object, a fiery explosion fills the area within a 5-foot radius around the spell's center. The flames deal 1d4 points of fire damage + 1 point per caster level (maximum +20). The item protected by the trap is not harmed by this explosion.\\
A fire-trapped item cannot have a second closure or warding spell placed on it. A \textit{knock }spell does not bypass a \textit{fire trap}. An unsuccessful \textit{dispel magic }spell does not detonate the spell. Underwater, this ward deals half damage and creates a large cloud of steam.\\
You can use the fire-trapped object without discharging it, as can any individual to whom the object was specifically attuned when cast. Attuning a fire-trapped object to an individual usually involves setting a password that you can share with friends.\\
Magic traps such as \textit{fire trap }are hard to detect and disable. A character with trapfinding can use the Perception skill to find a \textit{fire trap }and Disable Device to thwart it. The DC in each case is 25 + spell level (DC 27 for a druid's \textit{fire trap }or DC 29 for the arcane version).\\
