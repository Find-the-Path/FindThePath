\spellentry{Teleport}
\textbf{School} conjuration (teleportation); \textbf{Level} sorcerer/wizard 5\\
\textbf{Casting Time} 1 standard action\\
\textbf{Components} V\\
\textbf{Range} personal and touch\\
\textbf{Target} you and touched objects or other touched willing creatures\\
\textbf{Duration} instantaneous\\
\textbf{Saving Throw} none and Will negates (object); \textbf{Spell Resistance} no and yes (object)\\
This spell instantly transports you to a designated destination, which may be as distant as 100 miles per caster level. Interplanar travel is not possible. You can bring along objects as long as their weight doesn't exceed your maximum load. You may also bring one additional willing Medium or smaller creature (carrying gear or objects up to its maximum load) or its equivalent per three caster levels. A Large creature counts as two Medium creatures, a Huge creature counts as four Medium creatures, and so forth. All creatures to be transported must be in contact with one another, and at least one of those creatures must be in contact with you. As with all spells where the range is personal and the target is you, you need not make a saving throw, nor is spell resistance applicable to you. Only objects held or in use (attended) by another person receive saving throws and spell resistance.\\
You must have some clear idea of the location and layout of the destination. The clearer your mental image, the more likely the teleportation works. Areas of strong physical or magical energy may make teleportation more hazardous or even impossible.\\
To see how well the teleportation works, roll d% and consult the table at the end of this spell. Refer to the following information for definitions of the terms on the table.\\
\textit{Familiarity}: "Very familiar" is a place where you have been very often and where you feel at home. "Studied carefully" is a place you know well, either because you can currently physically see it or you've been there often. "Seen casually" is a place that you have seen more than once but with which you are not very familiar. "Viewed once" is a place that you have seen once, possibly using magic such as \textit{scrying}. \\
"False destination" is a place that does not truly exist or if you are teleporting to an otherwise familiar location that no longer exists as such or has been so completely altered as to no longer be familiar to you. When traveling to a false destination, roll 1d20+80 to obtain results on the table, rather than rolling d%, since there is no real destination for you to hope to arrive at or even be off target from.\\
\textit{On Target}: You appear where you want to be.\\
\textit{Off Target}: You appear safely a random distance away from the destination in a random direction. Distance off target is d% of the distance that was to be traveled. The direction off target is determined randomly.\\
\textit{Similar Area}: You wind up in an area that's visually or thematically similar to the target area. Generally, you appear in the closest similar place within range. If no such area exists within the spell's range, the spell simply fails instead.\\
\textit{Mishap}: You and anyone else teleporting with you have gotten "scrambled." You each take 1d10 points of damage, and you reroll on the chart to see where you wind up. For these rerolls, roll 1d20+80. Each time "Mishap" comes up, the characters take more damage and must reroll.\\
\textbf{Teleport, Greater}\\
\textbf{School} conjuration (teleportation); \textbf{Level} sorcerer/wizard 7\\
This spell functions like \textit{teleport, }except that there is no range limit and there is no chance you arrive off target. In addition, you need not have seen the destination, but in that case you must have at least a reliable description of the place to which you are teleporting. If you attempt to teleport with insufficient information (or with misleading information), you disappear and simply reappear in your original location. Interplanar travel is not possible.\\
