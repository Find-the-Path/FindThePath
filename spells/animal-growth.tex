\spellentry{Animal Growth}
\textbf{School }transmutation; \textbf{Level }druid 5, ranger 4, sorcerer/wizard 5\\
\textbf{Casting Time }1 standard action\\
\textbf{Components }V, S\\
\textbf{Range }medium (100 ft. + 10 ft./level)\\
\textbf{Target }one animal (Gargantuan or smaller)\\
\textbf{Duration }1 min./level\\
\textbf{Saving Throw} Fortitude negates; \textbf{Spell Resistance} yes\\
The target animal grows to twice its normal size and eight times its normal weight. This alteration changes the animal's size category to the next largest, grants it a +8 size bonus to Strength and a +4 size bonus to Constitution (and thus an extra 2 hit points per HD), and imposes a --2 size penalty to Dexterity. The creature's existing natural armor bonus increases by 2. The size change also affects the animal's modifier to AC, attack rolls, and its base damage. The animal's space and reach change as appropriate to the new size, but its speed does not change. If insufficient room is available for the desired growth, the creature attains the maximum possible size and may make a Strength check (using its increased Strength) to burst any enclosures in the process. If it fails, it is constrained without harm by the materials enclosing it—the spell cannot be used to crush a creature by increasing its size.\\
All equipment worn or carried by the animal is similarly enlarged by the spell, though this change has no effect on the magical properties of any such equipment.\\
Any enlarged item that leaves the enlarged creature's possession instantly returns to its normal size.\\
The spell gives no means of command over an enlarged animal.\\
Multiple magical effects that increase size do not stack.\\
