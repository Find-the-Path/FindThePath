\spellentry{Giant Form I}
\textbf{School }transmutation (polymorph); \textbf{Level }sorcerer/wizard 7\\
\textbf{Casting Time }1 standard action\\
\textbf{Components }V, S, M (a piece of the creature whose form you plan to assume)\\
\textbf{Range }personal\\
\textbf{Target }you\\
\textbf{Duration }1 min./level (D)\\
When you cast this spell you can assume the form of any Large humanoid creature of the giant subtype. Once you assume your new form, you gain the following abilities: a +6 size bonus to Strength, a –2 penalty to Dexterity, a +4 size bonus to Constitution, a +4 natural armor bonus, and low-light vision. If the form you assume has any of the following abilities, you gain the listed ability: darkvision 60 feet, rend (2d6 damage), regeneration 5, rock catching, and rock throwing (range 60 feet, 2d6 damage). If the creature has immunity or resistance to any elements, you gain resistance 20 to those elements. If the creature has vulnerability to an element, you gain that vulnerability.\\
\textbf{Giant Form II}\\
\textbf{School }Transmutation (polymorph); \textbf{Level }sorcerer/wizard 8\\
This spell functions as \textit{giant form I }except that it also allows you to assume the form of any Huge creature of the giant type. You gain the following abilities: a +8 size bonus to Strength, a –2 penalty to Dexterity, a +6 size bonus to Constitution, a +6 natural armor bonus, low-light vision, and a +10 foot enhancement bonus to your speed. If the form you assume has any of the following abilities, you gain the listed ability: swim 60 feet, darkvision 60 feet, rend (2d8 damage), regeneration 5, rock catching, and rock throwing (range 120 feet, 2d10 damage). If the creature has immunity or resistance to one element, you gain that immunity or resistance. If the creature has vulnerability to an element, you gain that vulnerability.\\
