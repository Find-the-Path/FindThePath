\spellentry{Wall of Stone}
\textbf{School} conjuration (creation) [earth]; \textbf{Level} cleric 5, druid 6, sorcerer/wizard 5\\
\textbf{Casting Time} 1 standard action\\
\textbf{Components} V, S, M/DF (a small block of granite)\\
\textbf{Range }medium (100 ft. + 10 ft./level)\\
\textbf{Effect} stone wall whose area is up to one 5-ft. square/level (S)\\
\textbf{Duration} instantaneous\\
\textbf{Saving Throw} see text; \textbf{Spell Resistance} no\\
This spell creates a wall of rock that merges into adjoining rock surfaces. A \textit{wall of stone }is 1 inch thick per four caster levels and composed of up to one 5-foot square per level. You can double the wall's area by halving its thickness. The wall cannot be conjured so that it occupies the same space as a creature or another object.\\
Unlike a \textit{wall of iron, }you can create a \textit{wall of stone }in almost any shape you desire. The wall created need not be vertical, nor rest upon any firm foundation; however, it must merge with and be solidly supported by existing stone. It can be used to bridge a chasm, for instance, or as a ramp. For this use, if the span is more than 20 feet, the wall must be arched and buttressed. This requirement reduces the spell's area by half. The wall can be crudely shaped to allow crenellations, battlements, and so forth by likewise reducing the area.\\
Like any other stone wall, this one can be destroyed by a \textit{disintegrate }spell or by normal means such as breaking and chipping. Each 5-foot square of the wall has hardness 8 and 15 hit points per inch of thickness. A section of wall whose hit points drop to 0 is breached. If a creature tries to break through the wall with a single attack, the DC for the Strength check is 20 + 2 per inch of thickness.\\
It is possible, but difficult, to trap mobile opponents within or under a \textit{wall of stone}, provided the wall is shaped so it can hold the creatures. Creatures can avoid entrapment with successful Reflex saves.\\
