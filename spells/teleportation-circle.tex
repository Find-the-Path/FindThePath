\spellentry{Teleportation Circle}
\textbf{School} conjuration (teleportation); \textbf{Level} sorcerer/wizard 9\\
\textbf{Casting Time} 10 minutes\\
\textbf{Components} V, M (amber dust to cover circle worth 1,000 gp)\\
\textbf{Range} 0 ft.\\
\textbf{Effect} 5-ft.-radius circle that teleports those who activate it\\
\textbf{Duration} 10 min./level (D)\\
\textbf{Saving Throw} none; \textbf{Spell Resistance} yes\\
You create a circle on the floor or other horizontal surface that teleports, as \textit{greater teleport, }any creature who stands on it to a designated spot. Once you designate the destination for the circle, you can't change it. The spell fails if you attempt to set the circle to teleport creatures into a solid object, to a place with which you are not familiar and have no clear description, or to another plane.\\
The circle itself is subtle and nearly impossible to notice. If you intend to keep creatures from activating it accidentally, you need to mark the circle in some way.\\
\textit{Teleportation circle }can be made permanent with a \textit{permanency }spell. A permanent \textit{teleportation circle }that is disabled becomes inactive for 10 minutes, then can be triggered again as normal.\\
Magic traps such as \textit{teleportation circle }are hard to detect and disable. A character with the trapfinding class feature can use Disable Device to disarm magic traps. The DC in each case is 25 + spell level, or 34 in the case of \textit{teleportation circle}.\\
