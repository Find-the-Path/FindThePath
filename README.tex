\chapter*{Preamble}
\section{Introduction}

In March 2018, Paizo announced a new edition of the Pathfinder RPG with sweeping changes
to the original rules. In the discussions that arose around this, I 
\href{https://paizo.com/threads/rzs2v5kr&page=2?Main-Thing-You-Want-From-PF2#82}{made a statement}

\begin{formal}
 What I want to see from PF2 is a system that \textbf{100\% of current Pathfinder players consider an Improvement} to Pathfinder. \\
 That is, a system where noone can say "PF1 is better than PF2 in \textit{this} aspect".
\end{formal}

This had me thinking: How much of Pathfinder could I change while remaining within these constraints? \\

This book - a rewritten PRD - attempts to answer that question.

\section{Design Philosophy and Goals}
One issue that the Playtest had was the lack of clear direction of design decisions and how they related
to design goals. It took half a year from the announcement for Paizo to actually publish their design
goals and even then, they were extremely vague. \\

For this project, I am trying the opposite. \\

\subsection{Every change must not make the system worse for anyone playing the game}

This is the most important goal. I acknowledge that it is a significant constraint, and some have said
that it might be impossible to achieve. \\

For this to work, \textbf{any change that could be controversial should not be applied}. \\

There may be some exteme cases where a rewording might make theorycrafting less effective (changes to 
\textit{Simularcrum} come to mind). As these do not (and should not) come up under actual play,
objections in this case may be discounted. \\

\subsection{Each change must be acknowledged}

Pathfinder is a very extensive ruleset, that builds on top of an old and venerable chassis. That it
has survived this long is a testament to its quality. However, there is much legacy cruft that is still
hanging off it. This still catches players and GMs by surprise. (The 1-round casting time of the \textit{Sleep}
spell continues to surprise players, and it has been around since 3.0). \\

For this reason, changes should be noted so that issues can be traced back to where they are from. \\

\subsection{Challenge Ratings should not require adjustment}

One of the strengths of Pathfinder is its wide array of first- and third-party content available. Changes to
the system that damage this will make the system of little use.


\section{Planned Changes}
In light of these goals, the following nonexhaustive changes are planned:

\begin{itemize}
 \item Remove some unnecessary rules, like the +1 BAB requirement to draw a weapon while moving.
 \item Remove or reword horribly misunderstood and unclear rules, like the "bless counters and dispels bane" and other "counters and dispels" clauses that never come up in matter of course.
 \item Reword the counterspell rules in general - in particular, remove the "same target" requirement that makes counterspelling personal-range spells impossible.
 \item Increase skill points for the sorcerer, fighter and cleric.
 \item Change the sorcerer's spell access - possibly by removing their one-level penalty compared to the wizard, and definitely by making their bloodline spells available as soon as the slots were unlocked.
 \item Possibly replace Arcane Spell Failure with a straight proficiency requirement.
 \item Add sidebar notes to commonly abused spells like Simulacrum to curb the theorycrafted abuses that arise. 
\end{itemize}

\section{Formatting Changes TODO}

Formatting changes should be made to the parent project (\url{https://github.com/Mekkiss/latexprd})

\begin{itemize}
 \item Fix multi-page, multi-column tables.
 \item Make font size slightly smaller in general. * Reduced to 9.5
 \item Overhanging table titles (is it possible in \LaTeX )
 \item Proper cross-references
 \item A bunch of Chapter 12 is missing --- rationalise against the PRD
\end{itemize}

% \section{LaTeX questions}
\begin{itemize}
 \item Find a table expert with multicolumn experience
 \item Balancing an end-of-chapter two-column layer
 \item LaTexLint
\end{itemize}

