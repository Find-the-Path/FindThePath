\section{Fighter}

\label{f0}				
Some take up arms for glory, wealth, or revenge. Others do battle to prove themselves, to protect others, or because they know nothing else. Still others learn the ways of weaponcraft to hone their bodies in battle and prove their mettle in the forge of war. Lords of the battlefield, fighters are a disparate lot, training with many weapons or just one, perfecting the uses of armor, learning the fighting techniques of exotic masters, and studying the art of combat, all to shape themselves into living weapons. Far more than mere thugs, these skilled warriors reveal the true deadliness of their weapons, turning hunks of metal into arms capable of taming kingdoms, slaughtering monsters, and rousing the hearts of armies. Soldiers, knights, hunters, and artists of war, fighters are unparalleled champions, and woe to those who dare stand against them.
				
\begin{formal}
The fighter is a somewhat-maligned class, often considered ``weak'' or otherwise unsuitable.
The changes that are happening here are:
\begin{itemize}
 \item Increased Skill points to four - they shouldn't be as high as the Ranger's six.
 \item Modified Bravery to apply to all will saves
 \item Combat feats ignore ability prereqs - This might have the side effect of 
     encouraging dipping too much? How to achieve this while still allowing
     Combat Expertise / TWF at first level?
\end{itemize}
More extreme changes that could happen:
\begin{itemize}
 \item Armour Mastery and Weapon Mastery - lower the level at which they are obtained.
 \item Weapon Training - Remove the staggered bonuses: when you add a new group, all
    groups you're trained in could get the full bonus.
\end{itemize}

\end{formal}
\textbf{Role}: Fighters excel at combat---defeating their enemies, controlling the flow of battle, and surviving such sorties themselves. While their specific weapons and methods grant them a wide variety of tactics, few can match fighters for sheer battle prowess.
				
\textbf{Alignment}: Any.
				
\textbf{Hit Die}: d10.
				
\subsection{Class Skills}

				
The fighter's class skills are Climb (Str), Craft (Int), Handle Animal (Cha), Intimidate (Cha), Knowledge (dungeoneering) (Int), Knowledge (engineering) (Int), Profession (Wis), Ride (Dex), Survival (Wis), and Swim (Str). 
				
\textbf{Skill Ranks per Level}: 4 + Int modifier.

\begin{table}[]
\sffamily
\fontsize{9.5}{11.4}\selectfont
\setlength{\tabcolsep}{1pt}
\caption{Table: Fighter}
\begin{tabular}{llllll}
               & \textbf{Base} & \textbf{Fort} & \textbf{Ref} & \textbf{Will}\\
\textbf{Level} & \textbf{Attack Bonus} & \textbf{Save} & \textbf{Save} & \textbf{Save} & \textbf{Special}\\
1st & +1 & +2 & +0 & +0 & Bonus feat\\
2nd & +2 & +3 & +0 & +0 & Bonus feat, bravery +1\\
3rd & +3 & +3 & +1 & +1 & Armor training 1 \\
4th & +4 & +4 & +1 & +1 & Bonus feat\\
5th & +5 & +4 & +1 & +1 & Weapon training 1\\
6th & +6/+1 & +5 & +2 & +2 & Bonus feat, bravery +2\\
7th & +7/+2 & +5 & +2 & +2 & Armor training 2\\
8th & +8/+3 & +6 & +2 & +2 & Bonus feat\\
9th & +9/+4 & +6 & +3 & +3 & Weapon training 2\\
10th & +10/+5 & +7 & +3 & +3 & Bonus feat, bravery +3\\
11th & +11/+6/+1 & +7 & +3 & +3 & Armor training 3\\
12th & +12/+7/+2 & +8 & +4 & +4 & Bonus feat\\
13th & +13/+8/+3 & +8 & +4 & +4 & Weapon training 3\\
14th & +14/+9/+4 & +9 & +4 & +4 & Bonus feat, bravery +4\\
15th & +15/+10/+5 & +9 & +5 & +5 & Armor training 4\\
16th & +16/+11/+6/+1 & +10 & +5 & +5 & Bonus feat\\
17th & +17/+12/+7/+2 & +10 & +5 & +5 & Weapon training 4\\
18th & +18/+13/+8/+3 & +11 & +6 & +6 & Bonus feat, bravery +5\\
19th & +19/+14/+9/+4 & +11 & +6 & +6 & Armor mastery 5\\
20th & +20/+15/+10/+5 & +12 & +6 & +6 & Bonus feat \\
     &                &     &    &    & weapon mastery 5\\
\end{tabular}
\end{table}
				
\subsection{Class Features}


The following are class features of the fighter.

\textbf{Weapon and Armor Proficiency}: A fighter is proficient with all simple and martial weapons 
    and with all armor (heavy, light, and medium) and shields (including tower shields).

\textbf{Bonus Feats}: At 1st level, and at every even level thereafter, a fighter gains a bonus
    feat in addition to those gained from normal advancement (meaning that the fighter gains
    a feat at every level). These bonus feats must be selected from those listed as combat feats,
    sometimes also called \texttt{{}"{}}fighter bonus feats.\texttt{{}"{}}. A fighter ignores
    ability score prerequesites for these feats.

Upon reaching 4th level, and every four levels thereafter (8th, 12th, and so on), a fighter can
choose to learn a new bonus feat in place of a bonus feat he has already learned. In effect, the
fighter loses the bonus feat in exchange for the new one. The old feat cannot be one that was used
as a prerequisite for another feat, prestige class, or other ability. A fighter can only change one
feat at any given level and must choose whether or not to swap the feat at the time he gains a new
bonus feat for the level.

\begin{formal}
 \textbf{On Retraining}
 The Pathfinder rules for Retraining are very convoluted - they are able to result in characters
 that are uncreatable organically. It is unclear as to whether this is by-design. I am open to
 opinions and arguments either way.
\end{formal}

\textbf{Bravery (Ex)}: Starting at 2nd level, a fighter gains a +1 bonus on Will saves. This bonus
increases by +1 for every four levels beyond 2nd.

\textbf{Armor Training (Ex)}: Starting at 3rd level, a fighter learns to be more maneuverable while
wearing armor. Whenever he is wearing armor, he reduces the armor check penalty by 1 (to a minimum of 0)
and gets a +1 dodge bonus to AC. Every four levels thereafter (7th, 11th, and 15th), these bonuses
increase by +1 each time, to a maximum --4 reduction of the armor check penalty and a +4 increase of the
maximum Dexterity bonus allowed.

In addition, a fighter can also move at his normal speed while wearing medium armor. At 7th level, a
fighter can move at his normal speed while wearing heavy armor.

\textbf{Weapon Training (Ex)}: Starting at 5th level, a fighter can select one group of weapons, as noted below.
Whenever he attacks with a weapon from this group, he gains a +1 bonus on attack and damage rolls.

Every four levels thereafter (9th, 13th, and 17th), a fighter becomes further trained in another group of weapons.
He gains a +1 bonus on attack and damage rolls when using a weapon from this group. In addition, the bonuses 
granted by previous weapon groups increase by +1 each. For example, when a fighter reaches 9th level, he
receives a +1 bonus on attack and damage rolls with one weapon group and a +2 bonus on attack and damage
rolls with the weapon group selected at 5th level. Bonuses granted from overlapping groups do not stack.
Take the highest bonus granted for a weapon if it resides in two or more groups.

A fighter also adds this bonus to any combat maneuver checks made with weapons from this group. This bonus also
applies to the fighter's Combat Maneuver Defense when defending against disarm and sunder attempts made against
weapons from this group.

Weapon groups are defined as follows (GMs may add other weapons to these groups, or add entirely new groups):

\textit{Axes}: battleaxe, dwarven waraxe, greataxe, handaxe, heavy pick, light pick, orc double axe, 
    and throwing axe. 

\textit{Blades, Heavy}: bastard sword, elven curve blade, falchion, greatsword, longsword, scimitar,
    scythe, and two-bladed sword.

\textit{Blades, Light}: dagger, kama, kukri, rapier, sickle, starknife, and short sword.

\textit{Bows}: composite longbow, composite shortbow, longbow, and shortbow.

\textit{Close}: gauntlet, heavy shield, light shield, punching dagger, sap, spiked armor, spiked gauntlet,
    spiked shield, and unarmed strike.

\textit{Crossbows}: hand crossbow, heavy crossbow, light crossbow, heavy repeating crossbow, and light 
    repeating crossbow.

\textit{Double}: dire flail, dwarven urgrosh, gnome hooked hammer, orc double axe, quarterstaff, and
    two-bladed sword.

\textit{Flails}: dire flail, flail, heavy flail, morningstar, nunchaku, spiked chain, and whip.

\textit{Hammers}: club, greatclub, heavy mace, light hammer, light mace, and warhammer.

\textit{Monk}: kama, nunchaku, quarterstaff, sai, shuriken, siangham, and unarmed strike.

\textit{Natural}: unarmed strike and all natural weapons, such as bite, claw, gore, tail, and wing.

\textit{Pole Arms}: glaive, guisarme, halberd, and ranseur.

\textit{Spears}: javelin, lance, longspear, shortspear, spear, and trident.

\textit{Thrown}: blowgun, bolas, club, dagger, dart, halfling sling staff, javelin, light hammer, net,
    shortspear, shuriken, sling, spear, starknife, throwing axe, and trident.

\textbf{Armor Mastery (Ex)}: At 19th level, a fighter gains DR 5/--- whenever he is wearing armor or using a shield.

\textbf{Weapon Mastery (Ex)}: At 20th level, a fighter chooses one weapon, such as the longsword, greataxe, 
    or longbow. Any attacks made with that weapon automatically confirm all critical threats and have their
    damage multiplier increased by 1 (\mbox{$\times$}2 becomes \mbox{$\times$}3, for example). In addition,
    he cannot be disarmed while wielding a weapon of this type.