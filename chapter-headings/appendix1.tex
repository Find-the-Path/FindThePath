\chapter{Appendix}
\section{Special Abilities}

\label{f0}				
The following special abilities include rules commonly used by a number of creatures, spells, and traps.
				
\textbf{Extraordinary Abilities (Ex)}: Extraordinary abilities are nonmagical. They are, however, not something that just anyone can do or even learn to do without extensive training. Effects or areas that suppress or negate magic have no effect on extraordinary abilities.
				
\textbf{Spell-Like Abilities (Sp)}: Spell-like abilities, as the name implies, are magical abilities that are very much like spells. Spell-like abilities are subject to spell resistance and \textit{dispel magic}. They do not function in areas where magic is suppressed or negated (such as an \textit{antimagic field}). Spell-like abilities can be dispelled but they cannot be counterspelled or used to counterspell.
				
\textbf{Supernatural Abilities (Su)}: Supernatural abilities are magical but not spell-like. Supernatural abilities are not subject to spell resistance and do not function in areas where magic is suppressed or negated (such as an \textit{antimagic field}). A supernatural ability's effect cannot be dispelled and is not subject to counterspells. See Table: Special Ability Types for a summary of the types of special abilities.
% <div class="table">


\begin{table}[]
\sffamily
\caption{Special Ability Types}
\begin{tabular}{llll}
\textbf{}             & \textbf{Extraordinary} & \textbf{Spell-Like} & \textbf{Supernatural} \\
Dispel                & No                     & Yes                 & No                    \\
Spell resistance      & No                     & Yes                 & No                    \\
Antimagic field       & No                     & Yes                 & Yes                   \\
Attack of             &                        &                     &     \\
opportunity & No                     & Yes                 & No    \\                
\end{tabular}\\
\textit{Dispel}: Can \textit{dispel magic }and similar spells dispel the effects of abilities of that type?\\
\textit{Spell Resistance}: Does spell resistance protect a creature from these abilities?\\
\textit{Antimagic Field}: Does an \textit{antimagic field }or similar magic suppress the ability?\\
\textit{Attack of Opportunity}: Does using the ability provoke attacks of opportunity the way that casting a spell does?\\
\end{table}
						
\subsection{Ability Score Bonuses}

				
Some spells and abilities increase your ability scores. Ability score increases with a duration of 1 day or less give only temporary bonuses. For every two points of increase to a single ability, apply a +1 bonus to the skills and statistics listed with the relevant ability.
				
\textbf{Strength}: Temporary increases to your Strength score give you a bonus on Strength-based skill checks, melee attack rolls, and weapon damage rolls (if they rely on Strength). The bonus also applies to your Combat Maneuver Bonus (if you are Small or larger) and to your Combat Maneuver Defense.
				
\textbf{Dexterity}: Temporary increases to your Dexterity score give you a bonus on Dexterity-based skill checks, ranged attack rolls, initiative checks, and Reflex saving throws. The bonus also applies to your Armor Class, your Combat Maneuver Bonus (if you are Tiny or smaller), and your Combat Maneuver Defense.
				
\textbf{Constitution}: Temporary increases to your Constitution score give you a bonus on your Fortitude saving throws. In addition, multiply your total Hit Dice by this bonus and add that amount to your current and total hit points. When the bonus ends, remove this total from your current and total hit points.
				
\textbf{Intelligence}: Temporary increases to your Intelligence score give you a bonus on Intelligence-based skill checks. This bonus also applies to any spell DCs based on Intelligence.
				
\textbf{Wisdom}: Temporary increases to your Wisdom score give you a bonus on Wisdom-based skill checks and Will saving throws. This bonus also applies to any spell DCs based on Wisdom.
				
\textbf{Charisma}: Temporary increases to your Charisma score give you a bonus on Charisma-based skill checks. This bonus also applies to any spell DCs based on Charisma and the DC to resist your channeled energy.
				
\textbf{Permanent Bonuses}: Ability bonuses with a duration greater than 1 day actually increase the relevant ability score after 24 hours. Modify all skills and statistics related to that ability. This might cause you to gain skill points, hit points, and other bonuses. These bonuses should be noted separately in case they are removed.
												
\subsection{Ability Score Damage, Penalty, and Drain}

				
Diseases, poisons, spells, and other abilities can all deal damage directly to your ability scores. This damage does not actually reduce an ability, but it does apply a penalty to the skills and statistics that are based on that ability.
				
For every 2 points of damage you take to a single ability, apply a --1 penalty to skills and statistics listed with the relevant ability. If the amount of ability damage you have taken equals or exceeds your ability score, you immediately fall unconscious until the damage is less than your ability score. The only exception to this is your Constitution score. If the damage to your Constitution is equal to or greater than your Constitution score, you die. Unless otherwise noted, damage to your ability scores is healed at the rate of 1 per day to each ability score that has been damaged. Ability damage can be healed through the use of spells, such as \textit{lesser restoration}. 
				
Some spells and abilities cause you to take an ability penalty for a limited amount of time. While in effect, these penalties function just like ability damage, but they cannot cause you to fall unconscious or die. In essence, penalties cannot decrease your ability score to less than 1.
				
\textbf{Strength}: Damage to your Strength score causes you to take penalties on Strength-based skill checks, melee attack rolls, and weapon damage rolls (if they rely on Strength). The penalty also applies to your Combat Maneuver Bonus (if you are Small or larger) and your Combat Maneuver Defense.
				
\textbf{Dexterity}: Damage to your Dexterity score causes you to take penalties on Dexterity-based skill checks, ranged attack rolls, initiative checks, and Reflex saving throws. The penalty also applies to your Armor Class, your Combat Maneuver Bonus (if you are Tiny or smaller), and to your Combat Maneuver Defense.
				
\textbf{Constitution}: Damage to your Constitution score causes you to take penalties on your Fortitude saving throws. In addition, multiply your total Hit Dice by this penalty and subtract that amount from your current and total hit points. Lost hit points are restored when the damage to your Constitution is healed.
				
\textbf{Intelligence}: Damage to your Intelligence score causes you to take penalties on Intelligence-based skill checks. This penalty also applies to any spell DCs based on Intelligence.
				
\textbf{Wisdom}: Damage to your Wisdom score causes you to take penalties on Wisdom-based skill checks and Will saving throws. This penalty also applies to any spell DCs based on Wisdom.
				
\textbf{Charisma}: Damage to your Charisma score causes you to take penalties on Charisma-based skill checks. This penalty also applies to any spell DCs based off Charisma and the DC to resist your channeled energy.
				
\textbf{Ability Drain}: Ability drain actually reduces the relevant ability score. Modify all skills and statistics related to that ability. This might cause you to lose skill points, hit points, and other bonuses. Ability drain can be healed through the use of spells such as \textit{restoration.}
								
\subsection{Afflictions}

				
From curses to poisons to diseases, there are a number of afflictions that can affect a creature. While each of these afflictions has a different effect, they all function using the same basic system. All afflictions grant a saving throw when they are contracted. If successful, the creature does not suffer from the affliction and does not need to make any further rolls. If the saving throw is a failure, the creature falls victim to the affliction and must deal with its effects.
				
Afflictions require a creature to make a saving throw after a period of time to avoid taking certain penalties. With most afflictions, if a number of saving throws are made consecutively, the affliction is removed and no further saves are necessary. Some afflictions, usually supernatural ones, cannot be cured through saving throws alone and require the aid of powerful magic to remove. Each affliction is presented as a short block of information to help you better adjudicate its results. 
				
\textbf{Name}: This is the name of the affliction. 
				
\textbf{Type}: This is the type of the affliction, such as curse, disease, or poison. It might also include the means by which it is contracted, such as contact, ingestion, inhalation, injury, spell, or trap.
				
\textbf{Save}: This gives the type of save necessary to avoid contracting the affliction, as well as the DC of that save. Unless otherwise noted, this is also the save to avoid the affliction's effects once it is contracted, as well as the DC of any caster level checks needed to end the affliction through magic, such as \textit{remove curse} or \textit{neutralize poison}.
				
\textbf{Onset}: Some afflictions have a variable amount of time before they set in. Creatures that come in contact with an affliction with an onset time must make a saving throw immediately. Success means that the affliction is avoided and no further saving throws must be made. Failure means that the creature has contracted the affliction and must begin making additional saves after the onset period has elapsed. The affliction's effect does not occur until after the onset period has elapsed and then only if further saving throws are failed.
				
\textbf{Frequency}: This is how often the periodic saving throw must be attempted after the affliction has been contracted (after the onset time, if the affliction has any). While some afflictions last until they are cured, others end prematurely, even if the character is not cured through other means. If an affliction ends after a set amount of time, it will be noted in the frequency. For example, a disease with a frequency of \texttt{{}"{}}1/day\texttt{{}"{}} lasts until cured, but a poison with a frequency of \texttt{{}"{}}1/round for 6 rounds\texttt{{}"{}} ends after 6 rounds have passed.
				
Afflictions without a frequency occur only once, immediately upon contraction (or after the onset time if one is listed). 
				
\textbf{Effect}: This is the effect that the character suffers each time if he fails his saving throw against the affliction. Most afflictions cause ability damage or hit point damage. These effects are cumulative, but they can be cured normally. Other afflictions cause the creature to take penalties or other effects. These effects are sometimes cumulative, with the rest only affecting the creature if it failed its most recent save. Some afflictions have different effects after the first save is failed. These afflictions have an initial effect, which occurs when the first save is failed, and a secondary effect, when additional saves are failed, as noted in the text. Hit point and ability score damage caused by an affliction cannot be healed naturally while the affliction persists.
				
\textbf{Cure}: This tells you how the affliction is cured. Commonly, this is a number of saving throws that must be made consecutively. Even if the affliction has a limited frequency, it might be cured prematurely if enough saving throws are made. Hit point damage and ability score damage is not removed when an affliction is cured. Such damage must be healed normally. Afflictions without a cure entry can only be cured through powerful spells, such as \textit{neutralize poison} and \textit{remove curse}. No matter how many saving throws are made, these afflictions continue to affect the target. 
				
 He failed a DC 15 Fortitude save to avoid contracting it, so after the onset period of 1d3 days has passed, he must make another DC 15 Fortitude save to avoid taking 1d6 points of Strength damage. From this point onward, he must make a DC 15 Fortitude save each day (according to the disease's frequency) to avoid further Strength damage. If, on two consecutive days, he makes his Fortitude saves, he is cured of the disease and any damage it caused begins to heal as normal.
								
\subsection{Curses}

				
Careless rogues plundering a tomb, drunken heroes insulting a powerful wizard, and foolhardy adventurers who pick up ancient swords all might suffer from curses. These magic afflictions can have a wide variety of effects, from a simple penalty to certain checks to transforming the victim into a toad. Some even cause the afflicted to slowly rot away, leaving nothing behind but dust. Unlike other afflictions, most curses cannot be cured through a number of successful saving throws. Curses can be cured through magic, however, usually via spells such as \textit{remove curse} and \textit{break enchantment}. While some curses cause a progressive deterioration, others inflict a static penalty from the moment they are contracted, neither fading over time nor growing worse. In addition, there are a number of magic items that act like curses. See Magic Items for a description of these cursed items.
				
The following samples present just some of the possibilities when creating curses.
				
\textbf{Baleful Polymorph Spell }
				
\textbf{Type} curse, spell; \textbf{Save} Fortitude DC 17 negates, Will DC 17 partial
				
\textbf{Effect }transforms target into a lizard; see \textit{baleful polymorph} description
				
\textbf{Bestow Curse Trap }
				
\textbf{Type} curse, spell, trap; \textbf{Save} Will DC 14
				
\textbf{Effect }--6 penalty to Strength
				
\textbf{Curse of the Ages }
				
\textbf{Type} curse; \textbf{Save} Will DC 17
				
\textbf{Frequency} 1/day
				
\textbf{Effect }age 1 year
				
\textbf{Mummy Rot }
				
\textbf{Type} curse, disease, injury; \textbf{Save} Fortitude DC 16
				
\textbf{Onset }1 minute; \textbf{Frequency} 1/day
				
\textbf{Effect }1d6 Con damage and 1d6 Cha damage;\textbf{ Cure} mummy rot can only be cured by successfully casting both \textit{remove curse} and \textit{remove disease} within 1 minute of each other.
				
\textbf{Unluck}
				
\textbf{Type} curse; \textbf{Save} Will DC 20 negates, no save to avoid effects
				
\textbf{Frequency} 1/hour
				
\textbf{Effect }target must reroll any roll decided by the GM and take the worse result
				
\textbf{Werewolf Lycanthropy }
				
\textbf{Type} curse, injury; \textbf{Save} Fortitude DC 15 negates, Will DC 15 to avoid effects 
				
\textbf{Onset }the next full moon; \textbf{Frequency} on the night of every full moon or whenever the target is injured
				
\textbf{Effect }target transforms into a wolf under the GM's control until the next morning
				
\subsection{Diseases}

				
From a widespread plague to the bite of a dire rat, disease is a serious threat to common folk and adventurers alike. Diseases rarely have a limited frequency, but most have a lengthy onset time. This onset time can also be variable. Most diseases can be cured by a number of consecutive saving throws or by spells such as \textit{remove disease}.
				
The following samples represent just some of the possibilities when creating diseases.
				
\textbf{Blinding Sickness }
				
\textbf{Type} disease, ingested; \textbf{Save} Fortitude DC 16 
				
\textbf{Onset }1d3 days; \textbf{Frequency} 1/day
				
\textbf{Effect }1d4 Str damage, if more than 2 Str damage, target must make an additional Fort save or be permanently blinded;\textbf{ Cure} 2 consecutive saves
				
\textbf{Bubonic Plague}
				
\textbf{Type} disease, injury or inhaled; \textbf{Save} Fortitude DC 17
				
\textbf{Onset }1 day; \textbf{Frequency} 1/day
				
\textbf{Effect }1d4 Con damage and 1 Cha damage and target is fatigued;\textbf{ Cure} 2 consecutive saves
				
\textbf{Cackle Fever }
				
\textbf{Type} disease, inhaled; \textbf{Save} Fortitude DC 16 
				
\textbf{Onset }1 day; \textbf{Frequency} 1/day
				
\textbf{Effect }1d6 Wis damage;\textbf{ Cure} 2 consecutive saves
				
\textbf{Demon Fever }
				
\textbf{Type} disease, injury; \textbf{Save} Fortitude DC 18 
				
\textbf{Onset }1 day; \textbf{Frequency} 1/day
				
\textbf{Effect }1d6 Con damage, target must make a second Fort save or 1 point of the damage is drain instead;\textbf{ Cure} 2 consecutive saves
				
\textbf{Devil Chills }
				
\textbf{Type} disease, injury; \textbf{Save} Fortitude DC 14 
				
\textbf{Onset }1d4 days; \textbf{Frequency} 1/day
				
\textbf{Effect }1d4 Str damage;\textbf{ Cure} 3 consecutive saves
				
\textbf{Filth Fever }
				
\textbf{Type} disease, injury; \textbf{Save} Fortitude DC 12 
				
\textbf{Onset }1d3 days; \textbf{Frequency} 1/day
				
\textbf{Effect }1d3 Dex damage and 1d3 Con damage;\textbf{ Cure} 2 consecutive saves
				
\textbf{Leprosy}
				
\textbf{Type} disease, contact, inhaled, or injury; \textbf{Save} Fortitude DC 12 negates, Fortitude DC 20 to avoid effects
				
\textbf{Onset }2d4 weeks; \textbf{Frequency} 1/week
				
\textbf{Effect }1d2 Cha damage;\textbf{ Cure} 2 consecutive saves
				
\textbf{Mindfire }
				
\textbf{Type} disease, inhaled; \textbf{Save} Fortitude DC 12 
				
\textbf{Onset }1 day; \textbf{Frequency} 1/day
				
\textbf{Effect }1d4 Int damage;\textbf{ Cure} 2 consecutive saves
				
\textbf{Red Ache }
				
\textbf{Type} disease, injury; \textbf{Save} Fortitude DC 15 
				
\textbf{Onset }1d3 days; \textbf{Frequency} 1/day
				
\textbf{Effect }1d6 Str damage;\textbf{ Cure} 2 consecutive saves
				
\textbf{Shakes }
				
\textbf{Type} disease, contact; \textbf{Save} Fortitude DC 13 
				
\textbf{Onset }1 day; \textbf{Frequency} 1/day
				
\textbf{Effect }1d8 Dex damage;\textbf{ Cure} 2 consecutive saves
				
\textbf{Slimy Doom }
				
\textbf{Type} disease, contact; \textbf{Save} Fortitude DC 14 
				
\textbf{Onset }1 day; \textbf{Frequency} 1/day
				
\textbf{Effect }1d4 Con damage, target must make a second Fort save or 1 point of the damage is drain instead;\textbf{ Cure} 2 consecutive saves
				
\subsection{Poison}

				
No other affliction is so prevalent as poison. From the fangs of a viper to the ichor-stained assassin's blade, poison is a constant threat. Poisons can be cured by successful saving throws and spells such as \textit{neutralize poison}. 
				
Contact poisons are contracted the moment someone touches the poison with his bare skin. Such poisons can be used as injury poisons. Contact poisons usually have an onset time of 1 minute and a frequency of 1 minute. Ingested poisons are contracted when a creature eats or drinks the poison. Ingested poisons usually have an onset time of 10 minutes and a frequency of 1 minute. Injury poisons are primarily contracted through the attacks of certain creatures and through weapons coated in the toxin. Injury poisons do not usually have an onset time and have a frequency of 1 round. Inhaled poisons are contracted the moment a creature enters an area containing such poisons. Most inhaled poisons fill a volume equal to a 10-foot cube per dose. Creatures can attempt to hold their breaths while inside to avoid inhaling the toxin. Creatures holding their breaths receive a 50\% chance of not having to make a Fortitude save each round. See the rules for holding your breath and suffocation in Environment. Note that a character that would normally suffocate while attempting to hold its breath instead begins to breathe normally again.
				
Unlike other afflictions, multiple doses of the same poison stack. Poisons delivered by injury and contact cannot inflict more than one dose of poison at a time, but inhaled and ingested poisons can inflict multiple doses at once. Each additional dose extends the total duration of the poison (as noted under frequency) by half its total duration. In addition, each dose of poison increases the DC to resist the poison by +2. This increase is cumulative. Multiple doses do not alter the cure conditions of the poison, and meeting these conditions ends the affliction for all the doses. For example, a character is bit three times in the same round by a trio of Medium monstrous spiders, injecting him with three doses of Medium spider venom. The unfortunate character must make a DC 18 Fortitude save for the next 8 rounds. Fortunately, just one successful save cures the character of all three doses of the poison.
				
Applying poison to a weapon or single piece of ammunition is a standard action. Whenever a character applies or readies a poison for use there is a 5\% chance that he exposes himself to the poison and must save against the poison as normal. This does not consume the dose of poison. Whenever a character attacks with a poisoned weapon, if the attack roll results in a natural 1, he exposes himself to the poison. This poison is consumed when the weapon strikes a creature or is touched by the wielder. Characters with the poison use class feature do not risk accidentally poisoning themselves.
				
Poisons can be made using Craft (alchemy). The DC to make a poison is equal to its Fortitude save DC. Rolling a natural 1 on a Craft skill check while making a poison exposes the crafter to the poison. Crafters with the poison use class feature do not risk poisoning themselves when using Craft to make poison.
				
The following samples represent just some of the possibilities when creating poisons.
				

\begin{table*}[]
\sffamily
\caption{Table: Sample Poisons}
\begin{tabular}{llllllll}
\textbf{Name} & \textbf{Type} & \textbf{Fort DC} & \textbf{Onset} & \textbf{Frequency} & \textbf{Effect} & \textbf{Cure} & \textbf{Cost}\\
Arsenic & ingested & 13 & 10 min. & 1/min. for 4 min. & 1d2  & 1 save & 120 gp\\
Belladonna & ingested & 14 & 10 min. & 1/min. for 6 min. & 1d2  & 1 save & 100 gp\\
Black adder venom & injury & 11 & --- & 1/rd. for 6 rds. & 1d2  & 1 save & 120 gp\\
Black lotus extract & contact & 20 & 1 min. & 1/rd. for 6 rds. & 1d6  & 2 saves & 4,500 gp\\
Bloodroot & injury & 12 & 1 rd. & 1/rd. for 4 rds. & 1  & 1 save & 100 gp\\
Blue whinnis & injury & 14 & --- & 1/rd. for 2 rds &  1  & 1 save & 120 gp\\
Burnt othur fumes & inhaled & 18 & --- & 1/rd. for 6 rds. & 1  & 2 saves & 2,100 gp\\
Dark reaver powder & ingested & 18 & 10 min. & 1/min. for 6 min. & 1d3  & 2 saves & 800 gp\\
Deathblade & injury & 20 & --- & 1/rd. for 6 rds. & 1d3  & 2 saves & 1,800 gp\\
Dragon bile & contact & 26 & --- & 1/rd. for 6 rds. & 1d3  & --- & 1,500 gp\\
Drow poison & injury & 13 & --- & 1/min. for 2 min. & unconscious 1 min./2d4 hours & 1 save & 75 gp\\
Giant wasp poison & injury & 18 & --- & 1/rd. for 6 rds. & 1d2  & 1 save & 210 gp\\
Greenblood oil & injury & 13 & --- & 1/rd. for 4 rds. & 1  & 1 save & 100 gp\\
Green prismatic poison & spell & varies & --- & 1/rd. & for 6 rds. Death/1  & 2 saves & ---\\
Hemlock & ingested & 18 & 10 min. & 1/min. for 6 min. & 1d6  & 2 saves & 2,500 gp\\
Id moss & ingested & 14 & 10 min. & 1/min. for 6 min. & 1d3  & 1 save & 125 gp\\
Insanity mist & inhaled & 15 & --- & 1/rd. for 6 rds. & 1d3  & 1 save & 1,500 gp\\
King's sleep & ingested & 19 & 1 day & 1/day & 1  & 2 saves & 5,000 gp\\
Large scorpion venom & injury & 17 & --- & 1/rd. for 6 rds. & 1d2  & 1 save & 200 gp\\
Lich dust & ingested & 17 & 10 min. & 1/min. for 6 min. & 1d3  & 2 saves & 400 gp\\
Malyass root paste & contact & 16 & 1 min. & 1/min. for 6 min. & 1d2  & 1 save & 250 gp\\
Medium spider venom & injury & 14 & --- & 1/rd. for 4 rds. & 1d2  & 1 save & 150 gp\\
Nightmare vapor & inhaled & 20 & --- & 1/rd. for 6 rds. & 1  & 2 saves & 1,800 gp\\
Nitharit & contact & 13 & 1 min. & 1/min. for 6 min. & 1d3  & 1 save & 650 gp\\
Oil of taggit & ingested & 15  &  min. & --- & unconscious 1d3 hours & 1 save & 90 gp\\
Purple worm poison & injury & 24 & --- & 1/rd. for 6 rds. & 1d3  & 2 saves & 700 gp\\
Sassone leaf residue & contact & 16 & 1 min. & 1/min. for 6 min. & 2d12 hp/1  & 1 save & 300 gp\\
Shadow essence & injury & 17 & --- & 1/rd. for 6 rds. & 1  & 1 save & 250 gp\\
Small centipede poison & injury & 11 & --- & 1/rd. for 4 rds. & 1  &  save & 90 gp\\
Striped toadstool & ingested & 11 & 10 min. & 1/min. for 4 min. & 1d3  & 1 save & 180 gp\\
Tears of death & contact & 22 & 1 min. & 1/min. for 6 min. & 1d6  & --- & 6,500 gp\\
Terinav root & contact & 16 & 1 min. & 1/min. for 6 min. & 1d3  & 1 save & 400 gp\\
Ungol dust & inhaled & 15 & --- & 1/rd. for 4 rds. & 1  & 1 save & 1,000 gp\\
Wolfsbane & ingested & 16 & 10 min. & 1/min. for 6 min. & 1d3  & 1 save & 500 gp\\
Wyvern poison & injury & 17 & --- & 1/rd. for 6 rds. & 1d4  & 2 saves & 3,000 gp\\
\end{tabular}
\end{table*}

\textbf{Arsenic}
				
\textbf{Type }poison, ingested; \textbf{Save} Fortitude DC 13 
				
\textbf{Onset }10 minutes; \textbf{Frequency} 1/minute for 4 minutes
				
\textbf{Effect }1d2 Con damage; \textbf{Cure} 1 save
				
\textbf{Belladonna}
				
\textbf{Type} poison, ingested; \textbf{Save} Fortitude DC 14 
				
\textbf{Onset }10 minutes; \textbf{Frequency} 1/minute for 6 minutes
				
\textbf{Effect }1d2 Str damage, target can attempt one save to cure a lycanthropy affliction contracted in the past hour;\textbf{ Cure} 1 save
				
\textbf{Black Adder Venom }
				
\textbf{Type} poison, injury; \textbf{Save} Fortitude DC 11 
				
\textbf{Frequency} 1/round for 6 rounds
				
\textbf{Effect }1d2 Con damage;\textbf{ Cure} 1 save
				
\textbf{Black Lotus Extract }
				
\textbf{Type} poison, contact; \textbf{Save} Fortitude DC 20 
				
\textbf{Onset }1 minute; \textbf{Frequency} 1/round for 6 rounds
				
\textbf{Effect }1d6 Con damage;\textbf{ Cure} 2 consecutive saves
				
\textbf{Bloodroot }
				
\textbf{Type} poison, injury; \textbf{Save} Fortitude DC 12 
				
\textbf{Onset }1 round; \textbf{Frequency} 1/round for 4 rounds
				
\textbf{Effect }1 Con damage and 1 Wis damage;\textbf{ Cure} 1 save
				
\textbf{Blue Whinnis }
				
\textbf{Type} poison, injury; \textbf{Save} Fortitude DC 14
				
\textbf{Frequency} 1/round for 2 rounds
				
\textbf{Initial Effect }1 Con damage; \textbf{Secondary Effect} unconsciousness for 1d3 hours;\textbf{ Cure} 1 save
				
\textbf{Burnt Othur Fumes }
				
\textbf{Type} poison, inhaled; \textbf{Save} Fortitude DC 18 
				
\textbf{Frequency} 1/round for 6 rounds
				
\textbf{Initial Effect }1 Con drain; \textbf{Secondary Effect} 1d3 Con damage;\textbf{ Cure} 2 consecutive saves
				
\textbf{Dark Reaver Powder }
				
\textbf{Type} poison, ingested; \textbf{Save} Fortitude DC 18 
				
\textbf{Onset }10 minutes; \textbf{Frequency} 1/minute for 6 minutes
				
\textbf{Effect }1d3 Con damage and 1 Str damage;\textbf{ Cure} 2 consecutive saves
				
\textbf{Deathblade }
				
\textbf{Type} poison, injury; \textbf{Save} Fortitude DC 20 
				
\textbf{Frequency} 1/round for 6 rounds
				
\textbf{Effect }1d3 Con damage;\textbf{ Cure} 2 consecutive saves
				
\textbf{Dragon Bile }
				
\textbf{Type} poison, contact; \textbf{Save} Fortitude DC 26 
				
\textbf{Frequency} 1/round for 6 rounds
				
\textbf{Effect }1d3 Str damage
				
\textbf{Drow Poison }
				
\textbf{Type} poison, injury; \textbf{Save} Fortitude DC 13 
				
\textbf{Frequency} 1/minute for 2 minutes
				
\textbf{Initial Effect }unconsciousness for 1 minute; \textbf{Secondary Effect }unconsciousness for 2d4 hours;\textbf{ Cure} 1 save
				
\textbf{Giant Wasp Poison }
				
\textbf{Type} poison, injury; \textbf{Save} Fortitude DC 18 
				
\textbf{Frequency} 1/round for 6 rounds
				
\textbf{Effect }1d2 Dex damage;\textbf{ Cure} 1 save
				
\textbf{Greenblood Oil }
				
\textbf{Type} poison, injury; \textbf{Save} Fortitude DC 13 
				
\textbf{Frequency} 1/round for 4 rounds
				
\textbf{Effect }1 Con damage;\textbf{ Cure} 1 save
				
\textbf{Green Prismatic Poison}
				
\textbf{Type} poison, spell; \textbf{Save} Fort DC varies by spell
				
\textbf{Frequency} 1/round for 6 rounds
				
\textbf{Initial Effect }death;\textbf{ Secondary Effect }1 Con damage; \textbf{Cure} 2 consecutive saves. See \textit{prismatic sphere, prismatic spray, }or \textit{prismatic wall} for more details.
				
\textbf{Hemlock}
				
\textbf{Type} poison, ingested; \textbf{Save} Fortitude DC 18
				
\textbf{Onset }10 minutes; \textbf{Frequency} 1/minute for 6 minutes
				
\textbf{Effect }1d6 Dex damage, creatures reduced to 0 Dexterity suffocate;\textbf{ Cure} 2 consecutive saves
				
\textbf{Id Moss }
				
\textbf{Type} poison, ingested; \textbf{Save} Fortitude DC 14 
				
\textbf{Onset }10 minutes; \textbf{Frequency} 1/minute for 6 minutes
				
\textbf{Effect }1d3 Int damage;\textbf{ Cure} 1 save
				
\textbf{Insanity Mist }
				
\textbf{Type} poison, inhaled; \textbf{Save} Fortitude DC 15 
				
\textbf{Frequency} 1/rounds for 6 rounds
				
\textbf{Effect }1d3 Wis damage;\textbf{ Cure} 1 save
				
\textbf{King's Sleep }
				
\textbf{Type} poison, ingested; \textbf{Save} Fortitude DC 19 
				
\textbf{Onset }1 day; \textbf{Frequency} 1/day
				
\textbf{Effect }1 Con drain;\textbf{ Cure} 2 consecutive saves
				
\textbf{Large Scorpion Venom }
				
\textbf{Type} poison, injury; \textbf{Save} Fortitude DC 17 
				
\textbf{Frequency} 1/round for 6 rounds
				
\textbf{Effect }1d2 Str damage;\textbf{ Cure} 1 save
				
\textbf{Lich Dust }
				
\textbf{Type} poison, ingested; \textbf{Save} Fortitude DC 17 
				
\textbf{Onset }10 minutes; \textbf{Frequency} 1/minute for 6 minutes
				
\textbf{Effect }1d3 Str damage;\textbf{ Cure} 2 consecutive saves
				
\textbf{Malyass Root Paste }
				
\textbf{Type} poison, contact; \textbf{Save} Fortitude DC 16 
				
\textbf{Onset }1 minute; \textbf{Frequency} 1/minute for 6 minutes
				
\textbf{Effect }1d2 Dex damage;\textbf{ Cure} 1 save
				
\textbf{Medium Spider Venom }
				
\textbf{Type} poison, injury; \textbf{Save} Fortitude DC 14 
				
\textbf{Frequency} 1/round for 4 rounds
				
\textbf{Effect }1d2 Str damage;\textbf{ Cure} 1 save
				
\textbf{Nightmare Vapor }
				
\textbf{Type} poison, inhaled; \textbf{Save} Fortitude DC 20 
				
\textbf{Frequency} 1/round for 6 rounds
				
\textbf{Effect }1 Wis damage and confused for 1 round;\textbf{ Cure} 2 consecutive saves
				
\textbf{Nitharit }
				
\textbf{Type} poison, contact; \textbf{Save} Fortitude DC 13 
				
\textbf{Onset }1 minute; \textbf{Frequency} 1/minute for 6 minutes
				
\textbf{Effect }1d3 Con damage;\textbf{ Cure} 1 save
				
\textbf{Oil of Taggit }
				
\textbf{Type} poison, ingested; \textbf{Save} Fortitude DC 15 
				
\textbf{Onset }1 minute
				
\textbf{Effect }unconsciousness for 1d3 hours;\textbf{ Cure} 1 save
				
\textbf{Purple Worm Poison }
				
\textbf{Type} poison, injury; \textbf{Save} Fortitude DC 24 
				
\textbf{Frequency} 1/round for 6 rounds
				
\textbf{Effect }1d3 Str damage;\textbf{ Cure} 2 consecutive saves
				
\textbf{Sassone Leaf Residue }
				
\textbf{Type} poison, contact; \textbf{Save} Fortitude DC 16 
				
\textbf{Onset }1 minute; \textbf{Frequency} 1/minute for 6 minutes
				
\textbf{Initial Effect }2d12 hit point damage; \textbf{Secondary Effect} 1 Con damage;\textbf{ Cure} 1 save
				
\textbf{Shadow Essence }
				
\textbf{Type} poison, injury; \textbf{Save} Fortitude DC 17 
				
\textbf{Frequency} 1/round for 6 rounds
				
\textbf{Initial Effect }1 Str drain; \textbf{Secondary Effect }1d2 Str damage;\textbf{ Cure} 1 save
				
\textbf{Small Centipede Poison }
				
\textbf{Type} poison, injury; \textbf{Save} Fortitude DC 11 
				
\textbf{Frequency} 1/round for 4 rounds
				
\textbf{Effect }1 Dex damage;\textbf{ Cure} 1 save
				
\textbf{Striped Toadstool }
				
\textbf{Type} poison, ingested; \textbf{Save} Fortitude DC 11 
				
\textbf{Onset }10 minutes; \textbf{Frequency} 1/minute for 4 minutes
				
\textbf{Effect }1d3 Wis damage and 1 Int damage;\textbf{ Cure} 1 save
				
\textbf{Tears of Death }
				
\textbf{Type} poison, contact; \textbf{Save} Fortitude DC 22 
				
\textbf{Onset }1 minute; \textbf{Frequency} 1/minute for 6 minutes
				
\textbf{Effect }1d6 Con damage and paralyzed for 1 minute
				
\textbf{Terinav Root }
				
\textbf{Type} poison, contact; \textbf{Save} Fortitude DC 16 
				
\textbf{Onset }1 minute; \textbf{Frequency} 1/minute for 6 minutes
				
\textbf{Effect }1d3 Dex damage;\textbf{ Cure} 1 save
				
\textbf{Ungol Dust }
				
\textbf{Type} poison, inhaled; \textbf{Save} Fortitude DC 15 
				
\textbf{Frequency} 1/round for 4 rounds
				
\textbf{Initial Effect }1 Cha drain; \textbf{Secondary Effect }1d2 Cha damage;\textbf{ Cure} 1 save
				
\textbf{Wolfsbane}
				
\textbf{Type} poison, ingested; \textbf{Save} Fortitude DC 16 
				
\textbf{Onset }10 minute; \textbf{Frequency} 1/minute for 6 minutes
				
\textbf{Effect }1d3 Con damage;\textbf{ Cure} 1 save
				
\textbf{Wyvern Poison }
				
\textbf{Type} poison, injury; \textbf{Save} Fortitude DC 17 
				
\textbf{Frequency} 1/round for 6 rounds
				
\textbf{Effect }1d4 Con damage;\textbf{ Cure} 2 consecutive saves
				
\subsection{Blindsight and Blindsense}

				
Some creatures possess blindsight, the extraordinary ability to use a nonvisual sense (or a combination senses) to operate effectively without vision. Such senses may include sensitivity to vibrations, acute scent, keen hearing, or echolocation. This makes invisibility and concealment (even magical darkness) irrelevant to the creature (though it still can't see ethereal creatures). This ability operates out to a range specified in the creature description.
				\begin{itemize}\item  Blindsight never allows a creature to distinguish color or visual contrast. A creature cannot read with blindsight.
				\item  Blindsight does not subject a creature to gaze attacks (even though darkvision does).
				\item  Blinding attacks do not penalize creatures that use blindsight.
				\item  Deafening attacks thwart blindsight if it relies on hearing.
				\item  Blindsight works underwater but not in a vacuum.
				\item  Blindsight negates displacement and blur effects.
\end{itemize}
				
\textbf{Blindsense}: Other creatures have blindsense, a lesser ability that lets the creature notice things it cannot see, but without the precision of blindsight. The creature with blindsense usually does not need to make Perception checks to notice and locate creatures within range of its blindsense ability, provided that it has line of effect to that creature. Any opponent that cannot be seen has total concealment (50\% miss chance) against a creature with blindsense, and the blindsensing creature still has the normal miss chance when attacking foes that have concealment. Visibility still affects the movement of a creature with blindsense. A creature with blindsense is still denied its Dexterity bonus to Armor Class against attacks from creatures it cannot see.
				
\subsection{Channel Resistance}

				
Creatures with channel resistance gain a bonus on Will saves made against channeled energy. They add their bonus to any Will saves made to halve the damage and resist the effect.
				
\subsection{Charm and Compulsion}

				
Many abilities and spells can cloud the minds of characters and monsters, leaving them unable to tell friend from foe---or worse yet, deceiving them into thinking that their former friends are now their worst enemies. Two general types of enchantments affect characters and creatures: charms and compulsions.
				
Charming another creature gives the charming character the ability to befriend and suggest courses of action to his minion, but the servitude is not absolute or mindless. Charms of this type include the various \textit{charm }spells and some monster abilities. Essentially, a \textit{charmed }character retains free will but makes choices according to a skewed view of the world.
				\begin{itemize}\item  A charmed creature doesn't gain any magical ability to understand his new friend's language.
				\item  A charmed character retains his original alignment and allegiances, generally with the exception that he now regards the charming creature as a dear friend and will give great weight to his suggestions and directions.
				\item  A charmed character fights his former allies only if they threaten his new friend, and even then he uses the least lethal means at his disposal as long as these tactics show any possibility of success (just as he would in a fight with an actual friend).
				\item  A charmed character is entitled to an opposed Charisma check against his master in order to resist instructions or commands that would make him do something he wouldn't normally do even for a close friend. If he succeeds, he decides not to go along with that order but remains charmed\textit{.}
				\item  A charmed character never obeys a command that is obviously suicidal or grievously harmful to him.
				\item  If the charming creature commands his minion to do something that the influenced character would be violently opposed to, the subject may attempt a new saving throw to break free of the influence altogether.
				\item  A charmed character who is openly attacked by the creature who charmed him or by that creature's apparent allies is automatically freed of the spell or effect.
\end{itemize}
				
Compulsion is a different matter altogether. A compulsion overrides the subject's free will in some way or simply changes the way the subject's mind works. A charm makes the subject a friend of the caster; a compulsion makes the subject obey the caster.
				
Regardless of whether a character is charmed or compelled, he does not volunteer information or tactics that his master doesn't ask for.
				
\subsection{Damage Reduction}

				
Some magic creatures have the supernatural ability to instantly heal damage from weapons or ignore blows altogether as though they were invulnerable.
				
The numerical part of a creature's damage reduction (or DR) is the amount of damage the creature ignores from normal attacks. Usually, a certain type of weapon can overcome this reduction (see Overcoming DR). This information is separated from the damage reduction number by a slash. For example, DR 5/magic means that a creature takes 5 less points of damage from all weapons that are not magic. If a dash follows the slash, then the damage reduction is effective against any attack that does not ignore damage reduction.
				
Whenever damage reduction completely negates the damage from an attack, it also negates most special effects that accompany the attack, such as injury poison, a monk's stunning, and injury-based disease. Damage reduction does not negate touch attacks, energy damage dealt along with an attack, or energy drains. Nor does it affect poisons or diseases delivered by inhalation, ingestion, or contact. 
				
Attacks that deal no damage because of the target's damage reduction do not disrupt spells.
				
Spells, spell-like abilities, and energy attacks (even nonmagical fire) ignore damage reduction.
				
Sometimes damage reduction represents instant healing. Sometimes it represents the creature's tough hide or body. In either case, other characters can see that conventional attacks won't work.
				
If a creature has damage reduction from more than one source, the two forms of damage reduction do not stack. Instead, the creature gets the benefit of the best damage reduction in a given situation. 
				
\textbf{Overcoming DR}: Damage reduction may be overcome by special materials, magic weapons (any weapon with a +1 or higher enhancement bonus, not counting the enhancement from masterwork quality), certain types of weapons (such as slashing or bludgeoning), and weapons imbued with an alignment. 
				
Ammunition fired from a projectile weapon with an enhancement bonus of +1 or higher is treated as a magic weapon for the purpose of overcoming damage reduction. Similarly, ammunition fired from a projectile weapon with an alignment gains the alignment of that projectile weapon (in addition to any alignment it may already have).
				
Weapons with an enhancement bonus of +3 or greater can ignore some types of damage reduction, regardless of their actual material or alignment. The following table shows what type of enhancement bonus is needed to overcome some common types of damage reduction.

\begin{tabular}{ll}
\textbf{DR Type} & \textbf{Weapon Enhancement Bonus Equivalent} \\
cold iron/silver & +3\\
adamantine* & +4 \\
alignment-based & +5\\
\end{tabular}
* Note that this does not give the ability to ignore hardness, like an actual adamantine weapon does
% </tfoot colspan="2">

				
\subsection{Darkvision}

				
Darkvision is the extraordinary ability to see with no light source at all, out to a range specified for the creature. Darkvision is black-and-white only (colors cannot be discerned). It does not allow characters to see anything that they could not see otherwise---invisible objects are still invisible, and illusions are still visible as what they seem to be. Likewise, darkvision subjects a creature to gaze attacks normally. The presence of light does not spoil darkvision.
				
\subsection{Death Attacks}

				
In most cases, a death attack allows the victim a Fortitude save to avoid the effect, but if the save fails, the creature takes a large amount of damage, which might cause it to die instantly.
				\begin{itemize}\item  \textit{Raise dead }doesn't work on someone killed by a death attack or effect.
				\item  Death attacks slay instantly. A victim cannot be made stable and thereby kept alive.
				\item  In case it matters, a dead character, no matter how he died, has hit points equal to or less than his negative Constitution score.
				\item  The spell \textit{death ward }protects against these attacks.
\end{itemize}
				
\subsection{Energy Drain and Negative Levels}

				
Some spells and a number of undead creatures have the ability to drain away life and energy; this dreadful attack results in \texttt{{}"{}}negative levels.\texttt{{}"{}} These cause a character to take a number of penalties.
				
For each negative level a creature has, it takes a cumulative --1 penalty on all ability checks, attack rolls, combat maneuver checks, Combat Maneuver Defense, saving throws, and skill checks. In addition, the creature reduces its current and total hit points by 5 for each negative level it possesses. The creature is also treated as one level lower for the purpose of level-dependent variables (such as spellcasting) for each negative level possessed. Spellcasters do not lose any prepared spells or slots as a result of negative levels. If a creature's negative levels equal or exceed its total Hit Dice, it dies.
				
A creature with temporary negative levels receives a new saving throw to remove the negative level each day. The DC of this save is the same as the effect that caused the negative levels.
				
Some abilities and spells (such as \textit{raise dead}) bestow permanent level drain on a creature. These are treated just like temporary negative levels, but they do not allow a new save each day to remove them. Level drain can be removed through spells like \textit{restoration}. Permanent negative levels remain after a dead creature is restored to life. A creature whose permanent negative levels equal its Hit Dice cannot be brought back to life through spells like \textit{raise dead} and \textit{resurrection} without also receiving a \textit{restoration }spell, cast the round after it is restored to life.
				
\subsection{Energy Immunity and Vulnerability}

				
A creature with energy immunity never takes damage from that energy type. Vulnerability means the creature takes half again as much (+50\%) damage as normal from that energy type, regardless of whether a saving throw is allowed or if the save is a success or failure.
				
\subsection{Energy Resistance}

				
A creature with resistance to energy has the ability (usually extraordinary) to ignore some damage of a certain type per attack, but it does not have total immunity. 
				
Each resistance ability is defined by what energy type it resists and how many points of damage are resisted. It doesn't matter whether the damage has a mundane or magical source.
				
When resistance completely negates the damage from an energy attack, the attack does not disrupt a spell. This resistance does not stack with the resistance that a spell might provide.
				
\subsection{Fear}

				
Spells, magic items, and certain monsters can affect characters with fear. In most cases, the character makes a Will saving throw to resist this effect, and a failed roll means that the character is shaken, frightened, or panicked.
				
\textbf{Shaken}: Characters who are shaken take a --2 penalty on attack rolls, saving throws, skill checks, and ability checks.
				
\textbf{Frightened}: Characters who are frightened are shaken, and in addition they flee from the source of their fear as quickly as they can. They can choose the paths of their flight. Other than that stipulation, once they are out of sight (or hearing) of the source of their fear, they can act as they want. If the duration of their fear continues, however, characters can be forced to flee if the source of their fear presents itself again. Characters unable to flee can fight (though they are still shaken).
				
\textbf{Panicked}: Characters who are panicked are shaken, and they run away from the source of their fear as quickly as they can, dropping whatever they are holding. Other than running away from the source, their paths are random. They flee from all other dangers that confront them rather than facing those dangers. Once they are out of sight (or hearing) of any source of danger, they can act as they want. Panicked characters cower if they are prevented from fleeing.
				
\textbf{Becoming Even More Fearful}: Fear effects are cumulative. A shaken character who is made shaken again becomes frightened, and a shaken character who is made frightened becomes panicked instead. A frightened character who is made shaken or frightened becomes panicked instead.
				
\subsection{Invisibility}

				
The ability to move about unseen is not foolproof. While they can't be seen, invisible creatures can be heard, smelled, or felt. 
				
Invisibility makes a creature undetectable by vision, including darkvision.
				
Invisibility does not, by itself, make a creature immune to critical hits, but it does make the creature immune to extra damage from being a ranger's favored enemy and from sneak attacks.
				
A creature can generally notice the presence of an active invisible creature within 30 feet with a DC 20 Perception check. The observer gains a hunch that \texttt{{}"{}}something's there\texttt{{}"{}} but can't see it or target it accurately with an attack. It's practically impossible (+20 DC) to pinpoint an invisible creature's location with a Perception check. Even once a character has pinpointed the square that contains an invisible creature, the creature still benefits from total concealment (50\% miss chance). There are a number of modifiers that can be applied to this DC if the invisible creature is moving or engaged in a noisy activity.
\begin{tabular}{ll}
\textbf{Invisible creature is...} & \textbf{Perception DC Modifier} \\
In combat or speaking             & --20                             \\
Moving at half speed              & --5                              \\
Moving at full speed              & --10                             \\
Running or charging               & --20                             \\
Not moving                        & +20                             \\
Using Stealth                     & Stealth check +20               \\
Some distance away                & +1 per 10 feet                  \\
Behind an obstacle (door)         & +5                              \\
Behind an obstacle (stone wall)   & +15                            
\end{tabular}
				
A creature can grope about to find an invisible creature. A character can make a touch attack with his hands or a weapon into two adjacent 5-foot squares using a standard action. If an invisible target is in the designated area, there is a 50\% miss chance on the touch attack. If successful, the groping character deals no damage but has successfully pinpointed the invisible creature's current location. If the invisible creature moves, its location, obviously, is once again unknown.
				
If an invisible creature strikes a character, the character struck knows the location of the creature that struck him (until, of course, the invisible creature moves). The only exception is if the invisible creature has a reach greater than 5 feet. In this case, the struck character knows the general location of the creature but has not pinpointed the exact location.
				
If a character tries to attack an invisible creature whose location he has pinpointed, he attacks normally, but the invisible creature still benefits from full concealment (and thus a 50\% miss chance). A particularly large and slow invisible creature might get a smaller miss chance.
				
If a character tries to attack an invisible creature whose location he has not pinpointed, have the player choose the space where the character will direct the attack. If the invisible creature is there, conduct the attack normally. If the enemy's not there, roll the miss chance as if it were there and tell him that the character has missed, regardless of the result. That way the player doesn't know whether the attack missed because the enemy's not there or because you successfully rolled the miss chance.
				
If an invisible character picks up a visible object, the object remains visible. An invisible creature can pick up a small visible item and hide it on his person (tucked in a pocket or behind a cloak) and render it effectively invisible. One could coat an invisible object with flour to at least keep track of its position (until the flour falls off or blows away). 
				
Invisible creatures leave tracks. They can be tracked normally. Footprints in sand, mud, or other soft surfaces can give enemies clues to an invisible creature's location.
				
An invisible creature in the water displaces water, revealing its location. The invisible creature, however, is still hard to see and benefits from concealment.
				
A creature with the scent ability can detect an invisible creature as it would a visible one.
				
A creature with the Blind-Fight feat has a better chance to hit an invisible creature. Roll the miss chance twice, and he misses only if both rolls indicate a miss. (Alternatively, make one 25\% miss chance roll rather than two 50\% miss chance rolls.)
				
A creature with blindsight can attack (and otherwise interact with) creatures regardless of invisibility.
				
An invisible burning torch still gives off light, as does an invisible object with a \textit{light} or similar spell cast upon it.
				
Ethereal creatures are invisible. Since ethereal creatures are not materially present, Perception checks, scent, Blind-Fight, and blindsight don't help locate them. Incorporeal creatures are often invisible. Scent, Blind-Fight, and blindsight don't help creatures find or attack invisible, incorporeal creatures, but Perception checks can help.
				
Invisible creatures cannot use gaze attacks.
				
Invisibility does not thwart divination spells.
				
Since some creatures can detect or even see invisible creatures, it is helpful to be able to hide even when invisible.
				
\subsection{Low-Light Vision}

				
Characters with low-light vision have eyes that are so sensitive to light that they can see twice as far as normal in dim light. Low-light vision is color vision. A spellcaster with low-light vision can read a scroll as long as even the tiniest candle flame is next to him as a source of light.
				
Characters with low-light vision can see outdoors on a moonlit night as well as they can during the day.
				
\subsection{Paralysis}

				
Some monsters and spells have the supernatural or spell-like ability to paralyze their victims, immobilizing them through magical means. Paralysis from poison is discussed in the Afflictions section.
				
A paralyzed character cannot move, speak, or take any physical action. He is rooted to the spot, frozen and helpless. Not even friends can move his limbs. He may take purely mental actions, such as casting a spell with no components.
				
A winged creature flying in the air at the time that it becomes paralyzed cannot flap its wings and falls. A swimmer can't swim and may drown.
				
\subsection{Scent}

				
This extraordinary ability lets a creature detect approaching enemies, sniff out hidden foes, and track by sense of smell.
				
A creature with the scent ability can detect opponents by sense of smell, generally within 30 feet. If the opponent is upwind, the range is 60 feet. If it is downwind, the range is 15 feet. Strong scents, such as smoke or rotting garbage, can be detected at twice the ranges noted above. Overpowering scents, such as skunk musk or troglodyte stench, can be detected at three times these ranges.
				
The creature detects another creature's presence but not its specific location. Noting the direction of the scent is a move action. If the creature moves within 5 feet (1 square) of the scent's source, the creature can pinpoint the area that the source occupies, even if it cannot be seen.
				
A creature with the Survival skill and the scent ability can follow tracks by smell, making a Survival check to find or follow a track. A creature with the scent ability can attempt to follow tracks using Survival untrained. The typical DC for a fresh trail is 10. The DC increases or decreases depending on how strong the quarry's odor is, the number of creatures, and the age of the trail. For each hour that the trail is cold, the DC increases by 2. The ability otherwise follows the rules for the Survival skill in regards to tracking. Creatures tracking by scent ignore the effects of surface conditions and poor visibility.
				
Creatures with the scent ability can identify familiar odors just as humans do familiar sights.
				
Water, particularly running water, ruins a trail for air-breathing creatures. Water-breathing creatures that have the scent ability, however, can use it in the water easily.
				
False, powerful odors can easily mask other scents. The presence of such an odor completely spoils the ability to properly detect or identify creatures, and the base Survival DC to track becomes 20 rather than 10.
				
\subsection{Spell Resistance}

				
Spell resistance is the extraordinary ability to avoid being affected by spells. Some spells also grant spell resistance.
				
To affect a creature that has spell resistance, a spellcaster must make a caster level check (1d20 + caster level) at least equal to the creature's spell resistance. The defender's spell resistance is like an Armor Class against magical attacks. If the caster fails the check, the spell doesn't affect the creature. The possessor does not have to do anything special to use spell resistance. The creature need not even be aware of the threat for its spell resistance to operate.
				
Only spells and spell-like abilities are subject to spell resistance. Extraordinary and supernatural abilities (including enhancement bonuses on magic weapons) are not. A creature can have some abilities that are subject to spell resistance and some that are not. Even some spells ignore spell resistance; see When Spell Resistance Applies, below. 
				
A creature can voluntarily lower its spell resistance. Doing so is a standard action that does not provoke an attack of opportunity. Once a creature lowers its resistance, it remains down until the creature's next turn. At the beginning of the creature's next turn, the creature's spell resistance automatically returns unless the creature intentionally keeps it down (also a standard action that does not provoke an attack of opportunity).
				
A creature's spell resistance never interferes with its own spells, items, or abilities.
				
A creature with spell resistance cannot impart this power to others by touching them or standing in their midst. Only the rarest of creatures and a few magic items have the ability to bestow spell resistance upon another.
				
Spell resistance does not stack, but rather overlaps. 
				
\subsection{When Spell Resistance Applies}

				
Each spell includes an entry that indicates whether spell resistance applies to the spell. In general, whether spell resistance applies depends on what the spell does.
				
\textbf{Targeted Spells}: Spell resistance applies if the spell is targeted at the creature. Some individually targeted spells can be directed at several creatures simultaneously. In such cases, a creature's spell resistance applies only to the portion of the spell actually targeted at that creature. If several different resistant creatures are subjected to such a spell, each checks its spell resistance separately.
				
\textbf{Area Spells}: Spell resistance applies if the resistant creature is within the spell's area. It protects the resistant creature without affecting the spell itself.
				
\textbf{Effect Spells}: Most effect spells summon or create something and are not subject to spell resistance. Sometimes, however, spell resistance applies to effect spells, usually to those that act upon a creature more or less directly, such as \textit{web.}
				
Spell resistance can protect a creature from a spell that's already been cast. Check spell resistance when the creature is first affected by the spell.
				
Check spell resistance only once for any particular casting of a spell or use of a spell-like ability. If spell resistance fails the first time, it fails each time the creature encounters that same casting of the spell. Likewise, if the spell resistance succeeds the first time, it always succeeds. If the creature has voluntarily lowered its spell resistance and is then subjected to a spell, the creature still has a single chance to resist that spell later, when its spell resistance is back up.
				
Spell resistance has no effect unless the energy created or released by the spell actually goes to work on the resistant creature's mind or body. If the spell acts on anything else and the creature is affected as a consequence, no roll is required. Spell-resistant creatures can be harmed by a spell when they are not being directly affected. 
				
Spell resistance does not apply if an effect fools the creature's senses or reveals something about the creature.
				
Magic actually has to be working for spell resistance to apply. Spells that have instantaneous durations but lasting results aren't subject to spell resistance unless the resistant creature is exposed to the spell the instant it is cast. 
				
\subsection{Successful Spell Resistance}

				
Spell resistance prevents a spell or a spell-like ability from affecting or harming the resistant creature, but it never removes a magical effect from another creature or negates a spell's effect on another creature. Spell resistance prevents a spell from disrupting another spell.
				
Against an ongoing spell that has already been cast, a failed check against spell resistance allows the resistant creature to ignore any effect the spell might have. The magic continues to affect others normally.
				
\section{Conditions}

				
If more than one condition affects a character, apply them all. If effects can't combine, apply the most severe effect.
				
\textbf{Bleed}: A creature that is taking bleed damage takes the listed amount of damage at the beginning of its turn. Bleeding can be stopped by a DC 15 Heal check or through the application of any spell that cures hit point damage (even if the bleed is ability damage). Some bleed effects cause ability damage or even ability drain. Bleed effects do not stack with each other unless they deal different kinds of damage. When two or more bleed effects deal the same kind of damage, take the worse effect. In this case, ability drain is worse than ability damage.
				
\textbf{Blinded}: The creature cannot see. It takes a --2 penalty to Armor Class, loses its Dexterity bonus to AC (if any), and takes a --4 penalty on most Strength- and Dexterity-based skill checks and on opposed Perception skill checks. All checks and activities that rely on vision (such as reading and Perception checks based on sight) automatically fail. All opponents are considered to have total concealment (50\% miss chance) against the blinded character. Blind creatures must make a DC 10 Acrobatics skill check to move faster than half speed. Creatures that fail this check fall prone. Characters who remain blinded for a long time grow accustomed to these drawbacks and can overcome some of them.
				
\textbf{Broken}: Items that have taken damage in excess of half their total hit points gain the broken condition, meaning they are less effective at their designated task. The broken condition has the following effects, depending upon the item.
				\begin{itemize}\item  If the item is a weapon, any attacks made with the item suffer a --2 penalty on attack and damage rolls. Such weapons only score a critical hit on a natural 20 and only deal \mbox{$\times$}2 damage on a confirmed critical hit.
				\item  If the item is a suit of armor or a shield, the bonus it grants to AC is halved, rounding down. Broken armor doubles its armor check penalty on skills.
				\item  If the item is a tool needed for a skill, any skill check made with the item takes a --2 penalty.
				\item  If the item is a wand or staff, it uses up twice as many charges when used.
				\item  If the item does not fit into any of these categories, the broken condition has no effect on its use. Items with the broken condition, regardless of type, are worth 75\% of their normal value. If the item is magical, it can only be repaired with a \textit{mending} or \textit{make whole }spell cast by a character with a caster level equal to or higher than the item's. Items lose the broken condition if the spell restores the object to half its original hit points or higher\textit{. }Non-magical items can be repaired in a similar fashion, or through the Craft skill used to create it. Generally speaking, this requires a DC 20 Craft check and 1 hour of work per point of damage to be repaired. Most craftsmen charge one-tenth the item's total cost to repair such damage (more if the item is badly damaged or ruined).
\end{itemize}
				
\textbf{Confused}: A confused creature is mentally befuddled and cannot act normally. A confused creature cannot tell the difference between ally and foe, treating all creatures as enemies. Allies wishing to cast a beneficial spell that requires a touch on a confused creature must succeed on a melee touch attack. If a confused creature is attacked, it attacks the creature that last attacked it until that creature is dead or out of sight.
				
Roll on the following table at the beginning of each confused subject's turn each round to see what the subject does in that round. 


\begin{tabularx}{\linewidth}{lX}
\textbf{d\%} & \textbf{Behavior} \\
01--25 & Act normally.\\
26--50 & Do nothing but babble incoherently.\\
51--75 & Deal 1d8 points of damage + Str modifier to self with item in hand.\\
76--100 & Attack nearest creature (for this purpose, a familiar counts as part of the subject's self).\\
\end{tabularx}

A confused creature who can't carry out the indicated action does nothing but babble incoherently. Attackers are not at any special advantage when attacking a confused creature. Any confused creature who is attacked automatically attacks its attackers on its next turn, as long as it is still confused when its turn comes. Note that a confused creature will not make attacks of opportunity against anything that it is not already devoted to attacking (either because of its most recent action or because it has just been attacked).
				
\textbf{Cowering}: The character is frozen in fear and can take no actions. A cowering character takes a --2 penalty to Armor Class and loses his Dexterity bonus (if any).
				
\textbf{Dazed}: The creature is unable to act normally. A dazed creature can take no actions, but has no penalty to AC.
				
A dazed condition typically lasts 1 round.
				
\textbf{Dazzled}: The creature is unable to see well because of overstimulation of the eyes. A dazzled creature takes a --1 penalty on attack rolls and sight-based Perception checks.
				
\textbf{Dead}: The character's hit points are reduced to a negative amount equal to his Constitution score, his Constitution drops to 0, or he is killed outright by a spell or effect. The character's soul leaves his body. Dead characters cannot benefit from normal or magical healing, but they can be restored to life via magic. A dead body decays normally unless magically preserved, but magic that restores a dead character to life also restores the body either to full health or to its condition at the time of death (depending on the spell or device). Either way, resurrected characters need not worry about rigor mortis, decomposition, and other conditions that affect dead bodies.
				
\textbf{Deafened}: A deafened character cannot hear. He takes a --4 penalty on initiative checks, automatically fails Perception checks based on sound, takes a --4 penalty on opposed Perception checks, and has a 20\% chance of spell failure when casting spells with verbal components. Characters who remain deafened for a long time grow accustomed to these drawbacks and can overcome some of them.
				
\textbf{Disabled}: A character with 0 hit points, or one who has negative hit points but has become stable and conscious, is disabled. A disabled character may take a single move action or standard action each round (but not both, nor can he take full-round actions, but he can still take swift, immediate, and free actions). He moves at half speed. Taking move actions doesn't risk further injury, but performing any standard action (or any other action the GM deems strenuous, including some free actions such as casting a quickened spell) deals 1 point of damage after the completion of the act. Unless the action increased the disabled character's hit points, he is now in negative hit points and dying.
				
A disabled character with negative hit points recovers hit points naturally if he is being helped. Otherwise, each day he can attempt a DC 10 Constitution check after resting for 8 hours, to begin recovering hit points naturally. The character takes a penalty on this roll equal to his negative hit point total. Failing this check causes the character to lose 1 hit point, but this does not cause the character to become unconscious. Once a character makes this check, he continues to heal naturally and is no longer in danger of losing hit points naturally.
				
\textbf{Dying}: A dying creature is unconscious and near death. Creatures that have negative hit points and have not stabilized are dying. A dying creature can take no actions. On the character's next turn, after being reduced to negative hit points (but not dead), and on all subsequent turns, the character must make a DC 10 Constitution check to become stable. The character takes a penalty on this roll equal to his negative hit point total. A character that is stable does not need to make this check. A natural 20 on this check is an automatic success. If the character fails this check, he loses 1 hit point. If a dying creature has an amount of negative hit points equal to its Constitution score, it dies.
				
\textbf{Energy Drained}: The character gains one or more negative levels, which might become permanent. If the subject has at least as many negative levels as Hit Dice, he dies. See Energy Drain and Negative levels for more information. 
				
\textbf{Entangled}: The character is ensnared. Being entangled impedes movement, but does not entirely prevent it unless the bonds are anchored to an immobile object or tethered by an opposing force. An entangled creature moves at half speed, cannot run or charge, and takes a --2 penalty on all attack rolls and a --4 penalty to Dexterity. An entangled character who attempts to cast a spell must make a concentration check (DC 15 + spell level) or lose the spell. 
				
\textbf{Exhausted}: An exhausted character moves at half speed, cannot run or charge, and takes a --6 penalty to Strength and Dexterity. After 1 hour of complete rest, an exhausted character becomes fatigued. A fatigued character becomes exhausted by doing something else that would normally cause fatigue.
				
\textbf{Fascinated}: A fascinated creature is entranced by a supernatural or spell effect. The creature stands or sits quietly, taking no actions other than to pay attention to the fascinating effect, for as long as the effect lasts. It takes a --4 penalty on skill checks made as reactions, such as Perception checks. Any potential threat, such as a hostile creature approaching, allows the fascinated creature a new saving throw against the fascinating effect. Any obvious threat, such as someone drawing a weapon, casting a spell, or aiming a ranged weapon at the fascinated creature, automatically breaks the effect. A fascinated creature's ally may shake it free of the spell as a standard action. 
				
\textbf{Fatigued}: A fatigued character can neither run nor charge and takes a --2 penalty to Strength and Dexterity. Doing anything that would normally cause fatigue causes the fatigued character to become exhausted. After 8 hours of complete rest, fatigued characters are no longer fatigued. 
				
\textbf{Flat-Footed}: A character who has not yet acted during a combat is flat-footed, unable to react normally to the situation. A flat-footed character loses his Dexterity bonus to AC (if any) and cannot make attacks of opportunity. 
				
\textbf{Frightened}: A frightened creature flees from the source of its fear as best it can. If unable to flee, it may fight. A frightened creature takes a --2 penalty on all attack rolls, saving throws, skill checks, and ability checks. A frightened creature can use special abilities, including spells, to flee; indeed, the creature must use such means if they are the only way to escape. 
				
Frightened is like shaken, except that the creature must flee if possible. Panicked is a more extreme state of fear.
				
\textbf{Grappled}: A grappled creature is restrained by a creature, trap, or effect. Grappled creatures cannot move and take a --4 penalty to Dexterity. A grappled creature takes a --2 penalty on all attack rolls and combat maneuver checks, except those made to grapple or escape a grapple. In addition, grappled creatures can take no action that requires two hands to perform. A grappled character who attempts to cast a spell or use a spell-like ability must make a concentration check (DC 10 + grappler's CMB + spell level), or lose the spell. Grappled creatures cannot make attacks of opportunity.
				
A grappled creature cannot use Stealth to hide from the creature grappling it, even if a special ability, such as hide in plain sight, would normally allow it to do so. If a grappled creature becomes invisible, through a spell or other ability, it gains a +2 circumstance bonus on its CMD to avoid being grappled, but receives no other benefit.
				
\textbf{Helpless}: A helpless character is paralyzed, held, bound, sleeping, unconscious, or otherwise completely at an opponent's mercy. A helpless target is treated as having a Dexterity of 0 (--5 modifier). Melee attacks against a helpless target get a +4 bonus (equivalent to attacking a prone target). Ranged attacks get no special bonus against helpless targets. Rogues can sneak attack helpless targets. 
				
As a full-round action, an enemy can use a melee weapon to deliver a coup de grace to a helpless foe. An enemy can also use a bow or crossbow, provided he is adjacent to the target. The attacker automatically hits and scores a critical hit. (A rogue also gets his sneak attack damage bonus against a helpless foe when delivering a coup de grace.) If the defender survives, he must make a Fortitude save (DC 10 + damage dealt) or die. Delivering a coup de grace provokes attacks of opportunity. 
				
Creatures that are immune to critical hits do not take critical damage, nor do they need to make Fortitude saves to avoid being killed by a coup de grace.
				
\textbf{Incorporeal}: Creatures with the incorporeal condition do not have a physical body. Incorporeal creatures are immune to all nonmagical attack forms. Incorporeal creatures take half damage (50\%) from magic weapons, spells, spell-like effects, and supernatural effects. Incorporeal creatures take full damage from other incorporeal creatures and effects, as well as all force effects. 
				
\textbf{Invisible}: Invisible creatures are visually undetectable. An invisible creature gains a +2 bonus on attack rolls against sighted opponents, and ignores its opponents' Dexterity bonuses to AC (if any). See Invisibility, under Special Abilities.
				
\textbf{Nauseated}: Creatures with the nauseated condition experience stomach distress. Nauseated creatures are unable to attack, cast spells, concentrate on spells, or do anything else requiring attention. The only action such a character can take is a single move actions per turn.
				
\textbf{Panicked}: A panicked creature must drop anything it holds and flee at top speed from the source of its fear, as well as any other dangers it encounters, along a random path. It can't take any other actions. In addition, the creature takes a --2 penalty on all saving throws, skill checks, and ability checks. If cornered, a panicked creature cowers and does not attack, typically using the total defense action in combat. A panicked creature can use special abilities, including spells, to flee; indeed, the creature must use such means if they are the only way to escape.
				
Panicked is a more extreme state of fear than shaken or frightened.
				
\textbf{Paralyzed}: A paralyzed character is frozen in place and unable to move or act. A paralyzed character has effective Dexterity and Strength scores of 0 and is helpless, but can take purely mental actions. A winged creature flying in the air at the time that it becomes paralyzed cannot flap its wings and falls. A paralyzed swimmer can't swim and may drown. A creature can move through a space occupied by a paralyzed creature---ally or not. Each square occupied by a paralyzed creature, however, counts as 2 squares to move through.
				
\textbf{Petrified}: A petrified character has been turned to stone and is considered unconscious. If a petrified character cracks or breaks, but the broken pieces are joined with the body as he returns to flesh, he is unharmed. If the character's petrified body is incomplete when it returns to flesh, the body is likewise incomplete and there is some amount of permanent hit point loss and/or debilitation.
				
\textbf{Pinned}: A pinned creature is tightly bound and can take few actions. A pinned creature cannot move and is denied its Dexterity bonus. A pinned character also takes an additional --4 penalty to his Armor Class. A pinned creature is limited in the actions that it can take. A pinned creature can always attempt to free itself, usually through a combat maneuver check or Escape Artist check. A pinned creature can take verbal and mental actions, but cannot cast any spells that require a somatic or material component. A pinned character who attempts to cast a spell or use a spell-like ability must make a concentration check (DC 10 + grappler's CMB + spell level) or lose the spell. Pinned is a more severe version of grappled, and their effects do not stack.
				
\textbf{Prone}: The character is lying on the ground. A prone attacker has a --4 penalty on melee attack rolls and cannot use a ranged weapon (except for a crossbow). A prone defender gains a +4 bonus to Armor Class against ranged attacks, but takes a --4 penalty to AC against melee attacks.
				
Standing up is a move-equivalent action that provokes an attack of opportunity.
				
\textbf{Shaken}: A shaken character takes a --2 penalty on attack rolls, saving throws, skill checks, and ability checks. Shaken is a less severe state of fear than frightened or panicked.
				
\textbf{Sickened}: The character takes a --2 penalty on all attack rolls, weapon damage rolls, saving throws, skill checks, and ability checks.
				
\textbf{Stable}: A character who was dying but who has stopped losing hit points each round and still has negative hit points is stable. The character is no longer dying, but is still unconscious. If the character has become stable because of aid from another character (such as a Heal check or magical healing), then the character no longer loses hit points. The character can make a DC 10 Constitution check each hour to become conscious and disabled (even though his hit points are still negative). The character takes a penalty on this roll equal to his negative hit point total. 
				
If a character has become stable on his own and hasn't had help, he is still at risk of losing hit points. Each hour he can make a Constitution check to become stable (as a character that has received aid), but each failed check causes him to lose 1 hit point.
				
\textbf{Staggered}: A staggered creature may take a single move action or standard action each round (but not both, nor can he take full-round actions). A staggered creature can still take free, swift and immediate actions. A creature with nonlethal damage exactly equal to its current hit points gains the staggered condition.
				
\textbf{Stunned}: A stunned creature drops everything held, can't take actions, takes a --2 penalty to AC, and loses its Dexterity bonus to AC (if any).
				
\textbf{Unconscious}: Unconscious creatures are knocked out and helpless. Unconsciousness can result from having negative hit points (but not more than the creature's Constitution score), or from nonlethal damage in excess of current hit points.
