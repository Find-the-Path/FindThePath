\chapter{Creating NPCs}

% Hopefully this is OGL: It wasn't in the PRD
Aside from the players, every other person encountered in the game world is a non-player character (NPC). These characters are
designed and controlled by the GM to fill every role from noble king to simple baker.
While some of these characters use player classes, most
rely upon basic NPC classes, allowing them to be easily
generated. The following rules govern all of the NPC
classes and include information on generating quick
NPCs for an evening's game.


\section{Adept}

\label{f0}				
\textbf{Alignment}: Any.
				
\textbf{Hit Die}: d6.
				
\subsection{Class Skills}

				
The adept's class skills (and the key ability for each skill) are Craft (Int), Handle Animal (Cha), Heal (Wis), Knowledge (all skills taken individually) (Int), Profession (Wis), Spellcraft (Int), and Survival (Wis). 
				
\textbf{Skill Ranks per Level}: 2 + Int modifier.
% <div class="table">
% <
\begin{table}[]
\sffamily
\caption{Table: Adept}
\begin{tabularx}{\linewidth}{lp{5em}p{1.5em}p{1.5em}p{1.5em}Xllllll}
\multirow{2}{*}{\textbf{Level}} & \multirow{2}{*}{\parbox{5em}{\textbf{Base Attack Bonus}}} & \multirow{2}{*}{\parbox{1.5em}{\textbf{Fort Save}}} & \multirow{2}{*}{\parbox{1.5em}{\textbf{Ref Save}}} & \multirow{2}{*}{\parbox{1.5em}{\textbf{Will Save}}} & \textbf{Special}                                                                                              & \multicolumn{6}{c}{\textbf{Spells per day}} \\
                       &                                    &                            &                           &                            &                                                                                                      & \textbf{1st}  & \textbf{2nd} & \textbf{3rd} &\textbf{4th} & \textbf{5th} & \textbf{6th} \\
1st & +0 & +0 & +0 & +2 &  & 3 & 1 & - & - & - & -\\
2nd & +1 & +0 & +0 & +3 & Summon familiar & 3 & 1 & - & - & - & -\\
3rd & +1 & +1 & +1 & +3 &  & 3 & 2 & - & - & - & -\\
4th & +2 & +1 & +1 & +4 &  & 3 & 2 & 0 & - & - & -\\
5th & +2 & +1 & +1 & +4 &  & 3 & 2 & 1 & - & - & -\\
6th & +3 & +2 & +2 & +5 &  & 3 & 2 & 1 & - & - & -\\
7th & +3 & +2 & +2 & +5 &  & 3 & 3 & 2 & - & - & -\\
8th & +4 & +2 & +2 & +6 &  & 3 & 3 & 2 & 0 & - & -\\
9th & +4 & +3 & +3 & +6 &  & 3 & 3 & 2 & 1 & - & -\\
10th & +5 & +3 & +3 & +7 &  & 3 & 3 & 2 & 1 & - & -\\
11th & +5 & +3 & +3 & +7 &  & 3 & 3 & 3 & 2 & - & -\\
12th & +6/+1 & +4 & +4 & +8 &  & 3 & 3 & 3 & 2 & 0 & -\\
13th & +6/+1 & +4 & +4 & +8 &  & 3 & 3 & 3 & 2 & 1 & -\\
14th & +7/+2 & +4 & +4 & +9 &  & 3 & 3 & 3 & 2 & 1 & -\\
15th & +7/+2 & +5 & +5 & +9 &  & 3 & 3 & 3 & 3 & 2 & -\\
16th & +8/+3 & +5 & +5 & +10 &  & 3 & 3 & 3 & 3 & 2 & 0\\
17th & +8/+3 & +5 & +5 & +10 &  & 3 & 3 & 3 & 3 & 2 & 1\\
18th & +9/+4 & +6 & +6 & +11 &  & 3 & 3 & 3 & 3 & 2 & 1\\
19th & +9/+4 & +6 & +6 & +11 &  & 3 & 3 & 3 & 3 & 3 & 2\\
20th & +10/+5 & +6 & +6 & +12 &  & 3 & 3 & 3 & 3 & 3 & 2\\
\end{tabularx}
\end{table}

				
\subsection{Class Features}

				
All of the following are class features of the adept NPC class.
				
\textbf{Weapon and Armor Proficiency}: Adepts are skilled with all simple weapons. Adepts are not proficient with any type of armor or shield.
				
\textbf{Spells}: An adept casts divine spells, which are drawn from the adept spell list. Like a cleric, an adept must choose and prepare her spells in advance. Unlike a cleric, an adept cannot spontaneously cast \textit{cure} or \textit{inflict} spells.
				
To prepare or cast a spell, an adept must have a Wisdom score equal to at least 10 + the spell level. The Difficulty Class for a saving throw against an adept's spell is 10 + the spell level + the adept's Wisdom modifier.
				
Adepts, unlike wizards, do not acquire their spells from books or scrolls, nor do they prepare them through study. Instead, they meditate or pray for their spells, receiving them as divine inspiration or through their own strength of faith. Each adept must choose a time each day during which she must spend an hour in quiet contemplation or supplication to regain her daily allotment of spells. Time spent resting has no effect on whether an adept can prepare spells.
				
Like other spellcasters, an adept can cast only a certain number of spells of each spell level per day. Her base daily spell allotment is given on Table 14--1. In addition, she receives bonus spells per day if she has a high Wisdom score.
				
Where Table 14--1 indicates that the adept gets 0 spells per day of a given spell level, she gains only the bonus spells she would be entitled to based on her Wisdom score for that spell level.
				
Each adept has a particular holy symbol (as a divine focus) depending on the adept's magical tradition.
				
\textbf{Summon Familiar}: At 2nd level, an adept can call a familiar, just as a wizard can using the arcane bond ability.
				
\subsection{Adept Spell List}

				
Adepts choose their spells from the following list.
				
0 Level: \textit{create water, detect magic, ghost sound, guidance, light, mending, purify food and drink, read magic, stabilize, touch of fatigue.}
				
1st Level: \textit{bless, burning hands, cause fear, command, comprehend languages, cure light wounds, detect chaos, detect evil, detect good, detect law, endure elements, obscuring mist, protection from chaos, protection from evil, protection from good, protection from law, sleep.}
				
2nd Level: \textit{aid, animal trance, bear's endurance, bull's strength, cat's grace, cure moderate wounds, darkness, delay poison, invisibility, mirror image, resist energy, scorching ray, see invisibility, web.}
				
3rd Level: \textit{animate dead, bestow curse, contagion, continual flame, cure serious wounds, daylight, deeper darkness, lightning bolt, neutralize poison, remove curse, remove disease, tongues.}
				
4th Level: \textit{cure critical wounds, minor creation, polymorph, restoration, stoneskin, wall of fire.}
				
5th Level: \textit{baleful polymorph, break enchantment, commune, heal, major creation, raise dead, true seeing, wall of stone.}
				
\section{Aristocrat}

				
\textbf{Alignment}: Any.
				
\textbf{Hit Die}: d8.
				
\subsection{Class Skills}

				
The aristocrat's class skills (and the key ability for each skill) are Appraise (Int), Bluff (Cha), Craft (Int), Diplomacy (Cha), Disguise (Cha), Handle Animal (Cha), Intimidate (Cha), Knowledge (all skills taken individually) (Int), Linguistics (Int), Perception (Wis), Perform (Cha), Profession (Wis), Ride (Dex), Sense Motive (Wis), Swim (Str), and Survival (Wis). 
				
\textbf{Skill Ranks per Level}: 4 + Int modifier.
					
\begin{table}[]
\sffamily
\caption{Aristocrat}
\begin{tabular}{lllll}
\textbf{Level} & \textbf{Base Attack Bonus} & \textbf{Fort Save} & \textbf{Ref Save} & \textbf{Will Save}\\
1st & +0 & +0 & +0 & +2\\
2nd & +1 & +0 & +0 & +3\\
3rd & +2 & +1 & +1 & +3\\
4th & +3 & +1 & +1 & +4\\
5th & +3 & +1 & +1 & +4\\
6th & +4 & +2 & +2 & +5\\
7th & +5 & +2 & +2 & +5\\
8th & +6/+1 & +2 & +2 & +6\\
9th & +6/+1 & +3 & +3 & +6\\
10th & +7/+2 & +3 & +3 & +7\\
11th & +8/+3 & +3 & +3 & +7\\
12th & +9/+4 & +4 & +4 & +8\\
13th & +9/+4 & +4 & +4 & +8\\
14th & +10/+5 & +4 & +4 & +9\\
15th & +11/+6/+1 & +5 & +5 & +9\\
16th & +12/+7/+2 & +5 & +5 & +10\\
17th & +12/+7/+2 & +5 & +5 & +10\\
18th & +13/+8/+3 & +6 & +6 & +11\\
19th & +14/+9/+4 & +6 & +6 & +11\\
20th & +15/+10/+5 & +6 & +6 & +12\\
\end{tabular}
\end{table}

\subsection{Class Features}

				
The following is a class feature of the aristocrat NPC class.
				
\textbf{Weapon and Armor Proficiency}: The aristocrat is proficient in the use of all simple and martial weapons and with all types of armor and shields.
				
\section{Commoner}

				
\textbf{Alignment}: Any.
				
\textbf{Hit Die}: d6.
				
\subsection{Class Skills}

				
The commoner's class skills (and the key ability for each skill) are Climb (Str), Craft (Int), Handle Animal (Cha), Perception (Wis), Profession (Wis), Ride (Dex), and Swim (Str).
				
\textbf{Skill Ranks per Level}: 2 + Int modifier.
% <
\begin{table}[]
\sffamily
\caption{Table: Commoner}
\begin{tabular}{lllll}
\textbf{Level} & \textbf{Base Attack Bonus} & \textbf{Fort Save} & \textbf{Ref Save} & \textbf{Will Save} \\
1st & +0 & +0 & +0 & +0\\
2nd & +1 & +0 & +0 & +0\\
3rd & +1 & +1 & +1 & +1\\
4th & +2 & +1 & +1 & +1\\
5th & +2 & +1 & +1 & +1\\
6th & +3 & +2 & +2 & +2\\
7th & +3 & +2 & +2 & +2\\
8th & +4 & +2 & +2 & +2\\
9th & +4 & +3 & +3 & +3\\
10th & +5 & +3 & +3 & +3\\
11th & +5 & +3 & +3 & +3\\
12th & +6/+1 & +4 & +4 & +4\\
13th & +6/+1 & +4 & +4 & +4\\
14th & +7/+2 & +4 & +4 & +4\\
15th & +7/+2 & +5 & +5 & +5\\
16th & +8/+3 & +5 & +5 & +5\\
17th & +8/+3 & +5 & +5 & +5\\
18th & +9/+4 & +6 & +6 & +6\\
19th & +9/+4 & +6 & +6 & +6\\
20th & +10/+5 & +6 & +6 & +6\\
\end{tabular}
\end{table}

				
\subsection{Class Features}

				
The following is a class feature of the commoner NPC class.
				
\textbf{Weapon and Armor Proficiency}: The commoner is proficient with one simple weapon. He is not proficient with any other weapons, nor is he proficient with any type of armor or shield.
				
\section{Expert}

				
\textbf{Alignment}: Any.
				
\textbf{Hit Die}: d8.
				
\subsection{Class Skills}

				
The expert can choose any 10 skills to be class skills.
				
\textbf{Skill Ranks per Level}: 6 + Int modifier.
				
% <
\begin{table}[]
\sffamily
\caption{Table: Expert}
\begin{tabular}{lllll}
\textbf{Level} & \textbf{Base Attack Bonus} & \textbf{Fort Save} & \textbf{Ref Save} & \textbf{Will Save}\\
1st & +0 & +0 & +0 & +2\\
2nd & +1 & +0 & +0 & +3\\
3rd & +2 & +1 & +1 & +3\\
4th & +3 & +1 & +1 & +4\\
5th & +3 & +1 & +1 & +4\\
6th & +4 & +2 & +2 & +5\\
7th & +5 & +2 & +2 & +5\\
8th & +6/+1 & +2 & +2 & +6\\
9th & +6/+1 & +3 & +3 & +6\\
10th & +7/+2 & +3 & +3 & +7\\
11th & +8/+3 & +3 & +3 & +7\\
12th & +9/+4 & +4 & +4 & +8\\
13th & +9/+4 & +4 & +4 & +8\\
14th & +10/+5 & +4 & +4 & +9\\
15th & +11/+6/+1 & +5 & +5 & +9\\
16th & +12/+7/+2 & +5 & +5 & +10\\
17th & +12/+7/+2 & +5 & +5 & +10\\
18th & +13/+8/+3 & +6 & +6 & +11\\
19th & +14/+9/+4 & +6 & +6 & +11\\
20th & +15/+10/+5 & +6 & +6 & +12\\
\end{tabular}
\end{table}
			
\subsection{Class Features}

				
The following is a class feature of the expert NPC class.
				
\textbf{Weapon and Armor Proficiency}: The expert is proficient in the use of all simple weapons and with light armor, but not with any type of shield. 
				
\section{Warrior}

				
\textbf{Alignment}: Any.
				
\textbf{Hit Die}: d10.
				
\subsection{Class Skills}

				
The warrior's class skills (and the key ability for each skill) are Climb (Str), Craft (Int), Handle Animal (Cha), Intimidate (Cha), Profession (Wis), Ride (Dex), and Swim (Str).
				
\textbf{Skill Ranks per Level}: 2 + Int modifier.
	
\begin{table}[]
\sffamily
\caption{Warrior}
\begin{tabular}{lllll}
\textbf{Level} & \textbf{Base Attack Bonus} & \textbf{Fort Save} & \textbf{Ref Save} & \textbf{Will Save} \\
1st & +1 & +2 & +0 & +0\\
2nd & +2 & +3 & +0 & +0\\
3rd & +3 & +3 & +1 & +1\\
4th & +4 & +4 & +1 & +1\\
5th & +5 & +4 & +1 & +1\\
6th & +6/+1 & +5 & +2 & +2\\
7th & +7/+2 & +5 & +2 & +2\\
8th & +8/+3 & +6 & +2 & +2\\
9th & +9/+4 & +6 & +3 & +3\\
10th & +10/+5 & +7 & +3 & +3\\
11th & +11/+6/+1 & +7 & +3 & +3\\
12th & +12/+7/+2 & +8 & +4 & +4\\
13th & +13/+8/+3 & +8 & +4 & +4\\
14th & +14/+9/+4 & +9 & +4 & +4\\
15th & +15/+10/+5 & +9 & +5 & +5\\
16th & +16/+11/+6/+1 & +10 & +5 & +5\\
17th & +17/+12/+7/+2 & +10 & +5 & +5\\
18th & +18/+13/+8/+3 & +11 & +6 & +6\\
19th & +19/+14/+9/+4 & +11 & +6 & +6\\
20th & +20/+15/+10/+5 & +12 & +6 & +6\\
\end{tabular}
\end{table}

				
\subsection{Class Features}

				
The following is a class feature of the warrior NPC class.
				
\textbf{Weapon and Armor Proficiency}: The warrior is proficient in the use of all simple and martial weapons and with all types of armor and shields.
        	


\section{Creating NPCs}

\label{f0}				
The world that the player characters inhabit should be full of rich and vibrant characters with whom they can interact. While most need little more than names and general descriptions, some require complete statistics, such as town guards, local clerics, and wizened sages. The PCs might find themselves in combat with these characters, either against them or as allies. Alternatively the PCs might find themselves relying on the skills and abilities of the NPCs. In either case, the process for creating these NPCs can be performed in seven simple steps.
				
\subsection{Step 1: The Basics}

				
The first step in making an NPC is to determine its basic role in your campaign. This includes its race, class, and basic concept.
				
\subsection{Step 2: Determine Ability Scores}

				
Once the character's basic concept has been determined, its ability scores must be assigned. Apply the NPC's racial modifiers after the scores have been assigned. For every four levels the NPC has attained, increase one of its scores by 1. If the NPC possesses levels in a PC class, it is considered a heroic NPC and receives better ability scores. These scores can be assigned in any order.
				
\textbf{Basic NPCs}: The ability scores for a basic NPC are: 13, 12, 11, 10, 9, and 8.
				
\textbf{Heroic NPCs}: The ability scores for a heroic NPC are: 15, 14, 13, 12, 10, and 8.
				
\textbf{Preset Ability Scores}: Instead of assigning the scores, you can use Table: NPC Ability Scores to determine the NPC's ability scores, adjusting them as necessary to fit. Use the Melee NPC ability scores for characters whose primary role involves melee combat, such as barbarians, fighters, monks, paladins, rangers, and warriors. The Ranged NPC ability scores are for characters that fight with ranged weapons or use their Dexterity to hit, such as fighters, rangers, and rogues. Use the Divine NPC ability scores for characters with divine spellcasting capabilities, such as adepts, clerics, and druids. The Arcane NPC ability scores should be used by characters with arcane spellcasting capabilities, such as bards, sorcerers, and wizards. Finally, the Skill NPC ability scores should be used for characters that focus on skill use, such as aristocrats, bards, commoners, experts, and rogues. Some NPCs might not fit into one of these categories and should have custom ability scores.
				
% <
\begin{table}[]
\sffamily
\caption{Table: NPC Ability Scores}
\begin{tabular}{llllllllllll}
\multicolumn{2}{c}{\textbf{Ability Score}} & \multicolumn{2}{c}{\textbf{Melee NPC}} & \multicolumn{2}{c}{\textbf{Ranged NPC}} & \multicolumn{2}{c}{\textbf{Divine NPC}} & \multicolumn{2}{c}{\textbf{Arcane NPC}} & \multicolumn{2}{c}{\textbf{Skill NPC}}\\
\textbf{Basic} & \textbf{Heroic} & \textbf{Basic} & \textbf{Heroic} &\textbf{Basic} & \textbf{Heroic} &\textbf{Basic} & \textbf{Heroic} &\textbf{Basic} & \textbf{Heroic} \\
Strength & 13 & 15 & 11 & 13 & 10 & 12 & 8 & 8 & 10 & 12 \\
 Dexterity & 11 & 13 & 13 & 15 & 8 & 8 & 12 & 14 & 12 & 14 \\
 Constitution & 12 & 14 & 12 & 14 & 12 & 14 & 10 & 12 & 11 & 13 \\
 Intelligence & 9 & 10 & 10 & 12 & 9 & 10 & 13* & 15* & 13 & 15 \\
 Wisdom & 10 & 12 & 9 & 10 & 13 & 15 & 9 & 10 & 8 & 8 \\
 Charisma & 8 & 8 & 8 & 8 & 11 & 13 & 11* & 13* & 9 & 10\\
\end{tabular}
* If the arcane caster's spellcasting relies on Charisma, exchange these scores with one another.\\
\end{table}

\begin{table}[]
\sffamily
\caption{Table: Racial Ability Adjustments}
\begin{tabular}{llllllll}
\textbf{Ability Score} & \textbf{Dwarf} & \textbf{Elf} & \textbf{Gnome} & \textbf{Half-Elf*} & \textbf{Half-Orc*} & \textbf{Halfling} & \textbf{Human*}\\
Strength & - & - & -2 & - & - & -2 & - \\
 Dexterity & - & +2 & - & - & - & +2 & - \\
 Constitution & +2 & -2 & +2 & - & - & - & - \\
 Intelligence & - & +2 & - & - & - & - & - \\
 Wisdom & +2 & - & - & - & - & - & - \\
 Charisma & -2 & - & +2 & - & - & +2 & -\\
\end{tabular}
* Half-elves, half-orcs, and humans receive a +2 bonus to one ability score of your choice.\\
\end{table}

\subsection{Step 3: Skills}

				
To assign skills precisely, total up the number of skill ranks possessed by the character and assign them normally. Remember that the number of ranks in an individual skill that a character can possess is limited by his total HD.
				
For simpler skill generation, refer to Table: NPC Skill Selections to determine the total number of skill selections the NPC possesses. After selecting that number of skills, mostly from the class skills lists of the NPC's class, the NPC receives a number of ranks in each skill equal to his level. 
				
If the NPC has two classes, start by selecting skills for the class with the fewest number of skill selections. The NPC receives a number of ranks in those skills equal to his total character level. Next, find the difference in the number of selections between the first class and the other class possessed by the NPC. Select that number of new skills and give the NPC a number of ranks in those skills equal to his level in the second class. For example, a human fighter 3/monk 4 with a +1 Intelligence modifier can select four skills for his fighter class (since it receives fewer selections). These four skills each have seven ranks (equal to his total level). Next, he selects a number of skills equal to the difference between the fighter and the monk classes, in this case two skills. These two skills each have four ranks (his monk level).
				
If the NPC has three or more classes, you must use the precise method for determining his skills. 
				
Once all of the NPC's ranks have been determined, assign class skill bonuses and apply the bonus or penalty from the NPC's relevant ability score.
	
% <
\begin{table}[]
\sffamily
\caption{Table: NPC Skill Selections}
\begin{tabular}{llll}
\multicolumn{2}{c}{\textbf{PC Skill Class Selections*}} & \multicolumn{2}{c}{\textbf{NPC Skill Class Selections*}}\\
Barbarian & 4 + Int Mod & Adept & 2 + Int Mod\\
Bard & 6 + Int Mod & Aristocrat &  4 + Int Mod\\
Cleric & 2 + Int Mod & Commoner & 2 + Int Mod\\
Druid & 4 + Int Mod & Expert & 6 + Int Mod\\
Fighter & 2 + Int Mod & Warrior & 2 + Int Mod\\
Monk & 4 + Int Mod &  & \\
Paladin & 2 + Int Mod &  & \\
Ranger & 6 + Int Mod &  & \\
Rogue & 8 + Int Mod &  & \\
Sorcerer & 2 + Int Mod &  & \\
Wizard & 2 + Int Mod &  & \\
\end{tabular}
* Humans receive one additional skill selection.\\
\end{table}

\subsection{Step 4: Feats}

				
After skills have been determined, the next step is to assign the NPC's feats. Start by assigning all of the feats granted through class abilities. Next, assign the feats garnered from the NPC's total character level (one feat for every two levels beyond 1st). Remember that humans receive an additional feat at 1st level. For simplified feat choices, select feats from the lists provided for the following character types.
				
\textbf{Arcane Caster}: Arcane Strike, Combat Casting, Eschew Materials, Greater Spell Focus, Greater Spell Penetration, Improved Initiative, Iron Will, item creation feats (all), Lightning Reflexes, metamagic feats (all), Spell Focus, Spell Mastery, Spell Penetration, and Toughness.
				
\textbf{Divine Caster (With Channeling)}: Alignment Channel, Channel Smite, Combat Casting, Command Undead, Elemental Channel, Extra Channel, Improved Initiative, Improved Channel, Iron Will, item creation feats (all), metamagic feats (all), Power Attack, Selective Channeling, Spell Focus, Spell Penetration, Toughness, and Turn Undead.
				
\textbf{Divine Caster} \textbf{(Without Channeling)}: Cleave, Combat Casting, Eschew Materials, Improved Initiative, Iron Will, item creation feats (all), Lightning Reflexes, metamagic feats (all), Natural Spell, Power Attack, Spell Focus, Spell Penetration, Toughness, and Weapon Focus.
				
\textbf{Melee (Finesse Fighter)}: Combat Expertise, Combat Reflexes, Dazzling Display, Deadly Stroke, Dodge, Greater Vital Strike, Improved Disarm, Improved Feint, Improved Trip, Improved Vital Strike, Mobility, Spring Attack, Shatter Defenses, Vital Strike, Weapon Finesse, and Whirlwind Attack.
				
\textbf{Melee (Unarmed Fighter)}: Combat Reflexes, Deflect Arrows, Dodge, Gorgon's Fist, Improved Grapple, Improved Initiative, Improved Unarmed Strike, Medusa's Wrath, Mobility, Scorpion Style, Snatch Arrows, Spring Attack, Stunning Fist, and Weapon Focus.
				
\textbf{Melee (Mounted)}: Improved Critical, Improved Initiative, Mounted Combat, Power Attack, Ride-By Attack, Skill Focus (Ride), Spirited Charge, Toughness, Trample, and Weapon Focus.
				
\textbf{Melee (Sword and Shield Fighter)}: Cleave, Great Cleave, Great Fortitude, Greater Vital Strike, Improved Bull Rush, Improved Critical, Improved Initiative, Improved Vital Strike, Power Attack, Shield Focus, Shield Master, Shield Slam, Two-Weapon Fighting, Vital Strike, and Weapon Focus.
				
\textbf{Melee (Two-Handed Fighter)}: Cleave, Great Cleave, Great Fortitude, Greater Vital Strike, Improved Bull Rush, Improved Critical, Improved Initiative, Improved Sunder, Improved Vital Strike, Power Attack, Vital Strike, and Weapon Focus.
				
\textbf{Melee (Two-Weapon Fighter)}: Combat Reflexes, Dodge, Double Slice, Greater Two-Weapon Fighting, Greater Vital Strike, Improved Critical, Improved Initiative, Improved Two-Weapon Fighting, Improved Vital Strike, Two-Weapon Defense, Two-Weapon Fighting, Two-Weapon Rend, Vital Strike, and Weapon Focus.
				
\textbf{Ranged}: Deadly Aim, Far Shot, Greater Vital Strike, Improved Initiative, Improved Vital Strike, Manyshot, Pinpoint Targeting, Point Blank Shot, Precise Shot, Rapid Reload, Rapid Shot, Shot on the Run, Vital Strike, and Weapon Focus.
				
\textbf{Skill (most NPC classes)}: Armor Proficiency (all), Great Fortitude, Improved Initiative, Iron Will, Lightning Reflexes, Martial Weapon Proficiency, Run, Shield Proficiency, Skill Focus, and Toughness.
				
\subsection{Step 5: Class Features}

				
After determining feats, the next step is to fill in all the class features possessed by the NPC. This is the time to make decisions about the NPC's spell selection, rage powers, rogue talents, and other class-based abilities. 
				
When it comes to spells, determine how many spell selections you need to make for each level. Choose a variety of spells for the highest two levels of spells possessed by the NPC. For all other levels, stick to a few basic spells, prepared multiple times (if possible). If this NPC is slated to appear in only one encounter (such as a combat), leaving off lower-level spells entirely is an acceptable way to speed up generation, especially if the NPC is unlikely to cast those spells. You can always choose a few during play if they are needed.
				
\subsection{Step 6: Gear}

				
After recording all of the NPC's class features, the next step is to outfit the character with gear appropriate to his level. Note that NPCs receive less gear than PCs of an equal level. If an NPC is a recurring character, his gear should be selected carefully. Use the total gp value found on Table: NPC Gear to determine how much gear he should carry. NPCs that are only scheduled to appear once can have a simpler gear selection. Table: NPC Gear includes a number of categories to make it easier to select an NPC's gear. When outfitting the character, spend the listed amount on each category by purchasing as few items as possible. Leftover gold from any category can be spent on any other category. Funds left over at the end represent coins and jewelry carried by the character.
				
Note that these values are approximate and based on the values for a campaign using the medium experience progression and a normal treasure allotment. If your campaign is using the fast experience progression, treat your NPCs as one level higher when determining their gear. If your campaign is using the slow experience progression, treat the NPCs as one level lower when determining their gear. If your campaign is high fantasy, double these values. Reduce them by half if your campaign is low fantasy. If the final value of an NPC's gear is a little over or under these amounts, that's okay.
					

\begin{table}[]
\sffamily
\caption{Table: NPC Gear}
\begin{tabular}{llllllll}
\textbf{Basic Level} & \textbf{Heroic Level} & \textbf{Total gp Value} & \textbf{Weapons} & \textbf{Protection} & \textbf{Magic} & \textbf{Limited Use} & \textbf{Gear}\\
1 & - & 260 gp & 50 gp & 130 gp & - & 40 gp & 40 gp \\
 2 & 1 & 390 gp & 100 gp & 150 gp & - & 40 gp & 100 gp \\
 3 & 2 & 780 gp & 350 gp & 200 gp & - & 80 gp & 150 gp \\
 4 & 3 & 1,650 gp & 650 gp & 800 gp & - & 100 gp & 200 gp \\
 5 & 4 & 2,400 gp & 900 gp & 1,000 gp & - & 300 gp & 200 gp \\
 6 & 5 & 3,450 gp & 1,400 gp & 1,400 gp & - & 450 gp & 200 gp \\
 7 & 6 & 4,650 gp & 2,350 gp & 1,650 gp & - & 450 gp & 200 gp \\
 8 & 7 & 6,000 gp & 2,700 gp & 2,000 gp & 500 gp & 600 gp & 200 gp \\
 9 & 8 & 7,800 gp & 3,000 gp & 2,500 gp & 1,000 gp & 800 gp & 500 gp \\
 10 & 9 & 10,050 gp & 3,500 gp & 3,000 gp & 2,000 gp & 1,050 gp & 500 gp \\
 11 & 10 & 12,750 gp & 4,000 gp & 4,000 gp & 3,000 gp & 1,250 gp & 500 gp \\
 12 & 11 & 16,350 gp & 6,000 gp & 4,500 gp & 4,000 gp & 1,350 gp & 500 gp \\
 13 & 12 & 21,000 gp & 8,500 gp & 5,500 gp & 5,000 gp & 1,500 gp & 500 gp \\
 14 & 13 & 27,000 gp & 9,000 gp & 8,000 gp & 7,000 gp & 2,500 gp & 500 gp \\
 15 & 14 & 34,800 gp & 12,000 gp & 10,500 gp & 9,000 gp & 2,800 gp & 500 gp \\
 16 & 15 & 45,000 gp & 17,000 gp & 13,500 gp & 11,000 gp & 3,000 gp & 500 gp \\
 17 & 16 & 58,500 gp & 19,000 gp & 18,000 gp & 16,000 gp & 4,000 gp & 1,500 gp \\
 18 & 17 & 75,000 gp & 24,000 gp & 23,000 gp & 20,000 gp & 6,500 gp & 1,500 gp \\
 19 & 18 & 96,000 gp & 30,000 gp & 28,000 gp & 28,000 gp & 8,000 gp & 2,000 gp \\
 20 & 19 & 123,000 gp & 40,000 gp & 35,000 gp & 35,000 gp & 11,000 gp & 2,000 gp \\
 - & 20 & 159,000 gp & 55,000 gp & 40,000 gp & 44,000 gp & 18,000 gp & 2,000 gp\\
\end{tabular}
\end{table}
			
\textbf{Weapons}: This includes normal, masterwork, and magic weapons, as well as magic staves and wands used by spellcasters to harm their enemies. For example, a \textit{wand of scorching ray} would count as a weapon, but a \textit{staff of life} would count as a piece of magic gear.
				
\textbf{Protection}: This category includes armor and shields, as well as any magic item that augments a character's Armor Class or saving throws.
				
\textbf{Magic}: This category includes all other permanent magic items. Most rings, rods, and wondrous items fit into this category.
				
\textbf{Limited Use}: Items that fall into this category include alchemical items, potions, scrolls, and wands with few charges. Charged wondrous items fall into this grouping as well.
				
\textbf{Gear}: Use the amount in this category to purchase standard nonmagical gear for the character. In most cases, this equipment can be omitted during creation and filled in as needed during play. You can assume that the character has whatever gear is needed for him to properly use his skills and class abilities. This category can also include jewelry, gems, or loose coins that the NPC might have on his person.
				
\subsection{Step 7: Details}

				
Once you have assigned all of the NPC's gear, all that remains is to fill out the details. Determine the character's attack and damage bonuses, CMB, CMD, initiative modifier, and Armor Class. If the character's magic items affect his skills or ability scores, make sure to take those changes into account. Determine the character's total hit points by assuming the average result. Finally, fill out any other important details, such as name, alignment, religion, and a few personality traits to round him out.
			        	