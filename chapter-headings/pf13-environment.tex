\chapter{Environment}
\section{Dungeons}

\label{f0}				
Of all the strange places that an adventurer might explore, none is deadlier than the dungeon. These labyrinths, full of deadly traps, hungry monsters, and priceless treasure, test every skill a character possesses. These rules can apply to dungeons of any type, from the wreck of a sunken ship to a vast cave complex.
				
\subsection{Types of Dungeons}

				
The four basic dungeon types are defined by their current status. Many dungeons are variations on these basic types or combinations of more than one of them. Sometimes old dungeons are used again and again by different inhabitants for different purposes.
				
\textbf{Ruined Structure}: Once occupied, this place is now abandoned (completely or in part) by its original creator or creators, and other creatures have wandered in. Many subterranean creatures look for abandoned underground constructions in which to make their lairs. Any traps that might exist have probably been set off, but wandering beasts might very well be common.
				
\textbf{Occupied Structure}: This type of dungeon is still in use. Creatures (usually intelligent) live there, although they might not be the dungeon's creators. An occupied structure might be a home, a fortress, a temple, an active mine, a prison, or a headquarters. This type of dungeon is less likely to have traps or wandering beasts, and more likely to have organized guards---both on watch and on patrol. Traps or wandering beasts that might be encountered are usually under the control of the occupants. Occupied structures have furnishings to suit the inhabitants, as well as decorations, supplies, and the ability for occupants to move around. The inhabitants might have a communication system, and they almost certainly control an exit to the outside.
				
Some dungeons are partially occupied and partially empty or in ruins. In such cases, the occupants are typically not the original builders, but instead a group of intelligent creatures that have set up their base, lair, or fortification within an abandoned dungeon.
				
\textbf{Safe Storage}: When people want to protect something, they sometimes bury it underground. Whether the item they want to protect is a fabulous treasure, a forbidden artifact, or the dead body of an important figure, these valuable objects are placed within a dungeon and surrounded by barriers, traps, and guardians.
				
The safe storage dungeon is the most likely to have traps but the least likely to have wandering beasts. This type of dungeon is normally built for function rather than appearance, but sometimes it has ornamentation in the form of statuary or painted walls. This is particularly true of the tombs of important people.
				
Sometimes, however, a vault or a crypt is constructed in such a way as to house living guardians. The problem with this strategy is that something must be done to keep the creatures alive between intrusion attempts. Magic is usually the best solution to provide food and water for these creatures. Builders of vaults or tombs often use undead creatures or constructs, both of which have no need for sustenance or rest, to guard their dungeons. Magic traps can attack intruders by summoning monsters into the dungeon that disappear when their task is done.
				
\textbf{Natural Cavern Complex}: Underground caves provide homes for all sorts of subterranean monsters. Created naturally and connected by labyrinthine tunnel systems, these caverns lack any sort of pattern, order, or decoration. With no intelligent force behind its construction, this type of dungeon is the least likely to have traps or even doors.
				
Fungi of all sorts thrive in caves, sometimes growing in huge forests of mushrooms and puffballs. Subterranean predators prowl these forests, looking for weaker creatures feeding upon the fungi. Some varieties of fungus give off a phosphorescent glow, providing a natural cavern complex with its own limited light source. In other areas, a \textit{daylight }spell or similar magical effect can provide enough light for green plants to grow.
				
Natural cavern complexes often connect with other types of dungeons, the caves having been discovered when the manufactured dungeons were delved. A cavern complex can connect two otherwise unrelated dungeons, sometimes creating a strange mixed environment. A natural cavern complex joined with another dungeon often provides a route by which subterranean creatures find their way into a manufactured dungeon and populate it.
				
\subsection{Dungeon Terrain}

				
The following rules cover the basics of terrain that can be found in a dungeon.
				
\subsection{Walls}

				
Masonry walls---stones piled on top of each other, usually but not always held in place with mortar---often divide dungeons into corridors and chambers. Dungeon walls can also be hewn from solid rock, leaving them with a rough, chiseled look. Still other dungeon walls can be the smooth, unblemished stone of a naturally occurring cave. Dungeon walls are difficult to break down or through, but they're generally easy to climb.

\begin{table}[]
\sffamily
\caption{Table: Walls}
\begin{tabular}{llllll}
\textbf{Wall Type} & \textbf{Typical Thickness\(^{1}\)} & \textbf{break DC} & \textbf{Hardness} & \textbf{Hit Points} & \textbf{Climb DC}\\
Masonry & 1 ft. & 35 & 8 & 90 hp & 20\\
Superior masonry & 1 ft. & 35 & 8 & 90 hp & 25\\
Reinforced masonry & 1 ft. & 45 & 8 & 180 hp & 20\\
Hewn stone & 3 ft. & 50 & 8 & 540 hp & 25\\
Unworked stone & 5 ft. & 65 & 8 & 900 hp & 15\\
Iron & 3 in. & 30 & 10 & 90 hp & 25\\
Paper & Paper-thin & 1 & - & 1 hp & 30\\
Wooden & 6 in. & 20 & 5 & 60 hp & 21\\
Magically treated\(^{2}\) & - & +20 & \mbox{$\times$}2 & \mbox{$\times$}2 & -\\
\end{tabular}
1 Per 10-foot-by-10-foot section.
2 This modifier can be applied to any of the other wall types.
3 Or an additional 50 hit points, whichever is greater.
\end{table}

				
\textbf{Masonry Walls}: The most common kind of dungeon wall, masonry walls are usually at least 1 foot thick. Often, these ancient walls sport cracks and crevices, and sometimes dangerous slimes or small monsters live in these areas and wait for prey. Masonry walls stop all but the loudest noises. It takes a DC 20 Climb check to travel along a masonry wall.
				
\textbf{Superior Masonry Walls}: Sometimes masonry walls are better built (smoother, with tighter-fitting stones and less cracking), and occasionally these superior walls are covered with plaster or stucco. Covered walls often bear paintings, carved reliefs, or other decoration. Superior masonry walls are no more difficult to destroy than regular masonry walls but are more difficult to climb (DC 25).
				
\textbf{Reinforced Masonry Walls}: These are masonry walls with iron bars on one or both sides of the wall, or placed within the wall to strengthen it. The hardness of a reinforced wall remains the same, but its hit points are doubled and the Strength check DC to break through it is increased by 10.
				
\textbf{Hewn Stone Walls}: Such walls usually result when a chamber or passage is tunneled out from solid rock. The rough surface of a hewn wall frequently provides minuscule ledges where fungus grows and fissures where vermin, bats, and subterranean snakes live. When such a wall has an \texttt{{}"{}}other side\texttt{{}"{}} (meaning it separates two chambers in the dungeon), the wall is usually at least 3 feet thick; anything thinner risks collapsing from the weight of all the stone overhead. It takes a DC 25 Climb check to climb a hewn stone wall.
				
\textbf{Unworked Stone Walls}: These surfaces are uneven and rarely flat. They are smooth to the touch but filled with tiny holes, hidden alcoves, and ledges at various heights. They're also usually wet or at least damp, since it's water that most frequently creates natural caves. When such a wall has an \texttt{{}"{}}other side,\texttt{{}"{}} the wall is usually at least 5 feet thick. It takes a DC 15 Climb check to move along an unworked stone wall. 
				
\textbf{Iron Walls}: These walls are placed within dungeons around important places, such as vaults. 
				
\textbf{Paper Walls}: Paper walls are placed as screens to block line of sight, but nothing more.
				
\textbf{Wooden Walls}: Wooden walls often exist as recent additions to older dungeons, used to create animal pens, storage bins, and temporary structures, or just to make a number of smaller rooms out of a larger one.
				
\textbf{Magically Treated Walls}: These walls are stronger than average, with a greater hardness, more hit points, and a higher break DC. Magic can usually double the hardness and hit points of a wall and add up to 20 to the break DC. A magically treated wall also gains a saving throw against spells that could affect it, with the save bonus equaling 2 + 1/2 the caster level of the magic reinforcing the wall. Creating a magic wall requires the Craft Wondrous Item feat and the expenditure of 1,500 gp for each 10-foot-by-10-foot wall section.
				
\textbf{Walls with Arrow Slits}: Walls with arrow slits can be made of any durable material but are most commonly masonry, hewn stone, or wood. Such a wall allows defenders to fire arrows or crossbow bolts at intruders from behind the safety of the wall. Archers behind arrow slits have improved cover that gives them a +8 bonus to Armor Class, a +4 bonus on Reflex saves, and the benefits of the improved evasion class feature.
				
\subsection{Floors}

				
As with walls, dungeon floors come in many types.
				
\textbf{Flagstone}: Like masonry walls, flagstone floors are made of fitted stones. They are usually cracked and only somewhat level. Slime and mold grows in the cracks. Sometimes water runs in rivulets between the stones or sits in stagnant puddles. Flagstone is the most common dungeon floor.
				
\textbf{Uneven Flagstone}: Over time, some floors can become so uneven that a DC 10 Acrobatics check is required to run or charge across the surface. Failure means the character can't move that round. Floors as treacherous as this should be the exception, not the rule.
				
\textbf{Hewn Stone Floors}: Rough and uneven, hewn floors are usually covered with loose stones, gravel, dirt, or other debris. A DC 10 Acrobatics check is required to run or charge across such a floor. Failure means the character can still act, but can't run or charge in this round.
				
\textbf{Light Rubble}: Small chunks of debris litter the ground. Light rubble adds 2 to the DC of Acrobatics checks.
				
\textbf{Dense Rubble}: The ground is covered with debris of all sizes. It costs 2 squares of movement to enter a square with dense rubble. Dense rubble adds 5 to the DC of Acrobatics checks, and it adds 2 to the DC of Stealth checks.
				
\textbf{Smooth Stone Floors}: Finished and sometimes even polished, smooth floors are found only in dungeons made by capable and careful builders. 
				
\textbf{Natural Stone Floors}: The floor of a natural cave is as uneven as the walls. Caves rarely have flat surfaces of any great size. Rather, their floors have many levels. Some adjacent floor surfaces might vary in elevation by only a foot, so that moving from one to the other is no more difficult than negotiating a stair step, but in other places the floor might suddenly drop off or rise up several feet or more, requiring Climb checks to get from one surface to the other. Unless a path has been worn and well marked in the floor of a natural cave, it takes 2 squares of movement to enter a square with a natural stone floor, and the DC of Acrobatics checks increases by 5. Running and charging are impossible, except along paths.
				
\textbf{Slippery}: Water, ice, slime, or blood can make any of the dungeon floors described in this section more treacherous. Slippery floors increase the DC of Acrobatics checks by 5. 
				
\textbf{Grate}: A grate often covers a pit or an area lower than the main floor. Grates are usually made from iron, but large ones can also be made from iron-bound timbers. Many grates have hinges to allow access to what lies below (such grates can be locked like any door), while others are permanent and designed to not move. A typical 1-inch-thick iron grate has 25 hit points, hardness 10, and a DC of 27 for Strength checks to break through it or tear it loose.
				
\textbf{Ledge}: Ledges allow creatures to walk above some lower area. They often circle around pits, run along underground streams, form balconies around large rooms, or provide a place for archers to stand while firing upon enemies below. Narrow ledges (12 inches wide or less) require those moving along them to make Acrobatics checks. Failure results in the moving character falling off the ledge. Ledges sometimes have railings along the wall. In such a case, characters gain a +5 circumstance bonus on Acrobatics checks to move along the ledge. A character who is next to a railing gains a +2 circumstance bonus on his opposed Strength check to avoid being bull rushed off the edge.
				
Ledges can also have low walls 2 to 3 feet high along their edges. Such walls provide cover against attackers within 30 feet on the other side of the wall, as long as the target is closer to the low wall than the attacker is.
				
\textbf{Transparent Floor}: Transparent floors, made of reinforced glass or magic materials (even a \textit{wall of force}), allow a dangerous setting to be viewed safely from above. Transparent floors are sometimes placed over lava pools, arenas, monster dens, and torture chambers. They can be used by defenders to watch key areas for intruders.
				
\textbf{Sliding Floors}: A sliding floor is a type of trap door, designed to be moved and thus reveal something that lies beneath it. A typical sliding floor moves so slowly that anyone standing on one can avoid falling into the gap it creates, assuming there's somewhere else to go. If such a floor slides quickly enough that there's a chance of a character falling into whatever lies beneath---a spiked pit, a vat of burning oil, or a pool filled with sharks---then it's a trap.
				
\textbf{Trap Floors}: Some floors are designed to become suddenly dangerous. With the application of just the right amount of weight, or the pull of a lever somewhere nearby, spikes protrude from the floor, gouts of steam or flame shoot up from hidden holes, or the entire floor tilts. These strange floors are sometimes found in arenas, designed to make combats more exciting and deadly. Construct these floors as you would any other trap. 
				
\subsection{Doors}

				
Doors in dungeons are much more than mere entrances and exits. Often they can be encounters all by themselves. Dungeon doors come in three basic types: wooden, stone, and iron.
\begin{table}[]
\sffamily
\caption{Table: Doors}
\begin{tabular}{llllll}
                   &                            &                   &                     & \multicolumn{2}{c}{\textbf{break DC}}\\
\textbf{Door Type} & \textbf{Typical Thickness} & \textbf{Hardness} & \textbf{Hit Points} & Stuck & Locked\\

Simple wooden & 1 in. & 5 & 10 hp & 13 & 15 \\
 Good wooden & 1-1/2 in. & 5 & 15 hp & 16 & 18 \\
 Strong wooden & 2 in. & 5 & 20 hp & 23 & 25 \\
 Stone & 4 in. & 8 & 60 hp & 28 & 28 \\
 Iron & 2 in. & 10 & 60 hp & 28 & 28 \\
 Portcullis, wooden & 3 in & 5 & 30 hp & 25* & 25* \\
 Portcullis, iron & 2 in. & 10 & 60 hp & 25* & 25* \\
 Lock & - & 15 & 30 hp & - & - \\
 Hinge & - & 10 & 30 hp & - & -\\
\end{tabular}
* DC to lift. Use appropriate door figure for breaking.\\
\end{table}
				
\textbf{Wooden Doors}: Constructed of thick planks nailed together, sometimes bound with iron for strength (and to reduce swelling from dungeon dampness), wooden doors are the most common type. Wooden doors come in varying strengths: simple, good, and strong. Simple doors (break DC 15) are not meant to keep out motivated attackers. Good doors (break DC 18), while sturdy and long-lasting, are still not meant to take much punishment. Strong doors (break DC 25) are bound in iron and are a sturdy barrier to those attempting to get past them. Iron hinges fasten the door to its frame, and typically a circular pull-ring in the center is there to help open it. Sometimes, instead of a pull-ring, a door has an iron pull-bar on one or both sides of the door to serve as a handle. In inhabited dungeons, these doors are usually well-maintained (not stuck) and unlocked, although important areas are locked up if possible.
				
\textbf{Stone}: Carved from solid blocks of stone, these heavy, unwieldy doors are often built so that they pivot when opened, although dwarves and other skilled craftsfolk are able to fashion hinges strong enough to hold up a stone door. Secret doors concealed within a stone wall are usually stone doors. Otherwise, such doors stand as tough barriers protecting something important beyond. Thus, they are often locked or barred.
				
\textbf{Iron}: Rusted but sturdy, iron doors in a dungeon are hinged like wooden doors. These doors are the toughest form of nonmagical door. They are usually locked or barred.
				
\textbf{Breaking Doors}: Dungeon doors might be locked, trapped, reinforced, barred, magically sealed, or sometimes just stuck. All but the weakest characters can eventually knock down a door with a heavy tool such as a sledgehammer, and a number of spells and magic items give characters an easy way around a locked door.
				
Attempts to literally chop down a door with a slashing or bludgeoning weapon use the hardness and hit points given in Table: Doors. When assigning a DC to an attempt to knock a door down, use the following as guidelines.
				
\textit{DC 10 or Lower}: a door just about anyone can break open.
				
\textit{DC 11--15}: a door that a strong person could break with one try and an average person might be able to break with one try. 
				
\textit{DC 16--20}: a door that almost anyone could break, given time.
				
\textit{DC 21--25}: a door that only a strong or very strong person has a hope of breaking, probably not on the first try.
				
\textit{DC 26 or Higher}: a door that only an exceptionally strong person has a hope of breaking.
				
\textbf{Locks}: Dungeon doors are often locked, and thus the Disable Device skill comes in very handy. Locks are usually built into the door, either on the edge opposite the hinges or right in the middle of the door. Built-in locks either control an iron bar that juts out of the door and into the wall of its frame, or else a sliding iron bar or heavy wooden bar that rests behind the entire door. By contrast, padlocks are not built-in but usually run through two rings, one on the door and the other on the wall. More complex locks, such as combination locks and puzzle locks, are usually built into the door itself. Because such keyless locks are larger and more complex, they are typically only found in sturdy doors (strong wooden, stone, or iron doors).
				
The Disable Device DC to pick a lock often falls within the range of 20 to 30, although locks with lower or higher DCs can exist. A door can have more than one lock, each of which must be unlocked separately. Locks are often trapped, usually with poison needles that extend out to prick a rogue's finger.
				
Breaking a lock is sometimes quicker than breaking the whole door. If a PC wants to whack at a lock with a weapon, treat the typical lock as having hardness 15 and 30 hit points. A lock can only be broken if it can be attacked separately from the door, which means that a built-in lock is immune to this sort of treatment. In an occupied dungeon, every locked door should have a key somewhere. 
				
A special door might have a lock with no key, instead requiring that the right combination of nearby levers must be manipulated or the right symbols must be pressed on a keypad in the correct sequence to open the door.
				
\textbf{Stuck Doors}: Dungeons are often damp, and sometimes doors get stuck, particularly wooden doors. Assume that about 10\% of wooden doors and 5\% of non-wooden doors are stuck. These numbers can be doubled (to 20\% and 10\%, respectively) for long-abandoned or neglected dungeons.
				
\textbf{Barred Doors}: When characters try to bash down a barred door, it's the quality of the bar that matters, not the material the door is made of. It takes a DC 25 Strength check to break through a door with a wooden bar, and a DC 30 Strength check if the bar is made of iron. Characters can attack the door and destroy it instead, leaving the bar hanging in the now-open doorway.
				
\textbf{Magic Seals}: Spells such as \textit{arcane lock }can discourage passage through a door. A door with an \textit{arcane lock }spell on it is considered locked even if it doesn't have a physical lock. It takes a \textit{knock }spell, a \textit{dispel magic }spell, or a successful Strength check to open such a door.
				
\textbf{Hinges}: Most doors have hinges, but sliding doors do not. They usually have tracks or grooves instead, allowing them to slide easily to one side.
				
\textit{Standard Hinges}: These hinges are metal, joining one edge of the door to the door frame or wall. Remember that the door swings open toward the side with the hinges. (So, if the hinges are on the PCs' side, the door opens toward them; otherwise it opens away from them.) Adventurers can take the hinges apart one at a time with successful Disable Device checks (assuming the hinges are on their side of the door, of course). Such a task has a DC of 20 because most hinges are rusted or stuck. Breaking a hinge is difficult. Most have hardness 10 and 30 hit points. The break DC for a hinge is the same as for breaking down the door.
				
\textit{Nested Hinges}: These hinges are much more complex than ordinary hinges, and are found only in areas of excellent construction. These hinges are built into the wall and allow the door to swing open in either direction. PCs can't get at the hinges to fool with them unless they break through the door frame or wall. Nested hinges are typically found on stone doors but sometimes on wooden or iron doors as well. 
				
\textit{Pivots}: Pivots aren't really hinges at all, but simple knobs jutting from the top and bottom of the door that fit into holes in the door frame, allowing the door to spin. The advantages of pivots are that they can't be dismantled like hinges and they're simple to make. The disadvantage is that since the door pivots on its center of gravity (typically in the middle), nothing larger than half the door's width can fit through without squeezing. Doors with pivots are usually stone and often quite wide to overcome this disadvantage. Another solution is to place the pivot toward one side and have the door be thicker at that end and thinner toward the other end so that it opens more like a normal door. Secret doors in walls often turn on pivots, since the lack of hinges makes it easier to hide the door's presence. Pivots also allow objects such as bookcases to be used as secret doors.
				
\textbf{Secret Doors}: Disguised as a bare patch of wall (or floor or ceiling), a bookcase, a fireplace, or a fountain, a secret door leads to a secret passage or room. Someone examining the area finds a secret door, if one exists, on a successful Perception check (DC 20 for a typical secret door to DC 30 for a well-hidden secret door).
				
Many secret doors require special methods of opening, such as hidden buttons or pressure plates. Secret doors can open like normal doors, or they might pivot, slide, sink, rise, or even lower like a drawbridge to permit access. Builders might put a secret door low near the floor or high in a wall, making it difficult to find or reach. Wizards and sorcerers have a spell, \textit{phase door}, that allows them to create a magic secret door that only they can use.
				
\textbf{Magic Doors}: Enchanted by the original builders, a door might speak to explorers, warning them away. It might be protected from harm, increasing its hardness or giving it more hit points as well as an improved saving throw bonus against \textit{disintegrate }and similar spells. A magic door might not lead into the space behind it, but instead might be a portal to a faraway place or even another plane of existence. Other magic doors might require passwords or special keys to open them. 
				
\textbf{Portcullises}: These special doors consist of iron or thick, ironbound wooden shafts that descend from recesses in the ceilings above archways. Sometimes a portcullis has crossbars that create a grid, sometimes not. Typically raised by means of a winch or a capstan, a portcullis can be dropped quickly, and the shafts end in spikes to discourage anyone from standing underneath (or from attempting to dive under it as it drops). Once it is dropped, a portcullis locks, unless it is so large that no normal person could lift it anyway. In any event, lifting a typical portcullis requires a DC 25 Strength check.
				
\subsection{Walls, Doors, and Detect Spells}

				
Stone walls, iron walls, and iron doors are usually thick enough to block most detect spells, such as \textit{detect thoughts. }Wooden walls, wooden doors, and stone doors are usually not thick enough to do so. A secret stone door built into a wall and as thick as the wall itself (at least 1 foot) does block most detect spells.
				
\subsection{Stairs}

				
Stairs are the most common means of traveling up and down within a dungeon. A character can move up or down stairs as part of their movement without penalty, but they cannot run on them. Increase the DC of any Acrobatics skill check made on stairs by 4. Some stairs are particularly steep and are treated as difficult terrain.
				
\subsection{Cave-Ins and Collapses (CR 8)}

				
Cave-ins and collapsing tunnels are extremely dangerous. Not only do dungeon explorers face the danger of being crushed by tons of falling rock, but even if they survive they might be buried beneath a pile of rubble or cut off from the only known exit. A cave-in buries anyone in the middle of the collapsing area, and then sliding debris damages anyone in the periphery of the collapse. A typical corridor subject to a cave-in might have a bury zone with a 15-foot radius and a 10-foot-wide slide zone extending beyond the bury zone. A weakened ceiling can be spotted with a DC 20 Knowledge (engineering) or DC 20 Craft (stonemasonry) check. Remember that Craft checks can be made untrained as Intelligence checks. A dwarf can make such a check if he simply passes within 10 feet of a weakened ceiling. 
				
A weakened ceiling might collapse when subjected to a major impact or concussion. A character can cause a cave-in by destroying half the pillars holding up the ceiling. 
				
Characters in the bury zone of a cave-in take 8d6 points of damage, or half that amount if they make a DC 15 Reflex save. They are subsequently buried. Characters in the slide zone take 3d6 points of damage, or no damage at all if they make a DC 15 Reflex save. Characters in the slide zone who fail their saves are buried.
				
Characters take 1d6 points of nonlethal damage per minute while buried. If such a character falls unconscious, he must make a DC 15 Constitution check each minute. If it fails, he takes 1d6 points of lethal damage each minute until freed or dead.
				
Characters who aren't buried can dig out their friends. In 1 minute, using only her hands, a character can clear rocks and debris equal to five times her heavy load limit. The amount of loose stone that fills a 5-foot-by-5-foot area weighs 1 ton (2,000 pounds). Armed with an appropriate tool, such as a pick, crowbar, or shovel, a digger can clear loose stone twice as quickly as by hand. A buried character can attempt to free himself with a DC 25 Strength check.
				
\subsection{Slimes, Molds, and Fungi}

				
In a dungeon's damp, dark recesses, molds and fungi thrive. For purposes of spells and other special effects, all slimes, molds, and fungi are treated as plants. Like traps, dangerous slimes and molds have CRs, and characters earn XP for encountering them.
				
A form of glistening organic sludge coats almost anything that remains in the damp and dark for too long. This kind of slime, though it might be repulsive, is not dangerous. Molds and fungi flourish in dark, cool, damp places. While some are as inoffensive as the normal dungeon slime, others are quite dangerous. Mushrooms, puffballs, yeasts, mildew, and other sorts of bulbous, fibrous, or flat patches of fungi can be found throughout most dungeons. They are usually inoffensive, and some are even edible (although most are unappealing or odd-tasting).
				
\textbf{Brown Mold (CR 2)}: Brown mold feeds on warmth, drawing heat from anything around it. It normally comes in patches 5 feet in diameter, and the temperature is always cold in a 30-foot radius around it. Living creatures within 5 feet of it take 3d6 points of nonlethal cold damage. Fire brought within 5 feet of brown mold causes the mold to instantly double in size. Cold damage, such as from a \textit{cone of cold, }instantly destroys it.
				
\textbf{Green Slime (CR 4)}: This dungeon peril is a dangerous variety of normal slime. Green slime devours flesh and organic materials on contact and is even capable of dissolving metal. Bright green, wet, and sticky, it clings to walls, floors, and ceilings in patches, reproducing as it consumes organic matter. It drops from walls and ceilings when it detects movement (and possible food) below.
				
A single 5-foot square of green slime deals 1d6 points of Constitution damage per round while it devours flesh. On the first round of contact, the slime can be scraped off a creature (destroying the scraping device), but after that it must be frozen, burned, or cut away (dealing damage to the victim as well). Anything that deals cold or fire damage, sunlight, or a \textit{remove disease }spell destroys a patch of green slime. Against wood or metal, green slime deals 2d6 points of damage per round, ignoring metal's hardness but not that of wood. It does not harm stone.
				
\textbf{Phosphorescent Fungus}: This strange underground fungus gives off a soft violet glow that illuminates underground caverns and passages as well as a candle does. Rare patches of fungus illuminate as well as a torch does.
				
\textbf{Shrieker}: This human-sized purple mushroom emits a piercing sound that lasts for 1d3 rounds whenever there is movement or a light source within 10 feet. This shriek makes it impossible to hear any other sound within 50 feet. The sound attracts nearby creatures that are disposed to investigate it. Some creatures that live near shriekers learn that this noise means there is food or an intruder nearby.
				
\textbf{Yellow Mold (CR 6)}: If disturbed, a 5-foot square of this mold bursts forth with a cloud of poisonous spores. All within 10 feet of the mold must make a DC 15 Fortitude save or take 1d3 points of Constitution damage. Another DC 15 Fortitude save is required once per round for the next 5 rounds, to avoid taking 1d3 points of Constitution damage each round. A successful Fortitude save ends this effect. Fire destroys yellow mold, and sunlight renders it dormant.
				
\section{Traps}

				
Traps are a common danger in dungeon environments. From gouts of white-hot flame to hails of poisoned darts, traps can serve to protect valuable treasure or stop intruders from proceeding.
				
\subsection{Elements of a Trap}

				
All traps---mechanical or magical---have the following elements: CR, type, Perception DC, Disable Device DC, trigger, reset, and effect. Some traps might also include optional elements, such as poison or a bypass. These characteristics are described below.
				
\subsection{Type}

				
A trap can be either mechanical or magical in nature. 
				
\textbf{Mechanical}: Dungeons are frequently equipped with deadly mechanical (nonmagical) traps. A trap typically is defined by its location and triggering conditions, how hard it is to spot before it goes off, how much damage it deals, and whether or not the characters receive a saving throw to mitigate its effects. Traps that attack with arrows, sweeping blades, and other types of weaponry make normal attack rolls, with specific attack bonuses dictated by the trap's design. A mechanical trap can be constructed by a PC through successful use of the Craft (traps) skill (see Designing a Trap and the Craft skill description).
				
Creatures that succeed on a Perception check detect a trap before it is triggered. The DC of this check depends on the trap itself. Success generally indicates that the creature has detected the mechanism that activates the trap, such as a pressure plate, odd gears attached to a door handle, and the like. Beating this check by 5 or more also gives some indication of what the trap is designed to do.
				
\textbf{Magic}: Many spells can be used to create dangerous traps. Unless the spell or item description states otherwise, assume the following to be true.
				\begin{itemize}\item  A successful Perception check (DC 25 + spell level) detects a magic trap before it goes off. 
				\item  Magic traps permit a saving throw in order to avoid the effect (DC 10 + spell level \mbox{$\times$} 1.5).
				\item  Magic traps may be disarmed by a character with the trapfinding class feature with a successful Disable Device skill check (DC 25 + spell level). Other characters have no chance to disarm a magic trap with a Disable Device check.
\end{itemize}
				
Magic traps are further divided into spell traps and magic device traps. Magic device traps initiate spell effects when activated, just as wands, rods, rings, and other magic items do. Creating a magic device trap requires the Craft Wondrous Item feat.
				
Spell traps are simply spells that themselves function as traps\textit{. }Creating a spell trap requires the services of a character who can cast the needed spell or spells, who is usually either the character creating the trap or an NPC spellcaster hired for that purpose. 
				
\subsection{Perception and Disable Device DCs}

				
The builder sets the Perception and Disable Device DCs for a mechanical trap. For a magic trap, the values depend on the highest-level spell used.
				
\textbf{Mechanical Trap}: The base DC for both Perception and Disable Device checks is 20. Raising or lowering either of these DCs affects the CR (Table: CR Modifiers for Mechanical Traps). 
				
\textbf{Magic Trap}: The DC for both Perception and Disable Device checks is equal to 25 + the spell level of the highest-level spell used. Only characters with the trapfinding class feature can attempt a Disable Device check involving a magic trap. 
				
\subsection{Trigger}

				
A trap's trigger determines how it is sprung.
				
\textbf{Location}: A location trigger springs a trap when someone stands in a particular square.
				
\textbf{Proximity}: This trigger activates the trap when a creature approaches within a certain distance of it. A proximity trigger differs from a location trigger in that the creature need not be standing in a particular square. Creatures that are flying can spring a trap with a proximity trigger but not one with a location trigger. Mechanical proximity triggers are extremely sensitive to the slightest change in the air. This makes them useful only in places such as crypts, where the air is unusually still.
				
The proximity trigger used most often for magic device traps is the \textit{alarm }spell. Unlike when the spell is cast, an \textit{alarm }spell used as a trigger can have an area that's no larger than the area the trap is meant to protect.
				
Some magic device traps have special proximity triggers that activate only when certain kinds of creatures approach. For example, a \textit{detect good }spell can serve as a proximity trigger on an evil altar, springing the attached trap only when someone of good alignment gets close enough to it.
				
\textbf{Sound}: This trigger springs a magic trap when it detects any sound. A sound trigger functions like an ear and has a +15 bonus on Perception checks. A successful Stealth check, magical \textit{silence, }and other effects that would negate hearing defeat it. A trap with a sound trigger requires the casting of \textit{clairaudience }during its construction.
				
\textbf{Visual}: This trigger for magic traps works like an actual eye, springing the trap whenever it \texttt{{}"{}}sees\texttt{{}"{}} something. A trap with a visual trigger requires the casting of \textit{arcane eye, clairvoyance, }or \textit{true seeing }during its construction. Sight range and the Perception bonus conferred on the trap depend on the spell chosen, as shown.

SpellSight RangePerception Bonus

\textit{arcane eye}Line of sight (unlimited range)+20
\textit{clairvoyance}One preselected location+15
\textit{true seeing}Line of sight (up to 120 ft.)+30
% </tbody id="trigger">

				
If you want the trap to see in the dark, you must either choose the \textit{true seeing }option or add \textit{darkvision }to the trap as well. (\textit{Darkvision }limits the trap's sight range in the dark to 60 feet.) If invisibility, disguises, or illusions can fool the spell being used, they can fool the visual trigger as well. 
				
\textbf{Touch}: A touch trigger, which springs the trap when touched, is one of the simplest kinds of trigger to construct. This trigger may be physically attached to the part of the mechanism that deals the damage or it may not. You can make a magic touch trigger by adding \textit{alarm }to the trap and reducing the area of the effect to cover only the trigger spot.
				
\textbf{Timed}: This trigger periodically springs the trap after a certain duration has passed.
				
\textbf{Spell}: All spell traps have this kind of trigger. The appropriate spell descriptions explain the trigger conditions for traps that contain spell triggers.
				
\subsection{Duration}

				
Unless otherwise stated, most traps have a duration of instantaneous; once triggered, they have their effect and then stop functioning. Some traps have a duration measured in rounds. Such traps continue to have their listed effect each round at the top of the initiative order (or whenever they were activated, if they were triggered during combat).
				
\subsection{Reset}

				
A reset element is the set of conditions under which a trap becomes ready to trigger again. Resetting a trap usually takes only a minute or so. For a trap with a more difficult reset method, you should set the time and labor required.
				
\textbf{No Reset}: Short of completely rebuilding the trap, there's no way to trigger it more than once. Spell traps have no reset element. 
				
\textbf{Repair}: To get the trap functioning again, you must repair it. Repairing a mechanical trap requires a Craft (traps) check against a DC equal to the one for building it. The cost for raw materials is one-fifth of the trap's original market price. To calculate how long it takes to fix a trap, use the same calculations you would for building it, but use the cost of the raw materials required for repair in place of the market price.
				
\textbf{Manual}: Resetting the trap requires someone to move the parts back into place. This is the kind of reset element most mechanical traps have.
				
\textbf{Automatic}: The trap resets itself, either immediately or after a timed interval.
				
\subsection{Bypass (Optional Element)}

				
If the builder of a trap wants to be able to move past the trap after it is created or placed, it's a good idea to build in a bypass mechanism: something that temporarily disarms the trap. Bypass elements are typically used only with mechanical traps; spell traps usually have built-in allowances for the caster to bypass them.
				
\textbf{Lock}: A lock bypass requires a DC 30 Disable Device check to open. 
				
\textbf{Hidden Switch}: A hidden switch requires a DC 25 Perception check to locate.
				
\textbf{Hidden Lock}: A hidden lock combines the features above, requiring a DC 25 Perception check to locate and a DC 30 Disable Device check to open.
				
\subsection{Effect}

				
The effect of a trap is what happens to those who spring it. This often takes the form of either damage or a spell effect, but some traps have special effects. A trap usually either makes an attack roll or forces a saving throw to avoid it. Occasionally a trap uses both of these options, or neither (see Never Miss).
				
\textbf{Pits}: These are holes (covered or not) that characters can fall into, causing them to take damage. A pit needs no attack roll, but a successful Reflex save (DC set by the builder) avoids it. Other save-dependent mechanical traps also fall into this category. Falling into a pit deals 1d6 points of damage per 10 feet of depth.
				
Pits in dungeons come in three basic varieties: uncovered, covered, and chasms. Pits and chasms can be defeated by judicious application of the Acrobatics skill, the Climb skill, or various mechanical or magical means.
				
Uncovered pits and natural chasms serve mainly to discourage intruders from going a certain way, although they cause much grief to characters who stumble into them in the dark, and they can greatly complicate nearby melee.
				
Covered pits are much more dangerous. They can be detected with a DC 20 Perception check, but only if the character is taking the time to carefully examine the area before walking across it. A character who fails to detect a covered pit is still entitled to a DC 20 Reflex save to avoid falling into it. If she was running or moving recklessly at the time, however, she gets no saving throw and falls automatically.
				
Trap coverings can be as simple as piled refuse (straw, leaves, sticks, garbage), a large rug, or an actual trap door concealed to appear as a normal part of the floor. Such a trap door usually swings open when enough weight (usually about 50 to 80 pounds) is placed upon it. Devious trap builders sometimes design trap doors so they spring back shut after they open. The trap door might lock once it's back in place, leaving the stranded character well and truly trapped. Opening such a trap door is just as difficult as opening a regular door (assuming the trapped character can reach it), and a DC 13 Strength check is needed to keep a spring-loaded door open.
				
Pit traps often have something nastier than just a hard floor at the bottom. A trap designer might put spikes, monsters, or a pool of acid, lava, or even water at the bottom. For rules on pit spikes and other such add-ons, see the Miscellaneous Trap Features section.
				
Monsters sometimes live in pits. Any monster that can fit into the pit might have been placed there by the dungeon's designer, or might simply have fallen in and not been able to climb back out. 
				
A secondary trap, mechanical or magical, at the bottom of a pit can be particularly deadly. Activated by a falling victim, the secondary trap attacks the already injured character when she's least ready for it.
				
\textbf{Ranged Attack Traps}: These traps fling darts, arrows, spears, or the like at whomever activated the trap. The builder sets the attack bonus. A ranged attack trap can be configured to simulate the effect of a composite bow with a high Strength rating, which provides the trap with a bonus on damage equal to its Strength rating. These traps deal whatever damage their ammunition normally does. If a trap is constructed with a high Strength rating, it has a corresponding bonus on damage.
				
\textbf{Melee Attack Traps}: These traps feature such obstacles as sharp blades that emerge from walls and stone blocks that fall from ceilings. Once again, the builder sets the attack bonus. These traps deal the same damage as the melee weapons they \texttt{{}"{}}wield.\texttt{{}"{}} In the case of a falling stone block, you can assign any amount of bludgeoning damage you like, but remember that whoever resets the trap has to lift that stone back into place. 
				
A melee attack trap can be constructed with a built-in bonus on damage rolls, just as if the trap itself had a high Strength score.
				
\textbf{Spell Traps}: Spell traps produce the spell's effect\textit{. }Like all spells, a spell trap that allows a saving throw has a save DC of 10 + spell level + caster's relevant ability modifier.
				
\textbf{Magic Device Traps}: These traps produce the effects of any spells included in their construction, as described in the appropriate entries\textit{. }If the spell in a magic device trap allows a saving throw, its save DC is (10 + spell level) \mbox{$\times$} 1.5. Some spells make attack rolls instead.
				
\textbf{Special}: Some traps have miscellaneous features that produce special effects, such as drowning for a water trap or ability damage for poison. Saving throws and damage depend on the poison or are set by the builder, as appropriate.
				
\subsection{Miscellaneous Trap Features}

				
Some traps include optional features that can make them considerably more deadly. The most common features are discussed below.
				
\textbf{Alchemical Item}: Mechanical traps might incorporate alchemical devices or other special substances or items, such as tanglefoot bags, alchemist's fire, thunderstones, and the like. Some such items mimic spell effects. If the item mimics a spell effect, it increases the CR as shown on Table: CR Modifiers for Mechanical Traps: CR Modifiers for Mechanical Traps.
				
\textbf{Gas}: With a gas trap, the danger is in the inhaled poison it delivers. Traps employing gas usually have the never miss and onset delay features.
				
\textbf{Liquid}: Any trap that involves a danger of drowning is in this category. Traps employing liquid usually have the never miss and onset delay features. 
				
\textbf{Multiple Targets}: Traps with this feature can affect more than one character.
				
\textbf{Never Miss}: When the entire dungeon wall moves to crush you, your quick reflexes won't help, since the wall can't possibly miss. A trap with this feature has neither an attack bonus nor a saving throw to avoid, but it does have an onset delay. Most traps involving liquid or gas are of the never miss variety. 
				
\textbf{Onset Delay}: An onset delay is the amount of time between when the trap is sprung and when it deals damage. A never miss trap always has an onset delay.
				
\textbf{Poison}: Traps that employ poison are deadlier than their nonpoisonous counterparts, so they have correspondingly higher CRs. To determine the CR modifier for a given poison, consult Table: CR Modifiers for Mechanical Traps. Only injury, contact, and inhaled poisons are suitable for traps; ingested types are not. Some traps simply deal the poison's damage. Others deal damage with ranged or melee attacks as well.
				
\textbf{Pit Spikes}: Treat spikes at the bottom of a pit as daggers, each with a +10 attack bonus. The damage bonus for each spike is +1 per 10 feet of pit depth (to a maximum of +5). Each character who falls into the pit is attacked by 1d4 spikes. This damage is in addition to any damage from the fall itself, and the statistics presented above are merely the most common variant---some traps might have far more dangerous spikes at their bottom. Pit spikes add to the average damage of the trap (see Average Damage, below).
				
\textbf{Pit Bottom}: If something other than spikes waits at the bottom of a pit, it's best to treat that as a separate trap (see Multiple Traps) with a location trigger that activates on any significant impact, such as a falling character. 
				
\textbf{Touch Attack}: This feature applies to any trap that needs only a successful touch attack (melee or ranged) to hit.
				
\section{Sample Traps}

				
The following sample traps represent just some of the possibilities when constructing traps to challenge the player characters.


\statblocktitle{Arrow Trap CR 1}\\
\textbf{Type }mechanical; \textbf{Perception} DC 20; \textbf{Disable Device} DC 20\\
\statblockbreaker{Effects}\\
\textbf{Trigger} touch; \textbf{Reset} none\\
\textbf{Effect }Atk +15 ranged (1d8+1/×3)\\

\statblocktitle{Pit Trap CR 1}\\
\textbf{Type }mechanical; \textbf{Perception} DC 20; \textbf{Disable Device} DC 20\\
\statblockbreaker{Effects}\\
\textbf{Trigger} location; \textbf{Reset} manual\\
\textbf{Effect }20-ft.-deep pit (2d6 falling damage); DC 20 Reflex avoids; multiple targets (all targets in a 10-ft.-square area)\\

\statblocktitle{Poisoned Dart Trap CR 1}\\
\textbf{Type }mechanical; \textbf{Perception} DC 20; \textbf{Disable Device} DC 20\\
\statblockbreaker{Effects}\\
\textbf{Trigger} touch; \textbf{Reset} none\\
\textbf{Effect }Atk +10 ranged (1d3 plus greenblood oil)\\

\statblocktitle{Swinging Axe Trap CR 1}\\
\textbf{Type }mechanical; \textbf{Perception} DC 20; \textbf{Disable Device} DC 20\\
\statblockbreaker{Effects}\\
\textbf{Trigger} location; \textbf{Reset} manual\\
\textbf{Effect }Atk +10 melee (1d8+1/×3); multiple targets (all targets in a 10-ft. line)\\

\statblocktitle{Burning Hands Trap CR 2}\\
\textbf{Type }magic; \textbf{Perception} DC 26; \textbf{Disable Device} DC 26\\
\statblockbreaker{Effects}\\
\textbf{Trigger} proximity (\textit{alarm}); \textbf{Reset} none\\
\textbf{Effect }spell effect (\textit{burning hands, }2d4 fire damage, DC 11 Reflex save for half damage); multiple targets (all targets in a 15-ft. cone)\\

\statblocktitle{Javelin Trap CR 2}\\
\textbf{Type }mechanical; \textbf{Perception} DC 20; \textbf{Disable Device} DC 20\\
\statblockbreaker{Effects}\\
\textbf{Trigger} location; \textbf{Reset} none\\
\textbf{Effect }Atk +15 ranged (1d6+6)\\

\statblocktitle{Spiked Pit Trap CR 2}\\
\textbf{Type }mechanical; \textbf{Perception} DC 20; \textbf{Disable Device} DC 20\\
\statblockbreaker{Effects}\\
\textbf{Trigger} location; \textbf{Reset} manual\\
\textbf{Effect }10-ft.-deep pit (1d6 falling damage); pit spikes (Atk +10 melee, 1d4 spikes per target for 1d4+2 damage each); DC 20 Reflex avoids; multiple targets (all targets in a 10-ft.-square area)\\

\statblocktitle{Acid Arrow Trap CR 3}\\
\textbf{Type }magic; \textbf{Perception} DC 27; \textbf{Disable Device} DC 27\\
\statblockbreaker{Effects}\\
\textbf{Trigger} proximity (\textit{alarm}); \textbf{Reset} none\\
\textbf{Effect }spell effect (\textit{acid arrow, }Atk +2 ranged touch, 2d4 acid damage for 4 rounds)\\

\statblocktitle{Camouflaged Pit Trap CR 3}\\
\textbf{Type }mechanical; \textbf{Perception} DC 25; \textbf{Disable Device} DC 20\\
\statblockbreaker{Effects}\\
\textbf{Trigger} location; \textbf{Reset} manual\\
\textbf{Effect }30-ft.-deep pit (3d6 falling damage); DC 20 Reflex avoids; multiple targets (all targets in a 10-ft.-square area)\\

\statblocktitle{Electricity Arc Trap CR 4}\\
\textbf{Type }mechanical; \textbf{Perception} DC 25; \textbf{Disable Device} DC 20\\
\statblockbreaker{Effects}\\
\textbf{Trigger} touch; \textbf{Reset} none\\
\textbf{Effect }electricity arc (4d6 electricity damage, DC 20 Reflex save for half damage); multiple targets (all targets in a 30-ft. line)\\

\statblocktitle{Wall Scythe Trap CR 4}\\
\textbf{Type }mechanical; \textbf{Perception} DC 20; \textbf{Disable Device} DC 20\\
\statblockbreaker{Effects}\\
\textbf{Trigger} location; \textbf{Reset} automatic reset\\
\textbf{Effect }Atk +20 melee (2d4+6/×4)\\

\statblocktitle{Falling Block Trap CR 5}\\
\textbf{Type }mechanical; \textbf{Perception} DC 20; \textbf{Disable Device} DC 20\\
\statblockbreaker{Effects}\\
\textbf{Trigger} location; \textbf{Reset} manual\\
\textbf{Effect }Atk +15 melee (6d6); multiple targets (all targets in a 10-ft. square)\\

\statblocktitle{Fireball Trap CR 5}\\
\textbf{Type }magic; \textbf{Perception} DC 28; \textbf{Disable Device} DC 28\\
\statblockbreaker{Effects}\\
\textbf{Trigger} proximity (\textit{alarm}); \textbf{Reset} none\\
\textbf{Effect }spell effect (\textit{fireball, }6d6 fire damage, DC 14 Reflex save for half damage); multiple targets (all targets in a 20-ft.-radius burst)\\

\statblocktitle{Flame Strike Trap CR 6}\\
\textbf{Type }magic; \textbf{Perception} DC 30; \textbf{Disable Device} DC 30\\
\statblockbreaker{Effects}\\
\textbf{Trigger} proximity (\textit{alarm}); \textbf{Reset} none\\
\textbf{Effect }spell effect (\textit{flame strike, }8d6 fire damage, DC 17 Reflex save for half damage); multiple targets (all targets in a 10-ft.-radius cylinder)\\

\statblocktitle{Wyvern Arrow Trap CR 6}\\
\textbf{Type }mechanical; \textbf{Perception} DC 20; \textbf{Disable Device} DC 20\\
\statblockbreaker{Effects}\\
\textbf{Trigger} location; \textbf{Reset} none\\
\textbf{Effect }Atk +15 ranged (1d6 plus wyvern poison/×3)\\

\statblocktitle{Frost Fangs Trap CR 7}\\
\textbf{Type }mechanical; \textbf{Perception} DC 25; \textbf{Disable Device} DC 20\\
\statblockbreaker{Effects}\\
\textbf{Trigger} location; \textbf{Duration }3 rounds; \textbf{Reset} none\\
\textbf{Effect }jets of freezing water (3d6 cold damage, DC 20 Reflex save for half damage); multiple targets (all targets in a 40-ft.-square chamber)\\

\statblocktitle{Summon Monster VI Trap CR 7}\\
\textbf{Type }magic; \textbf{Perception} DC 31; \textbf{Disable Device} DC 31\\
\statblockbreaker{Effects}\\
\textbf{Trigger} proximity (\textit{alarm}); \textbf{Reset} none\\
\textbf{Effect }spell effect (\textit{summon monster VI, }summons 1d3 Large elementals or 1 Huge elemental)\\

\statblocktitle{Camouflaged Spiked Pit Trap CR 8}\\
\textbf{Type }mechanical; \textbf{Perception} DC 25; \textbf{Disable Device} DC 20\\
\statblockbreaker{Effects}\\
\textbf{Trigger} location; \textbf{Reset} manual\\
\textbf{Effect }50-ft.-deep pit (5d6 falling damage); pit spikes (Atk +15 melee, 1d4 spikes per target for 1d6+5 damage each); DC 20 Reflex avoids; multiple targets (all targets in a 10-ft.-square area)\\

\statblocktitle{Insanity Mist Trap CR 8}\\
\textbf{Type }mechanical; \textbf{Perception} DC 25; \textbf{Disable Device} DC 20\\
\statblockbreaker{Effects}\\
\textbf{Trigger} location; \textbf{Reset} repair\\
\textbf{Effect }poison gas (insanity mist); never miss; onset delay (1 round); multiple targets (all targets in a 10-ft.-by-10-ft. room)\\

\statblocktitle{Hail of Arrows Trap CR 9}\\
\textbf{Type }mechanical; \textbf{Perception} DC 25; \textbf{Disable Device} DC 25\\
\statblockbreaker{Effects}\\
\textbf{Trigger} visual (\textit{arcane eye}); \textbf{Reset} repair\\
\textbf{Effect }Atk +20 ranged (6d6); multiple targets (all targets in a 20-ft. line)\\

\statblocktitle{Shocking Floor Trap CR 9}\\
\textbf{Type }magic; \textbf{Perception} DC 26; \textbf{Disable Device} DC 26\\
\statblockbreaker{Effects}\\
\textbf{Trigger} proximity (\textit{alarm}); \textbf{Duration }1d6 rounds; \textbf{Reset} none\\
\textbf{Effect }spell effect (\textit{shocking grasp, }Atk +9 melee touch [4d6 electricity damage]); multiple targets (all targets in a 40-ft.-square room)\\

\statblocktitle{Energy Drain Trap CR 10}\\
\textbf{Type }magic; \textbf{Perception} DC 34; \textbf{Disable Device} DC 34\\
\statblockbreaker{Effects}\\
\textbf{Trigger} visual (\textit{true seeing}); \textbf{Reset} none\\
\textbf{Effect }spell effect (\textit{energy drain, }Atk +10 ranged touch, 2d4 temporary negative levels\textit{, }DC 23 Fortitude negates after 24 hours)\\

\statblocktitle{Chamber of Blades Trap CR 10}\\
\textbf{Type }mechanical; \textbf{Perception} DC 25; \textbf{Disable Device} DC 20\\
\statblockbreaker{Effects}\\
\textbf{Trigger} location; \textbf{Duration }1d4 rounds; \textbf{Reset} repair\\
\textbf{Effect }Atk +20 melee (3d8+3); multiple targets (all targets in a 20-ft.-square chamber)\\

\statblocktitle{Cone of Cold Trap CR 11}\\
\textbf{Type }magic; \textbf{Perception} DC 30; \textbf{Disable Device} DC 30\\
\statblockbreaker{Effects}\\
\textbf{Trigger} proximity (\textit{alarm}); \textbf{Reset} none\\
\textbf{Effect }spell effect (\textit{cone of cold, }15d6 cold damage, DC 17 Reflex save for half damage); multiple targets (all targets in a 60-ft. cone)\\

\statblocktitle{Poisoned Pit Trap CR 12}\\
\textbf{Type }mechanical; \textbf{Perception} DC 25; \textbf{Disable Device} DC 20\\
\statblockbreaker{Effects}\\
\textbf{Trigger} location; \textbf{Reset} manual\\
\textbf{Effect }50-ft.-deep pit (5d6 falling damage); pit spikes (Atk +15 melee, 1d4 spikes per target for 1d6+5 damage each plus poison [shadow essence]); DC 25 Reflex avoids; multiple targets (all targets in a 10-ft.-square area)\\

\statblocktitle{Maximized Fireball Trap CR 13}\\
\textbf{Type }magic; \textbf{Perception} DC 31; \textbf{Disable Device} DC 31\\
\statblockbreaker{Effects}\\
\textbf{Trigger} proximity (\textit{alarm}); \textbf{Reset} none\\
\textbf{Effect }spell effect (\textit{fireball, }60 fire damage, DC 14 Reflex save for half damage); multiple targets (all targets in a 20-ft.-radius burst)\\

\statblocktitle{Harm Trap CR 14}\\
\textbf{Type }magic; \textbf{Perception} DC 31; \textbf{Disable Device} DC 31\\
\statblockbreaker{Effects}\\
\textbf{Trigger} touch; \textbf{Reset} none\\
\textbf{Effect }spell effect (\textit{harm, }+6 melee touch, 130 damage, DC 19 Will save for half, cannot be reduced to less than 1 hit point)\\

\statblocktitle{Crushing Stone Trap CR 15}\\
\textbf{Type }mechanical; \textbf{Perception} DC 30; \textbf{Disable Device} DC 20\\
\statblockbreaker{Effects}\\
\textbf{Trigger} location; \textbf{Reset} manual\\
\textbf{Effect }Atk +15 melee (16d6); multiple targets (all targets in a 10-ft. square)\\

\statblocktitle{Empowered Disintegrate Trap CR 16}\\
\textbf{Type }magic; \textbf{Perception} DC 33; \textbf{Disable Device} DC 33\\
\statblockbreaker{Effects}\\
\textbf{Trigger} sight (\textit{true seeing}); \textbf{Reset} none\\
\textbf{Effect }spell effect (empowered \textit{disintegrate, }+9 ranged touch, 30d6 damage plus 50%, DC 19 Fort save reduces the damage to 5d6 plus 50%)\\

\statblocktitle{Lightning Bolt Gallery Trap CR 17}\\
\textbf{Type }magic; \textbf{Perception} DC 29; \textbf{Disable Device} DC 29\\
\statblockbreaker{Effects}\\
\textbf{Trigger} proximity (\textit{alarm}); \textbf{Duration }1d6 rounds; \textbf{Reset} none\\
\textbf{Effect }spell effect (heightened \textit{lightning bolt, }8d6 electricity damage, DC 16 Reflex save for half); multiple targets (all targets in a 60-ft.-square chamber)\\

\statblocktitle{Deadly Spear Trap CR 18}\\
\textbf{Type }mechanical; \textbf{Perception} DC 30; \textbf{Disable Device} DC 30\\
\statblockbreaker{Effects}\\
\textbf{Trigger} sight (\textit{true seeing}); \textbf{Reset} manual\\
\textbf{Effect }Atk +20 ranged (1d8+6 plus black lotus extract)\\

\statblocktitle{Meteor Swarm Trap CR 19}\\
\textbf{Type }magic; \textbf{Perception} DC 34; \textbf{Disable Device} DC 34\\
\statblockbreaker{Effects}\\
\textbf{Trigger} sight (\textit{true seeing}); \textbf{Reset} none\\
\textbf{Effect }spell effect (meteor swarm\textit{, }4 meteors at separate targets, +9 ranged touch, 2d6 plus 6d6 fire [–4 save on a hit], DC 23 Reflex save for half fire damage, 18d6 fire damage from other meteors, DC 23 Reflex save for half damage); multiple targets (four targets, no two of which can be more than 40 ft. apart)\\

\statblocktitle{Destruction Trap CR 20}\\
\textbf{Type }magic; \textbf{Perception} DC 34; \textbf{Disable Device} DC 34\\
\statblockbreaker{Effects}\\
\textbf{Trigger} proximity (\textit{alarm}); \textbf{Reset} none\\
\textbf{Effect }spell effect (heightened \textit{destruction, }190 damage, DC 23 Fortitude save reduces damage to 10d6)\\


\subsection{Designing a Trap}

				
Designing new traps is a simple process. Start by deciding what type of trap you want to create.
% <
\begin{table}[]
\sffamily
\caption{Table: CR Modifiers for Mechanical Traps}
\begin{tabular}{ll}
Feature & CR Modifier\\
\multicolumn{2}{l}{\textit{Perception DC}}\\
15 or lower & -1\\
16-20 & -\\
21-25 & +1\\
26-29 & +2\\
30 or higher & +3\\
\multicolumn{2}{l}{\textit{Disable Device DC}}\\
15 or lower & -1\\
16-20 & -\\
21-25 & +1\\
26-29 & +2\\
30 or higher & +3\\
\multicolumn{2}{l}{\textit{Reflex Save DC (Pit or Other Save-Dependent Trap)}}\\
15 or lower & -1\\
16-20 & -\\
21-25 & +1\\
26-29 & +2\\
30 or higher & +3\\
\multicolumn{2}{l}{\textit{Attack Bonus (Melee or Ranged Attack Trap)}}\\
+0 or lower & -2\\
+1 to +5 & -1\\
+6 to +10 & -\\
+11 to +15 & +1\\
+16 to +20 & +2\\
Touch attack & +1\\
\multicolumn{2}{l}{\textit{Damage/Effect}}\\
Average damage & +1 per 10 points of average damage\\
\multicolumn{2}{l}{\textit{Miscellaneous Features}}\\
\textbf{Alchemical device} & \textbf{Level of spell mimicked}\\
Automatic reset & +1\\
Liquid & +5\\
Multiple targets (non-damage) & +1\\
Never miss & +2\\
Proximity or visual trigger & +1\\
\multicolumn{2}{l}{\textit{Poison}}\\
Black adder venom & +1\\
Black lotus extract & +8\\
Bloodroot & +1\\
Blue whinnis & +1\\
Burnt othur fumes & +6\\
Deathblade & +5\\
Dragon bile & +6\\
Giant wasp poison & +3\\
Greenblood oil & +1\\
Insanity mist & +4\\
Large scorpion venom & +3 \\
Malyass root paste & +3\\
Medium spider venom & +2\\
Nitharit & +4\\
Purple worm poison & +4\\
Sassone leaf residue & +3\\
Shadow essence & +3\\
Small centipede poison & +1\\
Terinav root & +5\\
Ungol dust & +3\\
Wyvern poison & +5\\
\end{tabular}
\end{table}

\begin{table}[]
\sffamily
\caption{Table: CR Modifiers for Magic Traps}
\begin{tabular}{ll}
\textbf{Feature} & \textbf{CR Modifier}\\
Highest-level spell effect & + Spell level \\
Damaging spell effect & +1 per 10 points of average damage\\
\end{tabular}
\end{table}

\begin{table}[]
\sffamily
\caption{Table: Cost Modifiers for Magic Device Traps}
\begin{tabular}{ll}
\textbf{Feature} & \textbf{Cost Modifier}\\
Alarm & -\\
One-Shot Trap \\
Each spell used & +50 gp $\times$ caster level $\times$ spell level\\
Material components & +Material component costs\\
Automatic Reset Trap \\
Each spell used & +500 gp $\times$ caster level $\times$ spell level\\
Material components & +Material component costs $\times$ 100\\
\end{tabular}
\end{table}

\begin{table}
 \sffamily
 \caption{Craft (Traps) DCs}
 \begin{tabular}{ll}
\textbf{Trap CR} & \textbf{Base Craft (Traps) DC} \\
1--5 & 20\\
6--10 & 25\\
11--15 & 30\\
16+ & 35\\
\textbf{Additional Components} & \textbf{Modifier to Craft (Traps) DC} \\
Proximity trigger & +5\\
Automatic reset & +5\\
 \end{tabular}

\end{table}

				
\textbf{Mechanical Traps}: Simply select the elements you want the trap to have and add up the adjustments to the trap's Challenge Rating that those elements require (see Table: CR Modifiers for Mechanical Traps) to arrive at the trap's final CR. From the CR you can derive the DC of the Craft (traps) checks a character must make to construct the trap.
				
\textbf{Magic Traps}: As with mechanical traps, decide what elements you want and then determine the CR of the resulting trap (see CR Modifiers for Magic Traps). If a player character wants to design and construct a magic trap, he, or an ally, must have the Craft Wondrous Item feat. In addition, he must be able to cast the spell or spells that the trap requires---or he must be able to hire an NPC to cast the spells for him.
				
\subsection{Challenge Rating of a Trap}

				
To calculate the Challenge Rating of a trap, add all the CR modifiers (see Table: CR Modifiers for Mechanical Traps or CR Modifiers for Magic Traps) to the base CR for the trap type.
				
\textbf{Mechanical Trap}: The base CR for a mechanical trap is 0. If your final CR is 0 or lower, add features until you get a CR of 1 or higher.
				
\textbf{Magic Trap}: For a spell trap or magic device trap, the base CR is 1. The highest-level spell used modifies the CR (see CR Modifiers for Magic Traps).
				
\textbf{Average Damage}: If a trap (mechanical or magical) does hit point damage, calculate the average damage for a successful hit and round that value to the nearest multiple of 10. If the trap is designed to hit more than one target, multiply this value by 2. If the trap is designed to deal damage over a number of rounds, multiply this value by the number of rounds the trap will be active (or the average number of rounds, if the duration is variable). Use this value to adjust the Challenge Rating of the trap, as indicated on Table: CR Modifiers for Mechanical Traps. Damage from poison does not count toward this value, but extra damage from pit spikes and multiple attacks does.
				
For a magic trap, only one modifier applies to the CR---either the level of the highest-level spell used in the trap, or the average damage figure, whichever is larger.
				
\textbf{Multiple Traps}: If a trap is really two or more connected traps that affect approximately the same area, determine the CR of each one separately.
				
\textit{Multiple Dependent Traps}: If one trap depends on the success of the other (that is, you can avoid the second trap by not falling victim to the first), characters earn XP for both traps by defeating the first one, regardless if the second one is also sprung.
				
\textit{Multiple Independent Traps}: If two or more traps act independently (they do not depend on one another to activate), characters only earn XP for traps that they defeat.
				
\subsection{Mechanical Trap Cost}

				
The cost of a mechanical trap is 1,000 gp \mbox{$\times$} the trap's Challenge Rating. If the trap uses spells in its trigger or reset, add those costs separately. If the trap cannot be reset, divide the cost in half. If the trap has an automatic reset, increase the cost by half (+50\%). Particularly simple traps, such as pit traps, might have a greatly reduced cost, subject to GM discretion. Such traps might cost as little as 250 gp \mbox{$\times$} the trap's Challenge Rating.
				
After you've determined the cost by Challenge Rating, add the price of any alchemical items or poison you incorporated into the trap. If the trap uses one of these elements and has an automatic reset, multiply the poison or alchemical item cost by 20 to provide an adequate supply of doses.
				
\textbf{Multiple Traps}: If a trap is really two or more connected traps, determine the final cost of each separately, then add those values together. This holds for both multiple dependent and multiple independent traps.
				
\subsection{Magic Device Trap Cost}

				
Building a magic device trap involves the expenditure of gp and requires the services of a spellcaster. Table: Cost Modifiers for Magic Device Traps summarizes the cost information for magic device traps. If the trap uses more than one spell (for instance, a sound or visual trigger spell in addition to the main spell effect), the builder must pay for them all (except \textit{alarm}, which is free unless it must be cast by an NPC).
				
The costs derived from Table: Cost Modifiers for Magic Device Traps assume that the builder is casting the necessary spells himself (or perhaps some other PC is providing the spells for free). If an NPC spellcaster must be hired to cast them, those costs must be factored in as well (see Equipment).
				
A magic device trap takes 1 day to construct per 500 gp of its cost.
				
\subsection{Spell Trap Cost}

				
A spell trap has a cost only if the builder must hire an NPC spellcaster to cast it.
				
\subsection{Craft DCs for Mechanical Traps}

				
Once you know the Challenge Rating of a trap, determine the Craft (traps) DC by referring to the values and modifiers given on Table: Craft (Traps) DCs.
				
\textbf{Making the Checks}: To determine how much progress a character makes on building a trap each week, that character makes a Craft (traps) check. See the Craft skill description for details on Craft checks and the circumstances that can affect them.
				
\section{Wilderness}

				
Outside the safety of city walls, the wilderness is a dangerous place, and many adventurers have gotten lost in its trackless wilds or fallen victim to deadly weather. The following rules give you guidelines on running adventures in a wilderness setting.
				
\subsection{Getting Lost}

				
There are many ways to get lost in the wilderness. Following an obvious road, trail, or feature such as a stream or shoreline prevents most from becoming lost, but travelers striking off cross-country might become disoriented---especially in conditions of poor visibility or in difficult terrain. 
				
\textbf{Poor Visibility}: Anytime characters cannot see at least 60 feet due to reduced visibility conditions, they might become lost. Characters traveling through fog, snow, or a downpour might easily lose the ability to see any landmarks not in their immediate vicinity. Similarly, characters traveling at night might be at risk, too, depending on the quality of their light sources, the amount of moonlight, and whether they have darkvision or low-light vision.
				
\textbf{Difficult Terrain}: Any character in forest, moor, hill, or mountain terrain might become lost if he moves away from a trail, road, stream, or other obvious path or track. Forests are especially dangerous because they obscure far-off landmarks and make it hard to see the sun or stars.
				
\textbf{Chance to Get Lost}: If conditions exist that make getting lost a possibility, the character leading the way must succeed on a Survival check or become lost. The difficulty of this check varies based on the terrain, the visibility conditions, and whether or not the character has a map of the area being traveled through. Refer to the table below and use the highest DC that applies.

\begin{tabular}{ll}
\textbf{Terrain} & \textbf{Survival DC} \\
Desert or plains & 14\\
Forest & 16\\
Moor or hill & 10\\
Mountain & 12\\
Open sea & 18\\
Urban, ruins, or dungeon & 8\\
 \textbf{Situation} & \textbf{Check Modifier}\\
Proper navigational tools (map, sextant) & +4\\
Poor visibility & --4\\
\end{tabular}

				
A character with at least 5 ranks in Knowledge (geography) or Knowledge (local) pertaining to the area being traveled through gains a +2 bonus on this check.
				
Check once per hour (or portion of an hour) spent in local or overland movement to see if travelers have become lost. In the case of a party moving together, only the character leading the way makes the check.
				
\textbf{Effects of Being Lost}: If a party becomes lost, it is no longer certain of moving in the direction it intended to travel. Randomly determine the direction in which the party actually travels during each hour of local or overland movement. The characters' movement continues to be random until they blunder into a landmark they can't miss, or until they recognize that they are lost and make an effort to regain their bearings.
				
\textit{Recognizing You're Lost}: Once per hour of random travel, each character in the party may attempt a Survival check (DC 20, --1 per hour of random travel) to recognize that he is no longer certain of his direction of travel. Some circumstances might make it obvious that the characters are lost.
				
\textit{Setting a New Course}: Determining the correct direction of travel once a party has become lost requires a Survival check (DC 15, +2 per hour of random travel). If a character fails this check, he chooses a random direction as the \texttt{{}"{}}correct\texttt{{}"{}} direction for resuming travel.
				
Once the characters are traveling along their new course, correct or incorrect, they might get lost again. If the conditions still make it possible for travelers to become lost, check once per hour of travel as described above to see if the party maintains its new course or begins to move at random again.
				
\textit{Conflicting Directions}: It's possible that several characters may attempt to determine the right direction to proceed after becoming lost. Make a Survival check for each character in secret, then tell the players whose characters succeeded the correct direction in which to travel, and tell the players whose characters failed a random direction they think is right, with no indication who is correct.
				
\textbf{Regaining Your Bearings}: There are several ways for characters to find their way after becoming lost. First, if the characters successfully set a new course and follow it to the destination they're trying to reach, they're not lost anymore. Second, the characters, through random movement, might run into an unmistakable landmark. Third, if conditions suddenly improve---the fog lifts or the sun comes up---lost characters may attempt to set a new course, as described above, with a +4 bonus on the Survival check.
				
\subsection{Forest Terrain}

				
Forest terrain can be divided into three categories: sparse, medium, and dense. An immense forest could have all three categories within its borders, with more sparse terrain at the outer edge of the forest and dense forest at its heart. 
				
The table below describes in general terms how likely it is that a given square has a terrain element in it.

\begin{tabular}{llll}
 & \multicolumn{3}{c}{\textbf{Category of Forest}} \\
 & \textbf{Sparse} & \textbf{Medium} & \textbf{Dense}\\
Typical trees& 50\% & 70\% & 80\%\\
Massive trees & --- & 10\% & 20\%\\
Light undergrowth & 50\%  & 70\% & 50\%\\
Heavy undergrowth & --- & 20\% & 50\%\\
\end{tabular}
				
\textbf{Trees}: The most important terrain element in a forest is the trees, obviously. A creature standing in the same square as a tree gains partial cover, which grants a +2 bonus to Armor Class and a +1 bonus on Reflex saves. The presence of a tree doesn't otherwise affect a creature's fighting space, because it's assumed that the creature is using the tree to its advantage when it can. The trunk of a typical tree has AC 4, hardness 5, and 150 hp. A DC 15 Climb check is sufficient to climb a tree. Medium and dense forests have massive trees as well. These trees take up an entire square and provide cover to anyone behind them. They have AC 3, hardness 5, and 600 hp. Like their smaller counterparts, it takes a DC 15 Climb check to climb them.
				
\textbf{Undergrowth}: Vines, roots, and short bushes cover much of the ground in a forest. A space covered with light undergrowth costs 2 squares of movement to move into, and provides concealment. Undergrowth increases the DC of Acrobatics and Stealth checks by 2 because the leaves and branches get in the way. Heavy undergrowth costs 4 squares of movement to move into and provides concealment with a 30\% miss chance (instead of the usual 20\%). It increases the DC of Acrobatics checks by 5. Heavy undergrowth is easy to hide in, granting a +5 circumstance bonus on Stealth checks. Running and charging are impossible. Squares with undergrowth are often clustered together. Undergrowth and trees aren't mutually exclusive; it's common for a 5-foot square to have both a tree and undergrowth.
				
\textbf{Forest Canopy}: It's common for elves and other forest dwellers to live on raised platforms far above the surface floor. These wooden platforms often have rope bridges between them. To get to the treehouses, characters ascend the trees' branches (Climb DC 15), use rope ladders (Climb DC 0), or take pulley elevators (which can be made to rise a number of feet equal to a Strength check, made each round as a full-round action). Creatures on platforms or branches in a forest canopy are considered to have cover when fighting creatures on the ground, and in medium or dense forests they have concealment as well.
				
\textbf{Other Forest Terrain Elements}: Fallen logs generally stand about 3 feet high and provide cover just as low walls do. They cost 5 feet of movement to cross. Forest streams average 5 to 10 feet wide and no more than 5 feet deep. Pathways wind through most forests, allowing normal movement and providing neither cover nor concealment. These paths are less common in dense forests, but even unexplored forests have occasional game trails.
				
\textbf{Stealth and Detection in a Forest}: In a sparse forest, the maximum distance at which a Perception check for detecting the nearby presence of others can succeed is 3d6 \mbox{$\times$} 10 feet. In a medium forest, this distance is 2d8 \mbox{$\times$} 10 feet, and in a dense forest it is 2d6 \mbox{$\times$} 10 feet.
				
Because any square with undergrowth provides concealment, it's usually easy for a creature to use the Stealth skill in the forest. Logs and massive trees provide cover, which also makes hiding possible.
				
The background noise in the forest makes Perception checks that rely on sound more difficult, increasing the DC of the check by 2 per 10 feet, not 1. 
				
\subsection{Forest Fires (CR 6)}

				
Most campfire sparks ignite nothing, but if conditions are dry, winds are strong, or the forest floor is dried out and flammable, a forest fire can result. Lightning strikes often set trees ablaze and start forest fires in this way. Whatever the cause of the fire, travelers can get caught in the conflagration.
				
A forest fire can be spotted from as far away as 2d6 \mbox{$\times$} 100 feet by a character who makes a Perception check, treating the fire as a Colossal creature (reducing the DC by 16). If all characters fail their Perception checks, the fire moves closer to them. They automatically see it when it closes to half the original distance. With proper elevation, the smoke from a forest fire can be spotted as far as 10 miles away.
				
Characters who are blinded or otherwise unable to make Perception checks can feel the heat of the fire (and thus automatically \texttt{{}"{}}spot\texttt{{}"{}} it) when it is 100 feet away.
				
The leading edge of a fire (the downwind side) can advance faster than a human can run (assume 120 feet per round for winds of moderate strength). Once a particular portion of the forest is ablaze, it remains so for 2d4 \mbox{$\times$} 10 minutes before dying to a smoking wasteland. Characters overtaken by a forest fire might find the leading edge of the fire advancing away from them faster than they can keep up, trapping them deeper and deeper within its grasp.
				
Within the bounds of a forest fire, a character faces three dangers: heat damage, catching on fire, and smoke inhalation. 
				
\textbf{Heat Damage}: Getting caught within a forest fire is even worse than being exposed to extreme heat (see Heat Dangers). Breathing the air causes a character to take 1d6 points of fire damage per round (no save). In addition, a character must make a Fortitude save every 5 rounds (DC 15, +1 per previous check) or take 1d4 points of nonlethal damage. A character who holds his breath can avoid the lethal damage, but not the nonlethal damage. Those wearing heavy clothing or any sort of armor take a --4 penalty on their saving throws. Those wearing metal armor or who come into contact with very hot metal are affected as if by a \textit{heat metal }spell.
				
\textbf{Catching on Fire}: Characters engulfed in a forest fire are at risk of catching on fire when the leading edge of the fire overtakes them, and continue to be at risk once per minute thereafter.
				
\textbf{Smoke Inhalation}: Forest fires naturally produce a great deal of smoke. A character who breathes heavy smoke must make a Fortitude save each round (DC 15, +1 per previous check) or spend that round choking and coughing. A character who chokes for 2 consecutive rounds takes 1d6 points of nonlethal damage. Smoke also provides concealment to characters within it.
				
\subsection{Marsh Terrain}

				
Two categories of marsh exist: relatively dry moors and watery swamps. Both are often bordered by lakes (described in Aquatic Terrain), which are effectively a third category of terrain found in marshes.

\begin{tabular}{lll}
 & \multicolumn{2}{c}{\textbf{Marsh Category}} \\
 & \textbf{Moor} & \textbf{Swamp} \\
Shallow bog & 20\% & 40\% \\
Deep bog & 5\% & 20\%\\
Light undergrowth & 30\% & 20\%\\
Heavy undergrowth & 10\% & 20\%\\
\end{tabular}
				
\textbf{Bogs}: If a square is part of a shallow bog, it has deep mud or standing water of about 1 foot in depth. It costs 2 squares of movement to move into a square with a shallow bog, and the DC of Acrobatics checks in such a square increases by 2. 
				
A square that is part of a deep bog has roughly 4 feet of standing water. It costs Medium or larger creatures 4 squares of movement to move into a square with a deep bog, or characters can swim if they wish. Small or smaller creatures must swim to move through a deep bog. Tumbling is impossible in a deep bog.
				
The water in a deep bog provides cover for Medium or larger creatures. Smaller creatures gain improved cover (+8 bonus to AC, +4 bonus on Reflex saves). Medium or larger creatures can crouch as a move action to gain this improved cover. Creatures with this improved cover take a --10 penalty on attacks against creatures that aren't underwater.
				
Deep bog squares are usually clustered together and surrounded by an irregular ring of shallow bog squares.
				
Both shallow and deep bogs increase the DC of Stealth checks by 2.
				
\textbf{Undergrowth}: The bushes, rushes, and other tall grasses in marshes function as undergrowth does in a forest. A square that is part of a bog does not also have undergrowth. 
				
\textbf{Quicksand}: Patches of quicksand present a deceptively solid appearance (appearing as undergrowth or open land) that might trap careless characters. A character approaching a patch of quicksand at a normal pace is entitled to a DC 8 Survival check to spot the danger before stepping in, but charging or running characters don't have a chance to detect a hidden patch before blundering into it. A typical patch of quicksand is 20 feet in diameter; the momentum of a charging or running character carries him 1d2 \mbox{$\times$} 5 feet into the quicksand.
				
\textit{Effects of Quicksand}: Characters in quicksand must make a DC 10 Swim check every round to simply tread water in place, or a DC 15 Swim check to move 5 feet in whatever direction is desired. If a trapped character fails this check by 5 or more, he sinks below the surface and begins to drown whenever he can no longer hold his breath (see the Swim skill description in Using Skills)\textit{.}
				
Characters below the surface of quicksand may swim back to the surface with a successful Swim check (DC 15, +1 per consecutive round of being under the surface).
				
\textit{Rescue}: Pulling out a character trapped in quicksand can be difficult. A rescuer needs a branch, spear haft, rope, or similar tool that enables him to reach the victim with one end of it. Then he must make a DC 15 Strength check to successfully pull the victim, and the victim must make a DC 10 Strength check to hold onto the branch, pole, or rope. If both checks succeed, the victim is pulled 5 feet closer to safety. If the victim fails to hold on, he must make a DC 15 Swim check immediately to stay above the surface. 
				
\textbf{Hedgerows}: Common in moors, hedgerows are tangles of stones, soil, and thorny bushes. Narrow hedgerows function as low walls, and it takes 3 squares of movement to cross them. Wide hedgerows are more than 5 feet tall and take up entire squares. They provide total cover, just as a wall does. It takes 4 squares of movement to move through a square with a wide hedgerow; creatures that succeed on a DC 10 Climb check need only 2 squares of movement to move through the square.
				
\textbf{Other Marsh Terrain Elements}: Some marshes, particularly swamps, have trees just as forests do, usually clustered in small stands. Paths lead across many marshes, winding to avoid bog areas. As in forests, paths allow normal movement and don't provide the concealment that undergrowth does.
				
\textbf{Stealth and Detection in a Marsh}: In a marsh, the maximum distance at which a Perception check for detecting the nearby presence of others can succeed is 6d6 \mbox{$\times$} 10 feet. In a swamp, this distance is 2d8 \mbox{$\times$} 10 feet.
				
Undergrowth and deep bogs provide plentiful concealment, so it's easy to use Stealth in a marsh.
				
\subsection{Hills Terrain}

				
A hill can exist in most other types of terrain, but hills can also dominate the landscape. Hills terrain is divided into two categories: gentle hills and rugged hills. Hills terrain often serves as a transition zone between rugged terrain such as mountains and flat terrain such as plains.
\begin{tabular}{lll}
 & \multicolumn{2}{c}{\textbf{Hills Category}}\\
 & \textbf{Gentle Hills} & \textbf{Rugged Hills}\\
Gradual slope & 75\% & 40\%\\
Steep slope & 20\% & 50\%\\
Cliff & 5\% & 10\%\\
Light undergrowth & 15\% & 15\%\\
\end{tabular}

				
\textbf{Gradual Slope}: This incline isn't steep enough to affect movement, but characters gain a +1 bonus on melee attacks against foes downhill from them.
				
\textbf{Steep Slope}: Characters moving uphill (to an adjacent square of higher elevation) must spend 2 squares of movement to enter each square of steep slope. Characters running or charging downhill (moving to an adjacent square of lower elevation) must succeed on a DC 10 Acrobatics check upon entering the first steep slope square. Mounted characters make a DC 10 Ride check instead. Characters who fail this check stumble and must end their movement 1d2 \mbox{$\times$} 5 feet later. Characters who fail by 5 or more fall prone in the square where they end their movement. A steep slope increases the DC of Acrobatics checks by 2.
				
\textbf{Cliff}: A cliff typically requires a DC 15 Climb check to scale and is 1d4 \mbox{$\times$} 10 feet tall, although the needs of your map might mandate a taller cliff. A cliff isn't perfectly vertical, taking up 5-foot squares if it's less than 30 feet tall and 10-foot squares if it's 30 feet or taller. 
				
\textbf{Light Undergrowth}: Sagebrush and other scrubby bushes grow on hills, although they rarely cover the landscape. Light undergrowth provides concealment and increases the DC of Acrobatics and Stealth checks by 2. 
				
\textbf{Other Hills Terrain Elements}: Trees aren't out of place in hills terrain, and valleys often have active streams (5 to 10 feet wide and no more than 5 feet deep) or dry streambeds (treat as a trench 5 to 10 feet across) in them. If you add a stream or streambed, remember that water always flows downhill.
				
\textbf{Stealth and Detection in Hills}: In gentle hills, the maximum distance at which a Perception check for detecting the nearby presence of others can succeed is 2d10 \mbox{$\times$} 10 feet. In rugged hills, this distance is 2d6 \mbox{$\times$} 10 feet.
				
Hiding in hills terrain can be difficult if there isn't undergrowth around. A hilltop or ridge provides enough cover to hide from anyone below the hilltop or ridge.
				
\subsection{Mountain Terrain}

				
The three mountain terrain categories are alpine meadows, rugged mountains, and forbidding mountains. As characters ascend into a mountainous area, they're likely to face each terrain category in turn, beginning with alpine meadows, extending through rugged mountains, and reaching forbidding mountains near the summit.
				
Mountains have an important terrain element, the rock wall, that is marked on the border between squares rather than taking up squares itself. 
								
% <div class="table">
\begin{tabular}{llll}
 & \multicolumn{3}{c}{\textbf{Mountain Category}}\\
 & \textbf{Alpine Meadow} & \textbf{Rugged} & \textbf{Forbidding} \\
Gradual slope & 50\% & 25\% & 15\%\\
Steep slope & 40\% & 55\% & 55\%\\
Cliff & 10\% & 15\% & 20\%\\
Chasm & --- & 5\% & 10\%\\
Light undergrowth & 20\% & 10\% & ---\\
Scree & --- & 20\% & 30\%\\
Dense rubble & --- & 20\% & 30\% \\
\end{tabular}
				
\textbf{Gradual and Steep Slopes}: These function as described in Hills Terrain.
				
\textbf{Cliff}: These terrain elements also function like their hills terrain counterparts, but they're typically 2d6 \mbox{$\times$} 10 feet tall. Cliffs taller than 80 feet take up 20 feet of horizontal space.
				
\textbf{Chasm}: Usually formed by natural geological processes, chasms function like pits in a dungeon setting. Chasms aren't hidden, so characters won't fall into them by accident (although bull rushes are another story). A typical chasm is 2d4 \mbox{$\times$} 10 feet deep, at least 20 feet long, and anywhere from 5 feet to 20 feet wide. It takes a DC 15 Climb check to climb out of a chasm. In forbidding mountain terrain, chasms are typically 2d8 \mbox{$\times$} 10 feet deep.
				
\textbf{Light Undergrowth}: This functions as described in Forest Terrain.
				
\textbf{Scree}: A field of shifting gravel, scree doesn't affect speed, but it can be treacherous on a slope. The DC of Acrobatics checks increases by 2 if there's scree on a gradual slope and by 5 if there's scree on a steep slope. The DC of Stealth checks increases by 2 if the scree is on a slope of any kind.
				
\textbf{Dense Rubble}: The ground is covered with rocks of all sizes. It costs 2 squares of movement to enter a square with dense rubble. The DC of Acrobatics checks on dense rubble increases by 5, and the DC of Stealth checks increases by 2. 
				
\textbf{Rock Wall}: A vertical plane of stone, rock walls require DC 25 Climb checks to ascend. A typical rock wall is 2d4 \mbox{$\times$} 10 feet tall in rugged mountains and 2d8 \mbox{$\times$} 10 feet tall in forbidding mountains. Rock walls are drawn on the edges of squares, not in the squares themselves.
				
\textbf{Cave Entrance}: Found in cliff and steep slope squares and next to rock walls, cave entrances are typically between 5 and 20 feet wide and 5 feet deep. A cave could be anything from a simple chamber to the entrance to an elaborate dungeon. Caves used as monster lairs typically have 1d3 rooms that are 1d4 \mbox{$\times$} 10 feet across. 
				
\textbf{Other Mountain Terrain Features}: Most alpine meadows begin above the treeline, so trees and other forest elements are rare in the mountains. Mountain terrain can include active streams (5 to 10 feet wide and no more than 5 feet deep) and dry streambeds (treat as a trench 5 to 10 feet across). Particularly high-altitude areas tend to be colder than the lowland areas that surround them, so they might be covered in ice sheets (described in Desert Terrain).
				
\textbf{Stealth and Detection in Mountains}: As a guideline, the maximum distance in mountain terrain at which a Perception check for detecting the nearby presence of others can succeed is 4d10 \mbox{$\times$} 10 feet. Certain peaks and ridgelines afford much better vantage points, of course, and twisting valleys and canyons have much shorter spotting distances. Because there's little vegetation to obstruct line of sight, the specifics on your map are your best guide for the range at which an encounter could begin. As in hills terrain, a ridge or peak provides enough cover to hide from anyone below the high point.
				
It's easier to hear faraway sounds in the mountains. The DC of Perception checks that rely on sound increase by 1 per 20 feet between listener and source, not per 10 feet.
				
\subsection{Avalanches (CR 7)}

				
The combination of high peaks and heavy snowfalls means that avalanches are a deadly peril in many mountainous areas. While avalanches of snow and ice are common, it's also possible to have an avalanche of rock and soil.
				
An avalanche can be spotted from as far away as 1d10 \mbox{$\times$} 500 feet by a character who makes a DC 20 Perception check, treating the avalanche as a Colossal creature. If all characters fail their Perception checks to determine the encounter distance, the avalanche moves closer to them, and they automatically become aware of it when it closes to half the original distance. It's possible to hear an avalanche coming even if you can't see it. Under optimum conditions (no other loud noises occurring), a character who makes a DC 15 Perception check can hear the avalanche or landslide when it is 1d6 \mbox{$\times$} 500 feet away. This check might have a DC of 20, 25, or higher in conditions where hearing is difficult (such as in the middle of a thunderstorm). 
				
A landslide or avalanche consists of two distinct areas: the bury zone (in the direct path of the falling debris) and the slide zone (the area the debris spreads out to encompass). Characters in the bury zone always take damage from the avalanche; characters in the slide zone might be able to get out of the way. Characters in the bury zone take 8d6 points of damage, or half that amount if they make a DC 15 Reflex save. They are subsequently buried. Characters in the slide zone take 3d6 points of damage, or no damage if they make a DC 15 Reflex save. Those who fail their saves are buried. 
				
Buried characters take 1d6 points of nonlethal damage per minute. If a buried character falls unconscious, he must make a DC 15 Constitution check or take 1d6 points of lethal damage each minute thereafter until freed or dead. See Cave-Ins and Collapses for rules on digging out buried creatures.
				
The typical avalanche has a width of 1d6 \mbox{$\times$} 100 feet, from one edge of the slide zone to the opposite edge. The bury zone in the center of the avalanche is half as wide as the avalanche's full width.
				
To determine the precise location of characters in the path of an avalanche, roll 1d6 \mbox{$\times$} 20; the result is the number of feet from the center of the path taken by the bury zone to the center of the party's location. Avalanches of snow and ice advance at a speed of 500 feet per round, while rock and soil avalanches travel at a speed of 250 feet per round.
				
\subsection{Mountain Travel}

				
High altitude travel can be extremely fatiguing---and sometimes deadly---to creatures that aren't used to it. Cold becomes extreme, and the lack of oxygen in the air can wear down even the most hardy of warriors.
				
\textbf{Acclimated Characters}: Creatures accustomed to high altitude generally fare better than lowlanders. Any creature with an Environment entry that includes mountains is considered native to the area and acclimated to the high altitude. Characters can also acclimate themselves by living at high altitude for a month. Characters who spend more than two months away from the mountains must reacclimate themselves when they return. Undead, constructs, and other creatures that do not breathe are immune to altitude effects.
				
\textbf{Altitude Zones}: In general, mountains present three possible altitude bands: low pass, low peak/high pass, and high peak. 
				
\textit{Low Pass (lower than 5,000 feet)}: Most travel in low mountains takes place in low passes, a zone consisting largely of alpine meadows and forests. Travelers might find the going difficult (which is reflected in the movement modifiers for traveling through mountains), but the altitude itself has no game effect.
				
\textit{Low Peak or High Pass (5,000 to 15,000 feet)}: Ascending to the highest slopes of low mountains, or most normal travel through high mountains, falls into this category. All non-acclimated creatures labor to breathe in the thin air at this altitude. Characters must succeed on a Fortitude save each hour (DC 15, +1 per previous check) or be fatigued. The fatigue ends when the character descends to an altitude with more air. Acclimated characters do not have to attempt the Fortitude save. 
				
\textit{High Peak (more than 15,000 feet)}: The highest mountains exceed 15,000 feet in height. At these elevations, creatures are subject to both high altitude fatigue (as described above) and altitude sickness, whether or not they're acclimated to high altitudes\textit{. }Altitude sickness represents long-term oxygen deprivation, and affects mental and physical ability scores. After each 6-hour period a character spends at an altitude of over 15,000 feet, he must succeed on a Fortitude save (DC 15, +1 per previous check) or take 1 point of damage to all ability scores. Creatures acclimated to high altitude receive a +4 competence bonus on their saving throws to resist high altitude effects and altitude sickness, but eventually even seasoned mountaineers must abandon these dangerous elevations. 
				
\subsection{Desert Terrain}

				
Desert terrain exists in warm, temperate, and cold climates, but all deserts share one common trait: little rain. The three categories of desert terrain are tundra (cold desert), rocky deserts (often temperate), and sandy deserts (often warm).
				
Tundra differs from the other desert categories in two important ways. Because snow and ice cover much of the landscape, it's easy to find water. During the height of summer, the permafrost thaws to a depth of a foot or so, turning the landscape into a vast field of mud. The muddy tundra affects movement and skill use as the shallow bogs described in Marsh Terrain, although there's little standing water.
				
The table below describes terrain elements found in each of the three desert categories. The terrain elements on this table are mutually exclusive; for instance, a square of tundra might contain either light undergrowth or an ice sheet, but not both.

\begin{tabular}{llll}
 & \multicolumn{3}{c}{\textbf{Desert Category}}\\
 & \textbf{Tundra} & \textbf{Rocky} & \textbf{Sandy} \\
Light undergrowth & 15\% & 5\% & 5\%\\
Ice sheet & 25\% & --- & ---\\
Light rubble & 5\% & 30\% & 10\%\\
Dense rubble & --- & 30\% & 5\%\\
Sand dunes & --- & --- & 50\%\\
\end{tabular}

				
\textbf{Light Undergrowth}: Consisting of scrubby, hardy bushes and cacti, light undergrowth functions as described for other terrain types.
				
\textbf{Ice Sheet}: The ground is covered with slippery ice. It costs 2 squares of movement to enter a square covered by an ice sheet, and the DC of Acrobatics checks there increases by 5. A DC 10 Acrobatics check is required to run or charge across an ice sheet. 
				
\textbf{Light Rubble}: Small rocks are strewn across the ground, making nimble movement more difficult. The DC of Acrobatics checks increases by 2. 
				
\textbf{Dense Rubble}: This terrain feature consists of more and larger stones. It costs 2 squares of movement to enter a square with dense rubble. The DC of Acrobatics checks increases by 5, and the DC of Stealth checks increases by 2.
				
\textbf{Sand Dunes}: Created by the action of wind on sand, dunes function as hills that move. If the wind is strong and consistent, a sand dune can move several hundred feet in a week's time. Sand dunes can cover hundreds of squares. They always have a gentle slope pointing in the direction of the prevailing wind and a steep slope on the leeward side.
				
\textbf{Other Desert Terrain Features}: Tundra is sometimes bordered by forests, and the occasional tree isn't out of place in the cold wastes. Rocky deserts have towers and mesas consisting of flat ground surrounded on all sides by cliffs and steep slopes (as described in Mountain Terrain). Sandy deserts sometimes have quicksand; this functions as described in Marsh Terrain, although desert quicksand is a waterless mixture of fine sand and dust. All desert terrain is crisscrossed with dry streambeds (treat as trenches 5 to 15 feet wide) that fill with water on the rare occasions when rain falls.
				
\textbf{Stealth and Detection in the Desert}: In general, the maximum distance in desert terrain at which a Perception check for detecting the nearby presence of others can succeed is 6d6 \mbox{$\times$} 20 feet; beyond this distance, elevation changes and heat distortion in warm deserts makes sight-based Perception impossible. The presence of dunes in sandy deserts limits spotting distance to 6d6 \mbox{$\times$} 10 feet. The scarcity of undergrowth or other elements that offer concealment or cover makes using Stealth more difficult.
				
\subsection{Sandstorms}

				
A sandstorm reduces visibility to 1d10 \mbox{$\times$} 5 feet and provides a --4 penalty on Perception checks. A sandstorm deals 1d3 points of nonlethal damage per hour to any creatures caught in the open, and leaves a thin coating of sand in its wake. Driving sand creeps in through all but the most secure seals and seams, chafing skin and contaminating carried gear. 
				
\subsection{Plains Terrain}

				
Plains come in three categories: farms, grasslands, and battlefields. Farms are common in settled areas, while grasslands represent untamed plains. The battlefields where large armies clash are temporary places, usually reclaimed by natural vegetation or the farmer's plow. Battlefields represent a third terrain category because adventurers tend to spend a lot of time there, not because they're particularly prevalent.
				
The table below shows the proportions of terrain elements in the different categories of plains. On a farm, light undergrowth represents most mature grain crops, so farms growing vegetable crops will have less light undergrowth, as will all farms during the time between harvest and a few months after planting.
				
The terrain elements in the table below are mutually exclusive.
				

\begin{tabular}{llll}
 & \textbf{Farm} & \textbf{Grassland} & \textbf{Battlefield}\\
Light undergrowth & 40\% & 20\% & 10\%\\
Heavy undergrowth & --- & 10\% & ---\\
Light rubble & --- &--- & 10\%\\
Trench & 5\% & --- & 5\%\\
Berm & --- & --- & 5\%\\
\end{tabular}
				
\textbf{Undergrowth}: Whether they're crops or natural vegetation, the tall grasses of the plains function like light undergrowth in a forest. Particularly thick bushes form patches of heavy undergrowth that dot the landscape in grasslands.
				
\textbf{Light Rubble}: On the battlefield, light rubble usually represents something that was destroyed: the ruins of a building or the scattered remnants of a stone wall, for example. It functions as described in the Desert Terrain section.
				
\textbf{Trench}: Often dug before a battle to protect soldiers, a trench functions as a low wall, except that it provides no cover against adjacent foes. It costs 2 squares of movement to leave a trench, but it costs nothing extra to enter one. Creatures outside a trench who make a melee attack against a creature inside the trench gain a +1 bonus on melee attacks because they have higher ground. In farm terrain, trenches are generally irrigation ditches.
				
\textbf{Berm}: A common defensive structure, a berm is a low, earthen wall that slows movement and provides a measure of cover. Put a berm on the map by drawing two adjacent rows of steep slope (described in Hills Terrain), with the edges of the berm on the downhill side. Thus, a character crossing a 2-square berm will travel uphill for 1 square, then downhill for 1 square. Two square berms provide cover as low walls for anyone standing behind them. Larger berms provide the low wall benefit for anyone standing 1 square downhill from the top of the berm. 
				
\textbf{Fences}: Wooden fences are generally used to contain livestock or impede oncoming soldiers. It costs an extra square of movement to cross a wooden fence. A stone fence provides a measure of cover as well, functioning as low walls. Mounted characters can cross a fence without slowing their movement if they succeed on a DC 15 Ride check. If the check fails, the steed crosses the fence, but the rider falls out of the saddle.
				
\textbf{Other Plains Terrain Features}: Occasional trees dot the landscape in many plains, although on battlefields they're often felled to provide raw material for siege engines (described in Urban Features). Hedgerows (described in Marsh Terrain) are found in plains as well. Streams, generally 5 to 20 feet wide and 5 to 10 feet deep, are commonplace.
				
\textbf{Stealth and Detection in Plains}: In plains terrain, the maximum distance at which a Perception check for detecting the nearby presence of others can succeed is 6d6 \mbox{$\times$} 40 feet, although the specifics of your map might restrict line of sight. Cover and concealment are not uncommon, so a good place of refuge is often nearby, if not right at hand.
				
\subsection{Aquatic Terrain}

				
Aquatic terrain is the least hospitable to most PCs, because they can't breathe there. Aquatic terrain doesn't offer the variety that land terrain does. The ocean floor holds many marvels, including undersea analogues of any of the terrain elements described earlier in this section, but if characters find themselves in the water because they were bull rushed off the deck of a pirate ship, the tall kelp beds hundreds of feet below them don't matter. Accordingly, these rules simply divide aquatic terrain into two categories: flowing water (such as streams and rivers) and non-flowing water (such as lakes and oceans).
				
\textbf{Flowing Water}: Large, placid rivers move at only a few miles per hour, so they function as still water for most purposes. But some rivers and streams are swifter; anything floating in them moves downstream at a speed of 10 to 40 feet per round. The fastest rapids send swimmers bobbing downstream at 60 to 90 feet per round. Fast rivers are always at least rough water (Swim DC 15), and whitewater rapids are stormy water (Swim DC 20). If a character is in moving water, move her downstream the indicated distance at the end of her turn. A character trying to maintain her position relative to the riverbank can spend some or all of her turn swimming upstream.
				
\textit{Swept Away}: Characters swept away by a river moving 60 feet per round or faster must make DC 20 Swim checks every round to avoid going under. If a character gets a check result of 5 or more over the minimum necessary, she arrests her motion by catching a rock, tree limb, or bottom snag---she is no longer being carried along by the flow of the water. Escaping the rapids by reaching the bank requires three DC 20 Swim checks in a row. Characters arrested by a rock, limb, or snag can't escape under their own power unless they strike out into the water and attempt to swim their way clear. Other characters can rescue them as if they were trapped in quicksand (described in Marsh Terrain). 
				
\textbf{Non-Flowing Water}: Lakes and oceans simply require a swim speed or successful Swim checks to move through (DC 10 in calm water, DC 15 in rough water, DC 20 in stormy water). Characters need a way to breathe if they're underwater; failing that, they risk drowning. When underwater, characters can move in any direction.
				
\textbf{Stealth and Detection Underwater}: How far you can see underwater depends on the water's clarity. As a guideline, creatures can see 4d8 \mbox{$\times$} 10 feet if the water is clear, and 1d8 \mbox{$\times$} 10 feet if it's murky. Moving water is always murky, unless it's in a particularly large, slow-moving river.
				
It's hard to find cover or concealment to hide underwater (except along the sea floor).
				
\textit{Invisibility}: An invisible creature displaces water and leaves a visible, body-shaped \texttt{{}"{}}bubble\texttt{{}"{}} where the water was displaced. The creature still has concealment (20\% miss chance), but not total concealment (50\% miss chance).
				
\subsection{Underwater Combat}

				
Land-based creatures can have considerable difficulty when fighting in water. Water affects a creature's attack rolls, damage, and movement. In some cases a creature's opponents might get a bonus on attacks. The effects are summarized on Table: Combat Adjustments Underwater. They apply whenever a character is swimming, walking in chest-deep water, or walking along the bottom of a body of water. 
				
\begin{table}[]
\sffamily
\caption{Combat Adjustments Underwater}
\begin{tabular}{lllll}
\textbf{Condition} & \multicolumn{2}{c}{\textbf{Attack/Damage}} & \textbf{Movement} & \textbf{Off Balance?\(^{1}\)}\\
                   & \textbf{Slashing or Bludgeoning} & \textbf{Piercing}\\
Freedom of movement & normal/normal & normal/normal & normal & No\\
Has a swim speed & -2/half & normal & normal & No\\
Successful Swim check& -2/half\(^{2}\) & normal & quarter or half\(^{3 }\) & No\\
Firm footing\(^{4 }\) & -2/half\(^{2}\) & normal & half & No\\
None of the above & -2/half\(^{2}\) &  -2/half & normal & Yes\\
\end{tabular}
1 Creatures flailing about in the water (usually because they failed their Swim checks) have a hard time fighting effectively. An off-balance creature loses its Dexterity bonus to Armor Class, and opponents gain a +2 bonus on attacks against it. 
2 A creature without \textit{freedom of movement }effects or a swim speed makes grapple checks underwater at a --2 penalty, but deals damage normally when grappling.
3 A successful Swim check lets a creature move one-quarter its speed as a move action or one-half its speed as a full-round action.
4 Creatures have firm footing when walking along the bottom, braced against a ship's hull, or the like. A creature can only walk along the bottom if it wears or carries enough gear to weigh itself down: at least 16 pounds for Medium creatures, twice that for each size category larger than Medium, and half that for each size category smaller than Medium. 
\end{table}

				
\textbf{Ranged Attacks Underwater}: Thrown weapons are ineffective underwater, even when launched from land. Attacks with other ranged weapons take a --2 penalty on attack rolls for every 5 feet of water they pass through, in addition to the normal penalties for range. 
				
\textbf{Attacks from Land}: Characters swimming, floating, or treading water on the surface, or wading in water at least chest deep, have improved cover (+8 bonus to AC, +4 bonus on Reflex saves) from opponents on land. Land-bound opponents who have \textit{freedom of movement }effects ignore this cover when making melee attacks against targets in the water. A completely submerged creature has total cover against opponents on land unless those opponents have \textit{freedom of movement }effects. Magical effects are unaffected except for those that require attack rolls (which are treated like any other effects) and fire effects.
				
\textbf{Fire}: Nonmagical fire (including alchemist's fire) does not burn underwater. Spells or spell-like effects with the fire descriptor are ineffective underwater unless the caster makes a caster level check (DC 20 + spell level). If the check succeeds, the spell creates a bubble of steam instead of its usual fiery effect, but otherwise the spell works as described. A supernatural fire effect is ineffective underwater unless its description states otherwise. The surface of a body of water blocks line of effect for any fire spell. If the caster has made the caster level check to make the fire spell usable underwater, the surface still blocks the spell's line of effect.
				
\textbf{Spellcasting Underwater}: Casting spells while submerged can be difficult for those who cannot breathe underwater. A creature that cannot breathe water must make a concentration check (DC 15 + spell level) to cast a spell underwater (this is in addition to the caster level check to successfully cast a fire spell underwater). Creatures that can breathe water are unaffected and can cast spells normally. Some spells might function differently underwater, subject to GM discretion.
				
\subsection{Floods}

				
In many wilderness areas, river floods are a common occurrence.
				
In spring, an enormous snowmelt can engorge the streams and rivers it feeds. Other catastrophic events such as massive rainstorms or the destruction of a dam can create floods as well.
				
During a flood, rivers become wider, deeper, and swifter. Assume that a river rises by 1d10+10 feet during the spring flood, and its width increases by a factor of 1d4 \mbox{$\times$} 50\%. Fords might disappear for days, bridges might be swept away, and even ferries might not be able to manage the crossing of a flooded river. A river in flood makes Swim checks one category harder (calm water becomes rough, and rough water becomes stormy). Rivers also become 50\% swifter.
				
\section{Urban Adventures}

				
At first glance, a city is much like a dungeon, made up of walls, doors, rooms, and corridors. Adventures that take place in cities have two salient differences from their dungeon counterparts, however. Characters have greater access to resources, and they must contend with law enforcement.
				
\textbf{Access to Resources}: Unlike in dungeons and the wilderness, characters can buy and sell gear quickly in a city. A large city or metropolis probably has high-level NPCs and experts in obscure fields of knowledge who can provide assistance and decipher clues. And when the PCs are battered and bruised, they can retreat to the comfort of a room at an inn.
				
The freedom to retreat and ready access to the marketplace means that the players have a greater degree of control over the pacing of an urban adventure.
				
\textbf{ Law Enforcement}: The other key distinctions between adventuring in a city and delving into a dungeon is that a dungeon is, almost by definition, a lawless place where the only law is that of the jungle: kill or be killed. A city, on the other hand, is held together by a code of laws, many of which are explicitly designed to prevent the sort of killing and looting that adventurers engage in all the time. Even so, most cities' laws recognize monsters as a threat to the stability the city relies on, and prohibitions about murder rarely apply to monsters such as aberrations or evil outsiders. Most evil humanoids, however, are typically protected by the same laws that protect all the citizens of the city. Having an evil alignment is not a crime (except in some severely theocratic cities, perhaps, with the magical power to back up the law); only evil deeds are against the law. Even when adventurers encounter an evildoer in the act of perpetrating some heinous evil upon the populace of the city, the law tends to frown on the sort of vigilante justice that leaves the evildoer dead or otherwise unable to testify at a trial.
				
\subsection{Weapon and Spell Restrictions}

				
Different cities have different laws about such issues as carrying weapons in public and restricting spellcasters.
				
The city's laws might not affect all characters equally. A monk isn't hampered at all by a law about peace-bonding weapons, but a cleric is reduced to a fraction of his power if all holy symbols are confiscated at the city's gates.
				
\subsection{Urban Features}

				
Walls, doors, poor lighting, and uneven footing: in many ways a city is much like a dungeon. Some special considerations for an urban setting are covered below.
				
\subsection{Walls and Gates}

				
Many cities are surrounded by walls. A typical small city wall is a fortified stone wall 5 feet thick and 20 feet high. Such a wall is fairly smooth, requiring a DC 30 Climb check to scale. The walls are crenellated on one side to provide a low wall for the guards atop it, and there is just barely room for guards to walk along the top of the wall. A typical small city wall has AC 3, hardness 8, and 450 hp per 10-foot section.
				
A typical large city wall is 10 feet thick and 30 feet high, with crenellations on both sides for the guards on top of the wall. It is likewise smooth, requiring a DC 30 Climb check to scale. Such a wall has AC 3, hardness 8, and 720 hp per 10-foot section.
				
A typical metropolis wall is 15 feet thick and 40 feet tall. It has crenellations on both sides and often has a tunnel and small rooms running through its interior. Metropolis walls have AC 3, hardness 8, and 1,170 hp per 10-foot section.
				
Unlike smaller cities, metropolises often have interior walls as well as surrounding walls---either old walls that the city has outgrown, or walls dividing individual districts from each other. Sometimes these walls are as large and thick as the outer walls, but more often they have the characteristics of a large city's or small city's walls.
				
\textbf{Watchtowers}: Some city walls are adorned with watchtowers set at irregular intervals. Few cities have enough guards to keep someone constantly stationed at every tower, unless the city is expecting attack from outside. The towers provide a superior view of the surrounding countryside as well as a point of defense against invaders.
				
Watchtowers are typically 10 feet higher than the wall they adjoin, and their diameter is 5 times the thickness of the wall. Arrow slits line the outer sides of the upper stories of a tower, and the top is crenellated like the surrounding walls are. In a small tower (25 feet in diameter adjoining a 5-foot-thick wall), a simple ladder typically connects the tower's stories and its roof. In a larger tower, stairs serve that purpose. 
				
Heavy wooden doors, reinforced with iron and bearing good locks (Disable Device DC 30), block entry to a tower, unless the tower is in regular use. As a rule, the captain of the guard keeps the keys to the towers secured on her person, and second copies are in the city's inner fortress or barracks.
				
\textbf{Gates}: A typical city gate is a gatehouse with two portcullises and murder holes above the space between them. In towns and some small cities, the primary entry is through iron double doors set into the city wall.
				
Gates are usually open during the day and locked or barred at night. Usually, one gate lets in travelers after sunset and is staffed by guards who will open it for someone who seems honest, presents proper papers, or offers a large enough bribe (depending on the city and the guards).
				
\subsection{Guards and Soldiers}

				
A city typically has full-time military personnel equal to 1\% of its adult population, in addition to militia or conscript soldiers equal to 5\% of the population. The full-time soldiers are city guards responsible for maintaining order within the city, similar to the role of modern police, and (to a lesser extent) for defending the city from outside assault. Conscript soldiers are called up to serve in case of an attack on the city.
				
A typical city guard force works on three 8-hour shifts, with 30\% of the force on a day shift (8 a.m. to 4 p.m.), 35\% on an evening shift (4 p.m. to 12 a.m.), and 35\% on a night shift (12 a.m. to 8 a.m.). At any given time, 80\% of the guards on duty are on the streets patrolling, while the remaining 20\% are stationed at various posts throughout the city where they can respond to nearby alarms. At least one such guard post is present within each neighborhood of a city (each neighborhood consisting of several districts).
				
The majority of a city guard force is made up of warriors, mostly 1st level. Officers include higher-level warriors, fighters, a fair number of clerics, and wizards or sorcerers, as well as multiclass fighter/spellcasters.
				
\subsection{Siege Engines}

				
Siege engines are large weapons, temporary structures, or pieces of equipment traditionally used in besieging castles or fortresses.

\begin{table}[]
\sffamily
\caption{Siege Engines}
\begin{tabular}{llllll}
\textbf{Item} & \textbf{Cost} & \textbf{Damage} & \textbf{Critical Range} & \textbf{Increment} & \textbf{Typical Crew}\\
Catapult, heavy & 800 gp & 6d6 & - & 200 ft. (100 ft. minimum) & 4\\
Catapult, light & 550 gp & 4d6 & - & 150 ft. (100 ft. minimum) & 2\\
Ballista & 500 gp & 3d8 & 19-20 & 120 ft. & 1\\
Ram & 1,000 gp & 3d6* & - & - & 10\\
Siege tower & 2,000 gp & - & - & - & 20\\
\end{tabular}
* See description for special rules.\\
\end{table}

\begin{table}
 \sffamily
 \caption{Catapult Attack Modifiers}
 \begin{tabular}{ll}
No line of sight to target square & --6\\
Successive shots (crew can see where most recent misses landed) & Cumulative +2 per previous miss (maximum +10)\\
Successive shots (crew can't see where most recent misses landed, but observer is providing feedback) & Cumulative +1 per previous miss (maximum +5)\\
 \end{tabular}

\end{table}

				
Siege engines are treated as difficult devices if someone tries to disable them using Disable Device. This takes 2d4 rounds and requires a DC 20 Disable Device check. Siege engines are typically made out of wood and have an AC of 3 (--5 Dex, --2 size), a Hardness of 5, and 80 hit points. Siege engines made up of a different material might have different values. Some siege engines are armored as well. Treat the siege engine as a Huge creature to determine the cost of such armor. Siege engines can be crafted as masterwork and enchanted as magic weapons, adding bonuses on attack rolls to the checks made to hit with the siege engine. A masterwork siege engine costs 300 gp more than the listed price. Enchanting a siege engine costs twice the normal amount. For example, a \textit{+1 flaming heavy catapult}, armored with full plate, would have an AC of 11 and would cost 23,100 gp (800 gp base + 6,000 gp for the armor + 300 gp masterwork + 16,000 gp for the enhancements).
				
\textbf{Catapult, Heavy}: A heavy catapult is a massive engine capable of throwing rocks or heavy objects with great force. Because the catapult throws its payload in a high arc, it can hit squares out of its line of sight. To fire a heavy catapult, the crew chief makes a special check against DC 15 using only his base attack bonus, Intelligence modifier, range increment penalty, and the appropriate modifiers from the lower section of Table: Siege Engines. If the check succeeds, the catapult stone hits the square the catapult was aimed at, dealing the indicated damage to any object or character in the square. Characters who succeed on a DC 15 Reflex save take half damage. Once a catapult stone hits a square, subsequent shots hit the same square unless the catapult is reaimed or the wind changes direction or speed.
				
If a catapult stone misses, roll 1d8 to determine where it lands. This determines the misdirection of the throw, with 1 being back toward the catapult and 2 through 8 counting clockwise around the target square. Finally, count 1d4 squares away from the target square for every range increment of the attack.
				
Loading a catapult requires a series of full-round actions. It takes a DC 15 Strength check to winch the throwing arm down; most catapults have wheels to allow up to two crew members to use the aid another action, assisting the main winch operator. A DC 15 Profession (siege engineer) check latches the arm into place, and then another DC 15 Profession (siege engineer) check loads the catapult ammunition. It takes four full-round actions to reaim a heavy catapult (multiple crew members can perform these full-round actions in the same round, so it would take a crew of four only 1 round to reaim the catapult).
				
A heavy catapult takes up a space 15 feet across.
				
\textbf{Catapult, Light}: This is a smaller, lighter version of the heavy catapult. It functions as the heavy catapult, except that it takes a DC 10 Strength check to winch the arm into place, and only two full-round actions are required to reaim the catapult.
				
A light catapult takes up a space 10 feet across.
								
\textbf{Ballista}: A ballista is essentially a Huge heavy crossbow fixed in place. Its size makes it hard for most creatures to aim it\textit{. }Thus, a Medium creature takes a --4 penalty on attack rolls when using a ballista, and a Small creature takes a --6 penalty. It takes a creature smaller than Large two full-round actions to reload the ballista after firing.
				
A ballista takes up a space 5 feet across.
				
\textbf{Ram}: This heavy pole is sometimes suspended from a movable scaffold that allows the crew to swing it back and forth against objects. As a full-round action, the character closest to the front of the ram makes an attack roll against the AC of the construction, applying the --4 penalty for lack of proficiency. It's not possible to be proficient with this device. In addition to the damage given on Table: Siege Engines, up to nine other characters holding the ram can add their Strength modifiers to the ram's damage, if they devote an attack action to doing so. It takes at least one Huge or larger creature, two Large creatures, four Medium creatures, or eight Small creatures to swing a ram.
				
A ram is typically 30 feet long. In a battle, the creatures wielding the ram stand in two adjacent columns of equal length, with the ram between them.
				
\textbf{Siege Tower}: This device is a massive wooden tower on wheels or rollers that can be rolled up against a wall to allow attackers to scale the tower and thus get to the top of the wall with cover. The wooden walls are usually 1 foot thick.
				
A typical siege tower takes up a space 15 feet across. The creatures inside push it at a base land speed of 10 feet (and a siege tower can't run). The eight creatures pushing on the ground floor have total cover, and those on higher floors get improved cover and can fire through arrow slits.
				
\subsection{City Streets}

				
Typical city streets are narrow and twisting. Most streets average 15 to 20 feet wide, while alleys range from 10 feet wide to only 5 feet. Cobblestones in good condition allow normal movement, but roads in poor repair and heavily rutted dirt streets are considered light rubble, increasing the DC of Acrobatics checks by 2.
				
Some cities have no larger thoroughfares, particularly cities that gradually grew from small settlements to larger cities. Cities that are planned, or perhaps have suffered a major fire that allowed authorities to construct new roads through formerly inhabited areas, might have a few larger streets through town. These main roads are 25 feet wide---offering room for wagons to pass each other---with 5-foot-wide sidewalks on either side.
				
\textbf{Crowds}: Urban streets are often full of people going about their daily lives. In most cases, it isn't necessary to put every 1st-level commoner on the map when a fight breaks out on the city's main thoroughfare. Instead, just indicate which squares on the map contain crowds. If crowds see something obviously dangerous, they'll move away at 30 feet per round at initiative count 0. It takes 2 squares of movement to enter a square with crowds. The crowds provide cover for anyone who does so, enabling a Stealth check and providing a bonus to Armor Class and on Reflex saves.
				
\textit{Directing Crowds}: It takes a DC 15 Diplomacy check or DC 20 Intimidate check to convince a crowd to move in a particular direction, and the crowd must be able to hear or see the character making the attempt. It takes a full-round action to make the Diplomacy check, but only a free action to make the Intimidate check.
				
If two or more characters are trying to direct a crowd in different directions, they make opposed Diplomacy or Intimidate checks to determine to whom the crowd listens. The crowd ignores everyone if none of the characters' check results beat the DCs given above.
				
\subsection{Above and Beneath the Streets}

				
\textbf{Rooftops}: Getting to a roof usually requires climbing a wall, unless the character can reach a roof by jumping down from a higher window, balcony, or bridge. Flat roofs, common only in warm climates (as accumulated snow can cause a flat roof to collapse), are easy to run across. Moving along the peak of a pitched roof requires a DC 20 Acrobatics check. Moving on an angled roof surface without changing altitude (moving parallel to the peak, in other words) requires a DC 15 Acrobatics check. Moving up and down across the peak of a roof requires a DC 10 Acrobatics check.
				
Eventually a character runs out of roof, requiring a long jump across to the next roof or down to the ground. The distance to the closest roof is usually 1d3 \mbox{$\times$} 5 feet horizontally, but the next roof is equally likely to be 5 feet higher, 5 feet lower, or the same height. Use the guidelines in the Acrobatics skill (a horizontal jump's peak height is one-fourth of the horizontal distance) to determine whether a character can make a jump.
				
\textbf{Sewers}: To get into the sewers, most characters open a grate (a full-round action) and jump down 10 feet. Sewers are built exactly like dungeons, except that they're much more likely to have floors that are slippery or covered with water. Sewers are also similar to dungeons in terms of creatures liable to be encountered therein. Some cities were built atop the ruins of older civilizations, so their sewers sometimes lead to treasures and dangers from a bygone age.
				
\subsection{City Buildings}

				
Most city buildings fall into three categories. The majority of buildings in the city are two to five stories high, built side-by-side to form long rows separated by secondary or main streets. These row houses usually have businesses on the ground floor, with offices or apartments above.
				
Inns, successful businesses, and large warehouses---as well as millers, tanners, and other businesses that require extra space---are generally large, free-standing buildings with up to five stories. 
				
Finally, small residences, shops, warehouses, or storage sheds are simple, one-story wooden buildings, especially if they're in poorer neighborhoods.
				
Most city buildings are made of a combination of stone or clay brick (on the lower one or two stories) and timbers (for the upper stories, interior walls, and floors). Roofs are a mixture of boards, thatch, and slates, sealed with pitch. A typical lower-story wall is 1 foot thick, with AC 3, hardness 8, 90 hp, and a Climb DC of 25. Upper-story walls are 6 inches thick, with AC 3, hardness 5, 60 hp, and a Climb DC of 21. Exterior doors on most buildings are good wooden doors that are usually kept locked, except on public buildings such as shops and taverns.
				
\subsection{City Lights}

				
If a city has main thoroughfares, they are lined with lanterns hanging at a height of 7 feet from building awnings. These lanterns are spaced 60 feet apart, so their illumination is all but continuous. Secondary streets and alleys are not lit; it is common for citizens to hire lantern-bearers when going out after dark.
				
Alleys can be dark places even in daylight, thanks to the shadows of the tall buildings that surround them. A dark alley in daylight is rarely dark enough to afford true concealment, but it can lend a +2 circumstance bonus on Stealth checks.
				
\section{Weather}

				
Weather can play an important role in an adventure.
				
\begin{table}[]
\sffamily
\caption{Table: Random Weather}
\begin{tabular}{lllll}
\textbf{d\%} & \textbf{Weather} & \textbf{Cold Climate} & \textbf{Temperate Climate\(^{1}\)} & \textbf{Desert}\\
01-70 & Normal weather & Cold, calm & Normal for season\(^{2}\) & Hot, calm\\
71-80 & Abnormal weather & Heat wave (01-30) or cold snap (31-100) & Heat wave (01-50) or cold snap (51-100) & Hot, windy\\
81-90 & Inclement weather & Precipitation (snow) & Precipitation (normal for season) & Hot, windy\\
91-99 & Storm & Snowstorm & Thunderstorm, snowstorm & Duststorm\\
100 & Powerful storm & Blizzard & Windstorm, blizzard, hurricane, tornado & Downpour\\
\end{tabular}
1 Temperate includes forests, hills, marshes, mountains, plains, and warm aquatic environments.
2 Winter is cold, summer is warm, spring and autumn are temperate. Marsh regions are slightly warmer in winter.
\end{table}
				
Table: Random Weather can be used as a simple local weather table. Terms on that table are defined as follows.
				
\textbf{Calm}: Wind speeds are light (0 to 10 mph).
				
\textbf{Cold}: Between 0\^A\mbox{${}^\circ$} and 40\^A\mbox{${}^\circ$} Fahrenheit during the day, 10 to 20 degrees colder at night.
				
\textbf{Cold Snap}: Lowers temperature by --10\^A\mbox{${}^\circ$} F.
				
\textbf{Downpour}: Treat as rain (see Precipitation, below), but conceals as fog. Can create floods. A downpour lasts for 2d4 hours.
				
\textbf{Heat Wave}: Raises temperature by +10\^A\mbox{${}^\circ$} F.
				
\textbf{Hot}: Between 85\^A\mbox{${}^\circ$} and 110\^A\mbox{${}^\circ$} Fahrenheit during the day, 10 to 20 degrees colder at night.
				
\textbf{Moderate}: Between 40\^A\mbox{${}^\circ$} and 60\^A\mbox{${}^\circ$} Fahrenheit during the day, 10 to 20 degrees colder at night.
				
\textbf{Powerful Storm (Windstorm/Blizzard/Hurricane/Tornado)}: Wind speeds are over 50 mph (see Table: Wind Effects). In addition, blizzards are accompanied by heavy snow (1d3 feet), and hurricanes are accompanied by downpours. Windstorms last for 1d6 hours. Blizzards last for 1d3 days. Hurricanes can last for up to a week, but their major impact on characters comes in a 24-to-48-hour period when the center of the storm moves through their area. Tornadoes are very short-lived (1d6 \mbox{$\times$} 10 minutes), typically forming as part of a thunderstorm system. 
				
\textbf{Precipitation}: Roll d\% to determine whether the precipitation is fog (01--30), rain/snow (31--90), or sleet/hail (91--00). Snow and sleet occur only when the temperature is 30\^A\mbox{${}^\circ$} Fahrenheit or below. Most precipitation lasts for 2d4 hours. By contrast, hail lasts for only 1d20 minutes but usually accompanies 1d4 hours of rain.
				
\textbf{Storm (Duststorm/Snowstorm/Thunderstorm)}: Wind speeds are severe (30 to 50 mph) and visibility is cut by three-quarters. Storms last for 2d4--1 hours. See Storms, below, for more details. 
				
\textbf{Warm}: Between 60\^A\mbox{${}^\circ$} and 85\^A\mbox{${}^\circ$} Fahrenheit during the day, 10 to 20 degrees colder at night.
				
\textbf{Windy}: Wind speeds are moderate to strong (10 to 30 mph); see Table: Wind Effects.
				
\subsection{Rain, Snow, Sleet, and Hail}

				
Bad weather frequently slows or halts travel and makes it virtually impossible to navigate from one spot to another. Torrential downpours and blizzards obscure vision as effectively as a dense fog.
				
Most precipitation is rain, but in cold conditions it can manifest as snow, sleet, or hail. Precipitation of any kind followed by a cold snap in which the temperature dips from above freezing to 30\^A\mbox{${}^\circ$} F or below might produce ice. 
				
\textit{Rain}: Rain reduces visibility ranges by half, resulting in a --4 penalty on Perception checks. It has the same effect on flames, ranged weapon attacks, and Perception checks as severe wind.
				
\textit{Snow}: Falling snow has the same effects on visibility, ranged weapon attacks, and skill checks as rain, and it costs 2 squares of movement to enter a snow-covered square. A day of snowfall leaves 1d6 inches of snow on the ground.
				
\textit{Heavy Snow}: Heavy snow has the same effects as normal snowfall but also restricts visibility as fog does (see Fog). A day of heavy snow leaves 1d4 feet of snow on the ground, and it costs 4 squares of movement to enter a square covered with heavy snow. Heavy snow accompanied by strong or severe winds might result in snowdrifts 1d4 \mbox{$\times$} 5 feet deep, especially in and around objects big enough to deflect the wind---a cabin or a large tent, for instance. There is a 10\% chance that a heavy snowfall is accompanied by lightning (see Thunderstorm). Snow has the same effect on flames as moderate wind.
				
\textit{Sleet}: Essentially frozen rain, sleet has the same effect as rain while falling (except that its chance to extinguish protected flames is 75\%) and the same effect as snow once on the ground. 
				
\textit{Hail}: Hail does not reduce visibility, but the sound of falling hail makes sound-based Perception checks more difficult (--4 penalty). Sometimes (5\% chance) hail can become large enough to deal 1 point of lethal damage (per storm) to anything in the open. Once on the ground, hail has the same effect on movement as snow.
				
\subsection{Storms}

				
The combined effects of precipitation (or dust) and wind that accompany all storms reduce visibility ranges by three-quarters, imposing a --8 penalty on Perception checks. Storms make ranged weapon attacks impossible, except for those using siege weapons, which have a --4 penalty on attack rolls. They automatically extinguish candles, torches, and similar unprotected flames. They cause protected flames, such as those of lanterns, to dance wildly and have a 50\% chance to extinguish these lights. See Table: Wind Effects for possible consequences to creatures caught outside without shelter during such a storm. Storms are divided into the following three types. 
				
\textit{Duststorm (CR 3)}: These desert storms differ from other storms in that they have no precipitation. Instead, a duststorm blows fine grains of sand that obscure vision, smother unprotected flames, and can even choke protected flames (50\% chance). Most duststorms are accompanied by severe winds and leave behind a deposit of 1d6 inches of sand. There is a 10\% chance for a greater duststorm to be accompanied by windstorm-magnitude winds (see Table: Wind Effects). These greater duststorms deal 1d3 points of nonlethal damage each round to anyone caught out in the open without shelter and also pose a choking hazard (see Drowning, except that a character with a scarf or similar protection across her mouth and nose does not begin to choke until after a number of rounds equal to 10 + her Constitution score). Greater duststorms leave 2d3--1 feet of fine sand in their wake.
				
\textit{Snowstorm}: In addition to the wind and precipitation common to other storms, snowstorms leave 1d6 inches of snow on the ground afterward. 
				
\textit{Thunderstorm}: In addition to wind and precipitation (usually rain, but sometimes also hail), thunderstorms are accompanied by lightning that can pose a hazard to characters without proper shelter (especially those in metal armor). As a rule of thumb, assume one bolt per minute for a 1-hour period at the center of the storm. Each bolt causes between 4d8 and 10d8 points of electricity damage. One in 10 thunderstorms is accompanied by a tornado. 
				
\textbf{Powerful Storms}: Very high winds and torrential precipitation reduce visibility to zero, making Perception checks and all ranged weapon attacks impossible. Unprotected flames are automatically extinguished, and protected flames have a 75\% chance of being doused. Creatures caught in the area must make a Fortitude save or face the effects based on the size of the creature (see Table: Wind Effects). Powerful storms are divided into the following four types.
				
\textit{Windstorm}: While accompanied by little or no precipitation, windstorms can cause considerable damage simply through the force of their winds.
				
\textit{Blizzard}: The combination of high winds, heavy snow (typically 1d3 feet), and bitter cold make blizzards deadly for all who are unprepared for them.
				
\textit{Hurricane}: In addition to very high winds and heavy rain, hurricanes are accompanied by floods. Most adventuring activity is impossible under such conditions.
				
\textit{Tornado}: In addition to incredibly high winds, tornadoes can severely injure and kill those that get pulled into their funnels.
				
\subsection{Fog}

				
Whether in the form of a low-lying cloud or a mist rising from the ground, fog obscures all sight beyond 5 feet, including darkvision. Creatures 5 feet away have concealment (attacks by or against them have a 20\% miss chance).
				
\subsection{Winds}

				
The wind can create a stinging spray of sand or dust, fan a large fire, keel over a small boat, and blow gases or vapors away. If powerful enough, it can even knock characters down (see Table: Wind Effects), interfere with ranged attacks, or impose penalties on some skill checks.
% <div class="table">

\begin{table}[]
\sffamily
\caption{Table: Wind Effects}
\begin{tabular}{llllll}
\textbf{Wind Force} & \textbf{Wind Speed} & \textbf{Ranged Attacks Normal/Siege Weapons\(^{1}\)} & \textbf{Checked Size\(^{2}\)} & \textbf{Blown Away Size} & \textbf{Fly Penalty}\\
Light & 0-10 mph & -/- & - & - & - \\
 Moderate & 11-20 mph & -/- & - & - & - \\
 Strong & 21-30 mph & -2/- & Tiny & - & -2 \\
 Severe & 31-50 mph & -4/- & Small & Tiny & -4 \\
 Windstorm & 51-74 mph & Impossible/-4 & Medium & Small & -8 \\
 Hurricane & 75-174 mph & Impossible/-8 & Large & Medium & -12 \\
 Tornado & 175-300 mph & Impossible/impossible & Huge & Large & -16\\
\end{tabular}
\textit{1} The siege weapon category includes ballista and catapult attacks as well as boulders tossed by giants.
\textit{2 Checked Size}: Creatures of this size or smaller are unable to move forward against the force of the wind unless they succeed on a DC 10 Strength check (if on the ground) or a DC 20 Fly skill check if airborne.
\textit{3 Blown Away Size}: Creatures on the ground are knocked prone and rolled 1d4 \mbox{$\times$} 10 feet, taking 1d4 points of nonlethal damage per 10 feet, unless they make a DC 15 Strength check. Flying creatures are blown back 2d6 \mbox{$\times$} 10 feet and take 2d6 points of nonlethal damage due to battering and buffeting, unless they succeed on a DC 25 Fly skill check.
\end{table}
				
\textit{Light Wind}: A gentle breeze, having little or no game effect.
				
\textit{Moderate Wind}: A steady wind with a 50\% chance of extinguishing small, unprotected flames, such as candles.
				
\textit{Strong Wind}: Gusts that automatically extinguish unprotected flames (candles, torches, and the like). Such gusts impose a --2 penalty on ranged attack rolls and on Perception checks.
				
\textit{Severe Wind}: In addition to automatically extinguishing any unprotected flames, winds of this magnitude cause protected flames (such as those of lanterns) to dance wildly and have a 50\% chance of extinguishing these lights. Ranged weapon attacks and Perception checks are at a --4 penalty. This is the velocity of wind produced by a \textit{gust of wind }spell.
				
\textit{Windstorm}: Powerful enough to bring down branches if not whole trees, windstorms automatically extinguish unprotected flames and have a 75\% chance of blowing out protected flames, such as those of lanterns. Ranged weapon attacks are impossible, and even siege weapons have a --4 penalty on attack rolls. Perception checks that rely on sound are at a --8 penalty due to the howling of the wind. 
				
\textit{Hurricane-Force Wind}: All flames are extinguished. Ranged attacks are impossible (except with siege weapons, which have a --8 penalty on attack rolls). Perception checks based on sound are impossible: all characters can hear is the roaring of the wind. Hurricane-force winds often fell trees.
				
\textit{Tornado (CR 10)}: All flames are extinguished. All ranged attacks are impossible (even with siege weapons), as are sound-based Perception checks. Instead of being blown away (see Table: Wind Effects), characters in close proximity to a tornado who fail their Fortitude saves are sucked toward the tornado. Those who come in contact with the actual funnel cloud are picked up and whirled around for 1d10 rounds, taking 6d6 points of damage per round, before being violently expelled (falling damage might apply). While a tornado's rotational speed can be as great as 300 mph, the funnel itself moves forward at an average of 30 mph (roughly 250 feet per round). A tornado uproots trees, destroys buildings, and causes similar forms of major destruction.
				
\section{The Planes}

				
While endless adventure awaits out in the game---there are other worlds beyond these---other continents, other planets, other galaxies. Yet even beyond this existence of countless planets exist more worlds---entirely different dimensions of reality known as the planes of existence. Except for rare linking points that allow travel between them, each plane is effectively its own universe with its own natural laws. Collectively, the entirety of these other dimensions and planes is known as the Great Beyond.
				
Although the number of planes is limited only by imagination, they can all be categorized into five general types: the Material Plane, the transitive planes, the Inner Planes, the Outer Planes, and the countless demiplanes.
				
\textbf{Material Plane}: The Material Plane tends to be the most Earth-like of all planes and operates under the same set of natural laws that our own real world does. The \texttt{{}"{}}size\texttt{{}"{}} of the Material Plane depends upon the campaign---it might conform only to the single world on which your game is set, or it might encompass an entire universe of planets, moons, stars, and galaxies. The Material Plane is the default plane for the Pathfinder Roleplaying Game.
				
\textbf{Transitive Planes}: Transitive planes have one important common characteristic: they \texttt{{}"{}}overlap\texttt{{}"{}} with other planes, and as such can be used to travel between these overlapping realities. These planes have the strongest regular interaction with the Material Plane and are often accessed by using various spells. They have native inhabitants as well. Example transitive planes include the following.
				
\textit{Astral Plane}: A silvery void that connects the Material and Inner Planes to the Outer Planes, the astral plane is the medium through which the souls of the departed travel to the afterlife. A traveler in the Astral Plane sees the plane as a vast empty void periodically dotted with tiny motes of physical reality calved off of the countless planes it overlaps. Powerful spellcasters utilize the Astral Plane for a tiny fraction of a second when they teleport, or they can use it to travel between planes with spells like \textit{astral projection}.
				
\textit{Ethereal Plane}: The Ethereal Plane is a ghostly realm that exists as a buffer between the Material Plane and the Shadow Plane, overlapping each. A traveler in the Ethereal plane experiences the real world as if the world were an insubstantial ghost, and can move through solid objects without being seen in the real world. Strange creatures dwell in the Ethereal Plane, as well as ghosts and dreams, many of which can sometimes extend their influence into the real world in mysterious and terrifying ways. Powerful spellcasters utilize the Ethereal Plane with spells like \textit{blink}, \textit{etherealness}, and \textit{ethereal jaunt}.
				
\textit{Shadow Plane}: The eerie and deadly Shadow Plane is a grim, colorless \texttt{{}"{}}duplicate\texttt{{}"{}} of the Material Plane. It overlaps with the Material Plane but is smaller in size, and is in many ways a warped and mocking \texttt{{}"{}}reflection\texttt{{}"{}} of the Material Plane, one infused with negative energy (see Inner Planes) and serving as home for strange monsters like undead shadows and worse. Powerful spellcasters utilize the Shadow Plane to swiftly travel immense distances on the Material Plane with \textit{shadow walk}, or draw upon the mutable essence of the Shadow Plane to create quasi-real effects and creatures with spells like \textit{shadow evocation} or \textit{shades}.
				
\textbf{Inner Planes}: The Inner Planes contain the building blocks of reality---it's easiest to envision these planes as \texttt{{}"{}}containing\texttt{{}"{}} the Material Plane, but they do not overlap with the Material Plane as do the transitive planes. Each Inner Plane is made up of a single type of energy or element that overwhelms all others. The natives of a particular Inner Plane are made of the same energy or element as the plane itself. Example Inner Planes include the following.
				
\textit{Elemental Planes}: The four classic Inner Planes are the Plane of Air, the Plane of Earth, the Plane of Fire, and the Plane of Water---it is from these planes that the creatures known as elementals hail, yet they house many other strange denizens as well, such as the genie races, strange metal-eating xorns, unseen invisible stalkers, and mischievous mephits.
				
\textit{Energy Planes}: Two energy planes exist---the Positive Energy Plane (from which the animating spark of life hails) and the Negative Energy Plane (from which the sinister taint of undeath hails). Energy from both planes infuses reality, the ebb and flow of this energy running through all creatures to bear them along the journey from birth to death. Clerics utilize power from these planes when they channel energy.
				
\textbf{Outer Planes}: Beyond the realm of the mortal world, beyond the building blocks of reality, lie the Outer Planes. Vast beyond imagining, it is to these realms that the souls of the dead travel, and it is upon these realms in which the gods themselves hold court. Each of the Outer Planes has an alignment, representing a particular moral or ethical outlook, and the natives of each plane tend to behave in agreement with that plane's alignment. The Outer Planes are also the final resting place of souls from the Material Plane, whether that final rest takes the form of calm introspection or eternal damnation. The denizens of the Outer Planes form the mythologies of civilization, comprising angels and demons, titans and devils, and countless other incarnations of possibility. Each campaign world should have different Outer Planes to match its themes and needs, but classic Outer Planes include lawful good Heaven, the chaos and evil of the Abyss, the regimented lawful evil of Hell, and the capricious freedom and joys of chaotic good Elysium. Powerful spellcasters can contact the Outer Planes for advice or guidance with spells like \textit{commune} and \textit{contact outer plane}, or can conjure allies with spells like \textit{planar ally} or \textit{summon monster}.
				
\textbf{Demiplanes}: This catchall category covers all extradimensional spaces that function like planes but have measurable size and limited access. Other kinds of planes are theoretically infinite in size, but a demiplane might be only a few hundred feet across. There are countless demiplanes adrift in reality, and while most are connected to the Astral Plane and Ethereal Plane, some are cut off entirely from the transitive planes and can only be accessed by well-hidden portals or obscure magic spells.
				
\subsection{Layered Planes}

				
Infinities may be broken into smaller infinities, and planes into smaller, related planes. These layers are effectively separate planes of existence, and each layer can have its own features and qualities. Layers are connected to each other through a variety of planar gates, natural vortices, paths, and shifting borders.
				
Access to a layered plane from elsewhere usually happens on the first layer of the plane, which can be either the top or bottom layer, depending on the specific plane. Most fixed access points (such as portals and natural vortices) reach this layer, which makes it the gateway for other layers of the plane. The \textit{plane shift} spell generally deposits the spellcaster on the first layer of the plane.
				
\subsection{How Planes Interact}

				
Two planes that are separate do not overlap or directly connect to each other. They are like planets in different orbits. The only way to get from one separate plane to the other is to go through a third plane, such as a Transitive Plane.
				
\textbf{Coterminous Planes}: Planes that touch at specific points are coterminous. Where they touch, a connection exists, and travelers can leave one reality behind and enter the other.
				
\textbf{Coexistent Planes}: If a link between two planes can be created at any point, the two planes are coexistent. These planes overlap each other completely. A coexistent plane can be reached from anywhere on the plane it overlaps. When moving on a coexistent plane, it is often possible to see into or interact with the plane with which it coexists. 
				
\section{Environmental Rules}

				
Environmental hazards specific to one kind of terrain are described in the Wilderness section. Environmental hazards common to more than one setting are detailed below.
				
\subsection{Acid Effects}

				
Corrosive acids deals 1d6 points of damage per round of exposure except in the case of total immersion (such as in a vat of acid), which deals 10d6 points of damage per round. An attack with acid, such as from a hurled vial or a monster's spittle, counts as a round of exposure.
				
The fumes from most acids are inhaled poisons. Those who are adjacent to a large body of acid must make a DC 13 Fortitude save or take 1 point of Constitution damage each round. This poison does not have a frequency, a creature is safe as soon as it moves away from the acid.
				
Creatures immune to acid's caustic properties might still drown in it if they are totally immersed (see Drowning).
				
\subsection{Cold Dangers}

				
Cold and exposure deal nonlethal damage to the victim. A character cannot recover from the damage dealt by a cold environment until she gets out of the cold and warms up again. Once a character has taken an amount of nonlethal damage equal to her total hit points, any further damage from a cold environment is lethal damage. 
				
An unprotected character in cold weather (below 40\^A\mbox{${}^\circ$} F) must make a Fortitude save each hour (DC 15, +1 per previous check) or take 1d6 points of nonlethal damage. A character who has the Survival skill may receive a bonus on this saving throw and might be able to apply this bonus to other characters as well (see the skill description).
				
In conditions of severe cold or exposure (below 0\^A\mbox{${}^\circ$} F), an unprotected character must make a Fortitude save once every 10 minutes (DC 15, +1 per previous check), taking 1d6 points of nonlethal damage on each failed save. A character who has the Survival skill may receive a bonus on this saving throw and might be able to apply this bonus to other characters as well. Characters wearing a cold weather outfit only need check once per hour for cold and exposure damage.
				
A character who takes any nonlethal damage from cold or exposure is beset by frostbite or hypothermia (treat her as fatigued). These penalties end when the character recovers the nonlethal damage she took from the cold and exposure.
				
Extreme cold (below --20\^A\mbox{${}^\circ$} F) deals 1d6 points of lethal damage per minute (no save). In addition, a character must make a Fortitude save (DC 15, +1 per previous check) or take 1d4 points of nonlethal damage.
				
\subsection{Ice Effects}

				
Characters walking on ice must spend 2 squares of movement to enter a square covered by ice, and the DC for Acrobatics checks increases by +5. Characters in prolonged contact with ice might run the risk of taking damage from severe cold.
				
\subsection{Darkness}

				
Darkvision allows many characters and monsters to see perfectly well without any light at all, but characters with normal or low-light vision can be rendered completely blind by putting out the lights. Torches or lanterns can be blown out by sudden gusts of subterranean wind, magical light sources can be dispelled or countered, or magical traps might create fields of impenetrable darkness.
				
In many cases, some characters or monsters might be able to see while others are blinded. For purposes of the following points, a blinded creature is one who simply can't see through the surrounding darkness.
				
Creatures blinded by darkness lose the ability to deal extra damage due to precision (for example, via sneak attack or a duelist's precise strike ability).
				
Blind creatures must make a DC 10 Acrobatics skill check to move faster than half speed. Creatures that fail this check fall prone. Blinded creatures can't run or charge.
				
All opponents have total concealment from a blinded creature, so the blinded creature has a 50\% miss chance in combat. A blinded creature must first pinpoint the location of an opponent in order to attack the right square; if the blinded creature launches an attack without pinpointing its foe, it attacks a random square within its reach. For ranged attacks or spells against a foe whose location is not pinpointed, roll to determine which adjacent square the blinded creature is facing; its attack is directed at the closest target that lies in that direction.
				
A blinded creature loses its Dexterity modifier to AC (if positive) and takes a --2 penalty to AC.
				
A blinded creature takes a --4 penalty on Perception checks and most Strength- and Dexterity-based skill checks, including any with an armor check penalty. A creature blinded by darkness automatically fails any skill check relying on vision.
				
Creatures blinded by darkness cannot use gaze attacks and are immune to gaze attacks.
				
A creature blinded by darkness can make a Perception check as a free action each round in order to locate foes (DC equal to opponents' Stealth checks). A successful check lets a blinded character hear an unseen creature \texttt{{}"{}}over there somewhere.\texttt{{}"{}} It's almost impossible to pinpoint the location of an unseen creature. A Perception check that beats the DC by 20 reveals the unseen creature's square (but the unseen creature still has total concealment from the blinded creature).
				
A blinded creature can grope about to find unseen creatures. A character can make a touch attack with his hands or a weapon into two adjacent squares using a standard action. If an unseen target is in the designated square, there is a 50\% miss chance on the touch attack. If successful, the groping character deals no damage but has pinpointed the unseen creature's current location. If the unseen creature moves, its location is once again unknown.
				
If a blinded creature is struck by an unseen foe, the blinded character pinpoints the location of the creature that struck him (until the unseen creature moves, of course). The only exception is if the unseen creature has a reach greater than 5 feet (in which case the blinded character knows the location of the unseen opponent, but has not pinpointed him) or uses a ranged attack (in which case the blinded character knows the general direction of the foe, but not his location).
				
A creature with the scent ability automatically pinpoints unseen creatures within 5 feet of its location.
				
\subsection{Falling}

				
Creatures that fall take 1d6 points of damage per 10 feet fallen, to a maximum of 20d6. Creatures that take lethal damage from a fall land in a prone position.
				
If a character deliberately jumps instead of merely slipping or falling, the damage is the same but the first 1d6 is nonlethal damage. A DC 15 Acrobatics check allows the character to avoid any damage from the first 10 feet fallen and converts any damage from the second 10 feet to nonlethal damage. Thus, a character who slips from a ledge 30 feet up takes 3d6 damage. If the same character deliberately jumps, he takes 1d6 points of nonlethal damage and 2d6 points of lethal damage. And if the character leaps down with a successful Acrobatics check, he takes only 1d6 points of nonlethal damage and 1d6 points of lethal damage from the plunge. 
				
Falls onto yielding surfaces (soft ground, mud) also convert the first 1d6 of damage to nonlethal damage. This reduction is cumulative with reduced damage due to deliberate jumps and the Acrobatics skill.
				
A character cannot cast a spell while falling, unless the fall is greater than 500 feet or the spell is an immediate action, such as \textit{feather fall}. Casting a spell while falling requires a concentration check with a DC equal to 20 + the spell's level. Casting \textit{teleport} or a similar spell while falling does not end your momentum, it just changes your location, meaning that you still take falling damage, even if you arrive atop a solid surface.
				
\textbf{Falling into Water}: Falls into water are handled somewhat differently. If the water is at least 10 feet deep, the first 20 feet of falling do no damage. The next 20 feet do nonlethal damage (1d3 per 10-foot increment). Beyond that, falling damage is lethal damage (1d6 per additional 10-foot increment).
				
Characters who deliberately dive into water take no damage on a successful DC 15 Swim check or DC 15 Acrobatics check, so long as the water is at least 10 feet deep for every 30 feet fallen. The DC of the check, however, increases by 5 for every 50 feet of the dive. 
				
\subsection{Falling Objects}

	
% <div class="right">
Table: Damage from Falling Objects
% <
\begin{table}[]
\sffamily
\caption{Table: Damage from Falling Objects}
\begin{tabular}{ll}
\textbf{Object Size} & \textbf{Damage}\\
Small & 2d6 \\
 Medium & 3d6 \\
 Large & 4d6 \\
 Huge & 6d6 \\
 Gargantuan & 8d6 \\
 Colossal & 10d6\\
\end{tabular}
\end{table}

				
Just as characters take damage when they fall more than 10 feet, so too do they take damage when they are hit by falling objects.
				
Objects that fall upon characters deal damage based on their size and the distance they have fallen. Table: Damage from Falling Objects determines the amount of damage dealt by an object based on its size. Note that this assumes that the object is made of dense, heavy material, such as stone. Objects made of lighter materials might deal as little as half the listed damage, subject to GM discretion. For example, a Huge boulder that hits a character deals 6d6 points of damage, whereas a Huge wooden wagon might deal only 3d6 damage. In addition, if an object falls less than 30 feet, it deals half the listed damage. If an object falls more than 150 feet, it deals double the listed damage. Note that a falling object takes the same amount of damage as it deals.
				
Dropping an object on a creature requires a ranged touch attack. Such attacks generally have a range increment of 20 feet. If an object falls on a creature (instead of being thrown), that creature can make a DC 15 Reflex save to halve the damage if he is aware of the object. Falling objects that are part of a trap use the trap rules instead of these general guidelines.
								
\subsection{Heat Dangers}

				
Heat deals nonlethal damage that cannot be recovered from until the character gets cooled off (reaches shade, survives until nightfall, gets doused in water, is targeted by \textit{endure elements}, and so forth). Once a character has taken an amount of nonlethal damage equal to her total hit points, any further damage from a hot environment is lethal damage.
				
A character in very hot conditions (above 90\^A\mbox{${}^\circ$} F) must make a Fortitude saving throw each hour (DC 15, +1 for each previous check) or take 1d4 points of nonlethal damage. Characters wearing heavy clothing or armor of any sort take a --4 penalty on their saves. A character with the Survival skill may receive a bonus on this saving throw and might be able to apply this bonus to other characters as well (see the skill description). Characters reduced to unconsciousness begin taking lethal damage (1d4 points per hour).
				
In severe heat (above 110\^A\mbox{${}^\circ$} F), a character must make a Fortitude save once every 10 minutes (DC 15, +1 for each previous check) or take 1d4 points of nonlethal damage. Characters wearing heavy clothing or armor of any sort take a --4 penalty on their saves. A character with the Survival skill may receive a bonus on this saving throw and might be able to apply this bonus to other characters as well (see the Survival skill in Using Skills). Characters reduced to unconsciousness begin taking lethal damage (1d4 points per each 10-minute period).
				
A character who takes any nonlethal damage from heat exposure now suffers from heatstroke and is fatigued. These penalties end when the character recovers from the nonlethal damage she took from the heat.
				
Extreme heat (air temperature over 140\^A\mbox{${}^\circ$} F, fire, boiling water, lava) deals lethal damage. Breathing air in these temperatures deals 1d6 points of fire damage per minute (no save). In addition, a character must make a Fortitude save every 5 minutes (DC 15, +1 per previous check) or take 1d4 points of nonlethal damage. Those wearing heavy clothing or any sort of armor take a --4 penalty on their saves.
				
Boiling water deals 1d6 points of scalding damage, unless the character is fully immersed, in which case it deals 10d6 points of damage per round of exposure.
				
\subsection{Catching on Fire}

				
Characters exposed to burning oil, bonfires, and non-instantaneous magic fires might find their clothes, hair, or equipment on fire. Spells with an instantaneous duration don't normally set a character on fire, since the heat and flame from these come and go in a flash.
				
Characters at risk of catching fire are allowed a DC 15 Reflex save to avoid this fate. If a character's clothes or hair catch fire, he takes 1d6 points of damage immediately. In each subsequent round, the burning character must make another Reflex saving throw. Failure means he takes another 1d6 points of damage that round. Success means that the fire has gone out---that is, once he succeeds on his saving throw, he's no longer on fire.
				
A character on fire may automatically extinguish the flames by jumping into enough water to douse himself. If no body of water is at hand, rolling on the ground or smothering the fire with cloaks or the like permits the character another save with a +4 bonus.
				
Those whose clothes or equipment catch fire must make DC 15 Reflex saves for each item. Flammable items that fail take the same amount of damage as the character.
				
\subsection{Lava Effects}

				
Lava or magma deals 2d6 points of fire damage per round of exposure, except in the case of total immersion (such as when a character falls into the crater of an active volcano), which deals 20d6 points of fire damage per round.
				
Damage from lava continues for 1d3 rounds after exposure ceases, but this additional damage is only half of that dealt during actual contact (that is, 1d6 or 10d6 points per round). Immunity or resistance to fire serves as an immunity or resistance to fire, lava or magma. A creature immune or resistant to fire might still drown if completely immersed in lava (see Drowning).
				
\subsection{Smoke Effects}

				
A character who breathes heavy smoke must make a Fortitude save each round (DC 15, +1 per previous check) or spend that round choking and coughing. A character who chokes for 2 consecutive rounds takes 1d6 points of nonlethal damage. Smoke obscures vision, giving concealment (20\% miss chance) to characters within it.
				
\subsection{Starvation and Thirst}

				
Characters might find themselves without food or water and with no means to obtain them. In normal climates, Medium characters need at least a gallon of fluids and about a pound of decent food per day to avoid starvation. (Small characters need half as much.) In very hot climates, characters need two or three times as much water to avoid dehydration.
				
A character can go without water for 1 day plus a number of hours equal to his Constitution score. After this time, the character must make a Constitution check each hour (DC 10, +1 for each previous check) or take 1d6 points of nonlethal damage. Characters that take an amount of nonlethal damage equal to their total hit points begin to take lethal damage instead.
				
A character can go without food for 3 days, in growing discomfort. After this time, the character must make a Constitution check each day (DC 10, +1 for each previous check) or take 1d6 points of nonlethal damage. Characters that take an amount of nonlethal damage equal to their total hit points begin to take lethal damage instead.
				
Characters who have taken nonlethal damage from lack of food or water are fatigued. Nonlethal damage from thirst or starvation cannot be recovered until the character gets food or water, as needed---not even magic that restores hit points heals this damage.
				
\subsection{Suffocation}

				
A character who has no air to breathe can hold her breath for 2 rounds per point of Constitution. If a character takes a standard or full-round action, the remaining duration that the character can hold her breath is reduced by 1 round. After this period of time, the character must make a DC 10 Constitution check in order to continue holding her breath. The check must be repeated each round, with the DC increasing by +1 for each previous success.
				
When the character fails one of these Constitution checks, she begins to suffocate. In the first round, she falls unconscious (0 hit points). In the following round, she drops to --1 hit points and is dying. In the third round, she suffocates.
				
\textbf{Slow Suffocation}: A Medium character can breathe easily for 6 hours in a sealed chamber measuring 10 feet on a side. After that time, the character takes 1d6 points of nonlethal damage every 15 minutes. Each additional Medium character or significant fire source (a torch, for example) proportionally reduces the time the air will last. Once rendered unconscious through the accumulation of nonlethal damage, the character begins to take lethal damage at the same rate. Small characters consume half as much air as Medium characters.
				
\subsection{Water Dangers}

				
Any character can wade in relatively calm water that isn't over his head, no check required. Similarly, swimming in calm water only requires Swim skill checks with a DC of 10. Trained swimmers can just take 10. Remember, however, that armor or heavy gear makes any attempt at swimming much more difficult (see the Swim skill description)\textit{.}
				
By contrast, fast-moving water is much more dangerous. Characters must make a successful DC 15 Swim check or a DC 15 Strength check to avoid going under. On a failed check, the character takes 1d3 points of nonlethal damage per round (1d6 points of lethal damage if flowing over rocks and cascades).
				
Very deep water is not only generally pitch black, posing a navigational hazard, but worse, deals water pressure damage of 1d6 points per minute for every 100 feet the character is below the surface. A successful Fortitude save (DC 15, +1 for each previous check) means the diver takes no damage in that minute. Very cold water deals 1d6 points of nonlethal damage from hypothermia per minute of exposure.
				
\subsection{Drowning}

				
Any character can hold her breath for a number of rounds equal to twice her Constitution score. If a character takes a standard or full-round action, the remaining duration that the character can hold her breath is reduced by 1 round. After this period of time, the character must make a DC 10 Constitution check every round in order to continue holding her breath. Each round, the DC increases by 1. 
				
When the character finally fails her Constitution check, she begins to drown. In the first round, she falls unconscious (0 hp). In the following round, she drops to --1 hit points and is dying. In the third round, she drowns.
				
Unconscious characters must begin making Constitution checks immediately upon being submerged (or upon becoming unconscious if the character was conscious when submerged). Once she fails one of these checks, she immediately drops to --1 (or loses 1 additional hit point, if her total is below --1). On the following round, she drowns.
				
It is possible to drown in substances other than water, such as sand, quicksand, fine dust, and silos full of grain.
