\chapter{Equipment}
\section{Wealth And Money}

\label{f0}	
Each character begins play with a number of gold pieces that he can spend on weapons, armor, and other equipment. As a character adventures, he accumulates more wealth that can be spent on better gear and magic items. Table: Starting Character Wealth lists the starting gold piece values by class. In addition, each character begins play with an outfit worth 10 gp or less. For characters above 1st level, see Table: Character Wealth by Level.
\begin{table}[]
\sffamily
\caption{Table: Starting Character Wealth}
\begin{tabular}{lll}
\textbf{Class} & \textbf{Starting Wealth} & \textbf{Average}\\
Barbarian& 3d6$\times$10 gp& 105 gp \\
Bard& 3d6$\times$10 gp& 105 gp \\
Cleric& 4d6$\times$10 gp& 140 gp \\
Druid& 2d6$\times$10 gp& 70 gp \\
Fighter& 5d6$\times$10 gp& 175 gp \\
Monk& 1d6$\times$10 gp& 35 gp \\
Paladin& 5d6$\times$10 gp& 175 gp \\
Ranger& 5d6$\times$10 gp& 175 gp \\
Rogue& 4d6$\times$10 gp& 140 gp \\
Sorcerer& 2d6$\times$10 gp& 70 gp \\
Wizard& 2d6$\times$10 gp& 70 gp\\
\end{tabular}
\end{table}

\subsection{Coins}


The most common coin is the gold piece (gp). A gold piece is worth 10 silver pieces (sp). Each silver piece is worth 10 copper pieces (cp). In addition to copper, silver, and gold coins, there are also platinum pieces (pp), which are each worth 10 gp.

The standard coin weighs about a third of an ounce (50 to the pound).

\begin{table}[]
\sffamily
\caption{Table: Coins}
\begin{tabular}{lllll}
\textbf{Value} & \textbf{cp} & \textbf{sp} & \textbf{gp} & \textbf{pp}\\
Copper piece (cp)&  1&  1/10&  1/100&  1/1000 \\
Silver piece (sp)&  10&  1&  1/10&  1/100 \\
Gold piece (gp)&  100&  10&  1&  1/10 \\
Platinum piece (pp)&  1,000&  100&  10&  1\\
\end{tabular}
\end{table}
			
\subsection{Other Wealth}

		
Merchants commonly exchange trade goods without using currency. As a means of comparison, some trade goods are detailed on Table: Trade Goods.
			
% <
\begin{table}[]
\sffamily
\caption{Table: Trade Goods}
\begin{tabular}{ll}
Cost & Item\\
1 cp& One pound of wheat \\ 
2 cp& One pound of flour, or one chicken \\
1 sp& One pound of iron \\
5 sp& One pound of tobacco or copper \\
1 gp& One pound of cinnamon, or one goat \\
2 gp& One pound of ginger or pepper, or one sheep \\
3 gp& One pig \\
4 gp& One square yard of linen \\
5 gp& One pound of salt or silver \\
10 gp& One square yard of silk, or one cow \\
15 gp& One pound of saffron or cloves, or one ox \\
50 gp& One pound of gold \\ 
500 gp& One pound of platinum\\
\end{tabular}
\end{table}

\subsection{Selling Treasure}

		
In general, a character can sell something for half its listed price, including weapons, armor, gear, and magic items. This also includes character-created items.
		
Trade goods are the exception to the half-price rule. A trade good, in this sense, is a valuable good that can be easily exchanged almost as if it were cash itself.
		
\section{Weapons}

		
From the common longsword to the exotic dwarven urgrosh, weapons come in a wide variety of shapes and sizes. 
		
All weapons deal hit point damage. This damage is subtracted from the current hit points of any creature struck by the weapon. When the result of the die roll to make an attack is a natural 20 (that is, the die actually shows a 20), this is known as a critical threat (although some weapons can score a critical threat on a roll of less than 20). If a critical threat is scored, another attack roll is made, using the same modifiers as the original attack roll. If this second attack roll is equal or greater than the target's AC, the hit becomes a critical hit, dealing additional damage.
		
Weapons are grouped into several interlocking sets of categories. These categories pertain to what training is needed to become proficient in a weapon's use (simple, martial, or exotic), the weapon's usefulness either in close combat (melee) or at a distance (ranged, which includes both thrown and projectile weapons), its relative encumbrance (light, one-handed, or two-handed), and its size (Small, Medium, or Large).
		
\textbf{Simple, Martial, and Exotic Weapons}: Anybody but a druid, monk, or wizard is proficient with all simple weapons. Barbarians, fighters, paladins, and rangers are proficient with all simple and all martial weapons. Characters of other classes are proficient with an assortment of simple weapons and possibly some martial or even exotic weapons. All characters are proficient with unarmed strikes and any natural weapons possessed by their race. A character who uses a weapon with which he is not proficient takes a --4 penalty on attack rolls.
		
\textbf{Melee and Ranged Weapons}: Melee weapons are used for making melee attacks, though some of them can be thrown as well. Ranged weapons are thrown weapons or projectile weapons that are not effective in melee.
		
\textit{Reach Weapons}: Glaives, guisarmes, lances, longspears, ranseurs, and whips are reach weapons. A reach weapon is a melee weapon that allows its wielder to strike at targets that aren't adjacent to him. Most reach weapons double the wielder's natural reach, meaning that a typical Small or Medium wielder of such a weapon can attack a creature 10 feet away, but not a creature in an adjacent square. A typical Large character wielding a reach weapon of the appropriate size can attack a creature 15 or 20 feet away, but not adjacent creatures or creatures up to 10 feet away.
		
\textit{Double Weapons}: Dire flails, dwarven urgroshes, gnome hooked hammers, orc double axes, quarterstaves, and two-bladed swords are double weapons. A character can fight with both ends of a double weapon as if fighting with two weapons, but he incurs all the normal attack penalties associated with two-weapon combat, just as though the character were wielding a one-handed weapon and a light weapon.
		
The character can also choose to use a double weapon two-handed, attacking with only one end of it. A creature wielding a double weapon in one hand can't use it as a double weapon---only one end of the weapon can be used in any given round.
		
\textit{Thrown Weapons}: Daggers, clubs, shortspears, spears, darts, javelins, throwing axes, light hammers, tridents, shuriken, and nets are thrown weapons. The wielder applies his Strength modifier to damage dealt by thrown weapons (except for splash weapons). It is possible to throw a weapon that isn't designed to be thrown (that is, a melee weapon that doesn't have a numeric entry in the Range column on Table: Weapons), and a character who does so takes a --4 penalty on the attack roll. Throwing a light or one-handed weapon is a standard action, while throwing a two-handed weapon is a full-round action. Regardless of the type of weapon, such an attack scores a threat only on a natural roll of 20 and deals double damage on a critical hit. Such a weapon has a range increment of 10 feet.
		
\textit{Projectile Weapons}: Blowguns, light crossbows, slings, heavy crossbows, shortbows, composite shortbows, longbows, composite longbows, halfling sling staves, hand crossbows, and repeating crossbows are projectile weapons. Most projectile weapons require two hands to use (see specific weapon descriptions). A character gets no Strength bonus on damage rolls with a projectile weapon unless it's a specially built composite shortbow or longbow, or a sling. If the character has a penalty for low Strength, apply it to damage rolls when he uses a bow or a sling.
		
\textit{Ammunition}: Projectile weapons use ammunition: arrows (for bows), bolts (for crossbows), darts (for blowguns), or sling bullets (for slings and halfling sling staves). When using a bow, a character can draw ammunition as a free action; crossbows and slings require an action for reloading (as noted in their descriptions). Generally speaking, ammunition that hits its target is destroyed or rendered useless, while ammunition that misses has a 50\% chance of being destroyed or lost.
		
Although they are thrown weapons, shuriken are treated as ammunition for the purposes of drawing them, crafting masterwork or otherwise special versions of them, and what happens to them after they are thrown.
		
\textbf{Light, One-Handed, and Two-Handed Melee Weapons}: This designation is a measure of how much effort it takes to wield a weapon in combat. It indicates whether a melee weapon, when wielded by a character of the weapon's size category, is considered a light weapon, a one-handed weapon, or a two-handed weapon.
		
\textit{Light}: A light weapon is used in one hand. It is easier to use in one's off hand than a one-handed weapon is, and can be used while grappling (see Combat). Add the wielder's Strength modifier to damage rolls for melee attacks with a light weapon if it's used in the primary hand, or half the wielder's Strength bonus if it's used in the off hand. Using two hands to wield a light weapon gives no advantage on damage; the Strength bonus applies as though the weapon were held in the wielder's primary hand only.
		
An unarmed strike is always considered a light weapon.
		
\textit{One-Handed}: A one-handed weapon can be used in either the primary hand or the off hand. Add the wielder's Strength bonus to damage rolls for melee attacks with a one-handed weapon if it's used in the primary hand, or 1/2 his Strength bonus if it's used in the off hand. If a one-handed weapon is wielded with two hands during melee combat, add 1-1/2 times the character's Strength bonus to damage rolls.
		
\textit{Two-Handed}: Two hands are required to use a two-handed melee weapon effectively. Apply 1-1/2 times the character's Strength bonus to damage rolls for melee attacks with such a weapon. 
		
\textbf{Weapon Size}: Every weapon has a size category. This designation indicates the size of the creature for which the weapon was designed.
		
A weapon's size category isn't the same as its size as an object. Instead, a weapon's size category is keyed to the size of the intended wielder. In general, a light weapon is an object two size categories smaller than the wielder, a one-handed weapon is an object one size category smaller than the wielder, and a two-handed weapon is an object of the same size category as the wielder.
		
\textit{Inappropriately Sized Weapons}: A creature can't make optimum use of a weapon that isn't properly sized for it. A cumulative --2 penalty applies on attack rolls for each size category of difference between the size of its intended wielder and the size of its actual wielder. If the creature isn't proficient with the weapon, a --4 nonproficiency penalty also applies.
		
The measure of how much effort it takes to use a weapon (whether the weapon is designated as a light, one-handed, or two-handed weapon for a particular wielder) is altered by one step for each size category of difference between the wielder's size and the size of the creature for which the weapon was designed. For example, a Small creature would wield a Medium one-handed weapon as a two-handed weapon. If a weapon's designation would be changed to something other than light, one-handed, or two-handed by this alteration, the creature can't wield the weapon at all.
		
\textbf{Improvised Weapons}: Sometimes objects not crafted to be weapons nonetheless see use in combat. Because such objects are not designed for this use, any creature that uses an improvised weapon in combat is considered to be nonproficient with it and takes a --4 penalty on attack rolls made with that object. To determine the size category and appropriate damage for an improvised weapon, compare its relative size and damage potential to the weapon list to find a reasonable match. An improvised weapon scores a threat on a natural roll of 20 and deals double damage on a critical hit. An improvised thrown weapon has a range increment of 10 feet.
\begin{table*}[]
\sffamily
\setlength{\tabcolsep}{1pt}
\caption{Table: Weapons}
\begin{tabular}{lllllllll}
\textbf{Simple Weapons} & \textbf{Cost} & \textbf{Dmg (S)} & \textbf{Dmg (M)} & \textbf{Critical} & \textbf{Range} & \textbf{Weight} & \textbf{Type} & \textbf{Special}\\
\textit{Unarmed Attacks} \\
 Gauntlet & 2 gp & 1d2 & 1d3 & $\times$2 & - & 1 lb. & B & - \\
 Unarmed strike & - & 1d2 & 1d3 & $\times$2 & - & - & B & nonlethal \\
\textit{Light Melee Weapons} \\
 Dagger & 2 gp & 1d3 & 1d4 & 19-20/$\times$2 & 10 ft. & 1 lb. & P or S & - \\
 Dagger, punching & 2 gp & 1d3 & 1d4  & $\times$3 & - & 1 lb. & P & - \\
 Gauntlet, spiked & 5 gp & 1d3 & 1d4 & $\times$2 & - & 1 lb. & P & - \\
 Mace, light & 5 gp & 1d4 & 1d6 & $\times$2 & - & 4 lbs. & B & - \\
 Sickle & 6 gp & 1d4 & 1d6 & $\times$2 & - & 2 lbs. & S & trip \\
 One-Handed Melee Weapons \\
 Club & - & 1d4 & 1d6 & $\times$2 & 10 ft. & 3 lbs. & B & - \\
 Mace, heavy & 12 gp & 1d6 & 1d8 & $\times$2 & - & 8 lbs. & B & - \\
 Morningstar & 8 gp & 1d6 & 1d8 & $\times$2 & - & 6 lbs. & B and P & - \\
 Shortspear & 1 gp & 1d4 & 1d6 & $\times$2 & 20 ft. & 3 lbs. & P & - \\
 Two-Handed Melee Weapons \\
 Longspear & 5 gp & 1d6 & 1d8 & $\times$3 & - & 9 lbs. & P & brace, reach \\
 Quarterstaff & - & 1d4/1d4 & 1d6/1d6 & $\times$2 & - & 4 lbs. & B & double, monk \\
 Spear & 2 gp & 1d6 & 1d8 & $\times$3 & 20 ft. & 6 lbs. & P & brace \\
 \textit{Ranged Weapons} \\
 Blowgun & 2 gp & 1 & 1d2 & $\times$2 & 20 ft. & 1 lb. & P & - \\
 Darts, blowgun (10) & 5 sp & - & - & - & - & - & - & - \\
 Crossbow, heavy & 50 gp & 1d8 & 1d10 & 19-20/$\times$2 & 120 ft. & 8 lbs. & P & - \\
 Bolts, crossbow (10) & 1 gp & - & - & - & - & 1 lb. & - & - \\
 Crossbow, light & 35 gp & 1d6 & 1d8 & 19-20/$\times$2 & 80 ft. & 4 lbs. & P & - \\
 Bolts, crossbow (10) & 1 gp & - & - & - & - & 1 lb. & - & - \\
 Dart & 5 sp & 1d3 & 1d4 & $\times$2 & 20 ft. & 1/2 lb. & P & - \\
 Javelin & 1 gp & 1d4 & 1d6 & $\times$2 & 30 ft. & 2 lbs. & P & - \\
 Sling & - & 1d3 & 1d4 & $\times$2 & 50 ft. & - & B & - \\
 Bullets, sling (10) & 1 sp & - & - & - & - & 5 lbs. & - & -\\
\textbf{Martial Weapons} & \textbf{Cost} & \textbf{Dmg (S)} & \textbf{Dmg (M)} & \textbf{Critical} & \textbf{Range} & \textbf{Weight} & \textbf{Type} & \textbf{Special}\\
\textit{Light Melee Weapons} \\
 Axe, throwing & 8 gp & 1d4 & 1d6 & $\times$2 & 10 ft. & 2 lbs. & S & - \\
 Hammer, light & 1 gp & 1d3 & 1d4 & $\times$2 & 20 ft. & 2 lbs. & B & - \\
 Handaxe & 6 gp & 1d4 & 1d6 & $\times$3 & - & 3 lbs. & S & - \\
 Kukri & 8 gp & 1d3 & 1d4 & 18-20/$\times$2 & - & 2 lbs. & S & - \\
 Pick, light & 4 gp & 1d3 & 1d4 & $\times$4 & - & 3 lbs. & P & - \\
 Sap & 1 gp & 1d4 & 1d6 & $\times$2 & - & 2 lbs. & B & nonlethal \\
 Shield, light & special & 1d2 & 1d3 & $\times$2 & - & special & B & - \\
 Spiked armor & special & 1d4 & 1d6 & $\times$2 & - & special & P & - \\
 Spiked shield & light, special & 1d3 & 1d4 & $\times$2 & - & special & P & - \\
 Starknife & 24 gp & 1d3 & 1d4 & $\times$3 & 20 ft. & 3 lbs. & P & - \\
 Sword, short & 10 gp & 1d4 & 1d6 & 19-20/$\times$2 & - & 2 lbs. & P & - \\
 \textit{One-Handed Melee Weapons} \\
 Battleaxe & 10 gp & 1d6 & 1d8 & $\times$3 & - & 6 lbs. & S & - \\
 Flail & 8 gp & 1d6 & 1d8 & $\times$2 & - & 5 lbs. & B & disarm, trip \\
 Longsword & 15 gp & 1d6 & 1d8 & 19-20/$\times$2 & - & 4 lbs. & S & - \\
 Pick, heavy & 8 gp & 1d4 & 1d6 & $\times$4 & - & 6 lbs. & P & - \\
 Rapier & 20 gp & 1d4 & 1d6 & 18-20/$\times$2 & - & 2 lbs. & P & - \\
 Scimitar & 15 gp & 1d4 & 1d6 & 18-20/$\times$2 & - & 4 lbs. & S & - \\
 Shield, heavy & special & 1d3 & 1d4 & $\times$2 & - & special & B & - \\
 Spiked shield & heavy, special & 1d4 & 1d6 & $\times$2 & - & special & P & - \\
 Trident & 15 gp & 1d6 & 1d8 & $\times$2 & 10 ft. & 4 lbs. & P & brace \\
 Warhammer & 12 gp & 1d6 & 1d8 & $\times$3 & - & 5 lbs. & B & - \\
 \end{tabular}
 \end{table*}
 
 \begin{table*}
  \sffamily
  \setlength{\tabcolsep}{1pt}

  \begin{tabular}{lllllllll}
\textbf{Martial Weapons (cont.)} & \textbf{Cost} & \textbf{Dmg (S)} & \textbf{Dmg (M)} & \textbf{Critical} & \textbf{Range} & \textbf{Weight} & \textbf{Type} & \textbf{Special}\\
 \textit{Two-Handed Melee Weapons} \\
 Falchion & 75 gp & 1d6 & 2d4 & 18-20/$\times$2 & - & 8 lbs. & S & - \\
 Glaive & 8 gp & 1d8 & 1d10 & $\times$3 & - & 10 lbs. & S & reach \\
 Greataxe & 20 gp & 1d10 & 1d12 & $\times$3 & - & 12 lbs. & S & - \\
 Greatclub & 5 gp & 1d8 & 1d10 & $\times$2 & - & 8 lbs. & B & - \\
 Flail, heavy & 15 gp & 1d8 & 1d10 & 19-20/$\times$2 & - & 10 lbs. & B & disarm, trip \\
 Greatsword & 50 gp & 1d10 & 2d6 & 19-20/$\times$2 & - & 8 lbs. & S & - \\
 Guisarme & 9 gp & 1d6  & 2d4 & $\times$3 & - & 12 lbs. & S & reach, trip \\
 Halberd & 10 gp & 1d8 & 1d10 & $\times$3 & - & 12 lbs. & P or S & brace, trip \\
 Lance & 10 gp & 1d6 & 1d8 & $\times$3 & - & 10 lbs. & P & reach \\
 Ranseur & 10 gp & 1d6 & 2d4 & $\times$3 & - & 12 lbs. & P & disarm, reach \\
 Scythe & 18 gp & 1d6 & 2d4 & $\times$4 & - & 10 lbs. & P or S & trip \\
 \textit{Ranged Weapons} \\
 Longbow & 75 gp & 1d6 & 1d8 & $\times$3 & 100 ft. & 3 lbs. & P & - \\
 Arrows (20) & 1 gp & - & - & - & - & 3 lbs. & - & - \\
 Longbow, composite & 100 gp & 1d6 & 1d8 & $\times$3 & 110 ft. & 3 lbs. & P & - \\
 Arrows (20) & 1 gp & - & - & - & - & 3 lbs. & - & - \\
 Shortbow & 30 gp & 1d4 & 1d6 & $\times$3 & 60 ft. & 2 lbs. & P & - \\
 Arrows (20) & 1 gp & - & - & - & - & 3 lbs. & - & - \\
 Shortbow, composite & 75 gp & 1d4 & 1d6 & $\times$3 & 70 ft. & 2 lbs. & P & - \\
 Arrows (20) & 1 gp & - & - & - & - & 3 lbs. & - & -\\
\textbf{Exotic Weapons} & \textbf{Cost} & \textbf{Dmg (S)} & \textbf{Dmg (M)} & \textbf{Critical} & \textbf{Range} & \textbf{Weight} & \textbf{Type} & \textbf{Special}\\
\textit{Light Melee Weapons} \\
 Kama & 2 gp & 1d4 & 1d6 & $\times$2 & - & 2 lbs. & S &  monk, trip \\
 Nunchaku & 2 gp & 1d4 & 1d6 & $\times$2 & - & 2 lbs. & B & disarm, monk \\
 Sai & 1 gp & 1d3 & 1d4 & $\times$2 & - & 1 lb. & B & disarm, monk \\
 Siangham & 3 gp & 1d4 & 1d6 & $\times$2 & - & 1 lb. & P & monk \\
\textit{One-Handed Melee Weapons} \\
 Sword, bastard & 35 gp & 1d8 & 1d10 & 19-20/$\times$2 & - & 6 lbs. & S & - \\
 Waraxe, dwarven & 30 gp & 1d8 & 1d10 & $\times$3 & - & 8 lbs. & S & - \\
 Whip & 1 gp & 1d2 & 1d3 & $\times$2 & - & 2 lbs. & S & disarm, nonlethal, reach, trip \\
 \textit{Two-Handed Melee Weapons} \\
 Axe, orc double & 60 gp & 1d6/1d6 & 1d8/1d8 & $\times$3 & - & 15 lbs. & S & double \\
 Chain, spiked & 25 gp & 1d6 & 2d4 & $\times$2 & - & 10 lbs. & P & disarm, trip \\
 Curve blade, elven & 80 gp & 1d8 & 1d10 & 18-20/$\times$2 & - & 7 lbs. & S & - \\
 Flail, dire & 90 gp & 1d6/1d6 & 1d8/1d8 & $\times$2 & - & 10 lbs. & B & disarm, double, trip \\
 Hammer, gnome hooked & 20 gp & 1d6/1d4 & 1d8/1d6 & $\times$3/$\times$4 & - & 6 lbs. & B or P & double, trip \\
 Sword, two-bladed & 100 gp & 1d6/1d6 & 1d8/1d8 & 19-20/$\times$2 & - & 10 lbs. & S & double \\
 Urgrosh, dwarven & 50 gp & 1d6/1d4 & 1d8/1d6 & $\times$3 & - & 12 lbs. & P or S & brace, double \\
 \textit{Ranged Weapons} \\
 Bolas & 5 gp & 1d3 & 1d4 & $\times$2 & 10 ft. & 2 lbs. & B & nonlethal, trip \\
 Crossbow, hand & 100 gp & 1d3 & 1d4 & 19-20/$\times$2 & 30 ft. & 2 lbs. & P & - \\
 Bolts (10) & 1 gp & - & - & - & - & 1 lb. & - & - \\
 Crossbow, repeating heavy & 400 gp & 1d8 & 1d10 & 19-20/$\times$2 & 120 ft. & 12 lbs. & P & - \\
 Bolts (5) & 1 gp & - & - & - & - & 1 lb. & - & - \\
 Crossbow, repeating light & 250 gp & 1d6 & 1d8 & 19-20/$\times$2 & 80 ft. & 6 lbs. & P & - \\
 Bolts (5) & 1 gp & - & - & - & - & 1 lb. & - & - \\
 Net & 20 gp & - & - & - & 10 ft. & 6 lbs. & - & - \\
 Shuriken (5) & 1 gp & 1 & 1d2 & $\times$2 & 10 ft. & 1/2 lb. & P & monk \\
 Sling staff, halfling & 20 gp & 1d6 & 1d8 & $\times$3 & 80 ft. & 3 lbs. & B & - \\
 Bullets, sling (10) & 1 sp & - & - & - & - & 5 lbs. & - & -\\
\end{tabular}\\
\textsuperscript{1} Weight figures are for Medium weapons. A Small weapon weighs half as much, and a Large weapon weighs twice as much.\\
\textsuperscript{2} A weapon with two types is both types if the entry specifies "and," or either type (wielder's choice) if the entry specifies "or."\\
\end{table*}


\subsection{Weapon Qualities}

Here is the format for weapon entries (given as column headings on Table: Weapons).
		
\textbf{Cost}: This value is the weapon's cost in gold pieces (gp) or silver pieces (sp). The cost includes miscellaneous gear that goes with the weapon, such as a scabbard or quiver.
		
This cost is the same for a Small or Medium version of the weapon. A Large version costs twice the listed price.
		

\begin{table}[]
\sffamily
\caption{Table: Tiny and Large Weapon Damage}
\begin{tabular}{lll}
\textbf{Medium Weapon} & \textbf{Tiny Weapon} & \textbf{Large Weapon} \\
\textbf{Damage} & \textbf{Damage} & \textbf{Damage}\\
1d2& -& 1d3 \\
1d3& 1& 1d4 \\
1d4& 1d2& 1d6 \\
1d6& 1d3& 1d8 \\
1d8& 1d4& 2d6 \\
1d10& 1d6& 2d8 \\
1d12& 1d8& 3d6 \\
2d4& 1d4& 2d6 \\
2d6& 1d8& 3d6 \\
2d8& 1d10& 3d8 \\
2d10& 2d6& 4d8\\
\end{tabular}
\end{table}
		
\textbf{Dmg}: These columns give the damage dealt by the weapon on a successful hit. The column labeled \texttt{{}"{}}Dmg (S)\texttt{{}"{}} is for Small weapons. The column labeled \texttt{{}"{}}Dmg (M)\texttt{{}"{}} is for Medium weapons. If two damage ranges are given, then the weapon is a double weapon. Use the second damage figure given for the double weapon's extra attack. Table: Tiny and Large Weapon Damage gives weapon damage values for Tiny and Large weapons.
		
\textbf{Critical}: The entry in this column notes how the weapon is used with the rules for critical hits. When your character scores a critical hit, roll the damage two, three, or four times, as indicated by its critical multiplier (using all applicable modifiers on each roll), and add all the results together.
		
Extra damage over and above a weapon's normal damage is not multiplied when you score a critical hit.
		
\mbox{$\times$}\textit{2}: The weapon deals double damage on a critical hit.
		
\mbox{$\times$}\textit{3}: The weapon deals triple damage on a critical hit.
		
\mbox{$\times$}\textit{3/}\mbox{$\times$}\textit{4}: One head of this double weapon deals triple damage on a critical hit. The other head deals quadruple damage on a critical hit.
		
\mbox{$\times$}\textit{4}: The weapon deals quadruple damage on a critical hit.
		
\textit{19--20/}\mbox{$\times$}\textit{2}: The weapon scores a threat on a natural roll of 19 or 20 (instead of just 20) and deals double damage on a critical hit. 
		
\textit{18--20/}\mbox{$\times$}\textit{2}: The weapon scores a threat on a natural roll of 18, 19, or 20 (instead of just 20) and deals double damage on a critical hit. 
		
\textbf{Range}: Any attack at more than this distance is penalized for range. Beyond this range, the attack takes a cumulative --2 penalty for each full range increment (or fraction thereof) of distance to the target. For example, a dagger (with a range of 10 feet) thrown at a target that is 25 feet away would incur a --4 penalty. A thrown weapon has a maximum range of five range increments. A projectile weapon can shoot to 10 range increments.
		
\textbf{Weight}: This column gives the weight of a Medium version of the weapon. Halve this number for Small weapons and double it for Large weapons. Some weapons have a special weight. See the weapon's description for details.
		
\textbf{Type}: Weapons are classified according to the type of damage they deal: B for bludgeoning, P for piercing, or S for slashing. Some monsters may be resistant or immune to attacks from certain types of weapons.
		
Some weapons deal damage of multiple types. If a weapon causes two types of damage, the type it deals is not half one type and half another; all damage caused is of both types. Therefore, a creature would have to be immune to both types of damage to ignore any of the damage caused by such a weapon.
		
In other cases, a weapon can deal either of two types of damage. In a situation where the damage type is significant, the wielder can choose which type of damage to deal with such a weapon.
		
\textbf{Special}: Some weapons have special features in addition to those noted in their descriptions. 
		
\textit{Brace}: If you use a readied action to set a brace weapon against a charge, you deal double damage on a successful hit against a charging character (see Combat).
		
\textit{Disarm}: When you use a disarm weapon, you get a +2 bonus on Combat Maneuver Checks to disarm an enemy.
		
\textit{Double}: You can use a double weapon to fight as if fighting with two weapons, but if you do, you incur all the normal attack penalties associated with fighting with two weapons, just as if you were using a one-handed weapon and a light weapon. You can choose to wield one end of a double weapon two-handed, but it cannot be used as a double weapon when wielded in this way---only one end of the weapon can be used in any given round.
		
\textit{Monk}: A monk weapon can be used by a monk to perform a flurry of blows (see Classes).
		
\textit{Nonlethal}: These weapons deal nonlethal damage (see Combat).
		
\textit{Reach}: You use a reach weapon to strike opponents 10 feet away, but you can't use it against an adjacent foe.
		
\textit{Trip}: You can use a trip weapon to make trip attacks. If you are tripped during your own trip attempt, you can drop the weapon to avoid being tripped.
	
\subsection{Weapon Descriptions}

\textbf{Arrows}: An arrow used as a melee weapon is treated as a light improvised weapon (--4 penalty on attack rolls) and deals damage as a dagger of its size (critical multiplier \mbox{$\times$}2). Arrows come in a leather quiver that holds 20 arrows.
		
\textbf{Axe, Orc Double}: A cruel weapon with blades placed at opposite ends of a long haft, an orc double axe is a double weapon.
		
\textbf{Blowgun}: Blowguns are generally used to deliver debilitating (but rarely fatal) poisons from a distance. They are nearly silent when fired. For a list of appropriate poisons, see Poison.
		
\textbf{Bolas}: A bolas is a pair of weights, connected by a thin rope or cord. You can use this weapon to make a ranged trip attack against an opponent. You can't be tripped during your own trip attempt when using a bolas.
		
\textbf{Bolts}: A crossbow bolt used as a melee weapon is treated as a light improvised weapon (--4 penalty on attack rolls) and deals damage as a dagger of its size (crit \mbox{$\times$}2). Bolts come in a case or quiver that holds 10 bolts (or 5, for a repeating crossbow). 
		
\textbf{Bullets, Sling}: Bullets are shaped metal balls, designed to be used by a sling or halfling sling staff. Bullets come in a leather pouch that holds 10 bullets.
		
\textbf{Chain, Spiked}: A spiked chain is about 4 feet in length, covered in wicked barbs. You can use the Weapon Finesse feat to apply your Dexterity modifier instead of your Strength modifier to attack rolls with a spiked chain sized for you, even though it isn't a light weapon.
		
\textbf{Crossbow, Hand}: You can draw a hand crossbow back by hand. Loading a hand crossbow is a move action that provokes attacks of opportunity.
		
You can shoot, but not load, a hand crossbow with one hand at no penalty. You can shoot a hand crossbow with each hand, but you take a penalty on attack rolls as if attacking with two light weapons.
		
\textbf{Crossbow, Heavy}: You draw a heavy crossbow back by turning a small winch. Loading a heavy crossbow is a full-round action that provokes attacks of opportunity.
		
Normally, operating a heavy crossbow requires two hands. However, you can shoot, but not load, a heavy crossbow with one hand at a --4 penalty on attack rolls. You can shoot a heavy crossbow with each hand, but you take a penalty on attack rolls as if attacking with two one-handed weapons. This penalty is cumulative with the penalty for one-handed firing.
		
\textbf{Crossbow, Light}: You draw a light crossbow back by pulling a lever. Loading a light crossbow is a move action that provokes attacks of opportunity.
		
Normally, operating a light crossbow requires two hands. However, you can shoot, but not load, a light crossbow with one hand at a --2 penalty on attack rolls. You can shoot a light crossbow with each hand, but you take a penalty on attack rolls as if attacking with two light weapons. This penalty is cumulative with the penalty for one-handed firing.
		
\textbf{Crossbow, Repeating}: The repeating crossbow (whether heavy or light) holds 5 crossbow bolts. As long as it holds bolts, you can reload it by pulling the reloading lever (a free action). Loading a new case of 5 bolts is a full-round action that provokes attacks of opportunity.
		
You can fire a repeating crossbow with one hand or fire a repeating crossbow in each hand in the same manner as you would a normal crossbow of the same size. However, you must fire the weapon with two hands in order to use the reloading lever, and you must use two hands to load a new case of bolts.
		
\textbf{Curve Blade, Elven}: Essentially a longer version of a scimitar, but with a thinner blade, the elven curve blade is exceptionally rare. You receive a +2 circumstance bonus to your Combat Maneuver Defense whenever a foe attempts to sunder your elven curve blade due to its flexible metal.
		
You can use the Weapon Finesse feat to apply your Dexterity modifier instead of your Strength modifier to attack rolls with an elven curve blade sized for you, even though it isn't a light weapon.
		
\textbf{Dagger}: A dagger has a blade that is about 1 foot in length. You get a +2 bonus on Sleight of Hand skill checks made to conceal a dagger on your body (see Using Skills).
		
\textbf{Dagger, Punching}: A punching dagger's blade is attached to a horizontal handle that projects out from the fist when held.
		
\textbf{Flail}: A flail consists of a spiked metal ball, connected to a handle by a sturdy chain.
		
\textbf{Flail, Dire}: A dire flail consists of two spheres of spiked iron dangling from chains at opposite ends of a long haft.
		
\textbf{Flail, Heavy}: Similar to a flail, a heavy flail has a larger metal ball and a longer handle. 
		
\textbf{Gauntlet}: This metal glove lets you deal lethal damage rather than nonlethal damage with unarmed strikes. A strike with a gauntlet is otherwise considered an unarmed attack. The cost and weight given are for a single gauntlet. Medium and heavy armors (except breastplate) come with gauntlets. Your opponent cannot use a disarm action to disarm you of gauntlets.
		
\textbf{Gauntlet, Spiked}: The cost and weight given are for a single gauntlet. An attack with a spiked gauntlet is considered an armed attack. Your opponent cannot use a disarm action to disarm you of spiked gauntlets.
		
\textbf{Glaive}: A glaive is a simple blade, mounted to the end of a pole about 7 feet in length.
		
\textbf{Greatsword}: This immense two-handed sword is about 5 feet in length.
		
\textbf{Guisarme}: A guisarme is an 8-foot-long shaft with a blade and a hook mounted at the tip.
		
\textbf{Halberd}: A halberd is similar to a 5-foot-long spear, but it also has a small, axe-like head mounted near the tip.
		
\textbf{Hammer, Gnome Hooked}: A gnome hooked hammer is a double weapon---an ingenious tool with a hammer head at one end of its haft and a long, curved pick at the other. The hammer's blunt head is a bludgeoning weapon that deals 1d6 points of damage (crit \mbox{$\times$}3). Its hook is a piercing weapon that deals 1d4 points of damage (crit \mbox{$\times$}4). You can use either head as the primary weapon. Gnomes treat hooked hammers as martial weapons.
		
\textbf{Javelin}: A javelin is a thin throwing spear. Since it is not designed for melee, you are treated as nonproficient with it and take a --4 penalty on attack rolls if you use a javelin as a melee weapon.
		
\textbf{Kama}: Similar to a sickle, a kama is a short, curved blade attached to a simple handle.
		
\textbf{Kukri}: A kukri is a curved blade, about 1 foot in length.
		
\textbf{Lance}: A lance deals double damage when used from the back of a charging mount. While mounted, you can wield a lance with one hand.
		
\textbf{Longbow}: At almost 5 feet in height, a longbow is made up of one solid piece of carefully curved wood. You need two hands to use a bow, regardless of its size. A longbow is too unwieldy to use while you are mounted. If you have a penalty for low Strength, apply it to damage rolls when you use a longbow. If you have a Strength bonus, you can apply it to damage rolls when you use a composite longbow (see below), but not when you use a regular longbow.
		
\textbf{Longbow, Composite}: You need at least two hands to use a bow, regardless of its size. You can use a composite longbow while mounted. All composite bows are made with a particular strength rating (that is, each requires a minimum Strength modifier to use with proficiency). If your Strength bonus is less than the strength rating of the composite bow, you can't effectively use it, so you take a --2 penalty on attacks with it. The default composite longbow requires a Strength modifier of +0 or higher to use with proficiency. A composite longbow can be made with a high strength rating to take advantage of an above-average Strength score; this feature allows you to add your Strength bonus to damage, up to the maximum bonus indicated for the bow. Each point of Strength bonus granted by the bow adds 100 gp to its cost. If you have a penalty for low Strength, apply it to damage rolls when you use a composite longbow.
		
For purposes of Weapon Proficiency and similar feats, a composite longbow is treated as if it were a longbow.
		
\textbf{Longspear}: A longspear is about 8 feet in length.
		
\textbf{Longsword}: This sword is about 3-1/2 feet in length.
		
\textbf{Mace}: A mace is made up of an ornate metal head attached to a simple wooden or metal shaft.
		
\textbf{Mace, Heavy}: A heavy mace has a larger head and a longer handle than a normal mace.
		
\textbf{Morningstar}: A morningstar is a spiked metal ball, affixed to the top of a long handle.
		
\textbf{Net}: A net is used to entangle enemies. When you throw a net, you make a ranged touch attack against your target. A net's maximum range is 10 feet. If you hit, the target is entangled. An entangled creature takes a --2 penalty on attack rolls and a --4 penalty on Dexterity, can move at only half speed, and cannot charge or run. If you control the trailing rope by succeeding on an opposed Strength check while holding it, the entangled creature can move only within the limits that the rope allows. If the entangled creature attempts to cast a spell, it must make a concentration check with a DC of 15 + the spell's level or be unable to cast the spell.
		
An entangled creature can escape with a DC 20 Escape Artist check (a full-round action). The net has 5 hit points and can be burst with a DC 25 Strength check (also a full-round action). A net is useful only against creatures within one size category of you.
		
A net must be folded to be thrown effectively. The first time you throw your net in a fight, you make a normal ranged touch attack roll. After the net is unfolded, you take a --4 penalty on attack rolls with it. It takes 2 rounds for a proficient user to fold a net and twice that long for a nonproficient one to do so.
		
\textbf{Nunchaku}: A nunchaku is made up of two wooden or metal bars connected by a small length of rope or chain.
		
\textbf{Quarterstaff}: A quarterstaff is a simple piece of wood, about 5 feet in length.
		
\textbf{Ranseur}: Similar in appearance to a trident, a ranseur has a single spear at its tip, flanked by a pair of short, curving blades.
		
\textbf{Rapier}: You can use the Weapon Finesse feat to apply your Dexterity modifier instead of your Strength modifier to attack rolls with a rapier sized for you, even though it isn't a light weapon. You can't wield a rapier in two hands in order to apply 1-1/2 times your Strength bonus to damage.
		
\textbf{Sai}: A sai is a metal spike flanked by a pair of prongs used to trap an enemy's weapon. With a sai, you get a +2 bonus on Combat Maneuver Checks to sunder an enemy's weapon. Though pointed, a sai is used primarily to bludgeon foes and to disarm weapons.
		
\textbf{Shield, Heavy or Light}: You can bash with a shield instead of using it for defense.
		
\textbf{Shortbow}: A shortbow is made up of one piece of wood, about 3 feet in length. You need two hands to use a bow, regardless of its size. You can use a shortbow while mounted. If you have a penalty for low Strength, apply it to damage rolls when you use a shortbow. If you have a bonus for high Strength, you can apply it to damage rolls when you use a composite shortbow (see below), but not a regular shortbow.
		
\textbf{Shortbow, Composite}: You need at least two hands to use a bow, regardless of its size. You can use a composite shortbow while mounted. All composite bows are made with a particular strength rating (that is, each requires a minimum Strength modifier to use with proficiency). If your Strength bonus is lower than the strength rating of the composite bow, you can't effectively use it, so you take a --2 penalty on attacks with it. The default composite shortbow requires a Strength modifier of +0 or higher to use with proficiency. A composite shortbow can be made with a high strength rating to take advantage of an above-average Strength score; this feature allows you to add your Strength bonus to damage, up to the maximum bonus indicated for the bow. Each point of Strength bonus granted by the bow adds 75 gp to its cost. If you have a penalty for low Strength, apply it to damage rolls when you use a composite shortbow. 
		
For purposes of Weapon Proficiency, Weapon Focus, and similar feats, a composite shortbow is treated as if it were a shortbow.
		
\textbf{Shortspear}: A shortspear is about 3 feet in length, making it a suitable thrown weapon.
		
\textbf{Shortsword}: This sword is about 2 feet in length.
		
\textbf{Shuriken}: A shuriken is a small piece of metal with sharpened edges, designed for throwing. A shuriken can't be used as a melee weapon. Although they are thrown weapons, shuriken are treated as ammunition for the purposes of drawing them, crafting masterwork or otherwise special versions of them, and what happens to them after they are thrown.
		
\textbf{Siangham}: This weapon is a handheld shaft fitted with a pointed tip for stabbing foes.
		
\textbf{Sling}: A sling is little more than a leather cup attached to a pair of strings. Your Strength modifier applies to damage rolls when you use a sling, just as it does for thrown weapons. You can fire, but not load, a sling with one hand. Loading a sling is a move action that requires two hands and provokes attacks of opportunity.
		
You can hurl ordinary stones with a sling, but stones are not as dense or as round as bullets. Thus, such an attack deals damage as if the weapon were designed for a creature one size category smaller than you and you take a --1 penalty on attack rolls.
		
\textbf{Sling Staff, Halfling}: Made from a specially designed sling attached to a short club, a halfling sling staff can be used by a proficient wielder to devastating effect. Your Strength modifier applies to damage rolls when you use a halfling sling staff, just as it does for thrown weapons. You can fire, but not load, a halfling sling staff with one hand. Loading a halfling sling staff is a move action that requires two hands and provokes attacks of opportunity.
		
 You can hurl ordinary stones with a halfling sling staff, but stones are not as dense or as round as bullets. Thus, such an attack deals damage as if the weapon were designed for a creature one size category smaller than you and you take a --1 penalty on attack rolls.
		
A halfling sling staff can be used as a simple weapon that deals bludgeoning damage equal to that of a club of its size. Halflings treat halfling sling staves as martial weapons.
		
\textbf{Spear}: A spear is 5 feet in length and can be thrown. 
		
\textbf{Spiked Armor}: You can outfit your armor with spikes, which can deal damage in a grapple or as a separate attack. See Armor, below, for details.
		
\textbf{Spiked Shield, Heavy or Light}: You can bash with a spiked shield instead of using it for defense.
		
\textbf{Starknife}: From a central metal ring, four tapering metal blades extend like points on a compass rose. A wielder can stab with the starknife or throw it. 
		
\textbf{Strike, Unarmed}: A Medium character deals 1d3 points of nonlethal damage with an unarmed strike. A Small character deals 1d2 points of nonlethal damage. A monk or any character with the Improved Unarmed Strike feat can deal lethal or nonlethal damage with unarmed strikes, at his discretion. The damage from an unarmed strike is considered weapon damage for the purposes of effects that give you a bonus on weapon damage rolls.
		
An unarmed strike is always considered a light weapon. Therefore, you can use the Weapon Finesse feat to apply your Dexterity modifier instead of your Strength modifier to attack rolls with an unarmed strike. Unarmed strikes do not count as natural weapons (see Combat).
		
\textbf{Sword, Bastard}: A bastard sword is about 4 feet in length, making it too large to use in one hand without special training; thus, it is an exotic weapon. A character can use a bastard sword two-handed as a martial weapon.
		
\textbf{Sword, Two-Bladed}: A two-bladed sword is a double weapon---twin blades extend from either side of a central, short haft, allowing the wielder to attack with graceful but deadly flourishes.
		
\textbf{Trident}: A trident has three metal prongs at end of a 4-foot-long shaft. This weapon can be thrown.
		
\textbf{Urgrosh, Dwarven}: A dwarven urgrosh is a double weapon---an axe head and a spear point on opposite ends of a long haft. The urgrosh's axe head is a slashing weapon that deals 1d8 points of damage. Its spear head is a piercing weapon that deals 1d6 points of damage. You can use either head as the primary weapon. The other becomes the off-hand weapon. If you use an urgrosh against a charging character, the spear head is the part of the weapon that deals damage. Dwarves treat dwarven urgroshes as martial weapons.
		
\textbf{Waraxe, Dwarven}: A dwarven waraxe has a large, ornate head mounted to a thick handle, making it too large to use in one hand without special training; thus, it is an exotic weapon. A Medium character can use a dwarven waraxe two-handed as a martial weapon, or a Large creature can use it one-handed in the same way. A dwarf treats a dwarven waraxe as a martial weapon even when using it in one hand.
		
\textbf{Whip}: A whip deals no damage to any creature with an armor bonus of +1 or higher or a natural armor bonus of +3 or higher. The whip is treated as a melee weapon with 15-foot reach, though you don't threaten the area into which you can make an attack. In addition, unlike most other weapons with reach, you can use it against foes anywhere within your reach (including adjacent foes).
		
Using a whip provokes an attack of opportunity, just as if you had used a ranged weapon.
		
You can use the Weapon Finesse feat to apply your Dexterity modifier instead of your Strength modifier to attack rolls with a whip sized for you, even though it isn't a light weapon.
	
\subsection{Masterwork Weapons}

		
A masterwork weapon is a finely crafted version of a normal weapon. Wielding it provides a +1 enhancement bonus on attack rolls.
		
You can't add the masterwork quality to a weapon after it is created; it must be crafted as a masterwork weapon (see the Craft skill). The masterwork quality adds 300 gp to the cost of a normal weapon (or 6 gp to the cost of a single unit of ammunition). Adding the masterwork quality to a double weapon costs twice the normal increase (+600 gp).
		
Masterwork ammunition is damaged (effectively destroyed) when used. The enhancement bonus of masterwork ammunition does not stack with any enhancement bonus of the projectile weapon firing it.
		
All magic weapons are automatically considered to be of masterwork quality. The enhancement bonus granted by the masterwork quality doesn't stack with the enhancement bonus provided by the weapon's magic.
		
Even though some types of armor and shields can be used as weapons, you can't create a masterwork version of such an item that confers an enhancement bonus on attack rolls. Instead, masterwork armor and shields have lessened armor check penalties.

\section{Armor}

	
For most, armor is the simplest way to protect oneself in a world of rampant threats and dangers. Many characters can wear only the simplest of armors, and only some can use shields. To wear heavier armor effectively, a character can select the Armor Proficiency feats, but most classes are automatically proficient with the armors that work best for them.
	
Here is the format for armor entries (given as column headings on Table: Armor and Shields).
	
\textbf{Cost}: The cost in gold pieces of the armor for Small or Medium humanoid creatures. See Table: Armor for Unusual Creatures for armor prices for other creatures.
	
\textbf{Armor/Shield Bonus}: Each type of armor grants an armor bonus to AC, while shields grant a shield bonus to AC. The armor bonus from a suit of armor doesn't stack with other effects or items that grant an armor bonus. Similarly, the shield bonus from a shield doesn't stack with other effects that grant a shield bonus.
	
\textbf{Maximum Dex Bonus}: This number is the maximum Dexterity bonus to AC that this type of armor allows. Dexterity bonuses in excess of this number are reduced to this number for the purposes of determining the wearer's AC. Heavier armors limit mobility, reducing the wearer's ability to dodge blows. This restriction doesn't affect any other Dexterity-related abilities.
	
Even if a character's Dexterity bonus to AC drops to 0 because of armor, this situation does not count as losing his Dexterity bonus to AC. 
	
A character's encumbrance (the amount of gear carried, including armor) may also restrict the maximum Dexterity bonus that can be applied to his Armor Class.
	
\textit{Shields}: Shields do not affect a character's maximum Dexterity bonus, except for tower shields.
	
\textbf{Armor Check Penalty}: Any armor heavier than leather, as well as any shield, hurts a character's ability to use Dexterity- and Strength-based skills. An armor check penalty applies to all Dexterity- and Strength-based skill checks. A character's encumbrance may also incur an armor check penalty.
	
\textit{Shields}: If a character is wearing armor and using a shield, both armor check penalties apply.
	
\textit{Nonproficient with Armor Worn}: A character who wears armor and/or uses a shield with which he is not proficient takes the armor's (and/or shield's) armor check penalty on attack rolls as well as on all Dexterity- and Strength-based ability and skill checks. The penalty for nonproficiency with armor stacks with the penalty for shields.
	
\textit{Sleeping in Armor}: A character who sleeps in medium or heavy armor is automatically fatigued the next day. He takes a --2 penalty on Strength and Dexterity and can't charge or run. Sleeping in light armor does not cause fatigue.
	
\textbf{Arcane Spell Failure Chance}: Armor interferes with the gestures that a spellcaster must make to cast an arcane spell that has a somatic component. Arcane spellcasters face the possibility of arcane spell failure if they're wearing armor. Bards can wear light armor and use shields without incurring any arcane spell failure chance for their bard spells.
	
\textit{Casting an Arcane Spell in Armor}: A character who casts an arcane spell while wearing armor must usually make an arcane spell failure check. The number in the Arcane Spell Failure Chance column on Table: Armor and Shields is the percentage chance that the spell fails and is ruined. If the spell lacks a somatic component, however, it can be cast with no chance of arcane spell failure.
	
\textit{Shields}: If a character is wearing armor and using a shield, add the two numbers together to get a single arcane spell failure chance.
	
\textbf{Speed}: Medium or heavy armor slows the wearer down. The number on Table: Armor and Shields is the character's speed while wearing the armor. Humans, elves, half-elves, and half-orcs have an unencumbered speed of 30 feet. They use the first column. Dwarves, gnomes, and halflings have an unencumbered speed of 20 feet. They use the second column. Remember, however, that a dwarf's land speed remains 20 feet even in medium or heavy armor or when carrying a medium or heavy load.
	
\textit{Shields}: Shields do not affect a character's speed.
	
\textbf{Weight}: This column gives the weight of the armor sized for a Medium wearer. Armor fitted for Small characters weighs half as much, and armor for Large characters weighs twice as much.
	
\subsection{Armor Descriptions}

\begin{table*}[]
\sffamily
\setlength{\tabcolsep}{1pt}
\caption{Table: Armor and Shields}
\begin{tabular}{lllllllll}
                        &      & \textbf{Armor/Shield} & \textbf{Maximum} & \textbf{Armor Check} & \textbf{Arcane Spell} & \multicolumn{2}{c}{\textbf{Speed}} &  \\
\textbf{Armor} & \textbf{Cost} & \textbf{Bonus}        & \textbf{Dex bonus} & \textbf{Penalty}   & \textbf{Failure Chance} & \textbf{30 ft.} & \textbf{20 ft.} & \textbf{Weight}\\
\textit{Light armor} \\
 Padded& 5 gp& +1& +8& 0& 5\%& 30 ft.& 20 ft.& 10 lbs. \\
 Leather& 10 gp& +2& +6& 0& 10\%& 30 ft.& 20 ft.& 15 lbs. \\
 Studded leather& 25 gp& +3& +5& -1& 15\%& 30 ft.& 20 ft.& 20 lbs. \\
 Chain shirt& 100 gp& +4& +4& -2& 20\%& 30 ft.& 20 ft.& 25 lbs. \\
\textit{Medium armor}   \\
 Hide& 15 gp& +4& +4& -3& 20\%& 20 ft.& 15 ft.& 25 lbs. \\
 Scale mail& 50 gp& +5& +3& -4& 25\%& 20 ft.& 15 ft.& 30 lbs. \\
 Chainmail& 150 gp& +6& +2& -5& 30\%& 20 ft.& 15 ft.& 40 lbs. \\
 Breastplate& 200 gp& +6& +3& -4& 25\%& 20 ft.& 15 ft.& 30 lbs. \\
\textit{Heavy armor}   \\
 Splint mail& 200 gp& +7& +0& -7& 40\%& 20 ft.& 15 ft.& 45 lbs. \\
 Banded mail& 250 gp& +7& +1& -6& 35\%& 20 ft.& 15 ft.& 35 lbs. \\
 Half-plate& 600 gp& +8& +0& -7& 40\%& 20 ft.& 15 ft.& 50 lbs. \\
 Full plate& 1,500 gp& +9& +1& -6& 35\%& 20 ft.& 15 ft.& 50 lbs. \\
 \textit{Shields}   \\
 Buckler& 5 gp& +1& -& -1& 5\%& -& -& 5 lbs. \\
 Shield, light wooden& 3 gp& +1& -& -1& 5\%& -& -& 5 lbs. \\
 Shield, light steel& 9 gp& +1& -& -1& 5\%& -& -& 6 lbs. \\
 Shield, heavy wooden& 7 gp& +2& -& -2& 15\%& -& -& 10 lbs. \\
 Shield, heavy steel& 20 gp& +2& -& -2& 15\%& -& -& 15 lbs. \\
 Shield, tower& 30 gp& +4& +2& -10& 50\%& -& -& 45 lbs. \\
 \textit{Extras}   \\
 Armor spikes& +50 gp& -& -& -& -& -& -& +10 lbs. \\
 Gauntlet, locked& 8 gp& -& -& special& n/a& -& -& +5 lbs. \\
 Shield spikes& +10 gp& -& -& -& -& -& -& +5 lbs.\\
\end{tabular}
\(^{1}\) Weight figures are for armor sized to fit Medium characters. Armor fitted for Small characters weighs half as much, and armor fitted for Large characters weighs twice as much.
\(^{2}\) When running in heavy armor, you move only triple your speed, not quadruple.
\(^{3}\) A tower shield can instead grant you cover. See the description.
\(^{4}\) Hand not free to cast spells.
\end{table*}
Any special benefits or accessories to the types of armor found on Table: Armor and Shields are described below.
		
\textbf{Armor Spikes}: You can have spikes added to your armor, which allow you to deal extra piercing damage (see \texttt{{}"{}}spiked armor\texttt{{}"{}} on Table: Weapons) on a successful grapple attack. The spikes count as a martial weapon. If you are not proficient with them, you take a --4 penalty on grapple checks when you try to use them. You can also make a regular melee attack (or off-hand attack) with the spikes, and they count as a light weapon in this case. (You can't also make an attack with armor spikes if you have already made an attack with another off-hand weapon, and vice versa.) An enhancement bonus to a suit of armor does not improve the spikes' effectiveness, but the spikes can be made into magic weapons in their own right.
		
\textbf{Banded Mail}: Banded mail is made up of overlapping strips of metal, fastened to a leather backing. The suit includes gauntlets.
		
\textbf{Breastplate}: Covering only the torso, a breastplate is made up of a single piece of sculpted metal.
		
\textbf{Buckler}: This small metal shield is worn strapped to your forearm. You can use a bow or crossbow without penalty while carrying it. You can also use your shield arm to wield a weapon (whether you are using an off-hand weapon or using your off hand to help wield a two-handed weapon), but you take a --1 penalty on attack rolls while doing so. This penalty stacks with those that may apply for fighting with your off hand and for fighting with two weapons. In any case, if you use a weapon in your off hand, you lose the buckler's AC bonus until your next turn. You can cast a spell with somatic components using your shield arm, but you lose the buckler's AC bonus until your next turn. You can't make a shield bash with a buckler.
		
\textbf{Chain Shirt}: Covering the torso, this shirt is made up of thousands of interlocking metal rings.
		
\textbf{Chainmail}: Unlike a chain shirt, chainmail covers the legs and arms of the wearer. The suit includes gauntlets.
		
\textbf{Full Plate}: This metal suit includes gauntlets, heavy leather boots, a visored helmet, and a thick layer of padding that is worn underneath the armor. Each suit of full plate must be individually fitted to its owner by a master armorsmith, although a captured suit can be resized to fit a new owner at a cost of 200 to 800 (2d4 \mbox{$\times$} 100) gold pieces.
		
\textbf{Gauntlet, Locked}: This armored gauntlet has small chains and braces that allow the wearer to attach a weapon to the gauntlet so that it cannot be dropped easily. It provides a +10 bonus to your Combat Maneuver Defense to keep from being disarmed in combat. Removing a weapon from a locked gauntlet or attaching a weapon to a locked gauntlet is a full-round action that provokes attacks of opportunity.
		
The price given is for a single locked gauntlet. The weight given applies only if you're wearing a breastplate, light armor, or no armor. Otherwise, the locked gauntlet replaces a gauntlet you already have as part of the armor.
		
While the gauntlet is locked, you can't use the hand wearing it for casting spells or employing skills. (You can still cast spells with somatic components, provided that your other hand is free.)
		
Like a normal gauntlet, a locked gauntlet lets you deal lethal damage rather than nonlethal damage with an unarmed strike.
		
\textbf{Half-Plate}: Combining elements of full plate and chainmail, half-plate includes gauntlets and a helm.
		
\textbf{Hide}: Hide armor is made up of the tanned and preserved skin of any thick-hided beast.
		
\textbf{Leather}: Leather armor is made up of pieces of hard boiled leather carefully sewn together.
		
\textbf{Padded}: Little more than heavy, quilted cloth, this armor provides only the most basic protection.
		
\textbf{Scale Mail}: Scale mail is made up of dozens of small overlapping metal plates. The suit includes gauntlets.
		
\textbf{Shield, Heavy; Wooden or Steel}: You strap a shield to your forearm and grip it with your hand. A heavy shield is so heavy that you can't use your shield hand for anything else.
		
\textit{Wooden or Steel}: Wooden and steel shields offer the same basic protection, though they respond differently to spells and effects.
		
\textit{Shield Bash Attacks}: You can bash an opponent with a heavy shield. See \texttt{{}"{}}shield, heavy\texttt{{}"{}} on Table: Weapons for the damage dealt by a shield bash. Used this way, a heavy shield is a martial bludgeoning weapon. For the purpose of penalties on attack rolls, treat a heavy shield as a one-handed weapon. If you use your shield as a weapon, you lose its AC bonus until your next turn. An enhancement bonus on a shield does not improve the effectiveness of a shield bash made with it, but the shield can be made into a magic weapon in its own right.
		
\textbf{Shield, Light; Wooden or Steel}: You strap a shield to your forearm and grip it with your hand. A light shield's weight lets you carry other items in that hand, although you cannot use weapons with it.
		
\textit{Wooden or Steel}: Wooden and steel shields offer the same basic protection, though they respond differently to some spells and effects.
		
\textit{Shield Bash Attacks}: You can bash an opponent with a light shield. See \texttt{{}"{}}shield, light\texttt{{}"{}} on Table: Weapons for the damage dealt by a shield bash. Used this way, a light shield is a martial bludgeoning weapon. For the purpose of penalties on attack rolls, treat a light shield as a light weapon. If you use your shield as a weapon, you lose its AC bonus until your next turn. An enhancement bonus on a shield does not improve the effectiveness of a shield bash made with it, but the shield can be made into a magic weapon in its own right.
		
\textbf{Shield, Tower}: This massive wooden shield is nearly as tall as you are. In most situations, it provides the indicated shield bonus to your AC. As a standard action, however, you can use a tower shield to grant you total cover until the beginning of your next turn. When using a tower shield in this way, you must choose one edge of your space. That edge is treated as a solid wall for attacks targeting you only. You gain total cover for attacks that pass through this edge and no cover for attacks that do not pass through this edge (see Combat). The shield does not, however, provide cover against targeted spells; a spellcaster can cast a spell on you by targeting the shield you are holding. You cannot bash with a tower shield, nor can you use your shield hand for anything else.
		
When employing a tower shield in combat, you take a --2 penalty on attack rolls because of the shield's encumbrance.
		
\textbf{Shield Spikes}: These spikes turn a shield into a martial piercing weapon and increase the damage dealt by a shield bash as if the shield were designed for a creature one size category larger than you (see \texttt{{}"{}}spiked shields\texttt{{}"{}} on Table: Weapons). You can't put spikes on a buckler or a tower shield. Otherwise, attacking with a spiked shield is like making a shield bash attack.
		
An enhancement bonus on a spiked shield does not improve the effectiveness of a shield bash made with it, but a spiked shield can be made into a magic weapon in its own right.
		
\textbf{Splint Mail}: Splint mail is made up of metal strips, like banded mail. The suit includes gauntlets.
		
\textbf{Studded Leather}: Similar to leather armor, this suit is reinforced with small metal studs.
	
\subsection{Masterwork Armor}

		
Just as with weapons, you can purchase or craft masterwork versions of armor or shields. Such a well-made item functions like the normal version, except that its armor check penalty is lessened by 1. 
		
A masterwork suit of armor or shield costs an extra 150 gp over and above the normal cost for that type of armor or shield.
		
The masterwork quality of a suit of armor or shield never provides a bonus on attack or damage rolls, even if the armor or shield is used as a weapon.
		
All magic armors and shields are automatically considered to be of masterwork quality.
		
You can't add the masterwork quality to armor or a shield after it is created; it must be crafted as a masterwork item.
	
\subsection{Armor for Unusual Creatures}

\begin{table}[]
\sffamily
\caption{Table: Armor for Unusual Creatures}
\begin{tabular}{lllll}
  & \multicolumn{2}{c}{\textbf{Humanoid}} & \multicolumn{2}{c}{\textbf{Nonhumanoid}}\\
\textbf{Size} & \textbf{Cost} & \textbf{Weight} & \textbf{Cost} & \textbf{Weight}\\
Tiny or smaller*& $\times$1/2& $\times$1/10& $\times$1& $\times$1/10 \\
 Small& $\times$1& $\times$1/2& $\times$2& $\times$1/2 \\
 Medium& $\times$1& $\times$1& $\times$2& $\times$1 \\
 Large& $\times$2& $\times$2& $\times$4& $\times$2 \\
 Huge& $\times$4& $\times$5& $\times$8& $\times$5 \\
 Gargantuan& $\times$8& $\times$8& $\times$16& $\times$8 \\
 Colossal& $\times$16& $\times$12& $\times$32& $\times$12\\
\end{tabular}
* Divide armor bonus by 2.\\
\end{table}
Armor and shields for unusually big creatures, unusually little creatures, and nonhumanoid creatures (such as horses) have different costs and weights from those given on Table: Armor and Shields. Refer to the appropriate line on Table: Armor for Unusual Creatures and apply the multipliers to cost and weight for the armor type in question.

\subsection{Getting Into and Out of Armor}

		
The time required to don armor depends on its type; see Table: Donning Armor.
		
\textbf{Don}: This column tells how long it takes a character to put the armor on. (One minute is 10 rounds.) Readying (strapping on) a shield is only a move action.
		
\textbf{Don Hastily}: This column tells how long it takes to put the armor on in a hurry. The armor check penalty and armor bonus for hastily donned armor are each 1 point worse than normal. 
		
\textbf{Remove}: This column tells how long it takes to get the armor off. Removing a shield from the arm and dropping it is only a move action.

\begin{table*}[]
\sffamily
\caption{Table: Donning Armor}
\begin{tabular}{llll}
\textbf{Armor Type} & \textbf{Don} & \textbf{Don Hastily} & \textbf{Remove}\\
Shield (any) & 1 move action &  n/a & 1 move action \\
Padded, leather, hide, studded leather, or chain shirt & 1 minute & 5 rounds & 1 minute \\
Breastplate, scale mail, chainmail, banded mail, or splint mail & 4 minutes & 1 minute & 1 minute \\
Half-plate or full plate & 4 minutes & 4 minutes & 1d4+1 minutes\\
\end{tabular}\\
\(^{1}\) If the character has some help, cut this time in half. A single character doing nothing else can help one or two adjacent characters. Two characters can't help each other don armor at the same time.\\
\(^{2}\) The wearer must have help to don this armor. Without help, it can be donned only hastily.\\
\end{table*}

\section{Special Materials}
	
Weapons and armor can be crafted using materials that possess innate special properties. If you make a suit of armor or weapon out of more than one special material, you get the benefit of only the most prevalent material. However, you can build a double weapon with each head made of a different special material. 
	
Each of the special materials described below has a definite game effect. Some creatures have damage reduction making them resistant to all but a special type of damage, such as that dealt by evil-aligned weapons or bludgeoning weapons. Others are vulnerable to weapons of a particular material. Characters may choose to carry several different types of weapons, depending upon the types of creatures they most commonly encounter. 
	
\textbf{Adamantine}: Mined from rocks that fell from the heavens, this ultrahard metal adds to the quality of a weapon or suit of armor. Weapons fashioned from adamantine have a natural ability to bypass hardness when sundering weapons or attacking objects, ignoring hardness less than 20 (see Additional Rules). Armor made from adamantine grants its wearer damage reduction of 1/--- if it's light armor, 2/--- if it's medium armor, and 3/--- if it's heavy armor. Adamantine is so costly that weapons and armor made from it are always of masterwork quality; the masterwork cost is included in the prices given below. Thus, adamantine weapons and ammunition have a +1 enhancement bonus on attack rolls, and the armor check penalty of adamantine armor is lessened by 1 compared to ordinary armor of its type. Items without metal parts cannot be made from adamantine. An arrow could be made of adamantine, but a quarterstaff could not.
	
Weapons and armor normally made of steel that are made of adamantine have one-third more hit points than normal. Adamantine has 40 hit points per inch of thickness and hardness 20.
			
\begin{table}
\sffamily
 \begin{tabular}{ll}
\textbf{Type of Adamantine Item} & \textbf{Item Cost Modifier}\\
Ammunition & +60 gp per missile \\
Light armor & +5,000 gp\\
Medium armor & +10,000 gp\\
Heavy armor & +15,000 gp\\
Weapon & +3,000 gp\\
 \end{tabular}

\end{table}

		
\textbf{Darkwood}: This rare magic wood is as hard as normal wood but very light. Any wooden or mostly wooden item (such as a bow or spear) made from darkwood is considered a masterwork item and weighs only half as much as a normal wooden item of that type. Items not normally made of wood or only partially of wood (such as a battleaxe or a mace) either cannot be made from darkwood or do not gain any special benefit from being made of darkwood. The armor check penalty of a darkwood shield is lessened by 2 compared to an ordinary shield of its type. To determine the price of a darkwood item, use the original weight but add 10 gp per pound to the price of a masterwork version of that item.
	
Darkwood has 10 hit points per inch of thickness and hardness 5.
	
\textbf{Dragonhide}: Armorsmiths can work with the hides of dragons to produce armor or shields of masterwork quality. One dragon produces enough hide for a single suit of masterwork hide armor for a creature one size category smaller than the dragon. By selecting only choice scales and bits of hide, an armorsmith can produce one suit of masterwork banded mail for a creature two sizes smaller, one suit of masterwork half-plate for a creature three sizes smaller, or one masterwork breastplate or suit of full plate for a creature four sizes smaller. In each case, enough hide is available to produce a light or heavy masterwork shield in addition to the armor, provided that the dragon is Large or larger. If the dragonhide comes from a dragon that had immunity to an energy type, the armor is also immune to that energy type, although this does not confer any protection to the wearer. If the armor or shield is later given the ability to protect the wearer against that energy type, the cost to add such protection is reduced by 25\%.
	
Because dragonhide armor isn't made of metal, druids can wear it without penalty.
	
Dragonhide armor costs twice as much as masterwork armor of that type, but it takes no longer to make than ordinary armor of that type (double all Craft results).
	
Dragonhide has 10 hit points per inch of thickness and hardness 10. The hide of a dragon is typically between 1/2 inch and 1 inch thick.
	
\textbf{Iron, Cold}: This iron, mined deep underground and known for its effectiveness against demons and fey creatures, is forged at a lower temperature to preserve its delicate properties. Weapons made of cold iron cost twice as much to make as their normal counterparts. Also, adding any magical enhancements to a cold iron weapon increases its price by 2,000 gp. This increase is applied the first time the item is enhanced, not once per ability added.
	
Items without metal parts cannot be made from cold iron. An arrow could be made of cold iron, but a quarterstaff could not. A double weapon with one cold iron half costs 50\% more than normal.
	
Cold iron has 30 hit points per inch of thickness and hardness 10.
	
\textbf{Mithral}: Mithral is a very rare silvery, glistening metal that is lighter than steel but just as hard. When worked like steel, it becomes a wonderful material from which to create armor, and is occasionally used for other items as well. Most mithral armors are one category lighter than normal for purposes of movement and other limitations. Heavy armors are treated as medium, and medium armors are treated as light, but light armors are still treated as light. This decrease does not apply to proficiency in wearing the armor. A character wearing mithral full plate must be proficient in wearing heavy armor to avoid adding the armor's check penalty to all his attack rolls and skill checks that involve moving. Spell failure chances for armors and shields made from mithral are decreased by 10\%, maximum Dexterity bonuses are increased by 2, and armor check penalties are decreased by 3 (to a minimum of 0).
	
An item made from mithral weighs half as much as the same item made from other metals. In the case of weapons, this lighter weight does not change a weapon's size category or the ease with which it can be wielded (whether it is light, one-handed, or two-handed). Items not primarily of metal are not meaningfully affected by being partially made of mithral. (A longsword can be a mithral weapon, while a quarterstaff cannot.) Mithral weapons count as silver for the purpose of overcoming damage reduction.
	
Weapons or armors fashioned from mithral are always masterwork items as well; the masterwork cost is included in the prices given below.
	
Mithral has 30 hit points per inch of thickness and hardness 15.
			
% <thead href="gettingStarted.html#dexterity">
\begin{table}
\sffamily
 \begin{tabular}{ll}
\textbf{Type of Mithral Item} & \textbf{Item Cost Modifier}\\
Light armor & +1,000 gp\\
Medium armor & +4,000 gp\\
Heavy armor & +9,000 gp\\
Shield & +1,000 gp\\
Other items & +500 gp/lb.\\
 \end{tabular}
\end{table}
		
\textbf{Silver, Alchemical}: A complex process involving metallurgy and alchemy can bond silver to a weapon made of steel so that it bypasses the damage reduction of creatures such as lycanthropes.
	
On a successful attack with a silvered slashing or piercing weapon, the wielder takes a --1 penalty on the damage roll (with a minimum of 1 point of damage). The alchemical silvering process can't be applied to nonmetal items, and it doesn't work on rare metals such as adamantine, cold iron, and mithral.
	
Alchemical silver has 10 hit points per inch of thickness and hardness 8.
			
\begin{table}
 \sffamily
 \begin{tabularx}{\linewidth}{Xl}
  \textbf{Type of Alchemical Silver Item} & \textbf{Item Cost Modifier} \\
  Ammunition & +2 gp\\
  Light weapon & +20 gp\\
  One-handed weapon, or one head of a double weapon & +90 gp \\
  Two-handed weapon, or both heads of a double weapon & +180 gp 
 \end{tabularx}
\end{table}
	
\section{Goods And Services}
\captionof{table}{Goods and Services}
\begin{xtabular}{lll}
\textit{Adventuring Gear}\\
\textbf{Item} & \textbf{Cost} & \textbf{Weight}\\
Backpack (empty) & 2 gp & 2 lbs. \\
 Barrel (empty) & 2 gp & 30 lbs. \\
 Basket (empty) & 4 sp & 1 lb. \\
 Bedroll & 1 sp & 5 lbs. \\
 Bell & 1 gp & - \\
 Blanket, winter & 5 sp & 3 lbs. \\
 Block and tackle & 5 gp & 5 lbs. \\
 Bottle, glass & 2 gp & 1 lb. \\
 Bucket (empty) & 5 sp & 2 lbs. \\
 Caltrops & 1 gp & 2 lbs. \\
 Candle & 1 cp & - \\
 Canvas (sq. yd.) & 1 sp & 1 lb. \\
 Case, map or scroll & 1 gp & 1/2 lb. \\
 Chain (10 ft.) & 30 gp & 2 lbs. \\
 Chalk, 1 piece & 1 cp & - \\
 Chest (empty) & 2 gp & 25 lbs. \\
 Crowbar & 2 gp & 5 lbs. \\
 Firewood (per day) & 1 cp & 20 lbs. \\
 Fishhook & 1 sp & - \\
 Fishing net, 25 sq. ft. & 4 gp & 5 lbs. \\
 Flask (empty) & 3 cp & 1-1/2 lbs. \\
 Flint and steel & 1 gp & - \\
 Grappling hook & 1 gp & 4 lbs. \\
 Hammer & 5 sp & 2 lbs. \\
 Hourglass & 25 gp & 1 lb. \\
 Ink (1 oz. vial) & 8 gp & - \\
 Inkpen & 1 sp & - \\
 Jug, clay & 3 cp & 9 lbs. \\
 Ladder, 10-foot & 2 sp & 20 lbs. \\
 Lamp, common & 1 sp & 1 lb. \\
 Lantern, bullseye & 12 gp & 3 lbs. \\
 Lantern, hooded & 7 gp & 2 lbs. \\
 Lock  \\
 Simple & 20 gp & 1 lb. \\
 Average & 40 gp & 1 lb. \\
 Good & 80 gp & 1 lb. \\
 Superior & 150 gp & 1 lb. \\
 Manacles & 15 gp & 2 lbs. \\
 Manacles, masterwork & 50 gp & 2 lbs. \\
 Mirror, small steel & 10 gp & 1/2 lb. \\
 Mug/Tankard, clay & 2 cp & 1 lb. \\
 Oil (1-pint flask) & 1 sp & 1 lb. \\
 Paper (sheet) & 4 sp & - \\
 Parchment (sheet) & 2 sp & - \\
 Pick, miner's & 3 gp & 10 lbs. \\
 Pitcher, clay & 2 cp & 5 lbs. \\
 Piton & 1 sp & 1/2 lb. \\
 Pole, 10-foot & 5 cp & 8 lbs. \\
 Pot, iron & 8 sp & 4 lbs. \\
 Pouch, belt (empty) & 1 gp & 1/2 lb. \\
 Ram, portable & 10 gp & 20 lbs. \\
 Rations, trail (per day) & 5 sp & 1 lb. \\
 Rope, hemp (50 ft.) & 1 gp & 10 lbs. \\
 Rope, silk (50 ft.) & 10 gp & 5 lbs. \\
 Sack (empty) & 1 sp & 1/2 lb. \\
 Sealing wax & 1 gp & 1 lb. \\
 Sewing needle & 5 sp & - \\
 Shovel or spade & 2 gp & 8 lbs. \\
 Signal whistle & 8 sp & - \\
 Signet ring & 5 gp & - \\
 Sledge & 1 gp & 10 lbs. \\
 Soap (per lb.) & 5 sp & 1 lb. \\
 Spyglass & 1,000 gp & 1 lb. \\
 Tent & 10 gp & 20 lbs. \\
 Torch & 1 cp & 1 lb. \\
 Vial, ink or potion & 1 gp & - \\
 Water clock & 1,000 gp & 200 lbs. \\
 Waterskin & 1 gp & 4 lbs. \\
 Whetstone & 2 cp & 1 lb.\\
\textit{Special Substances and Items} \\
Item & Cost & Weight\\
Acid (flask) & 10 gp & 1 lb. \\
 Alchemist's fire (flask) & 20 gp & 1 lb. \\
 Antitoxin (vial) & 50 gp & - \\
 Everburning torch & 110 gp & 1 lb. \\
 Holy water (flask) & 25 gp & 1 lb. \\
 Smokestick & 20 gp & 1/2 lb. \\
 Sunrod & 2 gp & 1 lb. \\
 Tanglefoot bag & 50 gp & 4 lbs. \\
 Thunderstone & 30 gp & 1 lb. \\
 Tindertwig & 1 gp & -\\
Tools and Skill Kits  \\
 Item & Cost & Weight\\
Alchemist's lab & 200 gp & 40 lbs. \\
 Artisan's tools & 5 gp & 5 lbs. \\
 Artisan's tools, masterwork & 55 gp & 5 lbs. \\
 Climber's kit & 80 gp & 5 lbs. \\
 Disguise kit & 50 gp & 8 lbs. \\
 Healer's kit & 50 gp & 1 lb. \\
 Holly and mistletoe & - & - \\
 Holy symbol, wooden & 1 gp & - \\
 Holy symbol, silver & 25 gp & 1 lb. \\
 Magnifying glass & 100 gp & - \\
 Musical instrument, common & 5 gp & 3 lbs. \\
 Musical instrument, masterwork & 100 gp & 3 lbs. \\
 Scale, merchant's & 2 gp & 1 lb. \\
 Spell component pouch & 5 gp & 2 lbs. \\
 Spellbook, wizard's (blank) & 15 gp & 3 lbs. \\
 Thieves' tools & 30 gp & 1 lb. \\
 Thieves' tools, masterwork & 100 gp & 2 lbs. \\
 Tool, masterwork & 50 gp & 1 lb.\\
Clothing \\
 Item & Cost & Weight\\
Artisan's outfit & 1 gp & 4 lbs. \\
 Cleric's vestments & 5 gp & 6 lbs. \\
 Cold-weather outfit & 8 gp & 7 lbs. \\
 Courtier's outfit & 30 gp & 6 lbs. \\
 Entertainer's outfit & 3 gp & 4 lbs. \\
 Explorer's outfit & 10 gp & 8 lbs. \\
 Monk's outfit & 5 gp & 2 lbs. \\
 Noble's outfit & 75 gp & 10 lbs. \\
 Peasant's outfit & 1 sp & 2 lbs. \\
 Royal outfit & 200 gp & 15 lbs. \\
 Scholar's outfit & 5 gp & 6 lbs. \\
 Traveler's outfit & 1 gp & 5 lbs.\\
\textit{Food, Drink and Lodging}  \\
 Item & Cost & Weight\\
Ale  \\
 Gallon & 2 sp & 8 lbs. \\
 Mug & 4 cp & 1 lb. \\
 Banquet (per person) & 10 gp & - \\
 Bread, loaf of & 2 cp & 1/2 lb. \\
 Cheese, hunk of & 1 sp & 1/2 lb. \\
 Inn stay (per day)  \\
 Good & 2 gp & - \\
 Common & 5 sp & - \\
 Poor & 2 sp & - \\
 Meals (per day)  \\
 Good & 5 sp & - \\
 Common & 3 sp & - \\
 Poor & 1 sp & - \\
 Meat, chunk of & 3 sp & 1/2 lb. \\
 Wine  \\
 Common (pitcher) & 2 sp & 6 lbs. \\
 Fine (bottle) & 10 gp & 1-1/2 lbs.\\
Mounts and Related Gear  \\
 Item & Cost & Weight\\
Barding  \\
 Medium creature & $\times$2 & $\times$1 \\
 Large creature & $\times$4 & $\times$2 \\
 Bit and bridle & 2 gp & 1 lb. \\
 Dog, guard & 25 gp & - \\
 Dog, riding & 150 gp & - \\
 Donkey or mule & 8 gp & - \\
 Feed (per day) & 5 cp & 10 lbs. \\
 Horse  \\
 Horse, heavy & 200 gp & - \\
 Horse, heavy, (combat trained) & 300 gp & - \\
 Horse, light & 75 gp & - \\
 Horse, light, (combat trained) & 110 gp & - \\
 Pony, 30 gp & - \\
 Pony (combat trained) & 45 gp & - \\
 Saddle \\
 Military & 20 gp & 30 lbs. \\
 Pack & 5 gp & 15 lbs. \\
 Riding & 10 gp & 25 lbs. \\
 Saddle, Exotic  \\
 Military & 60 gp & 40 lbs. \\
 Pack & 15 gp & 20 lbs. \\
 Riding & 30 gp & 30 lbs. \\
 Saddlebags & 4 gp & 8 lbs. \\
 Stabling (per day) & 5 sp & -\\
Transport  \\
 Item & Cost & Weight\\
Carriage & 100 gp & 600 lbs. \\
 Cart & 15 gp & 200 lbs. \\
 Galley & 30,000 gp & - \\
 Keelboat & 3,000 gp & - \\
 Longship & 10,000 gp & - \\
 Rowboat & 50 gp & 100 lbs. \\
 Oar & 2 gp & 10 lbs. \\
 Sailing ship & 10,000 gp & - \\
 Sled & 20 gp & 300 lbs. \\
 Wagon & 35 gp & 400 lbs. \\
 Warship & 25,000 gp & -\\
 \end{xtabular}
 \begin{xtabular}{ll}
\multicolumn{2}{l}{\textit{Spellcasting and Services}} \\
 Service & Cost\\
Coach cab & 3 cp per mile\\
Hireling, trained & 3 sp per day\\
Hireling, untrained & 1 sp per day\\
Messenger & 2 cp per mile\\
Road or gate toll & 1 cp\\
Ship's passage & 1 sp per mile\\
Spellcasting & Caster level $\times$ spell level $\times$ 10 gp\\
\end{xtabular}\\
--- No weight, or no weight worth noting.\\
\(^{1}\) These items weigh one-quarter this amount when made for Small characters. Containers for Small characters also carry one-quarter the normal amount.\\
\(^{2}\) Relative to similar armor made for a Medium humanoid.\\
\(^{3}\) See spell description for additional costs. If the additional costs put the spell's total cost above 3,000 gp, that spell is not generally available. Use a spell level of \mbox{$\frac12$} for 0-level spells to calculate the cost.\\

%\end{table}
	
Beyond armor and weapons, a character can carry a whole variety of gear, from rations (to sustain him on long travels), to rope (which is useful in countless circumstances). Most of the common gear carried by adventurers is summarized on Table: Goods and Services.
	
\subsection{Adventuring Gear}

		
Some of the pieces of adventuring gear found on Table: Goods and Services are described below, along with any special benefits they confer on the user (\texttt{{}"{}}you\texttt{{}"{}}).
		
\textbf{Caltrops}: A caltrop is a four-pronged metal spike crafted so that one prong faces up no matter how the caltrop comes to rest. You scatter caltrops on the ground in the hope that your enemies step on them or are at least forced to slow down to avoid them. One 2-pound bag of caltrops covers an area 5 feet square.
		
Each time a creature moves into an area covered by caltrops (or spends a round fighting while standing in such an area), it runs the risk of stepping on one. Make an attack roll for the caltrops (base attack bonus +0) against the creature. For this attack, the creature's shield, armor, and deflection bonuses do not count. If the creature is wearing shoes or other footwear, it gets a +2 armor bonus to AC. If the attack succeeds, the creature has stepped on a caltrop. The caltrop deals 1 point of damage, and the creature's speed is reduced by half because its foot is wounded. This movement penalty lasts for 24 hours, until the creature is successfully treated with a DC 15 Heal check, or until it receives at least 1 point of magical healing. A charging or running creature must immediately stop if it steps on a caltrop. Any creature moving at half speed or slower can pick its way through a bed of caltrops with no trouble.
		
Caltrops may not work against unusual opponents.
		
\textbf{Candle}: A candle dimly illuminates a small area, increasing the light level in a 5-foot radius by one step (darkness becomes dim light and dim light becomes normal light). A candle cannot increase the light level above normal light. A candle burns for 1 hour.
		
\textbf{Chain}: Chain has hardness 10 and 5 hit points. It can be burst with a DC 26 Strength check.
		
\textbf{Crowbar}: A crowbar grants a +2 circumstance bonus on Strength checks made to force open a door or chest. If used in combat, treat a crowbar as a one-handed improvised weapon that deals bludgeoning damage equal to that of a club of its size.
		
\textbf{Flint and Steel}: Lighting a torch with flint and steel is a full-round action, and lighting any other fire with them takes at least that long.
		
\textbf{Grappling Hook}: Throwing a grappling hook requires a ranged attack roll, treating the hook as a thrown weapon with a range increment of 10 feet. Objects with ample places to catch the hook are AC 5.
		
\textbf{Hammer}: If a hammer is used in combat, treat it as a one-handed improvised weapon that deals bludgeoning damage equal to that of a spiked gauntlet of its size.
		
\textbf{Ink}: Ink in colors other than black costs twice as much.
		
\textbf{Jug, Clay}: This basic jug is fitted with a stopper and holds 1 gallon of liquid.
		
\textbf{Lamp, Common}: A lamp illuminates a small area, providing normal light in a 15-foot radius and increasing the light level by one step for an additional 15 feet beyond that area (darkness becomes dim light and dim light becomes normal light). A lamp does not increase the light level in normal light or bright light. A lamp burns for 6 hours on one pint of oil. You can carry a lamp in one hand.
		
\textbf{ Lantern, Bullseye}: A bullseye lantern provides normal light in a 60-foot cone and increases the light level by one step in the area beyond that, out to a 120-foot cone (darkness becomes dim light and dim light becomes normal light). A bullseye lantern does not increase the light level in normal light or bright light. A lantern burns for 6 hours on one pint of oil. You can carry a lantern in one hand.
		
\textbf{Lantern, Hooded}: A hooded lantern sheds normal light in a 30-foot radius and increases the light level by one step for an additional 30 feet beyond that area (darkness becomes dim light and dim light becomes normal light). A hooded lantern does not increase the light level in normal light or bright light. A lantern burns for 6 hours on one pint of oil. You can carry a lantern in one hand.
		
\textbf{Lock}: The DC to open a lock with the Disable Device skill depends on the lock's quality: simple (DC 20), average (DC 25), good (DC 30), or superior (DC 40).
		
\textbf{Manacles, Standard and Masterwork}: Manacles can bind a Medium creature. A manacled creature can use the Escape Artist skill to slip free (DC 30, or DC 35 for masterwork manacles). Breaking the manacles requires a Strength check (DC 26, or DC 28 for masterwork manacles). Manacles have hardness 10 and 10 hit points.
		
Most manacles have locks; add the cost of the lock you want to the cost of the manacles.
		
For the same cost, you can buy manacles for a Small creature. For a Large creature, manacles cost 10 times the indicated amount, and for a Huge creature, 100 times the indicated amount. Gargantuan, Colossal, Tiny, Diminutive, and Fine creatures can be held only by specially made manacles, which cost at least 100 times the indicated amount.
		
\textbf{Oil}: A pint of oil burns for 6 hours in a lantern or lamp. You can also use a flask of oil as a splash weapon. Use the rules for alchemist's fire (see Special Substances and Items on Table: Goods and Services), except that it takes a full-round action to prepare a flask with a fuse. Once it is thrown, there is a 50\% chance of the flask igniting successfully.
		
You can pour a pint of oil on the ground to cover an area 5 feet square, provided that the surface is smooth. If lit, the oil burns for 2 rounds and deals 1d3 points of fire damage to each creature in the area.
		
\textbf{Pick, Miner's}: If a miner's pick is used in combat, treat it as a two-handed improvised weapon that deals piercing damage equal to that of a heavy pick of its size. 
		
\textbf{Ram, Portable}: This iron-shod wooden beam gives you a +2 circumstance bonus on Strength checks made to break open a door and allows a second person to help, automatically increasing your bonus by 2.
		
\textbf{Rope, Hemp}: This rope has 2 hit points and can be burst with a DC 23 Strength check.
		
\textbf{Rope, Silk}: This rope has 4 hit points and can be burst with a DC 24 Strength check.
		
\textbf{Shovel}: If a shovel is used in combat, treat it as a one-handed improvised weapon that deals bludgeoning damage equal to that of a club of its size.
		
\textbf{Spyglass}: Objects viewed through a spyglass are magnified to twice their size. Characters using a spyglass take a --1 penalty on Perception skill checks per 20 feet of distance to the target, if the target is visible.
		
\textbf{Torch}: A torch burns for 1 hour, shedding normal light in a 20-foot radius and increasing the light level by one step for an additional 20 feet beyond that area (darkness becomes dim light and dim light becomes normal light). A torch does not increase the light level in normal light or bright light. If a torch is used in combat, treat it as a one-handed improvised weapon that deals bludgeoning damage equal to that of a gauntlet of its size, plus 1 point of fire damage.
		
\textbf{Vial}: A vial is made out of glass or steel and holds 1 ounce of liquid.
		
\textbf{Water Clock}: This large, bulky contrivance gives the time accurately to within half an hour per day since it was last set. It requires a source of water, and it must be kept still because it marks time by the regulated flow of droplets of water.
	
\subsection{Special Substances and Items}

		
Any of these substances except for the everburning torch and holy water can be made by a character with the Craft (alchemy) skill.
		
\textbf{Acid}: You can throw a flask of acid as a splash weapon. Treat this attack as a ranged touch attack with a range increment of 10 feet. A direct hit deals 1d6 points of acid damage. Every creature within 5 feet of the point where the acid hits takes 1 point of acid damage from the splash.
		
\textbf{Alchemist's Fire}: You can throw a flask of alchemist's fire as a splash weapon. Treat this attack as a ranged touch attack with a range increment of 10 feet.
		
A direct hit deals 1d6 points of fire damage. Every creature within 5 feet of the point where the flask hits takes 1 point of fire damage from the splash. On the round following a direct hit, the target takes an additional 1d6 points of damage. If desired, the target can use a full-round action to attempt to extinguish the flames before taking this additional damage. Extinguishing the flames requires a DC 15 Reflex save. Rolling on the ground provides the target a +2 bonus on the save. Leaping into a lake or magically extinguishing the flames automatically smothers the fire.
		
\textbf{Antitoxin}: If you drink a vial of antitoxin, you get a +5 alchemical bonus on Fortitude saving throws against poison for 1 hour.
		
\textbf{Everburning Torch}: This otherwise normal torch has a \textit{continual flame} spell cast on it. This causes it to shed light like an ordinary torch, but it does not emit heat or deal fire damage if used as a weapon. 
		
\textbf{Holy Water}: Holy water damages undead creatures and evil outsiders almost as if it were acid. A flask of holy water can be thrown as a splash weapon.
		
Treat this attack as a ranged touch attack with a range increment of 10 feet. A flask breaks if thrown against the body of a corporeal creature, but to use it against an incorporeal creature, you must open the flask and pour the holy water out onto the target. Thus, you can douse an incorporeal creature with holy water only if you are adjacent to it. Doing so is a ranged touch attack that does not provoke attacks of opportunity.
		
A direct hit by a flask of holy water deals 2d4 points of damage to an undead creature or an evil outsider. Each such creature within 5 feet of the point where the flask hits takes 1 point of damage from the splash.
		
Temples to good deities sell holy water at cost (making no profit). Holy water is made using the \textit{bless water }spell.
		
\textbf{Smokestick}: This alchemically treated wooden stick instantly creates thick, opaque smoke when burned. The smoke fills a 10-foot cube (treat the effect as a \textit{fog cloud }spell, except that a moderate or stronger wind dissipates the smoke in 1 round). The stick is consumed after 1 round, and the smoke dissipates naturally after 1 minute.
		
\textbf{Sunrod}: This 1-foot-long, gold-tipped, iron rod glows brightly when struck as a standard action. It sheds normal light in a 30-foot radius and increases the light level by one step for an additional 30 feet beyond that area (darkness becomes dim light and dim light becomes normal light). A sunrod does not increase the light level in normal light or bright light. It glows for 6 hours, after which the gold tip is burned out and worthless.
		
\textbf{Tanglefoot Bag}: A tanglefoot bag is a small sack filled with tar, resin, and other sticky substances. When you throw a tanglefoot bag at a creature (as a ranged touch attack with a range increment of 10 feet), the bag comes apart and goo bursts out, entangling the target and then becoming tough and resilient upon exposure to air. An entangled creature takes a --2 penalty on attack rolls and a --4 penalty to Dexterity and must make a DC 15 Reflex save or be glued to the floor, unable to move. Even on a successful save, it can move only at half speed. Huge or larger creatures are unaffected by a tanglefoot bag. A flying creature is not stuck to the floor, but it must make a DC 15 Reflex save or be unable to fly (assuming it uses its wings to fly) and fall to the ground. A tanglefoot bag does not function underwater.
		
A creature that is glued to the floor (or unable to fly) can break free by making a DC 17 Strength check or by dealing 15 points of damage to the goo with a slashing weapon. A creature trying to scrape goo off itself, or another creature assisting, does not need to make an attack roll; hitting the goo is automatic, after which the creature that hit makes a damage roll to see how much of the goo was scraped off. Once free, the creature can move (including flying) at half speed. If the entangled creature attempts to cast a spell, it must make concentration check with a DC of 15 + the spell's level or be unable to cast the spell. The goo becomes brittle and fragile after 2d4 rounds, cracking apart and losing its effectiveness. An application of \textit{universal solvent} to a stuck creature dissolves the alchemical goo immediately.
		
\textbf{Thunderstone}: You can throw this stone as a ranged attack with a range increment of 20 feet. When it strikes a hard surface (or is struck hard), it creates a deafening bang that is treated as a sonic attack. Each creature within a 10-foot-radius spread must make a DC 15 Fortitude save or be deafened for 1 hour. A deafened creature, in addition to the obvious effects, takes a --4 penalty on initiative and has a 20\% chance to miscast and lose any spell with a verbal component that it tries to cast.
		
Since you don't need to hit a specific target, you can simply aim at a particular 5-foot square. Treat the target square as AC 5.
		
\textbf{Tindertwig}: The alchemical substance on the end of this small, wooden stick ignites when struck against a rough surface. Creating a flame with a tindertwig is much faster than creating a flame with flint and steel (or a magnifying glass) and tinder. Lighting a torch with a tindertwig is a standard action (rather than a full-round action), and lighting any other fire with one is at least a standard action.
	
\subsection{Tools and Skill Kits}

		
These items are particularly useful to characters with certain skills and class abilities.
		
\textbf{Alchemist's Lab}: This lab is used for making alchemical items, and provides a +2 circumstance bonus on Craft (alchemy) checks. It has no bearing on the costs related to the Craft (alchemy) skill. Without this lab, a character with the Craft (alchemy) skill is assumed to have enough tools to use the skill but not enough to get the +2 bonus that the lab provides.
		
\textbf{Artisan's Tools}: These special tools include the items needed to pursue any craft. Without them, you have to use improvised tools (--2 penalty on Craft checks), if you can do the job at all.
		
\textbf{Artisan's Tools, Masterwork}: These tools serve the same purpose as artisan's tools, but masterwork artisan's tools are the perfect tools for the job, so you get a +2 circumstance bonus on Craft checks made with them.
		
\textbf{Climber's Kit}: These crampons, pitons, ropes, and tools give you a +2 circumstance bonus on Climb checks.
		
\textbf{Disguise Kit}: The kit is the perfect tool for disguise and provides a +2 circumstance bonus on Disguise checks. A disguise kit is exhausted after 10 uses.
		
\textbf{Healer's Kit}: This collection of bandages and herbs provides a +2 circumstance bonus on Heal checks. A healer's kit is exhausted after 10 uses.
		
\textbf{Holly and Mistletoe}: Druids commonly use these plants as divine focuses when casting spells.
		
\textbf{Holy Symbol, Silver or Wooden}: A holy symbol focuses positive energy and is used by good clerics and paladins (or by neutral clerics who want to cast good spells or channel positive energy). Each religion has its own holy symbol.
		
\textit{Unholy Symbols}: An unholy symbol is like a holy symbol except that it focuses negative energy and is used by evil clerics (or by neutral clerics who want to cast evil spells or channel negative energy).
		
\textbf{Magnifying Glass}: This simple lens allows a closer look at small objects. It is also useful as a substitute for flint and steel when starting fires. Lighting a fire with a magnifying glass requires bright light, such as sunlight to focus, tinder to ignite, and at least a full-round action. A magnifying glass grants a +2 circumstance bonus on Appraise checks involving any item that is small or highly detailed.
		
\textbf{Musical Instrument, Common or Masterwork}: A masterwork instrument grants a +2 circumstance bonus on Perform checks involving its use.
		
\textbf{Scale, Merchant's}: A merchant's scale grants a +2 circumstance bonus on Appraise checks involving items that are valued by weight, including anything made of precious metals.
		
\textbf{Spell Component Pouch}: A spellcaster with a spell component pouch is assumed to have all the material components and focuses needed for spellcasting, except for those components that have a specific cost, divine focuses, and focuses that wouldn't fit in a pouch.
		
\textbf{Spellbook, Wizard's}: A spellbook has 100 pages of parchment, and each spell takes up one page per spell level (one page each for 0-level spells).
		
\textbf{Thieves' Tools}: This kit contains lockpicks and other tools you need to use the Disable Device skill. Without these tools, you must use improvised tools, and you take a --2 circumstance penalty on Disable Device checks.
		
\textbf{Thieves' Tools, Masterwork}: This kit contains extra tools and tools of better make, which grant a +2 circumstance bonus on Disable Device checks.
		
\textbf{Tool, Masterwork}: This well-made item is the perfect tool for the job. It grants a +2 circumstance bonus on a related skill check (if any). Bonuses provided by multiple masterwork items do not stack.
	
\subsection{Clothing}

		
All characters begin play with one outfit, valued at 10 gp or less. Additional outfits can be purchased normally.
		
\textbf{Artisan's Outfit}: This outfit includes a shirt with buttons, a skirt or pants with a drawstring, shoes, and perhaps a cap or hat. It may also include a belt or a leather or cloth apron for carrying tools.
		
\textbf{Cleric's Vestments}: These clothes are for performing priestly functions, not for adventuring. Cleric's vestments typically include a cassock, stole, and surplice. 
		
\textbf{Cold-Weather Outfit}: This outfit includes a wool coat, linen shirt, wool cap, heavy cloak, thick pants or skirt, and boots. This outfit grants a +5 circumstance bonus on Fortitude saving throws against exposure to cold weather.
		
\textbf{Courtier's Outfit}: This outfit includes fancy, tailored clothes in whatever fashion happens to be the current style in the courts of the nobles. Anyone trying to influence nobles or courtiers while wearing street dress will have a hard time of it (--2 penalty on Charisma-based skill checks to influence such individuals). If you wear this outfit without jewelry (costing an additional 50 gp), you look like an out-of-place commoner.
		
\textbf{Entertainer's Outfit}: This set of flashy---perhaps even gaudy---clothes is for entertaining. While the outfit looks whimsical, its practical design lets you tumble, dance, walk a tightrope, or just run (if the audience turns ugly).
		
\textbf{Explorer's Outfit}: This set of clothes is for someone who never knows what to expect. It includes sturdy boots, leather breeches or a skirt, a belt, a shirt (perhaps with a vest or jacket), gloves, and a cloak. Rather than a leather skirt, a leather overtunic may be worn over a cloth skirt. The clothes have plenty of pockets (especially the cloak). The outfit also includes any extra accessories you might need, such as a scarf or a wide-brimmed hat.
		
\textbf{Monk's Outfit}: This simple outfit includes sandals, loose breeches, and a loose shirt, and is bound together with sashes. The outfit is designed to give you maximum mobility, and it's made of high-quality fabric. You can conceal small weapons in pockets hidden in the folds, and the sashes are strong enough to serve as short ropes.
		
\textbf{Noble's Outfit}: These clothes are designed specifically to be expensive and gaudy. Precious metals and gems are worked into the clothing. A would-be noble also needs a signet ring and jewelry (worth at least 100 gp) to accessorize this outfit.
		
\textbf{Peasant's Outfit}: This set of clothes consists of a loose shirt and baggy breeches, or a loose shirt and skirt or overdress. Cloth wrappings are used for shoes.
		
\textbf{Royal Outfit}: This is just the clothing, not the royal scepter, crown, ring, and other accoutrements. Royal clothes are ostentatious, with gems, gold, silk, and fur in abundance.
		
\textbf{Scholar's Outfit}: Perfect for a scholar, this outfit includes a robe, a belt, a cap, soft shoes, and possibly a cloak.
		
\textbf{Traveler's Outfit}: This set of clothes consists of boots, a wool skirt or breeches, a sturdy belt, a shirt (perhaps with a vest or jacket), and an ample cloak with a hood.
	
\subsection{Food, Drink, and Lodging}

		
These prices are for meals and accommodations at establishments in an average city.
		
\textbf{Inn}: Poor accommodations at an inn amount to a place on the floor near the hearth. Common accommodations consist of a place on a raised, heated floor and the use of a blanket and a pillow. Good accommodations consist of a small, private room with one bed, some amenities, and a covered chamber pot in the corner.
		
\textbf{Meals}: Poor meals might be composed of bread, baked turnips, onions, and water. Common meals might consist of bread, chicken stew, carrots, and watered-down ale or wine. Good meals might be composed of bread and pastries, beef, peas, and ale or wine.
	
\subsection{Mounts and Related Gear}

		
These are the common mounts available in most cities. Some markets might have additional creatures available, such as camels or even griffons, depending on the terrain. Such additional choices are up to GM discretion.
		
\textbf{Barding, Medium Creature and Large Creature}: Barding is a type of armor that covers the head, neck, chest, body, and possibly legs of a horse or other mount. Barding made of medium or heavy armor provides better protection than light barding, but at the expense of speed. Barding can be made of any of the armor types found on Table: Armor and Shields.
		
Armor for a horse (a Large nonhumanoid creature) costs four times as much as human armor (a Medium humanoid creature) and also weighs twice as much (see Table: Armor for Unusual Creatures). If the barding is for a pony or other Medium mount, the cost is only double, and the weight is the same as for Medium armor worn by a humanoid. Medium or heavy barding slows a mount that wears it, as shown on the table below.
		
Flying mounts can't fly in medium or heavy barding.
		
Removing and fitting barding takes five times as long as the figures given on Table: Donning Armor. A barded animal cannot be used to carry any load other than a rider and normal saddlebags.

\begin{table}
 \sffamily
 \begin{tabular}{llll}
\textbf{Barding} & \multicolumn{3}{c}{\textbf{Modifier}}\\
                 & \textbf{(40 ft.)} & \textbf{(50 ft.)} & \textbf{(60 ft.)} \\
Medium           & 30 ft. & 35 ft. & 40 ft.\\
Heavy            & 30 ft.* & 35 ft.* & 40 ft.*  \\
 \end{tabular}
 * A mount wearing heavy armor moves at only triple its normal speed when running instead of quadruple.
\end{table}
\textbf{Dog, Riding}: This Medium dog is specially trained to carry a Small humanoid rider. It is brave in combat like a combat-trained horse. Due to its smaller stature, you take no damage when you fall from a riding dog.
		
\textbf{Donkey or Mule}: Donkeys and mules are stolid in the face of danger, hardy, surefooted, and capable of carrying heavy loads over vast distances. Unlike a horse, a donkey or a mule is willing (though not eager) to enter dungeons and other strange or threatening places.
		
\textbf{Feed}: Horses, donkeys, mules, and ponies can graze to sustain themselves, but providing feed for them is better. If you have a riding dog, you have to feed it meat.
		
\textbf{Horse}: A horse is suitable as a mount for a human, dwarf, elf, half-elf, or half-orc. A pony is smaller than a horse and is a suitable mount for a gnome or halfling.
		
A war-trained horse can be ridden into combat without danger. See the Handle Animal skill for a list of tricks known by horses and ponies with combat training.
		
\textbf{Saddle, Exotic}: An exotic saddle is designed for an unusual mount. Exotic saddles come in military, pack, and riding styles.
		
\textbf{Saddle, Military}: This saddle braces the rider, providing a +2 circumstance bonus on Ride checks related to staying in the saddle. If you're knocked unconscious while in a military saddle, you have a 75\% chance to stay in the saddle.
		
\textbf{Saddle, Pack}: A pack saddle holds gear and supplies, but not a rider. It holds as much gear as the mount can carry.
		
\textbf{Saddle, Riding}: If you are knocked unconscious while in a riding saddle, you have a 50\% chance to stay in the saddle.
	
\subsection{Transport}

		
The prices listed are to purchase the vehicle. These prices generally exclude crew or animals.
		
\textbf{Carriage}: This four-wheeled vehicle can transport as many as four people within an enclosed cab, plus two drivers. In general, two horses (or other beasts of burden) draw it. A carriage comes with the harness needed to pull it.
		
\textbf{Cart}: This two-wheeled vehicle can be drawn by a single horse (or other beast of burden). It comes with a harness.
		
\textbf{Galley}: This three-masted ship has 70 oars on either side and requires a total crew of 200. A galley is 130 feet long and 20 feet wide, and can carry 150 tons of cargo or 250 soldiers. For 8,000 gp more, it can be fitted with a ram and castles with firing platforms fore, aft, and amidships. This ship cannot make sea voyages and sticks to the coast. It moves about 4 miles per hour when being rowed or under sail.
		
\textbf{Keelboat}: This 50- to 75-foot-long ship is 15 to 20 feet wide and has a few oars to supplement its single mast with a square sail. It has a crew of 8 to 15 and can carry 40 to 50 tons of cargo or 100 soldiers. It can make sea voyages, as well as sail down rivers (thanks to its flat bottom). It moves about 1 mile per hour.
		
\textbf{Longship}: This 75-foot-long ship with 40 oars requires a total crew of 50. It has a single mast and a square sail, and it can carry 50 tons of cargo or 120 soldiers. A longship can make sea voyages. It moves about 3 miles per hour when being rowed or under sail.
		
\textbf{Rowboat}: This 8- to 12-foot-long boat with two oars holds two or three Medium passengers. It moves about 1-1/2 miles per hour.
		
\textbf{Sailing Ship}: This large, seaworthy ship is 75 to 90 feet long and 20 feet wide, and has a crew of 20. It can carry 150 tons of cargo. It has square sails on its two masts and can make sea voyages. It moves about 2 miles per hour.
		
\textbf{Sled}: This is a wagon on runners for snow and ice travel. In general, two horses (or other beasts of burden) draw it. A sled comes with the harness needed to pull it.
		
\textbf{Wagon}: A four-wheeled, open vehicle for transporting heavy loads. Two horses (or other beasts of burden) must draw it. A wagon comes with the harness needed to pull it.
		
\textbf{Warship}: This 100-foot-long ship has a single mast, although oars can also propel it. It has a crew of 60 to 80 rowers. This ship can carry 160 soldiers, but not for long distances, since there isn't room for supplies to support that many people. The warship cannot make sea voyages and sticks to the coast. It is not used for cargo. It moves about 2-1/2 miles per hour when being rowed or under sail.
	
\subsection{Spellcasting and Services}

		
Sometimes the best solution to a problem is to hire someone else to take care of it.
		
\textbf{Coach Cab}: The price given is for a ride in a coach that transports people (and light cargo) between towns. For a ride in a cab that transports passengers within a city, 1 copper piece usually takes you anywhere you need to go.
		
\textbf{Hireling, Trained}: The amount given is the typical daily wage for mercenary warriors, masons, craftsmen, cooks, scribes, teamsters, and other trained hirelings. This value represents a minimum wage; many such hirelings require significantly higher pay.
		
\textbf{Hireling, Untrained}: The amount shown is the typical daily wage for laborers, maids, and other menial workers.
		
\textbf{Messenger}: This includes horse-riding messengers and runners. Those willing to carry a message to a place they were going anyway may ask for only half the indicated amount.
		
\textbf{Road or Gate Toll}: A toll is sometimes charged to cross a well-kept and well-guarded road to pay for patrols on it and for its upkeep. Occasionally, a large, walled city charges a toll to enter or exit (or sometimes just to enter).
		
\textbf{Ship's Passage}: Most ships do not specialize in passengers, but many have the capability to take a few along when transporting cargo. Double the given cost for creatures larger than Medium or creatures that are otherwise difficult to bring aboard a ship.
		
\textbf{Spellcasting}: The indicated amount is how much it costs to get a spellcaster to cast a spell for you. This cost assumes that you can go to the spellcaster and have the spell cast at his convenience (generally at least 24 hours later, so that the spellcaster has time to prepare the spell in question). If you want to bring the spellcaster somewhere to cast a spell you need to negotiate with him, and the default answer is no.
		
The cost given is for any spell that does not require a costly material component. If the spell includes a material component, add the cost of that component to the cost of the spell. If the spell has a focus component (other than a divine focus), add 1/10 the cost of that focus to the cost of the spell.
		
Furthermore, if a spell has dangerous consequences, the spellcaster will certainly require proof that you can and will pay for dealing with any such consequences (that is, assuming that the spellcaster even agrees to cast such a spell, which isn't certain). In the case of spells that transport the caster and characters over a distance, you will likely have to pay for two castings of the spell, even if you aren't returning with the caster.
		
In addition, not every town or village has a spellcaster of sufficient level to cast any spell. In general, you must travel to a small town (or larger settlement) to be reasonably assured of finding a spellcaster capable of casting 1st-level spells, a large town for 2nd-level spells, a small city for 3rd- or 4th-level spells, a large city for 5th- or 6th-level spells, and a metropolis for 7th- or 8th-level spells. Even a metropolis isn't guaranteed to have a local spellcaster able to cast 9th-level spells.
	
