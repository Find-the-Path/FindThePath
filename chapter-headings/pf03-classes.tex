% Generated by html2tex: Version 2.7 of February 6, 2012.
% Written by F.J. Faase.  http://www.iwriteiam.nl/
		
\chapter{Classes}

\label{f0}				
\textbf{Barbarian}: The barbarian is a brutal berserker from beyond the edge of civilized lands.
				
\textbf{Bard}: The bard uses skill and spell alike to bolster his allies, confound his enemies, and build upon his fame.
				
\textbf{Cleric}: A devout follower of a deity, the cleric can heal wounds, raise the dead, and call down the wrath of the gods.
				
\textbf{Druid}: The druid is a worshiper of all things natural---a spellcaster, a friend to animals, and a skilled shapechanger.
				
\textbf{Fighter}: Brave and stalwart, the fighter is a master of all manner of arms and armor.
				
\textbf{Monk}: A student of martial arts, the monk trains his body to be his greatest weapon and defense.
				
\textbf{Paladin}: The paladin is the knight in shining armor, a devoted follower of law and good.
				
\textbf{Ranger}: A tracker and hunter, the ranger is a creature of the wild and of tracking down his favored foes.
				
\textbf{Rogue}: The rogue is a thief and a scout, an opportunist capable of delivering brutal strikes against unwary foes.
				
\textbf{Sorcerer}: The spellcasting sorcerer is born with an innate knack for magic and has strange, eldritch powers.
				
\textbf{Wizard}: The wizard masters magic through constant study that gives him incredible magical power.
				
\section{Character Advancement}

				
As player characters overcome challenges, they gain experience points. As these points accumulate, PCs advance in level and power. The rate of this advancement depends on the type of game that your group wants to play. Some prefer a fast-paced game, where characters gain levels every few sessions, while others prefer a game where advancement occurs less frequently. In the end, it is up to your group to decide what rate fits you best. Characters advance in level according to Table: Character Advancement and Level-Dependent Bonuses.
				
\subsection{Advancing Your Character}

				
A character advances in level as soon as he earns enough experience points to do so---typically, this occurs at the end of a game session, when your GM hands out that session's experience point awards.
				
The process of advancing a character works in much the same way as generating a character, except that your ability scores, race, and previous choices concerning class, skills, and feats cannot be changed. Adding a level generally gives you new abilities, additional skill points to spend, more hit points, and possibly an ability score increase or additional feat (see Table: Character Advancement and Level-Dependent Bonuses). Over time, as your character rises to higher levels, he becomes a truly powerful force in the game world, capable of ruling nations or bringing them to their knees.
				
When adding new levels of an existing class or adding levels of a new class (see Multiclassing, below), make sure to take the following steps in order. First, select your new class level. You must be able to qualify for this level before any of the following adjustments are made. Second, apply any ability score increases due to gaining a level. Third, integrate all of the level's class abilities and then roll for additional hit points. Finally, add new skills and feats. For more information on when you gain new feats and ability score increases, see Table: Character Advancement and Level-Dependent Bonuses.
% <div class="table">


\begin{table}[]
\sffamily
\caption{Character Advancement and Level-Dependent Bonuses}
\begin{tabular}{llllll}
      & \multicolumn{3}{c}{\textbf{Experience Point Total}} &       & \textbf{Ability}  \\
\textbf{Level} & \textbf{Slow} & \textbf{Medium} & \textbf{Fast} & \textbf{Feats} & \textbf{Score} \\
1st  & --         & --         & --         & 1st  & --   \\
2nd  & 3,000     & 2,000     & 1,300     & --    & --   \\
3rd  & 7,500     & 5,000     & 3,300     & 2nd  & --   \\
4th  & 14,000    & 9,000     & 6,000     & --    & 1st \\
5th  & 23,000    & 15,000    & 10,000    & 3rd  & --   \\
6th  & 35,000    & 23,000    & 15,000    & --    & --   \\
7th  & 53,000    & 35,000    & 23,000    & 4th  & --   \\
8th  & 77,000    & 51,000    & 34,000    & --    & 2nd \\
9th  & 115,000   & 75,000    & 50,000    & 5th  & --   \\
10th & 160,000   & 105,000   & 71,000    & --    & --   \\
11th & 235,000   & 155,000   & 105,000   & 6th  & --   \\
12th & 330,000   & 220,000   & 145,000   & --    & 3rd \\
13th & 475,000   & 315,000   & 210,000   & 7th  & --   \\
14th & 665,000   & 445,000   & 295,000   & --    & --   \\
15th & 955,000   & 635,000   & 425,000   & 8th  & --   \\
16th & 1,350,000 & 890,000   & 600,000   & --    & 4th \\
17th & 1,900,000 & 1,300,000 & 850,000   & 9th  & --   \\
18th & 2,700,000 & 1,800,000 & 1,200,000 & --    & --   \\
19th & 3,850,000 & 2,550,000 & 1,700,000 & 10th & --   \\
20th & 5,350,000 & 3,600,000 & 2,400,000 & --    & 5th
\end{tabular}
\end{table}

				
\subsection{Multiclassing}

Instead of gaining the abilities granted by the next level in your character's current class, he can instead gain the 1st-level abilities of a new class, adding all of those abilities to his existing ones. This is known as "multiclassing."
				
For example, let's say a 5th-level fighter decides to dabble in the arcane arts, and adds one level of wizard when he advances to 6th level. Such a character would have the powers and abilities of both a 5th-level fighter and a 1st-level wizard, but would still be considered a 6th-level character. (His class levels would be 5th and 1st, but his total character level is 6th.) He keeps all of his bonus feats gained from 5 levels of fighter, but can now also cast 1st-level spells and picks an arcane school. He adds all of the hit points, base attack bonuses, and saving throw bonuses from a 1st-level wizard on top of those gained from being a 5th-level fighter.
				
Note that there are a number of effects and prerequisites that rely on a character's level or Hit Dice. Such effects are always based on the total number of levels or Hit Dice a character possesses, not just those from one class. The exception to this is class abilities, most of which are based on the total number of class levels that a character possesses of that particular class.

\subsection{Favored Class}


Each character begins play with a single favored class of his choosing---typically, this is the same class as the one he chooses at 1st level. Whenever a character gains a level in his favored class, he receives either + 1 hit point or + 1 skill rank. The choice of favored class cannot be changed once the character is created, and the choice of gaining a hit point or a skill rank each time a character gains a level (including his first level) cannot be changed once made for a particular level. Prestige classes (see Prestige Classes) can never be a favored class. 

\begin{formal}
 \textbf{On Capstones}
 One more radical idea is to move class capstones to 15th level. The reasoning behind this is that it will make them
 more playable, and in addition, allow base classes to better compete with prestige classes that grant 10th-level
 capstones.
\end{formal}


\section{Barbarian}

\label{f0}				
For some, there is only rage. In the ways of their people, in the fury of their passion, in the howl of battle, conflict is all these brutal souls know. Savages, hired muscle, masters of vicious martial techniques, they are not soldiers or professional warriors---they are the battle possessed, creatures of slaughter and spirits of war. Known as barbarians, these warmongers know little of training, preparation, or the rules of warfare; for them, only the moment exists, with the foes that stand before them and the knowledge that the next moment might hold their death. They possess a sixth sense in regard to danger and the endurance to weather all that might entail. These brutal warriors might rise from all walks of life, both civilized and savage, though whole societies embracing such philosophies roam the wild places of the world. Within barbarians storms the primal spirit of battle, and woe to those who face their rage.
				
\textbf{Role}: Barbarians excel in combat, possessing the martial prowess and fortitude to take on foes seemingly far superior to themselves. With rage granting them boldness and daring beyond that of most other warriors, barbarians charge furiously into battle and ruin all who would stand in their way.
				
\textbf{Alignment}: Any nonlawful.
				
\textbf{Hit Die}: d12.
				
\subsection{Class Skills}

				
The barbarian's class skills are Acrobatics (Dex), Climb (Str), Craft (Int), Handle Animal (Cha), Intimidate (Cha), Knowledge (nature) (Int), Perception (Wis), Ride (Dex), Survival (Wis), and Swim (Str).
				
\textbf{Skill Ranks per Level}: 4 + Int modifier.
% <div class="table">

\begin{table*}[]
\caption{Barbarian}
\sffamily
\begin{tabular}{llllll}
\textbf{Level} & \textbf{Base Attack Bonus} & \textbf{Fort Save} & \textbf{Ref Save} & \textbf{Will Save} & \textbf{Special}\\
1st   & +1                & +2        & +0       & +0        & Fast movement, rage              \\
2nd   & +2                & +3        & +0       & +0        & Rage power, uncanny dodge        \\
3rd   & +3                & +3        & +1       & +1        & Trap sense +1                    \\
4th   & +4                & +4        & +1       & +1        & Rage power                       \\
5th   & +5                & +4        & +1       & +1        & Improved uncanny dodge           \\
6th   & +6/+1             & +5        & +2       & +2        & Rage power, Trap sense +2        \\
7th   & +7/+2             & +5        & +2       & +2        & Damage reduction 1/-             \\
8th   & +8/+3             & +6        & +2       & +2        & Rage power                       \\
9th   & +9/+4             & +6        & +3       & +3        & Trap sense +3                    \\
10th  & +10/+5            & +7        & +3       & +3        & Damage reduction 2/-, Rage power \\
11th  & +11/+6/+1         & +7        & +3       & +3        & Greater rage                     \\
12th  & +12/+7/+2         & +8        & +4       & +4        & Rage power, Trap sense +4        \\
13th  & +13/+8/+3         & +8        & +4       & +4        & Damage reduction 3/-             \\
14th  & +14/+9/+4         & +9        & +4       & +4        & Indomitable will, Rage power     \\
15th  & +15/+10/+5        & +9        & +5       & +5        & Trap sense +5                    \\
16th  & +16/+11/+6/+1     & +10       & +5       & +5        & Damage reduction 4/-, Rage power \\
17th  & +17/+12/+7/+2     & +10       & +5       & +5        & Tireless rage                    \\
18th  & +18/+13/+8/+3     & +11       & +6       & +6        & Rage power, Trap sense +6        \\
19th  & +19/+14/+9/+4     & +11       & +6       & +6        & Damage reduction 5/-             \\
20th  & +20/+15/+10/+5    & +12       & +6       & +6        & Mighty rage, Rage power         
\end{tabular}
\end{table*}

				
\subsection{Class Features}

				
All of the following are class features of the barbarian.
				
\textbf{Weapon and Armor Proficiency}: A barbarian is proficient with all simple and martial weapons, light armor, medium armor, and shields (except tower shields).
				
\textbf{Fast Movement (Ex)}: A barbarian's land speed is faster than the norm for her race by +10 feet. This benefit applies only when she is wearing no armor, light armor, or medium armor, and not carrying a heavy load. Apply this bonus before modifying the barbarian's speed because of any load carried or armor worn. This bonus stacks with any other bonuses to the barbarian's land speed.
				
\textbf{Rage (Ex)}: A barbarian can call upon inner reserves of strength and ferocity, granting her additional combat prowess. Starting at 1st level, a barbarian can rage for a number of rounds per day equal to 4 + her Constitution modifier. At each level after 1st, she can rage for 2 additional rounds. Temporary increases to Constitution, such as those gained from rage and spells like \textit{bear's endurance,} do not increase the total number of rounds that a barbarian can rage per day. A barbarian can enter rage as a free action. The total number of rounds of rage per day is renewed after resting for 8 hours, although these hours do not need to be consecutive.
				
While in rage, a barbarian gains a +4 morale bonus to her Strength and Constitution, as well as a +2 morale bonus on Will saves. In addition, she takes a --2 penalty to Armor Class. The increase to Constitution grants the barbarian 2 hit points per Hit Dice, but these disappear when the rage ends and are not lost first like temporary hit points. While in rage, a barbarian cannot use any Charisma-, Dexterity-, or Intelligence-based skills (except Acrobatics, Fly, Intimidate, and Ride) or any ability that requires patience or concentration.
				
A barbarian can end her rage as a free action and is fatigued after rage for a number of rounds equal to 2 times the number of rounds spent in the rage. A barbarian cannot enter a new rage while fatigued or exhausted but can otherwise enter rage multiple times during a single encounter or combat. If a barbarian falls unconscious, her rage immediately ends, placing her in peril of death.
				
\textbf{Rage Powers (Ex)}: As a barbarian gains levels, she learns to use her rage in new ways. Starting at 2nd level, a barbarian gains a rage power. She gains another rage power for every two levels of barbarian attained after 2nd level. A barbarian gains the benefits of rage powers only while raging, and some of these powers require the barbarian to take an action first. Unless otherwise noted, a barbarian cannot select an individual power more than once.
				
\textit{Animal Fury (Ex)}: While raging, the barbarian gains a bite attack. If used as part of a full attack action, the bite attack is made at the barbarian's full base attack bonus --5. If the bite hits, it deals 1d4 points of damage (assuming the barbarian is Medium; 1d3 points of damage if Small) plus half the barbarian's Strength modifier. A barbarian can make a bite attack as part of the action to maintain or break free from a grapple. This attack is resolved before the grapple check is made. If the bite attack hits, any grapple checks made by the barbarian against the target this round are at a +2 bonus.
				
\textit{Clear Mind} \textit{(Ex)}: A barbarian may reroll a failed Will save. This power is used as an immediate action after the first save is attempted, but before the results are revealed by the GM. The barbarian must take the second result, even if it is worse. A barbarian must be at least 8th level before selecting this power. This power can only be used once per rage.
				
\textit{Fearless Rage (Ex)}: While raging, the barbarian is immune to the shaken and frightened conditions. A barbarian must be at least 12th level before selecting this rage power.
				
\textit{Guarded Stance} \textit{(Ex)}: The barbarian gains a +1 dodge bonus to her Armor Class against melee attacks for a number of rounds equal to the barbarian's current Constitution modifier (minimum 1). This bonus increases by +1 for every 6 levels the barbarian has attained. Activating this ability is a move action that does not provoke an attack of opportunity.
				
\textit{Increased Damage Reduction} \textit{(Ex)}: The barbarian's damage reduction increases by 1/---. This increase is always active while the barbarian is raging. A barbarian can select this rage power up to three times. Its effects stack. A barbarian must be at least 8th level before selecting this rage power.
				
\textit{Internal Fortitude (Ex)}: While raging, the barbarian is immune to the sickened and nauseated conditions. A barbarian must be at least 8th level before selecting this rage power.
				
\textit{Intimidating Glare} \textit{(Ex)}: The barbarian can make an Intimidate check against one adjacent foe as a move action. If the barbarian successfully demoralizes her opponent, the foe is shaken for 1d4 rounds + 1 round for every 5 points by which the barbarian's check exceeds the DC.
				
\textit{Knockback} \textit{(Ex)}: Once per round, the barbarian can make a bull rush attempt against one target in place of a melee attack. If successful, the target takes damage equal to the barbarian's Strength modifier and is moved back as normal. The barbarian does not need to move with the target if successful. This does not provoke an attack of opportunity.
				
\textit{Low-Light Vision} \textit{(Ex)}: The barbarian's senses sharpen and she gains low-light vision while raging.
				
\textit{Mighty Swing} \textit{(Ex)}: The barbarian automatically confirms a critical hit. This power is used as an immediate action once a critical threat has been determined. A barbarian must be at least 12th level before selecting this power. This power can only be used once per rage.
				
\textit{Moment of Clarity} \textit{(Ex)}: The barbarian does not gain any benefits or take any of the penalties from rage for 1 round. Activating this power is a swift action. This includes the penalty to Armor Class and the restriction on what actions can be performed. This round still counts against her total number of rounds of rage per day. This power can only be used once per rage.
				
\textit{Night Vision} \textit{(Ex)}: The barbarian's senses grow incredibly sharp while raging and she gains darkvision 60 feet. A barbarian must have low-light vision as a rage power or a racial trait to select this rage power.
				
\textit{No Escape (Ex)}: The barbarian can move up to double her normal speed as an immediate action but she can only use this ability when an adjacent foe uses a withdraw action to move away from her. She must end her movement adjacent to the enemy that used the withdraw action. The barbarian provokes attacks of opportunity as normal during this movement. This power can only be used once per rage.
				
\textit{Powerful Blow (Ex)}: The barbarian gains a +1 bonus on a single damage roll. This bonus increases by +1 for every 4 levels the barbarian has attained. This power is used as a swift action before the roll to hit is made. This power can only be used once per rage.
				
\textit{Quick Reflexes (Ex)}: While raging, the barbarian can make one additional attack of opportunity per round.
				
\textit{Raging Climber (Ex)}: When raging, the barbarian adds her level as an enhancement bonus on all Climb skill checks.
				
\textit{Raging Leaper (Ex)}: When raging, the barbarian adds her level as an enhancement bonus on all Acrobatics skill checks made to jump. When making a jump in this way, the barbarian is always considered to have a running start.
				
\textit{Raging Swimmer (Ex)}: When raging, the barbarian adds her level as an enhancement bonus on all Swim skill checks.
				
\textit{Renewed Vigor} \textit{(Ex)}: As a standard action, the barbarian heals 1d8 points of damage + her Constitution modifier. For every four levels the barbarian has attained above 4th, this amount of damage healed increases by 1d8, to a maximum of 5d8 at 20th level. A barbarian must be at least 4th level before selecting this power. This power can be used only once per day and only while raging.
				
\textit{Rolling Dodge (Ex)}: The barbarian gains a +1 dodge bonus to her Armor Class against ranged attacks for a number of rounds equal to the barbarian's current Constitution modifier (minimum 1). This bonus increases by +1 for every 6 levels the barbarian has attained. Activating this ability is a move action that does not provoke an attack of opportunity.
				
\textit{Roused Anger} \textit{(Ex)}: The barbarian may enter a rage even if fatigued. While raging after using this ability, the barbarian is immune to the fatigued condition. Once this rage ends, the barbarian is exhausted for 10 minutes per round spent raging.
				
\textit{Scent} \textit{(Ex)}: The barbarian gains the scent ability while raging and can use this ability to locate unseen foes (see Special Abilities for rules on the scent ability).
				
\textit{Strength Surge} \textit{(Ex)}: The barbarian adds her barbarian level on one Strength check or combat maneuver check, or to her Combat Maneuver Defense when an opponent attempts a maneuver against her. This power is used as an immediate action. This power can only be used once per rage.
				
\textit{Superstition (Ex)}: The barbarian gains a +2 morale bonus on saving throws made to resist spells, supernatural abilities, and spell-like abilities. This bonus increases by +1 for every 4 levels the barbarian has attained. While raging, the barbarian cannot be a willing target of any spell and must make saving throws to resist all spells, even those cast by allies.
				
\textit{Surprise Accuracy} \textit{(Ex)}: The barbarian gains a +1 morale bonus on one attack roll. This bonus increases by +1 for every 4 levels the barbarian has attained. This power is used as a swift action before the roll to hit is made. This power can only be used once per rage.
				
\textit{Swift Foot} \textit{(Ex)}: The barbarian gains a 5-foot enhancement bonus to her speed. This increase is always active while the barbarian is raging. A barbarian can select this rage power up to three times. Its effects stack. 
				
\textit{Terrifying Howl} \textit{(Ex)}: The barbarian unleashes a terrifying howl as a standard action. All shaken enemies within 30 feet must make a Will save (DC equal to 10 + 1/2 the barbarian's level + the barbarian's Strength modifier) or be panicked for 1d4+1 rounds. Once an enemy has made a save versus terrifying howl (successful or not), it is immune to this power for 24 hours. A barbarian must have the intimidating glare rage power to select this rage power. A barbarian must be at least 8th level before selecting this power.
				
\textit{Unexpected Strike (Ex)}: The barbarian can make an attack of opportunity against a foe that moves into any square threatened by the barbarian, regardless of whether or not that movement would normally provoke an attack of opportunity. This power can only be used once per rage. A barbarian must be at least 8th level before selecting this power.
				
\textbf{Uncanny Dodge (Ex)}: At 2nd level, a barbarian gains the ability to react to danger before her senses would normally allow her to do so. She cannot be caught flat-footed, nor does she lose her Dexterity bonus to AC if the attacker is invisible. She still loses her Dexterity bonus to AC if immobilized. A barbarian with this ability can still lose her Dexterity bonus to AC if an opponent successfully uses the feint action against her.
				
If a barbarian already has uncanny dodge from a different class, she automatically gains improved uncanny dodge (see below) instead.
				
\textbf{Trap Sense (Ex)}: At 3rd level, a barbarian gains a +1 bonus on Reflex saves made to avoid traps and a +1 dodge bonus to AC against attacks made by traps. These bonuses increase by +1 every three barbarian levels thereafter (6th, 9th, 12th, 15th, and 18th level). Trap sense bonuses gained from multiple classes stack.
				
\textbf{Improved Uncanny Dodge (Ex)}: At 5th level and higher, a barbarian can no longer be flanked. This defense denies a rogue the ability to sneak attack the barbarian by flanking her, unless the attacker has at least four more rogue levels than the target has barbarian levels.
				
If a character already has uncanny dodge (see above) from another class, the levels from the classes that grant uncanny dodge stack to determine the minimum rogue level required to flank the character.
				
\textbf{Damage Reduction (Ex)}: At 7th level, a barbarian gains damage reduction. Subtract 1 from the damage the barbarian takes each time she is dealt damage from a weapon or a natural attack. At 10th level, and every three barbarian levels thereafter (13th, 16th, and 19th level), this damage reduction rises by 1 point. Damage reduction can reduce damage to 0 but not below 0.
				
\textbf{Greater Rage (Ex)}: At 11th level, when a barbarian enters rage, the morale bonus to her Strength and Constitution increases to +6 and the morale bonus on her Will saves increases to +3.
				
\textbf{Indomitable Will (Ex)}: While in rage, a barbarian of 14th level or higher gains a +4 bonus on Will saves to resist enchantment spells. This bonus stacks with all other modifiers, including the morale bonus on Will saves she also receives during her rage.
				
\textbf{Tireless Rage (Ex)}: Starting at 17th level, a barbarian no longer becomes fatigued at the end of her rage.
				
\textbf{Mighty Rage (Ex)}: At 20th level, when a barbarian enters rage, the morale bonus to her Strength and Constitution increases to +8 and the morale bonus on her Will saves increases to +4.
				
\subsection{Ex-Barbarians}

				
A barbarian who becomes lawful loses the ability to rage and cannot gain more levels as a barbarian. She retains all other benefits of the class.
        	

\classentry{Bard}

\label{f0}				
Untold wonders and secrets exist for those skillful enough to discover them. Through cleverness, talent, and magic, these cunning few unravel the wiles of the world, becoming adept in the arts of persuasion, manipulation, and inspiration. Typically masters of one or many forms of artistry, bards possess an uncanny ability to know more than they should and use what they learn to keep themselves and their allies ever one step ahead of danger. Bards are quick-witted and captivating, and their skills might lead them down many paths, be they gamblers or jacks-of-all-trades, scholars or performers, leaders or scoundrels, or even all of the above. For bards, every day brings its own opportunities, adventures, and challenges, and only by bucking the odds, knowing the most, and being the best might they claim the treasures of each.
				
\textbf{Role}: Bards capably confuse and confound their foes while inspiring their allies to ever-greater daring. While accomplished with both weapons and magic, the true strength of bards lies outside melee, where they can support their companions and undermine their foes without fear of interruptions to their performances.
				
\textbf{Alignment}: Any.
				
\textbf{Hit Die}: d8.
				
\subsection{Class Skills}

				
The bard's class skills are Acrobatics (Dex), Appraise (Int), Bluff (Cha), Climb (Str), Craft (Int), Diplomacy (Cha), Disguise (Cha), Escape Artist (Dex), Intimidate (Cha), Knowledge (all) (Int), Linguistics (Int), Perception (Wis), Perform (Cha), Profession (Wis), Sense Motive (Wis), Sleight of Hand (Dex), Spellcraft (Int), Stealth (Dex), and Use Magic Device (Cha).
				
\textbf{Skill Ranks per Level}: 6 + Int modifier.
								
% <div class="table">
Table: Bard
% <
\begin{table}[]
\caption{Table: Bard}
\sffamily
\begin{tabularx}{\linewidth}{lp{5em}p{1.5em}p{1.5em}p{1.5em}Xllllll}
\multirow{2}{*}{\textbf{Level}} & \multirow{2}{*}{\parbox{5em}{\textbf{Base Attack Bonus}}} & \multirow{2}{*}{\parbox{1.5em}{\textbf{Fort Save}}} & \multirow{2}{*}{\parbox{1.5em}{\textbf{Ref Save}}} & \multirow{2}{*}{\parbox{1.5em}{\textbf{Will Save}}} & \textbf{Special}                                                                                              & \multicolumn{6}{c}{\textbf{Spells per day}} \\
                       &                                    &                            &                           &                            &                                                                                                      & \textbf{1st}  & \textbf{2nd} & \textbf{3rd} &\textbf{4th} & \textbf{5th} & \textbf{6th} \\
1st & +0 & +0 & +2 & +2 & Bardic knowledge, bardic performance, cantrips, countersong, distraction, fascinate, inspire courage & 1 & $-$ & $-$ & $-$ & $-$ & $-$\\
2nd & +1 & +0 & +3 & +3 & Versatile performance, well-versed & 2 & $-$ & $-$ & $-$ & $-$ & $-$\\
3rd & +2 & +1 & +3 & +3 & Inspire competence & 3 & $-$ & $-$ & $-$ & $-$ & $-$\\
4th & +3 & +1 & +4 & +4 &  & 3 & 1 & $-$ & $-$ & $-$ & $-$\\
5th & +3 & +1 & +4 & +4 & inspire courage, lore master & 4 & 2 & $-$ & $-$ & $-$ & $-$\\
6th & +4 & +2 & +5 & +5 & Suggestion, Versatile performance & 4 & 3 & $-$ & $-$ & $-$ & $-$\\
7th & +5 & +2 & +5 & +5 & Inspire competence & 4 & 3 & 1 & $-$ & $-$ & $-$\\
8th & +6/+1 & +2 & +6 & +6 & Dirge of doom & 4 & 4 & 2 & $-$ & $-$ & $-$\\
9th & +6/+1 & +3 & +6 & +6 & Inspire greatness & 5 & 4 & 3 & $-$ & $-$ & $-$\\
10th & +7/+2 & +3 & +7 & +7 & Jack-of-all-trades, Versatile performance & 5 & 4 & 3 & 1 & $-$ & $-$\\
11th & +8/+3 & +3 & +7 & +7 & Inspire competence, inspire courage, lore master & 5 & 4 & 4 & 2 & $-$ & $-$\\
12th & +9/+4 & +4 & +8 & +8 & Soothing performance & 5 & 5 & 4 & 3 & $-$ & $-$\\
13th & +9/+4 & +4 & +8 & +8 &  & 5 & 5 & 4 & 3 & 1 & $-$\\
14th & +10/+5 & +4 & +9 & +9 & Frightening tune, Versatile performance & 5 & 5 & 4 & 4 & 2 & $-$\\
15th & +11/+6/+1 & +5 & +9 & +9 & Inspire competence, inspire heroics & 5 & 5 & 5 & 4 & 3 & $-$\\
16th & +12/+7/+2 & +5 & +10 & +10 &  & 5 & 5 & 5 & 4 & 3 & 1\\
17th & +12/+7/+2 & +5 & +10 & +10 & inspire courage, lore master & 5 & 5 & 5 & 4 & 4 & 2\\
18th & +13/+8/+3 & +6 & +11 & +11 & Mass suggestion, Versatile performance & 5 & 5 & 5 & 5 & 4 & 3\\
19th & +14/+9/+4 & +6 & +11 & +11 & Inspire competence & 5 & 5 & 5 & 5 & 5 & 4\\
20th & +15/+10/+5 & +6 & +12 & +12 & Deadly performance & 5 & 5 & 5 & 5 & 5 & 5\\
\end{tabularx}
\end{table}

\begin{table}[]
\caption{Table: Bard Spells Known}
\sffamily
\begin{tabular}{llllllll}
Level&0&1st&2nd&3rd&4th&5th&6th\\
\hline
1st & 4 & 2 & $-$ & $-$ & $-$ & $-$ & $-$\\
2nd & 5 & 3 & $-$ & $-$ & $-$ & $-$ & $-$\\
3rd & 6 & 4 & $-$ & $-$ & $-$ & $-$ & $-$\\
4th & 6 & 4 & 2 & $-$ & $-$ & $-$ & $-$\\
5th & 6 & 4 & 3 & $-$ & $-$ & $-$ & $-$\\
6th & 6 & 4 & 4 & $-$ & $-$ & $-$ & $-$\\
7th & 6 & 5 & 4 & 2 & $-$ & $-$ & $-$\\
8th & 6 & 5 & 4 & 3 & $-$ & $-$ & $-$\\
9th & 6 & 5 & 4 & 4 & $-$ & $-$ & $-$\\
10th & 6 & 5 & 5 & 4 & 2 & $-$ & $-$\\
11th & 6 & 6 & 5 & 4 & 3 & $-$ & $-$\\
12th & 6 & 6 & 5 & 4 & 4 & $-$ & $-$\\
13th & 6 & 6 & 5 & 5 & 4 & 2 & $-$\\
14th & 6 & 6 & 6 & 5 & 4 & 3 & $-$\\
15th & 6 & 6 & 6 & 5 & 4 & 4 & $-$\\
16th & 6 & 6 & 6 & 5 & 5 & 4 & 2\\
17th & 6 & 6 & 6 & 6 & 5 & 4 & 3\\
18th & 6 & 6 & 6 & 6 & 5 & 4 & 4\\
19th & 6 & 6 & 6 & 6 & 5 & 5 & 4\\
20th & 6 & 6 & 6 & 6 & 6 & 5 & 5\\
\end{tabular}
\end{table}

\subsection{Class Features}

				
All of the following are class features of the bard.
				
\textbf{Weapon and Armor Proficiency}: A bard is proficient with all simple weapons, plus the longsword, rapier, sap, short sword, shortbow, and whip. Bards are also proficient with light armor and shields (except tower shields). A bard can cast bard spells while wearing light armor and use a shield without incurring the normal arcane spell failure chance. Like any other arcane spellcaster, a bard wearing medium or heavy armor incurs a chance of arcane spell failure if the spell in question has a somatic component. A multiclass bard still incurs the normal arcane spell failure chance for arcane spells received from other classes.
				
\textbf{Spells}: A bard casts arcane spells drawn from the bard spell list presented in Spell Lists. He can cast any spell he knows without preparing it ahead of time. Every bard spell has a verbal component (song, recitation, or music). To learn or cast a spell, a bard must have a Charisma score equal to at least 10 + the spell level. The Difficulty Class for a saving throw against a bard's spell is 10 + the spell level + the bard's Charisma modifier.
				
Like other spellcasters, a bard can cast only a certain number of spells of each spell level per day. His base daily spell allotment is given on Table: Bard. In addition, he receives bonus spells per day if he has a high Charisma score (see Table: Ability Modifiers and Bonus Spells).
				
The bard's selection of spells is extremely limited. A bard begins play knowing four 0-level spells and two 1st-level spells of the bard's choice. At each new bard level, he gains one or more new spells, as indicated on Table: Bard Spells Known. (Unlike spells per day, the number of spells a bard knows is not affected by his Charisma score. The numbers on Table: Bard Spells Known are fixed.)
				
Upon reaching 5th level, and at every third bard level after that (8th, 11th, and so on), a bard can choose to learn a new spell in place of one he already knows. In effect, the bard \texttt{{}"{}}loses\texttt{{}"{}} the old spell in exchange for the new one. The new spell's level must be the same as that of the spell being exchanged, and it must be at least one level lower than the highest-level bard spell the bard can cast. A bard may swap only a single spell at any given level and must choose whether or not to swap the spell at the same time that he gains new spells known for the level.
				
A bard need not prepare his spells in advance. He can cast any spell he knows at any time, assuming he has not yet used up his allotment of spells per day for the spell's level. 
				
\textbf{Bardic Knowledge (Ex)}: A bard adds half his class level (minimum 1) on all Knowledge skill checks and may make all Knowledge skill checks untrained. 
				
\textbf{Bardic Performance}: A bard is trained to use the Perform skill to create magical effects on those around him, including himself if desired. He can use this ability for a number of rounds per day equal to 4 + his Charisma modifier. At each level after 1st a bard can use bardic performance for 2 additional rounds per day. Each round, the bard can produce any one of the types of bardic performance that he has mastered, as indicated by his level. 
				
Starting a bardic performance is a standard action, but it can be maintained each round as a free action. Changing a bardic performance from one effect to another requires the bard to stop the previous performance and start a new one as a standard action. A bardic performance cannot be disrupted, but it ends immediately if the bard is killed, paralyzed, stunned, knocked unconscious, or otherwise prevented from taking a free action to maintain it each round. A bard cannot have more than one bardic performance in effect at one time.
				
At 7th level, a bard can start a bardic performance as a move action instead of a standard action. At 13th level, a bard can start a bardic performance as a swift action. 
				
Each bardic performance has audible components, visual components, or both.
				
If a bardic performance has audible components, the targets must be able to hear the bard for the performance to have any effect, and many such performances are language dependent (as noted in the description). A deaf bard has a 20\% chance to fail when attempting to use a bardic performance with an audible component. If he fails this check, the attempt still counts against his daily limit. Deaf creatures are immune to bardic performances with audible components.
				
If a bardic performance has a visual component, the targets must have line of sight to the bard for the performance to have any effect. A blind bard has a 50\% chance to fail when attempting to use a bardic performance with a visual component. If he fails this check, the attempt still counts against his daily limit. Blind creatures are immune to bardic performances with visual components.
				
\textit{Countersong (Su)}: At 1st level, a bard learns to counter magic effects that depend on sound (but not spells that have verbal components). Each round of the countersong he makes a Perform (keyboard, percussion, wind, string, or sing) skill check. Any creature within 30 feet of the bard (including the bard himself) that is affected by a sonic or language-dependent magical attack may use the bard's Perform check result in place of its saving throw if, after the saving throw is rolled, the Perform check result proves to be higher. If a creature within range of the countersong is already under the effect of a noninstantaneous sonic or language-dependent magical attack, it gains another saving throw against the effect each round it hears the countersong, but it must use the bard's Perform skill check result for the save. Countersong does not work on effects that don't allow saves. Countersong relies on audible components.
				
\textit{Distraction (Su)}: At 1st level, a bard can use his performance to counter magic effects that depend on sight. Each round of the distraction, he makes a Perform (act, comedy, dance, or oratory) skill check. Any creature within 30 feet of the bard (including the bard himself) that is affected by an illusion (pattern) or illusion (figment) magical attack may use the bard's Perform check result in place of its saving throw if, after the saving throw is rolled, the Perform skill check proves to be higher. If a creature within range of the distraction is already under the effect of a noninstantaneous illusion (pattern) or illusion (figment) magical attack, it gains another saving throw against the effect each round it sees the distraction, but it must use the bard's Perform skill check result for the save. Distraction does not work on effects that don't allow saves. Distraction relies on visual components.
				
\textit{Fascinate (Su)}: At 1st level, a bard can use his performance to cause one or more creatures to become fascinated with him. Each creature to be fascinated must be within 90 feet, able to see and hear the bard, and capable of paying attention to him. The bard must also be able to see the creatures affected. The distraction of a nearby combat or other dangers prevents this ability from working. For every three levels the bard has attained beyond 1st, he can target one additional creature with this ability.
				
Each creature within range receives a Will save (DC 10 + 1/2 the bard's level + the bard's Cha modifier) to negate the effect. If a creature's saving throw succeeds, the bard cannot attempt to fascinate that creature again for 24 hours. If its saving throw fails, the creature sits quietly and observes the performance for as long as the bard continues to maintain it. While fascinated, a target takes a --4 penalty on all skill checks made as reactions, such as Perception checks. Any potential threat to the target allows the target to make a new saving throw against the effect. Any obvious threat, such as someone drawing a weapon, casting a spell, or aiming a weapon at the target, automatically breaks the effect.
				
Fascinate is an enchantment (compulsion), mind-affecting ability. Fascinate relies on audible and visual components in order to function. 
				
\textit{Inspire Courage (Su)}: A 1st-level bard can use his performance to inspire courage in his allies (including himself), bolstering them against fear and improving their combat abilities. To be affected, an ally must be able to perceive the bard's performance. An affected ally receives a +1 morale bonus on saving throws against charm and fear effects and a +1 competence bonus on attack and weapon damage rolls. At 5th level, and every six bard levels thereafter, this bonus increases by +1, to a maximum of +4 at 17th level. Inspire courage is a mind-affecting ability. Inspire courage can use audible or visual components. The bard must choose which component to use when starting his performance.
				
\textit{Inspire Competence (Su)}: A bard of 3rd level or higher can use his performance to help an ally succeed at a task. That ally must be within 30 feet and be able to hear the bard. The ally gets a +2 competence bonus on skill checks with a particular skill as long as she continues to hear the bard's performance. This bonus increases by +1 for every four levels the bard has attained beyond 3rd (+3 at 7th, +4 at 11th, +5 at 15th, and +6 at 19th). Certain uses of this ability are infeasible, such as Stealth, and may be disallowed at the GM's discretion. A bard can't inspire competence in himself. Inspire competence relies on audible components.
				
\textit{Suggestion (Sp)}: A bard of 6th level or higher can use his performance to make a \textit{suggestion} (as per the spell) to a creature he has already fascinated (see above). Using this ability does not disrupt the fascinate effect, but it does require a standard action to activate (in addition to the free action to continue the fascinate effect). A bard can use this ability more than once against an individual creature during an individual performance.
				
Making a \textit{suggestion} does not count against a bard's daily use of bardic performance. A Will saving throw (DC 10 + 1/2 the bard's level + the bard's Cha modifier) negates the effect. This ability affects only a single creature. \textit{Suggestion }is an enchantment (compulsion), mind affecting, language-dependent ability and relies on audible components.
				
\textit{Dirge of Doom (Su)}: A bard of 8th level or higher can use his performance to foster a sense of growing dread in his enemies, causing them to become shaken. To be affected, an enemy must be within 30 feet and able to see and hear the bard's performance. The effect persists for as long as the enemy is within 30 feet and the bard continues his performance. This performance cannot cause a creature to become frightened or panicked, even if the targets are already shaken from another effect. Dirge of doom is a mind-affecting fear effect, and it relies on audible and visual components.
				
\textit{Inspire Greatness (Su)}: A bard of 9th level or higher can use his performance to inspire greatness in himself or a single willing ally within 30 feet, granting extra fighting capability. For every three levels the bard attains beyond 9th, he can target an additional ally while using this performance (up to a maximum of four targets at 18th level). To inspire greatness, all of the targets must be able to see and hear the bard. A creature inspired with greatness gains 2 bonus Hit Dice (d10s), the commensurate number of temporary hit points (apply the target's Constitution modifier, if any, to these bonus Hit Dice), a +2 competence bonus on attack rolls, and a +1 competence bonus on Fortitude saves. The bonus Hit Dice count as regular Hit Dice for determining the effect of spells that are Hit Dice dependent. Inspire greatness is a mind-affecting ability and it relies on audible and visual components.
				
\textit{Soothing Performance} \textit{(Su)}: A bard of 12th level or higher can use his performance to create an effect equivalent to a \textit{mass cure serious wounds, }using the bard's level as the caster level. In addition, this performance removes the fatigued, sickened, and shaken conditions from all those affected. Using this ability requires 4 rounds of continuous performance, and the targets must be able to see and hear the bard throughout the performance. Soothing performance affects all targets that remain within 30 feet throughout the performance. Soothing performance relies on audible and visual components.
				
\textit{Frightening Tune (Sp)}: A bard of 14th level or higher can use his performance to cause fear in his enemies. To be affected, an enemy must be able to hear the bard perform and be within 30 feet. Each enemy within range receives a Will save (DC 10 + 1/2 the bard's level + the bard's Cha modifier) to negate the effect. If the save succeeds, the creature is immune to this ability for 24 hours. If the save fails, the target becomes frightened and flees for as long as the target can hear the bard's performance. Frightening tune relies on audible components.
				
\textit{Inspire Heroics} \textit{(Su)}: A bard of 15th level or higher can inspire tremendous heroism in himself or a single ally within 30 feet. For every three bard levels the character attains beyond 15th, he can inspire heroics in an additional creature. To inspire heroics, all of the targets must be able to see and hear the bard. Inspired creatures gain a +4 morale bonus on saving throws and a +4 dodge bonus to AC. This effect lasts for as long as the targets are able to witness the performance. Inspire heroics is a mind-affecting ability that relies on audible and visual components.
				
\textit{Mass Suggestion (Sp)}: This ability functions just like \textit{suggestion, }but allows a bard of 18th level or higher to make a \textit{suggestion} simultaneously to any number of creatures that he has already fascinated. \textit{Mass suggestion} is an  enchantment (compulsion), mind-affecting, language-dependent ability that relies on audible components.
				
\textit{Deadly Performance (Su)}: A bard of 20th level or higher can use his performance to cause one enemy to die from joy or sorrow. To be affected, the target must be able to see and hear the bard perform for 1 full round and be within 30 feet. The target receives a Will save (DC 10 + 1/2 the bard's level + the bard's Cha modifier) to negate the effect. If a creature's saving throw succeeds, the target is staggered for 1d4 rounds, and the bard cannot use deadly performance on that creature again for 24 hours. If a creature's saving throw fails, it dies. Deadly performance is a mind-affecting death effect that relies on audible and visual components.
				
\textbf{Cantrips}: Bards learn a number of cantrips, or 0-level spells, as noted on Table: Bard Spells Known under \texttt{{}"{}}Spells Known.\texttt{{}"{}} These spells are cast like any other spell, but they do not consume any slots and may be used again.
				
\textbf{Versatile Performance (Ex)}: At 2nd level, a bard can choose one type of Perform skill. He can use his bonus in that skill in place of his bonus in associated skills. When substituting in this way, the bard uses his total Perform skill bonus, including class skill bonus, in place of its associated skill's bonus, whether or not he has ranks in that skill or if it is a class skill. At 6th level, and every 4 levels thereafter, the bard can select an additional type of Perform to substitute.
				
The types of Perform and their associated skills are: Act (Bluff, Disguise), Comedy (Bluff, Intimidate), Dance (Acrobatics, Fly), Keyboard Instruments (Diplomacy, Intimidate), Oratory (Diplomacy, Sense Motive), Percussion (Handle Animal, Intimidate), Sing (Bluff, Sense Motive), String (Bluff, Diplomacy), and Wind (Diplomacy, Handle Animal).
				
\textbf{Well-Versed (Ex)}: At 2nd level, the bard becomes resistant to the bardic performance of others, and to sonic effects in general. The bard gains a +4 bonus on saving throws made against bardic performance, sonic, and language-dependent effects. 
				
\textbf{Lore Master (Ex)}: At 5th level, the bard becomes a master of lore and can take 10 on any Knowledge skill check that he has ranks in. A bard can choose not to take 10 and can instead roll normally. In addition, once per day, the bard can take 20 on any Knowledge skill check as a standard action. He can use this ability one additional time per day for every six levels he possesses beyond 5th, to a maximum of three times per day at 17th level.
				
\textbf{Jack-of-All-Trades (Ex)}: At 10th level, the bard can use any skill, even if the skill normally requires him to be trained. At 16th level, the bard considers all skills to be class skills. At 19th level, the bard can take 10 on any skill check, even if it is not normally allowed.
        	

\section{Cleric}

\label{f0}				
In faith and the miracles of the divine, many find a greater purpose. Called to serve powers beyond most mortal understanding, all priests preach wonders and provide for the spiritual needs of their people. Clerics are more than mere priests, though; these emissaries of the divine work the will of their deities through strength of arms and the magic of their gods. Devoted to the tenets of the religions and philosophies that inspire them, these ecclesiastics quest to spread the knowledge and influence of their faith. Yet while they might share similar abilities, clerics prove as different from one another as the divinities they serve, with some offering healing and redemption, others judging law and truth, and still others spreading conflict and corruption. The ways of the cleric are varied, yet all who tread these paths walk with the mightiest of allies and bear the arms of the gods themselves.
				
\textbf{Role}: More than capable of upholding the honor of their deities in battle, clerics often prove stalwart and capable combatants. Their true strength lies in their capability to draw upon the power of their deities, whether to increase their own and their allies' prowess in battle, to vex their foes with divine magic, or to lend healing to companions in need.
				
As their powers are influenced by their faith, all clerics must focus their worship upon a divine source. While the vast majority of clerics revere a specific deity, a small number dedicate themselves to a divine concept worthy of devotion---such as battle, death, justice, or knowledge---free of a deific abstraction. (Work with your GM if you prefer this path to selecting a specific deity.)
				
\textbf{Alignment}: A cleric's alignment must be within one step of her deity's, along either the law/chaos axis or the good/evil axis (see Additional Rules).
				
\textbf{Hit Die}: d8.
				
\subsection{Class Skills}

				
The cleric's class skills are Appraise (Int), Craft (Int), Diplomacy (Cha), Heal (Wis), Knowledge (arcana) (Int), Knowledge (history) (Int), Knowledge (nobility) (Int), Knowledge (planes) (Int), Knowledge (religion) (Int), Linguistics (Int), Profession (Wis), Sense Motive (Wis), and Spellcraft (Int).
				
\textbf{Skill Ranks per Level: }2 + Int modifier.

\begin{table*}[]
\caption{Table: Cleric}
\sffamily
\setlength{\tabcolsep}{1pt}
\begin{tabularx}{\linewidth}{lp{6em}p{2.5em}p{2.5em}p{2.5em}Xllllllllll}
\multirow{2}{*}{\textbf{Level}} & \multirow{2}{*}{\parbox{5em}{\textbf{Base Attack Bonus}}} & \multirow{2}{*}{\parbox{1.5em}{\textbf{Fort Save}}} & \multirow{2}{*}{\parbox{1.5em}{\textbf{Ref Save}}} & \multirow{2}{*}{\parbox{1.5em}{\textbf{Will Save}}} & \textbf{Special} & \multicolumn{10}{c}{\textbf{Spells per day}} \\
                       &                                    &                            &                           &                            &                                                                                                  &  \textbf{0} & \textbf{1st} & \textbf{2nd} & \textbf{3rd} & \textbf{4th} & \textbf{5th} & \textbf{6th} & \textbf{7th} & \textbf{8th} & \textbf{9th} \\
1st & +0 & +2 & +0 & +2 & Aura, channel energy 1d6, domains, orisons & 3 & 1+1 & - & - & - & - & - & - & - & -\\
2nd & +1 & +3 & +0 & +3 &  & 4 & 2+1 & - & - & - & - & - & - & - & -\\
3rd & +2 & +3 & +1 & +3 & Channel energy 2d6 & 4 & 2+1 & 1+1 & - & - & - & - & - & - & -\\
4th & +3 & +4 & +1 & +4 &  & 4 & 3+1 & 2+1 & - & - & - & - & - & - & -\\
5th & +3 & +4 & +1 & +4 & Channel energy 3d6 & 4 & 3+1 & 2+1 & 1+1 & - & - & - & - & - & -\\
6th & +4 & +5 & +2 & +5 &  & 4 & 3+1 & 3+1 & 2+1 & - & - & - & - & - & -\\
7th & +5 & +5 & +2 & +5 & Channel energy 4d6 & 4 & 4+1 & 3+1 & 2+1 & 1+1 & - & - & - & - & -\\
8th & +6/+1 & +6 & +2 & +6 &  & 4 & 4+1 & 3+1 & 3+1 & 2+1 & - & - & - & - & -\\
9th & +6/+1 & +6 & +3 & +6 & Channel energy 5d6 & 4 & 4+1 & 4+1 & 3+1 & 2+1 & 1+1 & - & - & - & -\\
10th & +7/+2 & +7 & +3 & +7 &  & 4 & 4+1 & 4+1 & 3+1 & 3+1 & 2+1 & - & - & - & -\\
11th & +8/+3 & +7 & +3 & +7 & Channel energy 6d6 & 4 & 4+1 & 4+1 & 4+1 & 3+1 & 2+1 & 1+1 & - & - & -\\
12th & +9/+4 & +8 & +4 & +8 &  & 4 & 4+1 & 4+1 & 4+1 & 3+1 & 3+1 & 2+1 & - & - & -\\
13th & +9/+4 & +8 & +4 & +8 & Channel energy 7d6 & 4 & 4+1 & 4+1 & 4+1 & 4+1 & 3+1 & 2+1 & 1+1 & - & -\\
14th & +10/+5 & +9 & +4 & +9 &  & 4 & 4+1 & 4+1 & 4+1 & 4+1 & 3+1 & 3+1 & 2+1 & - & -\\
15th & +11/+6/+1 & +9 & +5 & +9 & Channel energy 8d6 & 4 & 4+1 & 4+1 & 4+1 & 4+1 & 4+1 & 3+1 & 2+1 & 1+1 & -\\
16th & +12/+7/+2 & +10 & +5 & +10 &  & 4 & 4+1 & 4+1 & 4+1 & 4+1 & 4+1 & 3+1 & 3+1 & 2+1 & -\\
17th & +12/+7/+2 & +10 & +5 & +10 & Channel energy 9d6 & 4 & 4+1 & 4+1 & 4+1 & 4+1 & 4+1 & 4+1 & 3+1 & 2+1 & 1+1\\
18th & +13/+8/+3 & +11 & +6 & +11 &  & 4 & 4+1 & 4+1 & 4+1 & 4+1 & 4+1 & 4+1 & 3+1 & 3+1 & 2+1\\
19th & +14/+9/+4 & +11 & +6 & +11 & Channel energy 10d6 & 4 & 4+1 & 4+1 & 4+1 & 4+1 & 4+1 & 4+1 & 4+1 & 3+1 & 3+1\\
20th & +15/+10/+5 & +12 & +6 & +12 &  & 4 & 4+1 & 4+1 & 4+1 & 4+1 & 4+1 & 4+1 & 4+1 & 4+1 & 4+1\\
\end{tabularx}
 Note: ``+1'' represents the domain spell slot\\
\end{table*}

				
\subsection{Class Features}

				
The following are class features of the cleric.
				
\textbf{Weapon and Armor Proficiency}: Clerics are proficient with all simple weapons, light armor, medium armor, and shields (except tower shields). Clerics are also proficient with the favored weapon of their deity.
				
\textbf{Aura (Ex)}: A cleric of a chaotic, evil, good, or lawful deity has a particularly powerful aura corresponding to the deity's alignment (see the \textit{detect evil} spell for details).
				
\textbf{Spells}: A cleric casts divine spells which are drawn from the cleric spell list presented in Spell Lists. Her alignment, however, may restrict her from casting certain spells opposed to her moral or ethical beliefs; see chaotic, evil, good, and lawful spells. A cleric must choose and prepare her spells in advance.
				
To prepare or cast a spell, a cleric must have a Wisdom score equal to at least 10 + the spell level. The Difficulty Class for a saving throw against a cleric's spell is 10 + the spell level + the cleric's Wisdom modifier.
				
Like other spellcasters, a cleric can cast only a certain number of spells of each spell level per day. Her base daily spell allotment is given on Table: Cleric. In addition, she receives bonus spells per day if she has a high Wisdom score (see Table: Ability Modifiers and Bonus Spells). 
				
Clerics meditate or pray for their spells. Each cleric must choose a time when she must spend 1 hour each day in quiet contemplation or supplication to regain her daily allotment of spells. A cleric may prepare and cast any spell on the cleric spell list, provided that she can cast spells of that level, but she must choose which spells to prepare during her daily meditation.
				
\textbf{Channel Energy (Su)}: Regardless of alignment, any cleric can release a wave of energy by channeling the power of her faith through her holy (or unholy) symbol. This energy can be used to cause or heal damage, depending on the type of energy channeled and the creatures targeted. 
				
A good cleric (or one who worships a good deity) channels positive energy and can choose to deal damage to undead creatures or to heal living creatures. An evil cleric (or one who worships an evil deity) channels negative energy and can choose to deal damage to living creatures or to heal undead creatures. A neutral cleric who worships a neutral deity (or one who is not devoted to a particular deity) must choose whether she channels positive or negative energy. Once this choice is made, it cannot be reversed. This decision also determines whether the cleric casts spontaneous cure or inflict spells (see spontaneous casting). 
				
Channeling energy causes a burst that affects all creatures of one type (either undead or living) in a 30-foot radius centered on the cleric. The amount of damage dealt or healed is equal to 1d6 points of damage plus 1d6 points of damage for every two cleric levels beyond 1st (2d6 at 3rd, 3d6 at 5th, and so on). Creatures that take damage from channeled energy receive a Will save to halve the damage. The DC of this save is equal to 10 + 1/2 the cleric's level + the cleric's Charisma modifier. Creatures healed by channeled energy cannot exceed their maximum hit point total---all excess healing is lost. A cleric may channel energy a number of times per day equal to 3 + her Charisma modifier. This is a standard action that does not provoke an attack of opportunity. A cleric can choose whether or not to include herself in this effect. A cleric must be able to present her holy symbol to use this ability. 
				
\textbf{Domains}: A cleric's deity influences her alignment, what magic she can perform, her values, and how others see her. A cleric chooses two domains from among those belonging to her deity. A cleric can select an alignment domain (Chaos, Evil, Good, or Law) only if her alignment matches that domain. If a cleric is not devoted to a particular deity, she still selects two domains to represent her spiritual inclinations and abilities (subject to GM approval). The restriction on alignment domains still applies.
				
Each domain grants a number of domain powers, dependent upon the level of the cleric, as well as a number of bonus spells. A cleric gains one domain spell slot for each level of cleric spell she can cast, from 1st on up. Each day, a cleric can prepare one of the spells from her two domains in that slot. If a domain spell is not on the cleric spell list, a cleric can prepare it only in her domain spell slot. Domain spells cannot be used to cast spells spontaneously. 
				
In addition, a cleric gains the listed powers from both of her domains, if she is of a high enough level. Unless otherwise noted, using a domain power is a standard action. Cleric domains are listed at the end of this class entry.
				
\textbf{Orisons}: Clerics can prepare a number of orisons, or 0-level spells, each day, as noted on Table: Cleric under \^aSpells per Day.\^a These spells are cast like any other spell, but they are not expended when cast and may be used again.
				
\textbf{Spontaneous Casting}: A good cleric (or a neutral cleric of a good deity) can channel stored spell energy into healing spells that she did not prepare ahead of time. The cleric can \^alose\^a any prepared spell that is not an orison or domain spell in order to cast any cure spell of the same spell level or lower (a cure spell is any spell with \^acure\^a in its name). 
				
An evil cleric (or a neutral cleric who worships an evil deity) can't convert prepared spells to cure spells but can convert them to inflict spells (an inflict spell is one with \^ainflict\^a in its name).
				
A cleric who is neither good nor evil and whose deity is neither good nor evil can convert spells to either cure spells or inflict spells (player's choice). Once the player makes this choice, it cannot be reversed. This choice also determines whether the cleric channels positive or negative energy (see Channel Energy).
				
\textbf{Chaotic, Evil, Good, and Lawful Spells}: A cleric can't cast spells of an alignment opposed to her own or her deity's (if she has one). Spells associated with particular alignments are indicated by the chaotic, evil, good, and lawful descriptors in their spell descriptions.
				
\textbf{Bonus Languages}: A cleric's bonus language options include Celestial, Abyssal, and Infernal (the languages of good, chaotic evil, and lawful evil outsiders, respectively). These choices are in addition to the bonus languages available to the character because of her race.
				
\subsection{Ex-Clerics}

				
A cleric who grossly violates the code of conduct required by her god loses all spells and class features, except for armor and shield proficiencies and proficiency with simple weapons. She cannot thereafter gain levels as a cleric of that god until she atones for her deeds (see the \textit{atonement} spell description).
				
\subsection{Domains}

				
Clerics may select any two of the domains granted by their deity. Clerics without a deity may select any two domains (choice are subject to GM approval).
				
\subsection{Air Domain}

				
\textbf{Granted Powers}: You can manipulate lightning, mist, and wind, traffic with air creatures, and are resistant to electricity damage.
				
\textit{Lightning Arc (Sp)}: As a standard action, you can unleash an arc of electricity targeting any foe within 30 feet as a ranged touch attack. This arc of electricity deals 1d6 points of electricity damage + 1 point for every two cleric levels you possess. You can use this ability a number of times per day equal to 3 + your Wisdom modifier. 
				
\textit{Electricity Resistance (Ex)}: At 6th level, you gain resist electricity 10. This resistance increases to 20 at 12th level. At 20th level, you gain immunity to electricity.
				
\textbf{Domain Spells}: 1st---\textit{obscuring mist, }2nd---\textit{wind wall, }3rd---\textit{gaseous form, }4th---\textit{air walk, }5th---\textit{control winds, }6th---\textit{chain lightning, }7th---\textit{elemental body IV} (air only)\textit{, }8th---\textit{whirlwind, }9th---\textit{elemental swarm} (air spell only).
				
\subsection{Animal Domain}

				
\textbf{Granted Powers}: You can speak with and befriend animals with ease. In addition, you treat Knowledge (nature) as a class skill.
				
\textit{Speak with Animals (Sp)}: You can \textit{speak with animals}, as per the spell, for a number of rounds per day equal to 3 + your cleric level. 
				
\textit{Animal Companion (Ex)}: At 4th level, you gain the service of an animal companion. Your effective druid level for this animal companion is equal to your cleric level -- 3. (Druids who take this ability through their nature bond class feature use their druid level -- 3 to determine the abilities of their animal companions).
				
\textbf{Domain Spells}: 1st---\textit{calm animals, }2nd---\textit{hold animal, }3rd---\textit{dominate animal, }4th---\textit{summon nature's ally IV} (animals only), 5th---\textit{beast shape III }(animals only)\textit{, }6th---\textit{antilife shell, }7th---\textit{animal shapes, }8th---\textit{summon nature's ally VIII }(animals only), 9th---\textit{shapechange.}
				
\subsection{Artifice Domain}

				
\textbf{Granted Powers}: You can repair damage to objects, animate objects with life, and create objects from nothing.
				
\textit{Artificer's Touch (Sp)}: You can cast \textit{mending }at will, using your cleric level as the caster level to repair damaged objects. In addition, you can cause damage to objects and construct creatures by striking them with a melee touch attack. Objects and constructs take 1d6 points of damage +1 for every two cleric levels you possess. This attack bypasses an amount of damage reduction and hardness equal to your cleric level. You can use this ability a number of times per day equal to 3 + your Wisdom modifier.
				
\textit{Dancing Weapons (Su)}: At 8th level, you can give a weapon touched the \textit{dancing }special weapon quality for 4 rounds. You can use this ability once per day at 8th level, and an additional time per day for every four levels beyond 8th.
				
\textbf{Domain Spells}: 1st---\textit{animate rope, }2nd---\textit{wood shape, }3rd---\textit{stone shape, }4th---\textit{minor creation, }5th---\textit{fabricate, }6th---\textit{major creation, }7th---\textit{wall of iron, }8th---\textit{statue, }9th---\textit{prismatic sphere.}
				
\subsection{Chaos Domain}

				
\textbf{Granted Powers}: Your touch infuses life and weapons with chaos, and you revel in all things anarchic.
				
\textit{Touch of Chaos (Sp)}: You can imbue a target with chaos as a melee touch attack. For the next round, anytime the target rolls a d20, he must roll twice and take the less favorable result. You can use this ability a number of times per day equal to 3 + your Wisdom modifier.
				
\textit{Chaos Blade (Su)}: At 8th level, you can give a weapon touched the\textit{ anarchic }special weapon quality for a number of rounds equal to 1/2 your cleric level. You can use this ability once per day at 8th level, and an additional time per day for every four levels beyond 8th.
				
\textbf{Domain Spells}: 1st---\textit{protection from law, }2nd---\textit{align weapon }(chaos only), 3rd---\textit{magic circle against law, }4th---\textit{chaos hammer, }5th---\textit{dispel law, }6th---\textit{animate objects, }7th---\textit{word of chaos, }8th---\textit{cloak of chaos, }9th---\textit{summon monster IX} (chaos spell only).
				
\subsection{Charm Domain}

				
\textbf{Granted Powers}: You can baffle and befuddle foes with a touch or a smile, and your beauty and grace are divine.
				
\textit{Dazing Touch (Sp)}: You can cause a living creature to become dazed for 1 round as a melee touch attack. Creatures with more Hit Dice than your cleric level are unaffected. You can use this ability a number of times per day equal to 3 + your Wisdom modifier.
				
\textit{Charming Smile (Sp)}: At 8th level, you can cast \textit{charm person} as a swift action, with a DC of 10 + 1/2 your cleric level + your Wisdom modifier. You can only have one creature charmed in this way at a time. The total number of rounds of this effect per day is equal to your cleric level. The rounds do not need to be consecutive, and you can dismiss the charm at any time as a free action. Each attempt to use this ability consumes 1 round of its duration, whether or not the creature succeeds on its save to resist the effect.
				
\textbf{Domain Spells}: 1st---\textit{charm person, }2nd---\textit{calm emotions, }3rd---\textit{suggestion, }4th---\textit{heroism, }5th---\textit{charm monster, }6th---\textit{geas/quest, }7th---\textit{insanity, }8th---\textit{demand, }9th---\textit{dominate monster}. 
				
\subsection{Community Domain}

				
\textbf{Granted Powers}: Your touch can heal wounds, and your presence instills unity and strengthens emotional bonds.
				
\textit{Calming Touch (Sp)}: You can touch a creature as a standard action to heal it of 1d6 points of nonlethal damage + 1 point per cleric level. This touch also removes the fatigued, shaken, and sickened conditions (but has no effect on more severe conditions). You can use this ability a number of times per day equal to 3 + your Wisdom modifier.
				
\textit{Unity (Su)}: At 8th level, whenever a spell or effect targets you and one or more allies within 30 feet, you can use this ability to allow your allies to use your saving throw against the effect in place of their own. Each ally must decide individually before the rolls are made. Using this ability is an immediate action. You can use this ability once per day at 8th level, and one additional time per day for every four cleric levels beyond 8th.
				
\textbf{Domain Spells}: 1st---\textit{bless, }2nd---\textit{shield other, }3rd---\textit{prayer}, 4th---\textit{imbue with spell ability, }5th---\textit{telepathic bond, }6th---\textit{heroes' feast, }7th---\textit{refuge, }8th---\textit{mass cure critical wounds, }9th---\textit{miracle}.
				
\subsection{Darkness Domain}

				
\textbf{Granted Power}: You manipulate shadows and darkness. In addition, you receive Blind-Fight as a bonus feat.
				
\textit{Touch of Darkness (Sp)}: As a melee touch attack, you can cause a creature's vision to be fraught with shadows and darkness. The creature touched treats all other creatures as if they had concealment, suffering a 20\% miss chance on all attack rolls. This effect lasts for a number of rounds equal to 1/2 your cleric level (minimum 1). You can use this ability a number of times per day equal to 3 + your Wisdom modifier.
				
\textit{Eyes of Darkness (Su)}: At 8th level, your vision is not impaired by lighting conditions, even in absolute darkness and magic darkness. You can use this ability for a number of rounds per day equal to 1/2 your cleric level. These rounds do not need to be consecutive.
				
\textbf{Domain Spells}: 1st---\textit{obscuring mist, }2nd---\textit{blindness/deafness} (only to cause blindness)\textit{, }3rd---\textit{deeper darkness, }4th---\textit{shadow conjuration, }5th---\textit{summon monster V }(summons 1d3 shadows), 6th---\textit{shadow walk, }7th---\textit{power word blind, }8th---\textit{greater shadow evocation, }9th---\textit{shades. } 
				
\subsection{Death Domain}

				
\textbf{Granted Powers}: You can cause the living to bleed at a touch, and find comfort in the presence of the dead.
				
\textit{Bleeding Touch (Sp)}: As a melee touch attack, you can cause a living creature to take 1d6 points of damage per round. This effect persists for a number of rounds equal to 1/2 your cleric level (minimum 1) or until stopped with a DC 15 Heal check or any spell or effect that heals damage. You can use this ability a number of times per day equal to 3 + your Wisdom modifier.
				
\textit{Death's Embrace (Ex)}: At 8th level, you heal damage instead of taking damage from channeled negative energy. If the channeled negative energy targets undead, you heal hit points just like undead in the area.
				
\textbf{Domain Spells}: 1st---\textit{cause fear, }2nd---\textit{death knell, }3rd---\textit{animate dead, }4th---\textit{death ward, }5th---\textit{slay living, }6th---\textit{create undead, }7th---\textit{destruction, }8th---\textit{create greater undead, }9th---\textit{wail of the banshee}. 
				
\subsection{Destruction Domain}

				
\textbf{Granted Powers}: You revel in ruin and devastation, and can deliver particularly destructive attacks.
				
\textit{Destructive Smite (Su)}: You gain the destructive smite power: the supernatural ability to make a single melee attack with a morale bonus on damage rolls equal to 1/2 your cleric level (minimum 1). You must declare the destructive smite before making the attack. You can use this ability a number of times per day equal to 3 + your Wisdom modifier.
				
\textit{Destructive Aura (Su)}: At 8th level, you can emit a 30-foot aura of destruction for a number of rounds per day equal to your cleric level. All attacks made against targets in this aura (including you) gain a morale bonus on damage equal to 1/2 your cleric level and all critical threats are automatically confirmed. These rounds do not need to be consecutive.
				
\textbf{Domain Spells}: 1st---\textit{true strike, }2nd---\textit{shatter, }3rd---\textit{rage, }4th---\textit{inflict critical wounds, }5th---\textit{shout}, 6th---\textit{harm, }7th---\textit{disintegrate, }8th---\textit{earthquake, }9th---\textit{implosion.}
				
\subsection{Earth Domain}

				
\textbf{Granted Powers}: You have mastery over earth, metal, and stone, can fire darts of acid, and command earth creatures.
				
\textit{Acid Dart (Sp)}: As a standard action, you can unleash an acid dart targeting any foe within 30 feet as a ranged touch attack. This acid dart deals 1d6 points of acid damage + 1 point for every two cleric levels you possess. You can use this ability a number of times per day equal to 3 + your Wisdom modifier. 
				
\textit{Acid Resistance (Ex)}: At 6th level, you gain resist acid 10. This resistance increases to 20 at 12th level. At 20th level, you gain immunity to acid.
				
\textbf{Domain Spells}: 1st---\textit{magic stone, }2nd---\textit{soften earth and stone, }3rd---\textit{stone shape, }4th---\textit{spike stones, }5th---\textit{wall of stone, }6th---\textit{stoneskin, }7th---\textit{elemental body IV }(earth only)\textit{, }8th---\textit{earthquake, }9th---\textit{elemental swarm }(earth spell only).
				
\subsection{Evil Domain}

				
\textbf{Granted Powers}: You are sinister and cruel, and have wholly pledged your soul to the cause of evil.
				
\textit{Touch of Evil} \textit{(Sp)}: You can cause a creature to become sickened as a melee touch attack. Creatures sickened by your touch count as good for the purposes of spells with the evil descriptor. This ability lasts for a number of rounds equal to 1/2 your cleric level (minimum 1). You can use this ability a number of times per day equal to 3 + your Wisdom modifier.
				
\textit{Scythe of Evil (Su)}: At 8th level, you can give a weapon touched the\textit{ unholy }special weapon quality for a number of rounds equal to 1/2 your cleric level. You can use this ability once per day at 8th level, and an additional time per day for every four levels beyond 8th.
				
\textbf{Domain Spells}: 1st---\textit{protection from good, }2nd---\textit{align weapon }(evil only), 3rd---\textit{magic circle against good, }4th---\textit{unholy blight, }5th---\textit{dispel good, }6th---\textit{create undead, }7th---\textit{blasphemy, }8th---\textit{unholy aura, }9th---\textit{summon monster IX} (evil spell only). 
				
\subsection{Fire Domain}

				
\textbf{Granted Powers}: You can call forth fire, command creatures of the inferno, and your flesh does not burn.
				
\textit{Fire Bolt (Sp)}: As a standard action, you can unleash a scorching bolt of divine fire from your outstretched hand. You can target any single foe within 30 feet as a ranged touch attack with this bolt of fire. If you hit the foe, the fire bolt deals 1d6 points of fire damage + 1 point for every two cleric levels you possess. You can use this ability a number of times per day equal to 3 + your Wisdom modifier. 
				
\textit{Fire Resistance (Ex)}: At 6th level, you gain resist fire 10. This resistance increases to 20 at 12th level. At 20th level, you gain immunity to fire.
				
\textbf{Domain Spells}: 1st---\textit{burning hands, }2nd---\textit{produce flame, }3rd---\textit{fireball, }4th---\textit{wall of fire, }5th---\textit{fire shield, }6th---\textit{fire seeds, }7th---\textit{elemental body IV }(fire only)\textit{, }8th---\textit{incendiary cloud, }9th---\textit{elemental swarm }(fire spell only).
				
\subsection{Glory Domain}

				
\textbf{Granted Powers}: You are infused with the glory of the divine, and are a true foe of the undead. In addition, when you channel positive energy to harm undead creatures, the save DC to halve the damage is increased by 2.
				
\textit{Touch of Glory (Sp)}: You can cause your hand to shimmer with divine radiance, allowing you to touch a creature as a standard action and give it a bonus equal to your cleric level on a single Charisma-based skill check or Charisma ability check. This ability lasts for 1 hour or until the creature touched elects to apply the bonus to a roll. You can use this ability to grant the bonus a number of times per day equal to 3 + your Wisdom modifier.
				
\textit{Divine Presence (Su)}: At 8th level, you can emit a 30-foot aura of divine presence for a number of rounds per day equal to your cleric level. All allies within this aura are treated as if under the effects of a \textit{sanctuary }spell with a DC equal to 10 + 1/2 your cleric level + your Wisdom modifier. These rounds do not need to be consecutive. Activating this ability is a standard action. If an ally leaves the area or makes an attack, the effect ends for that ally. If you make an attack, the effect ends for you and your allies.
				
\textbf{Domain Spells}: 1st---\textit{shield of faith, }2nd---\textit{bless weapon, }3rd---\textit{searing light, }4th---\textit{holy smite, }5th---\textit{righteous might, }6th---\textit{undeath to death, }7th---\textit{holy sword, }8th---\textit{holy aura, }9th---\textit{gate.}
				
\subsection{Good Domain}

				
\textbf{Granted Powers}: You have pledged your life and soul to goodness and purity.
				
\textit{Touch of Good} \textit{(Sp)}: You can touch a creature as a standard action, granting a sacred bonus on attack rolls, skill checks, ability checks, and saving throws equal to half your cleric level (minimum 1) for 1 round. You can use this ability a number of times per day equal to 3 + your Wisdom modifier.
				
\textit{Holy Lance (Su)}: At 8th level, you can give a weapon you touch the\textit{ holy }special weapon quality for a number of rounds equal to 1/2 your cleric level. You can use this ability once per day at 8th level, and an additional time per day for every four levels beyond 8th.
				
\textbf{Domain Spells}: 1st---\textit{protection from evil, }2nd---\textit{align weapon }(good only), 3rd---\textit{magic circle against evil, }4th---\textit{holy smite, }5th---\textit{dispel evil, }6th---\textit{blade barrier, }7th---\textit{holy word}, 8th---\textit{holy aura, }9th---\textit{summon monster IX} (good spell only).
				
\subsection{Healing Domain}

				
\textbf{Granted Powers}: Your touch staves off pain and death, and your healing magic is particularly vital and potent.
				
\textit{Rebuke Death (Sp)}: You can touch a living creature as a standard action, healing it for 1d4 points of damage plus 1 for every two cleric levels you possess. You can only use this ability on a creature that is below 0 hit points. You can use this ability a number of times per day equal to 3 + your Wisdom modifier.
				
\textit{Healer's Blessing} \textit{(Su)}: At 6th level, all of your cure spells are treated as if they were empowered, increasing the amount of damage healed by half (+50\%). This does not apply to damage dealt to undead with a cure spell. This does not stack with the Empower Spell metamagic feat.
				
\textbf{Domain Spells}: 1st---\textit{cure light wounds, }2nd---\textit{cure moderate wounds}, 3rd---\textit{cure serious wounds, }4th---\textit{cure critical wounds, }5th---\textit{breath of life, }6th---\textit{heal, }7th---\textit{regenerate, }8th---\textit{mass cure critical wounds, }9th---\textit{mass heal}.
				
\subsection{Knowledge Domain}

				
\textbf{Granted Powers}: You are a scholar and a sage of legends. In addition, you treat all Knowledge skills as class skills.
				
\textit{Lore Keeper (Sp)}: You can touch a creature to learn about its abilities and weaknesses. With a successful touch attack, you gain information as if you made the appropriate Knowledge skill check with a result equal to 15 + your cleric level + your Wisdom modifier.
				
\textit{Remote Viewing} \textit{(Sp)}: Starting at 6th level, you can use \textit{clairvoyance/clairaudience} at will as a spell-like ability using your cleric level as the caster level. You can use this ability for a number of rounds per day equal to your cleric level. These rounds do not need to be consecutive.
				
\textbf{Domain Spells}: 1st---\textit{comprehend languages, }2nd---\textit{detect thoughts}, 3rd---\textit{speak with dead, }4th---\textit{divination, }5th---\textit{true seeing, }6th---\textit{find the path, }7th---\textit{legend lore, }8th---\textit{discern location, }9th---\textit{foresight}.
				
\subsection{Law Domain}

				
\textbf{Granted Powers}: You follow a strict and ordered code of laws, and in so doing, achieve enlightenment.
				
\textit{Touch of Law} \textit{(Sp)}: You can touch a willing creature as a standard action, infusing it with the power of divine order and allowing it to treat all attack rolls, skill checks, ability checks, and saving throws for 1 round as if the natural d20 roll resulted in an 11. You can use this ability a number of times per day equal to 3 + your Wisdom modifier. 
				
\textit{Staff of Order (Su)}: At 8th level, you can give a weapon touched the\textit{ axiomatic }special weapon quality for a number of rounds equal to 1/2 your cleric level. You can use this ability once per day at 8th level, and an additional time per day for every four levels beyond 8th.
				
\textbf{Domain Spells}: 1st---\textit{protection from chaos, }2nd---\textit{align weapon }(law only), 3rd---\textit{magic circle against chaos, }4th---\textit{order's wrath, }5th---\textit{dispel chaos, }6th---\textit{hold monster, }7th---\textit{dictum, }8th---\textit{shield of law, }9th---\textit{summon monster IX} (law spell only). 
				
\subsection{Liberation Domain}

				
\textbf{Granted Powers}: You are a spirit of freedom and a staunch foe against all who would enslave and oppress.
				
\textit{Liberation (Su)}: You have the ability to ignore impediments to your mobility. For a number of rounds per day equal to your cleric level, you can move normally regardless of magical effects that impede movement, as if you were affected by \textit{freedom of movement.} This effect occurs automatically as soon as it applies. These rounds do not need to be consecutive.
				
\textit{Freedom's Call (Su)}: At 8th level, you can emit a 30-foot aura of freedom for a number of rounds per day equal to your cleric level. Allies within this aura are not affected by the confused, grappled, frightened, panicked, paralyzed, pinned, or shaken conditions. This aura only suppresses these effects, and they return once a creature leaves the aura or when the aura ends, if applicable. These rounds do not need to be consecutive. 
				
\textbf{Domain Spells}: 1st---\textit{remove fear, }2nd---\textit{remove paralysis}, 3rd---\textit{remove curse, }4th---\textit{freedom of movement, }5th---\textit{break enchantment, }6th---\textit{greater dispel magic, }7th---\textit{refuge, }8th---\textit{mind blank, }9th---\textit{freedom}.
				
\subsection{Luck Domain}

				
\textbf{Granted Powers}: You are infused with luck, and your mere presence can spread good fortune.
				
\textit{Bit of Luck (Sp)}: You can touch a willing creature as a standard action, giving it a bit of luck. For the next round, any time the target rolls a d20, he may roll twice and take the more favorable result. You can use this ability a number of times per day equal to 3 + your Wisdom modifier.
				
\textit{Good Fortune (Ex)}: At 6th level, as an immediate action, you can reroll any one d20 roll that you have just made before the results of the roll are revealed. You must take the result of the reroll, even if it's worse than the original roll. You can use this ability once per day at 6th level, and one additional time per day for every six cleric levels beyond 6th.
				
\textbf{Domain Spells}: 1st---\textit{true strike, }2nd---\textit{aid}, 3rd---\textit{protection from energy, }4th---\textit{freedom of movement, }5th---\textit{break enchantment, }6th---\textit{mislead, }7th---\textit{spell turning, }8th---\textit{moment of prescience, }9th---\textit{miracle}.
				
\subsection{Madness Domain}

				
\textbf{Granted Powers}: You embrace the madness that lurks deep in your heart, and can unleash it to drive your foes insane or to sacrifice certain abilities to hone others.
				
\textit{Vision of Madness (Sp)}: You can give a creature a \textit{vision of madness} as a melee touch attack. Choose one of the following: attack rolls, saving throws, or skill checks. The target receives a bonus to the chosen rolls equal to 1/2 your cleric level (minimum +1) and a penalty to the other two types of rolls equal to 1/2 your cleric level (minimum --1). This effect fades after 3 rounds. You can use this ability a number of times per day equal to 3 + your Wisdom modifier.
				
\textit{Aura of Madness (Su)}: At 8th level, you can emit a 30-foot aura of madness for a number of rounds per day equal to your cleric level. Enemies within this aura are affected by \textit{confusion} unless they make a Will save with a DC equal to 10 + 1/2 your cleric level + your Wisdom modifier. The \textit{confusion }effect ends immediately when the creature leaves the area or the aura expires. Creatures that succeed on their saving throw are immune to this aura for 24 hours. These rounds do not need to be consecutive.
				
\textbf{Domain Spells}: 1st---\textit{lesser confusion, }2nd---\textit{touch of idiocy,} 3rd---\textit{rage, }4th---\textit{confusion, }5th---\textit{nightmare, }6th---\textit{phantasmal killer, }7th---\textit{insanity, }8th---\textit{scintillating pattern, }9th---\textit{weird}.
				
\subsection{Magic Domain}

				
\textbf{Granted Powers}: You are a true student of all things mystical, and see divinity in the purity of magic.
				
\textit{Hand of the Acolyte (Su)}: You can cause your melee weapon to fly from your grasp and strike a foe before instantly returning. As a standard action, you can make a single attack using a melee weapon at a range of 30 feet. This attack is treated as a ranged attack with a thrown weapon, except that you add your Wisdom modifier to the attack roll instead of your Dexterity modifier (damage still relies on Strength). This ability cannot be used to perform a combat maneuver. You can use this ability a number of times per day equal to 3 + your Wisdom modifier.
				
\textit{Dispelling Touch} \textit{(Sp)}: At 8th level, you can use a targeted \textit{dispel magic} effect as a melee touch attack. You can use this ability once per day at 8th level and one additional time per day for every four cleric levels beyond 8th.
				
\textbf{Domain Spells}: 1st---\textit{identify, }2nd---\textit{magic mouth,} 3rd---\textit{dispel magic, }4th---\textit{imbue with spell ability, }5th---\textit{spell resistance, }6th---\textit{antimagic field, }7th---\textit{spell turning, }8th---\textit{protection from spells, }9th---\textit{mage's disjunction}.
				
\subsection{Nobility Domain}

				
\textbf{Granted Powers}: You are a great leader, an inspiration to all who follow the teachings of your faith.
				
\textit{Inspiring Word (Sp)}: As a standard action, you can speak an inspiring word to a creature within 30 feet. That creature receives a +2 morale bonus on attack rolls, skill checks, ability checks, and saving throws for a number of rounds equal to 1/2 your cleric level (minimum 1). You can use this power a number of times per day equal to 3 + your Wisdom modifier.
				
\textit{Leadership (Ex)}: At 8th level, you receive Leadership as a bonus feat. In addition, you gain a +2 bonus on your leadership score as long as you uphold the tenets of your deity (or divine concept if you do not venerate a deity). 
				
\textbf{Domain Spells}: 1st---\textit{divine favor, }2nd---\textit{enthrall,} 3rd---\textit{magic vestment, }4th---\textit{discern lies, }5th---\textit{greater command, }6th---\textit{geas/quest, }7th---\textit{repulsion, }8th---\textit{demand, }9th---\textit{storm of vengeance}.
				
\subsection{Plant Domain}

				
\textbf{Granted Powers}: You find solace in the green, can grow defensive thorns, and can communicate with plants.
				
\textit{Wooden Fist (Su)}: As a free action, your hands can become as hard as wood, covered in tiny thorns. While you have wooden fists, your unarmed strikes do not provoke attacks of opportunity, deal lethal damage, and gain a bonus on damage rolls equal to 1/2 your cleric level (minimum +1). You can use this ability for a number of rounds per day equal to 3 + your Wisdom modifier. These rounds do not need to be consecutive.
				
\textit{Bramble Armor (Su)}: At 6th level, you can cause a host of wooden thorns to burst from your skin as a free action. While bramble armor is in effect, any foe striking you with an unarmed strike or a melee weapon without reach takes 1d6 points of piercing damage + 1 point per two cleric levels you possess. You can use this ability for a number of rounds per day equal to your cleric level. These rounds do not need to be consecutive.
				
\textbf{Domain Spells}: 1st---\textit{entangle, }2nd---\textit{barkskin,} 3rd---\textit{plant growth, }4th---\textit{command plants, }5th---\textit{wall of thorns, }6th---\textit{repel wood, }7th---\textit{animate plants, }8th---\textit{control plants, }9th---\textit{shambler}.
				
\subsection{Protection Domain}

				
\textbf{Granted Powers}: Your faith is your greatest source of protection, and you can use that faith to defend others. In addition, you receive a +1 resistance bonus on saving throws. This bonus increases by 1 for every 5 levels you possess. 
				
\textit{Resistant Touch (Sp)}: As a standard action, you can touch an ally to grant him your resistance bonus for 1 minute. When you use this ability, you lose your resistance bonus granted by the Protection domain for 1 minute. You can use this ability a number of times per day equal to 3 + your Wisdom modifier.
				
\textit{Aura of Protection (Su)}: At 8th level, you can emit a 30-foot aura of protection for a number of rounds per day equal to your cleric level. You and your allies within this aura gain a +1 deflection bonus to AC and resistance 5 against all elements (acid, cold, electricity, fire, and sonic). The deflection bonus increases by +1 for every four cleric levels you possess beyond 8th. At 14th level, the resistance against all elements increases to 10. These rounds do not need to be consecutive.
				
\textbf{Domain Spells}: 1st---\textit{sanctuary, }2nd---\textit{shield other,} 3rd---\textit{protection from energy, }4th---\textit{spell immunity, }5th---\textit{spell resistance, }6th---\textit{antimagic field, }7th---\textit{repulsion, }8th---\textit{mind blank, }9th---\textit{prismatic sphere}.
				
\subsection{Repose Domain}

				
\textbf{Granted Powers}: You see death not as something to be feared, but as a final rest and reward for a life well spent. The taint of undeath is a mockery of what you hold dear.
				
\textit{Gentle Rest (Sp)}: Your touch can fill a creature with lethargy, causing a living creature to become staggered for 1 round as a melee touch attack. If you touch a staggered living creature, that creature falls asleep for 1 round instead. Undead creatures touched are staggered for a number of rounds equal to your Wisdom modifier. You can use this ability a number of times per day equal to 3 + your Wisdom modifier.
				
\textit{Ward Against Death (Su)}: At 8th level, you can emit a 30-foot aura that wards against death for a number of rounds per day equal to your cleric level. Living creatures in this area are immune to all death effects, energy drain, and effects that cause negative levels. This ward does not remove negative levels that a creature has already gained, but the negative levels have no effect while the creature is inside the warded area. These rounds do not need to be consecutive.
				
\textbf{Domain Spells}: 1st---\textit{deathwatch, }2nd---\textit{gentle repose,} 3rd---\textit{speak with dead, }4th---\textit{death ward, }5th---\textit{slay living, }6th---\textit{undeath to death, }7th---\textit{destruction, }8th---\textit{waves of exhaustion, }9th---\textit{wail of the banshee}.
				
\subsection{Rune Domain}

				
\textbf{Granted Powers}: In strange and eldritch runes you find potent magic. You gain Scribe Scroll as a bonus feat.
				
\textit{Blast Rune (Sp)}: As a standard action, you can create a blast rune in any adjacent square. Any creature entering this square takes 1d6 points of damage + 1 point for every two cleric levels you possess. This rune deals either acid, cold, electricity, or fire damage, decided when you create the rune. The rune is invisible and lasts a number of rounds equal to your cleric level or until discharged. You cannot create a blast rune in a square occupied by another creature. This rune counts as a 1st-level spell for the purposes of dispelling. It can be discovered with a DC 26 Perception skill check and disarmed with a DC 26 Disable Device skill check. You can use this ability a number of times per day equal to 3 + your Wisdom modifier.
				
\textit{Spell Rune (Sp)}: At 8th level, you can attach another spell that you cast to one of your blast runes, causing that spell to affect the creature that triggers the rune, in addition to the damage. This spell must be of at least one level lower than the highest-level cleric spell you can cast and it must target one or more creatures. Regardless of the number of targets the spell can normally affect, it only affects the creature that triggers the rune.
				
\textbf{Domain Spells}: 1st---\textit{erase, }2nd---\textit{secret page,} 3rd---\textit{glyph of warding, }4th---\textit{explosive runes, }5th---\textit{lesser planar binding, }6th---\textit{greater glyph of warding, }7th---\textit{instant summons, }8th---\textit{symbol of death, }9th---\textit{teleportation circle}.
				
\subsection{Strength Domain}

				
\textbf{Granted Powers}: In strength and brawn there is truth---your faith gives you incredible might and power. 
				
\textit{Strength Surge (Sp)}: As a standard action, you can touch a creature to give it great strength. For 1 round, the target gains an enhancement bonus equal to 1/2 your cleric level (minimum +1) to melee attacks, combat maneuver checks that rely on Strength, Strength-based skills, and Strength checks. You can use this ability a number of times per day equal to 3 + your Wisdom modifier.
				
\textit{Might of the Gods (Su)}: At 8th level, you can add your cleric level as an enhancement bonus to your Strength score for a number of rounds per day equal to your cleric level. This bonus only applies on Strength checks and Strength-based skill checks. These rounds do not need to be consecutive.
				
\textbf{Domain Spells}: 1st---\textit{enlarge person, }2nd---\textit{bull's strength,} 3rd---\textit{magic vestment, }4th---\textit{spell immunity, }5th---\textit{righteous might, }6th---\textit{stoneskin, }7th---\textit{grasping hand, }8th---\textit{clenched fist, }9th---\textit{crushing hand.}
				
\subsection{Sun Domain}

				
\textbf{Granted Powers}: You see truth in the pure and burning light of the sun, and can call upon its blessing or wrath to work great deeds.
				
\textit{Sun's Blessing (Su)}: Whenever you channel positive energy to harm undead creatures, add your cleric level to the damage dealt. Undead do not add their channel resistance to their saves when you channel positive energy.
				
\textit{Nimbus of Light (Su)}: At 8th level, you can emit a 30-foot nimbus of light for a number of rounds per day equal to your cleric level. This acts as a \textit{daylight }spell. In addition, undead within this radius take an amount of damage equal to your cleric level each round that they remain inside the nimbus. Spells and spell-like abilities with the darkness descriptor are automatically dispelled if brought inside this nimbus. These rounds do not need to be consecutive.
				
\textbf{Domain Spells}: 1st---\textit{endure elements, }2nd---\textit{heat metal,} 3rd---\textit{searing light, }4th---\textit{fire shield, }5th---\textit{flame strike, }6th---\textit{fire seeds, }7th---\textit{sunbeam, }8th---\textit{sunburst, }9th---\textit{prismatic sphere.}
				
\subsection{Travel Domain}

				
\textbf{Granted Powers}: You are an explorer and find enlightenment in the simple joy of travel, be it by foot or conveyance or magic. Increase your base speed by 10 feet.
				
\textit{Agile Feet (Su)}: As a free action, you can gain increased mobility for 1 round. For the next round, you ignore all difficult terrain and do not take any penalties for moving through it. You can use this ability a number of times per day equal to 3 + your Wisdom modifier.
				
\textit{Dimensional Hop (Sp)}: At 8th level, you can teleport up to 10 feet per cleric level per day as a move action. This teleportation must be used in 5-foot increments and such movement does not provoke attacks of opportunity. You must have line of sight to your destination to use this ability. You can bring other willing creatures with you, but you must expend an equal amount of distance for each creature brought.
				
\textbf{Domain Spells}: 1st---\textit{longstrider, }2nd---\textit{locate object,} 3rd---\textit{fly, }4th---\textit{dimension door, }5th---\textit{teleport, }6th---\textit{find the path, }7th---\textit{greater teleport, }8th---\textit{phase door, }9th---\textit{astral projection.}
				
\subsection{Trickery Domain}

				
\textbf{Granted Powers}: You are a master of illusions and deceptions. Bluff, Disguise, and Stealth are class skills.
				
\textit{Copycat (Sp)}: You can create an illusory double of yourself as a move action. This double functions as a single \textit{mirror image} and lasts for a number of rounds equal to your cleric level, or until the illusory duplicate is dispelled or destroyed. You can have no more than one copycat at a time. This ability does not stack with the \textit{mirror image} spell. You can use this ability a number of times per day equal to 3 + your Wisdom modifier.
				
\textit{Master's Illusion (Sp)}: At 8th level, you can create an illusion that hides the appearance of yourself and any number of allies within 30 feet for 1 round per cleric level. This ability otherwise functions like the spell \textit{veil}. The save DC to disbelieve this effect is equal to 10 + 1/2 your cleric level + your Wisdom modifier. The rounds do not need to be consecutive.
				
\textbf{Domain Spells}: 1st---\textit{disguise self, }2nd---\textit{invisibility,} 3rd---\textit{nondetection, }4th---\textit{confusion, }5th---\textit{false vision, }6th---\textit{mislead, }7th---\textit{screen, }8th---\textit{mass invisibility, }9th---\textit{time stop.}
				
\subsection{War Domain}

				
\textbf{Granted Powers}: You are a crusader for your god, always ready and willing to fight to defend your faith.
				
\textit{Battle Rage (Sp)}: You can touch a creature as a standard action to give it a bonus on melee damage rolls equal to 1/2 your cleric level for 1 round (minimum +1). You can do so a number of times per day equal to 3 + your Wisdom modifier.
				
\textit{Weapon Master (Su)}: At 8th level, as a swift action, you gain the use of one combat feat for a number of rounds per day equal to your cleric level. These rounds do not need to be consecutive and you can change the feat chosen each time you use this ability. You must meet the prerequisites to use this feat.
				
\textbf{Domain Spells}: 1st---\textit{magic weapon, }2nd---\textit{spiritual weapon,} 3rd---\textit{magic vestment, }4th---\textit{divine power, }5th---\textit{flame strike, }6th---\textit{blade barrier, }7th---\textit{power word blind, }8th---\textit{power word stun, }9th---\textit{power word kill.}
				
\subsection{Water Domain}

				
\textbf{Granted Powers}: You can manipulate water and mist and ice, conjure creatures of water, and resist cold.
				
\textit{Icicle (Sp)}: As a standard action, you can fire an icicle from your finger, targeting any foe within 30 feet as a ranged touch attack. The icicle deals 1d6 points of cold damage + 1 point for every two cleric levels you possess. You can use this ability a number of times per day equal to 3 + your Wisdom modifier.
				
\textit{Cold Resistance (Ex)}: At 6th level, you gain resist cold 10. This resistance increases to 20 at 12th level. At 20th level, you gain immunity to cold.
				
\textbf{Domain Spells}: 1st---\textit{obscuring mist, }2nd---\textit{fog cloud, }3rd---\textit{water breathing, }4th---\textit{control water, }5th---\textit{ice storm, }6th---\textit{cone of cold, }7th---\textit{elemental body IV }(water only)\textit{, }8th---\textit{horrid wilting, }9th---\textit{elemental swarm }(water spell only).
				
\subsection{Weather Domain}

				
\textbf{Granted Powers}: With power over storm and sky, you can call down the wrath of the gods upon the world below.
				
\textit{Storm Burst (Sp)}: As a standard action, you can create a storm burst targeting any foe within 30 feet as a ranged touch attack. The storm burst deals 1d6 points of nonlethal damage + 1 point for every two cleric levels you possess. In addition, the target is buffeted by winds and rain, causing it to take a --2 penalty on attack rolls for 1 round. You can use this ability a number of times per day equal to 3 + your Wisdom modifier.
				
\textit{Lightning Lord (Sp)}: At 8th level, you can call down a number of bolts of lightning per day equal to your cleric level. You can call down as many bolts as you want with a single standard action, but no creature can be the target of more than one bolt and no two targets can be more than 30 feet apart. This ability otherwise functions as \textit{call lightning}.
				
\textbf{Domain Spells}: 1st---\textit{obscuring mist, }2nd---\textit{fog cloud, }3rd---\textit{call lightning, }4th---\textit{sleet storm, }5th---\textit{ice storm, }6th---\textit{control winds, }7th---\textit{control weather, }8th---\textit{whirlwind, }9th---\textit{storm of vengeance}.
        	

\section{Druid}

\label{f0}				
Within the purity of the elements and the order of the wilds lingers a power beyond the marvels of civilization. Furtive yet undeniable, these primal magics are guarded over by servants of philosophical balance known as druids. Allies to beasts and manipulators of nature, these often misunderstood protectors of the wild strive to shield their lands from all who would threaten them and prove the might of the wilds to those who lock themselves behind city walls. Rewarded for their devotion with incredible powers, druids gain unparalleled shape-shifting abilities, the companionship of mighty beasts, and the power to call upon nature's wrath. The mightiest temper powers akin to storms, earthquakes, and volcanoes with primeval wisdom long abandoned and forgotten by civilization. 
				
\textbf{Role}: While some druids might keep to the fringe of battle, allowing companions and summoned creatures to fight while they confound foes with the powers of nature, others transform into deadly beasts and savagely wade into combat. Druids worship personifications of elemental forces, natural powers, or nature itself. Typically this means devotion to a nature deity, though druids are just as likely to revere vague spirits, animalistic demigods, or even specific awe-inspiring natural wonders.
				
\textbf{Alignment}: Any neutral.
				
\textbf{Hit Die}: d8.
				
\subsection{Class Skills}

				
The druid's class skills are Climb (Str), Craft (Int), Fly (Dex), Handle Animal (Cha), Heal (Wis), Knowledge (geography) (Int), Knowledge (nature) (Int), Perception (Wis), Profession (Wis), Ride (Dex), Spellcraft (Int), Survival (Wis), and Swim (Str).
				
\textbf{Skill Ranks per Level}: 4 + Int modifier.
% <div class="table">
% <
\begin{table*}[]
\caption{Table: Druid}
\sffamily
\setlength{\tabcolsep}{1pt}
\begin{tabularx}{\linewidth}{lp{6em}p{2.5em}p{2.5em}p{2.5em}Xllllllllll}
\multirow{2}{*}{\textbf{Level}} & \multirow{2}{*}{\parbox{5em}{\textbf{Base Attack Bonus}}} & \multirow{2}{*}{\parbox{1.5em}{\textbf{Fort Save}}} & \multirow{2}{*}{\parbox{1.5em}{\textbf{Ref Save}}} & \multirow{2}{*}{\parbox{1.5em}{\textbf{Will Save}}} & \textbf{Special}     & \multicolumn{10}{c}{\textbf{Spells per day}} \\
                       &                                    &                            &                           &                            &                                                                                                  &  \textbf{0} & \textbf{1st} & \textbf{2nd} & \textbf{3rd} & \textbf{4th} & \textbf{5th} & \textbf{6th} & \textbf{7th} & \textbf{8th} & \textbf{9th} \\
1st & +0 & +2 & +0 & +2 & Nature bond, nature sense, orisons, wild empathy & 3 & 1 & - & - & - & - & - & - & - & -\\
2nd & +1 & +3 & +0 & +3 & Woodland stride & 4 & 2 & - & - & - & - & - & - & - & -\\
3rd & +2 & +3 & +1 & +3 & Trackless step & 4 & 2 & 1 & - & - & - & - & - & - & -\\
4th & +3 & +4 & +1 & +4 & Resist nature's lure, Wild shape & 4 & 3 & 2 & - & - & - & - & - & - & -\\
5th & +3 & +4 & +1 & +4 &  & 4 & 3 & 2 & 1 & - & - & - & - & - & -\\
6th & +4 & +5 & +2 & +5 & Wild shape & 4 & 3 & 3 & 2 & - & - & - & - & - & -\\
7th & +5 & +5 & +2 & +5 &  & 4 & 4 & 3 & 2 & 1 & - & - & - & - & -\\
8th & +6/+1 & +6 & +2 & +6 & Wild shape & 4 & 4 & 3 & 3 & 2 & - & - & - & - & -\\
9th & +6/+1 & +6 & +3 & +6 & Venom immunity & 4 & 4 & 4 & 3 & 2 & 1 & - & - & - & -\\
10th & +7/+2 & +7 & +3 & +7 & Wild shape & 4 & 4 & 4 & 3 & 3 & 2 & - & - & - & -\\
11th & +8/+3 & +7 & +3 & +7 &  & 4 & 4 & 4 & 4 & 3 & 2 & 1 & - & - & -\\
12th & +9/+4 & +8 & +4 & +8 & Wild shape & 4 & 4 & 4 & 4 & 3 & 3 & 2 & - & - & -\\
13th & +9/+4 & +8 & +4 & +8 & A thousand faces & 4 & 4 & 4 & 4 & 4 & 3 & 2 & 1 & - & -\\
14th & +10/+5 & +9 & +4 & +9 & Wild shape & 4 & 4 & 4 & 4 & 4 & 3 & 3 & 2 & - & -\\
15th & +11/+6/+1 & +9 & +5 & +9 & Timeless body & 4 & 4 & 4 & 4 & 4 & 4 & 3 & 2 & 1 & -\\
16th & +12/+7/+2 & +10 & +5 & +10 & Wild shape & 4 & 4 & 4 & 4 & 4 & 4 & 3 & 3 & 2 & -\\
17th & +12/+7/+2 & +10 & +5 & +10 &  & 4 & 4 & 4 & 4 & 4 & 4 & 4 & 3 & 2 & 1\\
18th & +13/+8/+3 & +11 & +6 & +11 & Wild shape & 4 & 4 & 4 & 4 & 4 & 4 & 4 & 3 & 3 & 2\\
19th & +14/+9/+4 & +11 & +6 & +11 &  & 4 & 4 & 4 & 4 & 4 & 4 & 4 & 4 & 3 & 3\\
20th & +15/+10/+5 & +12 & +6 & +12 & Wild shape & 4 & 4 & 4 & 4 & 4 & 4 & 4 & 4 & 4 & 4\\
\end{tabularx}
\end{table*}


				
\subsection{Class Features}

				
All of the following are class features of the druid.
				
\textbf{Weapon and Armor Proficiency}: Druids are proficient with the following weapons: club, dagger, dart, quarterstaff, scimitar, scythe, sickle, shortspear, sling, and spear. They are also proficient with all natural attacks (claw, bite, and so forth) of any form they assume with wild shape (see below).
				
Druids are proficient with light and medium armor but are prohibited from wearing metal armor; thus, they may wear only padded, leather, or hide armor. A druid may also wear wooden armor that has been altered by the \textit{ironwood} spell so that it functions as though it were steel. Druids are proficient with shields (except tower shields) but must use only those crafted from wood.
				
A druid who wears prohibited armor or uses a prohibited shield is unable to cast druid spells or use any of her supernatural or spell-like class abilities while doing so and for 24 hours thereafter.
				
\textbf{Spells}: A druid casts divine spells which are drawn from the druid spell list presented in Spell Lists. Her alignment may restrict her from casting certain spells opposed to her moral or ethical beliefs; see Chaotic, Evil, Good, and Lawful Spells. A druid must choose and prepare her spells in advance.
				
To prepare or cast a spell, the druid must have a Wisdom score equal to at least 10 + the spell level. The Difficulty Class for a saving throw against a druid's spell is 10 + the spell level + the druid's Wisdom modifier.
				
Like other spellcasters, a druid can cast only a certain number of spells of each spell level per day. Her base daily spell allotment is given on Table: Druid. In addition, she receives bonus spells per day if she has a high Wisdom score (see Table: Ability Modifiers and Bonus Spells).
				
A druid must spend 1 hour each day in a trance-like meditation on the mysteries of nature to regain her daily allotment of spells. A druid may prepare and cast any spell on the druid spell list, provided that she can cast spells of that level, but she must choose which spells to prepare during her daily meditation.
				
\textbf{Spontaneous Casting}: A druid can channel stored spell energy into summoning spells that she hasn't prepared ahead of time. She can \^alose\^a a prepared spell in order to cast any \textit{summon nature's ally} spell of the same level or lower. 
				
\textbf{Chaotic, Evil, Good, and Lawful Spells}: A druid can't cast spells of an alignment opposed to her own or her deity's (if she has one). Spells associated with particular alignments are indicated by the chaos, evil, good, and law descriptors in their spell descriptions.
				
\textbf{Orisons}: Druids can prepare a number of orisons, or 0-level spells, each day, as noted on Table: Druid under \^aSpells per Day.\^a These spells are cast like any other spell, but they are not expended when cast and may be used again.
				
\textbf{Bonus Languages}: A druid's bonus language options include Sylvan, the language of woodland creatures. This choice is in addition to the bonus languages available to the character because of her race.
				
A druid also knows Druidic, a secret language known only to druids, which she learns upon becoming a 1st-level druid. Druidic is a free language for a druid; that is, she knows it in addition to her regular allotment of languages and it doesn't take up a language slot. Druids are forbidden to teach this language to nondruids.
				
Druidic has its own alphabet.
				
\textbf{Nature Bond (Ex)}: At 1st level, a druid forms a bond with nature. This bond can take one of two forms. The first is a close tie to the natural world, granting the druid one of the following cleric domains: Air, Animal, Earth, Fire, Plant, Water, or Weather. When determining the powers and bonus spells granted by this domain, the druid's effective cleric level is equal to her druid level. A druid that selects this option also receives additional domain spell slots, just like a cleric. She must prepare the spell from her domain in this slot and this spell cannot be used to cast a spell spontaneously.
				
The second option is to form a close bond with an animal companion. A druid may begin play with any of the animals listed in Animal Choices. This animal is a loyal companion that accompanies the druid on her adventures.
				
Unlike normal animals of its kind, an animal companion's Hit Dice, abilities, skills, and feats advance as the druid advances in level. If a character receives an animal companion from more than one source, her effective druid levels stack for the purposes of determining the statistics and abilities of the companion. Most animal companions increase in size when their druid reaches 4th or 7th level, depending on the companion. If a druid releases her companion from service, she may gain a new one by performing a ceremony requiring 24 uninterrupted hours of prayer in the environment where the new companion typically lives. This ceremony can also replace an animal companion that has perished.
				
\textbf{Nature Sense (Ex)}: A druid gains a +2 bonus on Knowledge (nature) and Survival checks.
				
\textbf{Wild Empathy (Ex)}: A druid can improve the attitude of an animal. This ability functions just like a Diplomacy check made to improve the attitude of a person (see Using Skills). The druid rolls 1d20 and adds her druid level and her Charisma modifier to determine the wild empathy check result. The typical domestic animal has a starting attitude of indifferent, while wild animals are usually unfriendly.
				
To use wild empathy, the druid and the animal must be within 30 feet of one another under normal conditions. Generally, influencing an animal in this way takes 1 minute but, as with influencing people, it might take more or less time.
				
A druid can also use this ability to influence a magical beast with an Intelligence score of 1 or 2, but she takes a --4 penalty on the check.
				
\textbf{Woodland Stride (Ex)}: Starting at 2nd level, a druid may move through any sort of undergrowth (such as natural thorns, briars, overgrown areas, and similar terrain) at her normal speed and without taking damage or suffering any other impairment. Thorns, briars, and overgrown areas that have been magically manipulated to impede motion, however, still affect her.
				
\textbf{Trackless Step (Ex)}: Starting at 3rd level, a druid leaves no trail in natural surroundings and cannot be tracked. She may choose to leave a trail if so desired.
				
\textbf{Resist Nature's Lure (Ex)}: Starting at 4th level, a druid gains a +4 bonus on saving throws against the spell-like and supernatural abilities of fey. This bonus also applies to spells and effects that utilize or target plants, such as \textit{blight, entangle, spike growth, }and \textit{warp wood.}
				
\textbf{Wild Shape (Su)}: At 4th level, a druid gains the ability to turn herself into any Small or Medium animal and back again once per day. Her options for new forms include all creatures with the animal type. This ability functions like the \textit{beast shape I} spell, except as noted here. The effect lasts for 1 hour per druid level, or until she changes back. Changing form (to animal or back) is a standard action and doesn't provoke an attack of opportunity. The form chosen must be that of an animal with which the druid is familiar. 
				
A druid loses her ability to speak while in animal form because she is limited to the sounds that a normal, untrained animal can make, but she can communicate normally with other animals of the same general grouping as her new form. (The normal sound a wild parrot makes is a squawk, so changing to this form does not permit speech.)
				
A druid can use this ability an additional time per day at 6th level and every two levels thereafter, for a total of eight times at 18th level. At 20th level, a druid can use wild shape at will. As a druid gains levels, this ability allows the druid to take on the form of larger and smaller animals, elementals, and plants. Each form expends one daily use of this ability, regardless of the form taken. 
				
At 6th level, a druid can also use wild shape to change into a Large or Tiny animal or a Small elemental. When taking the form of an animal, a druid's wild shape now functions as \textit{beast shape II}. When taking the form of an elemental, the druid's wild shape functions as \textit{elemental body I}.
				
At 8th level, a druid can also use wild shape to change into a Huge or Diminutive animal, a Medium elemental, or a Small or Medium plant creature. When taking the form of animals, a druid's wild shape now functions as \textit{beast shape III}. When taking the form of an elemental, the druid's wild shape now functions as \textit{elemental body II}. When taking the form of a plant creature, the druid's wild shape functions as \textit{plant shape I}.
				
At 10th level, a druid can also use wild shape to change into a Large elemental or a Large plant creature. When taking the form of an elemental, the druid's wild shape now functions as \textit{elemental body III}. When taking the form of a plant, the druid's wild shape now functions as \textit{plant shape II}.
				
At 12th level, a druid can also use wild shape to change into a Huge elemental or a Huge plant creature. When taking the form of an elemental, the druid's wild shape now functions as \textit{elemental body IV}. When taking the form of a plant, the druid's wild shape now functions as \textit{plant shape III}.
				
\textbf{Venom Immunity (Ex)}: At 9th level, a druid gains immunity to all poisons. 
				
\textbf{A Thousand Faces (Su)}: At 13th level, a druid gains the ability to change her appearance at will, as if using the \textit{alter self} spell, but only while in her normal form.
				
\textbf{Timeless Body (Ex)}: After attaining 15th level, a druid no longer takes ability score penalties for aging and cannot be magically aged. Any penalties she may have already incurred, however, remain in place. Bonuses still accrue, and the druid still dies of old age when her time is up.
				
\subsection{Ex-Druids}

				
A druid who ceases to revere nature, changes to a prohibited alignment, or teaches the Druidic language to a nondruid loses all spells and druid abilities (including her animal companion, but not including weapon, armor, and shield proficiencies). She cannot thereafter gain levels as a druid until she atones (see the \textit{atonement} spell description).
				
\subsection{Animal Companions}

				
An animal companion's abilities are determined by the druid's level and its animal racial traits. Table: Animal Companion Base Statistics determines many of the base statistics of the animal companion. They remain creatures of the animal type for purposes of determining which spells can affect them.
				
% <div class="table">

\begin{table*}[]
\sffamily
\caption{Table: Animal Companion Base Statistics}
\setlength{\tabcolsep}{1pt}
\begin{tabular}{llllllllllllllllllll}
      &    &     &      &     &      &        &       & \textbf{Natural} &         &        & \\
\textbf{Class} &    &     &      &     &      &        &       & \textbf{Armour}  & \textbf{Str/Dex} & \textbf{Bonus}  & \\
\textbf{Level} & \textbf{HD} & \textbf{BAB} & \textbf{Fort} & \textbf{Ref} & \textbf{Will} & \textbf{Skills} & \textbf{Feats} & \textbf{Bonus} & \textbf{Bonus}   & \textbf{Tricks} & \textbf{Special} \\
1st & 2 & +1 & +3 & +3 & +0 & 2 & 1 & +0 & +0 & 1 & Link, share spells \\
2nd & 3 & +2 & +3 & +3 & +1 & 3 & 2 & +0 & +0 & 1 & - \\
3rd & 3 & +2 & +3 & +3 & +1 & 3 & 2 & +2 & +1 & 2 & Evasion \\
4th & 4 & +3 & +4 & +4 & +1 & 4 & 2 & +2 & +1 & 2 & Ability score increase \\
 5th & 5 & +3 & +4 & +4 & +1 & 5 & 3 & +2 & +1 & 2 & - \\
 6th & 6 & +4 & +5 & +5 & +2 & 6 & 3 & +4 & +2 & 3 & Devotion \\
 7th & 6 & +4 & +5 & +5 & +2 & 6 & 3 & +4 & +2 & 3 & - \\
 8th & 7 & +5 & +5 & +5 & +2 & 7 & 4 & +4 & +2 & 3 & - \\
 9th & 8 & +6 & +6 & +6 & +2 & 8 & 4 & +6 & +3 & 4 & Ability score increase, Multiattack \\
 10th & 9 & +6 & +6 & +6 & +3 & 9 & 5 & +6 & +3 & 4 & - \\
 11th & 9 & +6 & +6 & +6 & +3 & 9 & 5 & +6 & +3 & 4 & - \\
 12th & 10 & +7 & +7 & +7 & +3 & 10 & 5 & +8 & +4 & 5 & - \\
 13th & 11 & +8 & +7 & +7 & +3 & 11 & 6 & +8 & +4 & 5 & - \\
 14th & 12 & +9 & +8 & +8 & +4 & 12 & 6 & +8 & +4 & 5 & Ability score increase \\
 15th & 12 & +9 & +8 & +8 & +4 & 12 & 6 & +10 & +5 & 6 & Improved evasion \\
 16th & 13 & +9 & +8 & +8 & +4 & 13 & 7 & +10 & +5 & 6 & - \\
 17th & 14 & +10 & +9 & +9 & +4 & 14 & 7 & +10 & +5 & 6 & - \\
 18th & 15 & +11 & +9 & +9 & +5 & 15 & 8 & +12 & +6 & 7 & - \\
 19th & 15 & +11 & +9 & +9 & +5 & 15 & 8 & +12 & +6 & 7 & - \\
 20th & 16 & +12 & +10 & +10 & +5 & 16 & 8 & +12 & +6 & 7 & Ability score increase\\
\end{tabular}
\end{table*}

% </div href="#companions-ability-score-increase">

				
\textbf{Class Level}: This is the character's druid level. The druid's class levels stack with levels of any other classes that are entitled to an animal companion for the purpose of determining the companion's statistics.
				
\textbf{HD}: This is the total number of eight-sided (d8) Hit Dice the animal companion possesses, each of which gains a Constitution modifier, as normal. 
				
\textbf{BAB}: This is the animal companion's base attack bonus. An animal companion's base attack bonus is the same as that of a druid of a level equal to the animal's HD. Animal companions do not gain additional attacks using their natural weapons for a high base attack bonus.
				
\textbf{Fort/Ref/Will}: These are the animal companion's base saving throw bonuses. An animal companion has good Fortitude and Reflex saves.
				
\textbf{Skills}: This lists the animal's total skill ranks. Animal companions can assign skill ranks to any skill listed under Animal Skills. If an animal companion increases its Intelligence to 10 or higher, it gains bonus skill ranks as normal. Animal companions with an Intelligence of 3 or higher can purchase ranks in any skill. An animal companion cannot have more ranks in a skill than it has Hit Dice.
				
\textbf{Feats}: This is the total number of feats possessed by an animal companion. Animal companions should select their feats from those listed under Animal Feats. Animal companions can select other feats, although they are unable to utilize some feats (such as Martial Weapon Proficiency). Note that animal companions cannot select a feat with a requirement of base attack bonus +1 until they gain their second feat at 3 Hit Dice.
				
\textbf{Natural Armor Bonus}: The number noted here is an improvement to the animal companion's existing natural armor bonus.
				
\textbf{Str/Dex Bonus}: Add this modifier to the animal companion's Strength and Dexterity scores.
				
\textbf{Bonus Tricks}: The value given in this column is the total number of \^abonus\^a tricks that the animal knows in addition to any that the druid might choose to teach it (see the Handle Animal skill for more details on how to teach an animal tricks). These bonus tricks don't require any training time or Handle Animal checks, and they don't count against the normal limit of tricks known by the animal. The druid selects these bonus tricks, and once selected, they can't be changed. 
				
\textbf{Special}: This includes a number of abilities gained by animal companions as they increase in power. Each of these bonuses is described below.
				
\textit{Link (Ex):} A druid can handle her animal companion as a free action, or push it as a move action, even if she doesn't have any ranks in the Handle Animal skill. The druid gains a +4 circumstance bonus on all wild empathy checks and Handle Animal checks made regarding an animal companion.
				
\textit{Share Spells (Ex):} The druid may cast a spell with a target of \^aYou\^a on her animal companion (as a spell with a range of touch) instead of on herself. A druid may cast spells on her animal companion even if the spells normally do not affect creatures of the companion's type (animal). Spells cast in this way must come from a class that grants an animal companion. This ability does not allow the animal to share abilities that are not spells, even if they function like spells.
				
\textit{Evasion (Ex):} If an animal companion is subjected to an attack that normally allows a Reflex save for half damage, it takes no damage if it makes a successful saving throw.
				
\textit{Ability Score Increase (Ex):} The animal companion adds +1 to one of its ability scores.
				
\textit{Devotion (Ex):} An animal companion gains a +4 morale bonus on Will saves against enchantment spells and effects.
				
\textit{Multiattack: }An animal companion gains Multiattack as a bonus feat if it has three or more natural attacks and does not already have that feat. If it does not have the requisite three or more natural attacks, the animal companion instead gains a second attack with one of its natural weapons, albeit at a --5 penalty.
				
\textit{Improved Evasion (Ex):} When subjected to an attack that allows a Reflex saving throw for half damage, an animal companion takes no damage if it makes a successful saving throw and only half damage if the saving throw fails.
				
\subsection{Animal Skills}

				
Animal companions can have ranks in any of the following skills: Acrobatics* (Dex), Climb* (Str), Escape Artist (Dex), Fly* (Dex), Intimidate (Cha), Perception* (Wis), Stealth* (Dex), Survival (Wis), and Swim* (Str). All of the skills marked with an (*) are class skills for animal companions. Animal companions with an Intelligence of 3 or higher can put ranks into any skill. 
				
\subsection{Animal Feats}

				
Animal companions can select from the following feats: Acrobatic, Agile Maneuvers, Armor Proficiency (light, medium, and heavy), Athletic, Blind-Fight, Combat Reflexes, Diehard, Dodge, Endurance, Great Fortitude, Improved Bull Rush, Improved Initiative, Improved Natural Armor, Improved Natural Attack, Improved Overrun, Intimidating Prowess, Iron Will, Lightning Reflexes, Mobility, Power Attack, Run, Skill Focus, Spring Attack, Stealthy, Toughness, Weapon Finesse, and Weapon Focus. Animal companions with an Intelligence of 3 or higher can select any feat they are physically capable of using. GMs might expand this list to include feats from other sources. 
				
\subsection{Animal Choices}

				
Each animal companion has different starting sizes, speed, attacks, ability scores, and special qualities. All animal attacks are made using the creature's full base attack bonus unless otherwise noted. Animal attacks add the animal's Strength modifier to the damage roll, unless it is its only attack, in which case it adds 1-1/2 its Strength modifier. Some have special abilities, such as scent. See Special Abilities for more information on these abilities. As you gain levels, your animal companion improves as well, usually at 4th or 7th level, in addition to the standard bonuses noted on Table: Animal Companion Base Statistics. Instead of taking the listed benefit at 4th or 7th level, you can instead choose to increase the companion's Dexterity and Constitution by 2.
				
\subsection{Ape}

				
\textbf{Starting Statistics}:\textbf{ Size} Medium; \textbf{Speed} 30 ft., Climb 30 ft.; \textbf{AC} +1 natural armor; \textbf{Attack} bite (1d4), 2 claws (1d4); \textbf{Ability Scores }Str 13, Dex 17, Con 10, Int 2, Wis 12, Cha 7; \textbf{Special Qualities} low-light vision, scent.
				
\textbf{4th-Level Advancement}: \textbf{Size }Large; \textbf{AC }+2 natural armor; \textbf{Attack} bite (1d6), 2 claws (1d6); \textbf{Ability Scores }Str +8, Dex --2, Con +4.
				
\subsection{Badger (Wolverine)}

				
\textbf{Starting Statistics}:\textbf{ Size} Small; \textbf{Speed} 30 ft., burrow 10 ft., climb 10 ft.; \textbf{AC} +2 natural armor; \textbf{Attack} bite (1d4), 2 claws (1d3); \textbf{Ability Scores }Str 10, Dex 17, Con 15, Int 2, Wis 12, Cha 10; \textbf{Special Attacks }rage (as a barbarian for 6 rounds per day); \textbf{Special Qualities} low-light vision, scent.
				
\textbf{4th-Level Advancement}: \textbf{Size }Medium; \textbf{Attack }bite (1d6), 2 claws (1d4); \textbf{Ability Scores }Str +4, Dex --2, Con +2. 
				
\subsection{Bear}

				
\textbf{Starting Statistics}:\textbf{ Size} Small; \textbf{Speed} 40 ft.; \textbf{AC} +2 natural armor; \textbf{Attack} bite (1d4), 2 claws (1d3); \textbf{Ability Scores }Str 15, Dex 15, Con 13, Int 2, Wis 12, Cha 6; \textbf{Special Qualities} low-light vision, scent.
				
\textbf{4th-Level Advancement}: \textbf{Size }Medium; \textbf{Attack} bite (1d6), 2 claws (1d4); \textbf{Ability Scores }Str +4, Dex --2, Con +2. 
				
\subsection{Bird (Eagle/Hawk/Owl)}

				
\textbf{Starting Statistics}:\textbf{ Size} Small; \textbf{Speed} 10 ft., fly 80 ft. (average); \textbf{AC} +1 natural armor; \textbf{Attack} bite (1d4), 2 talons (1d4); \textbf{Ability Scores }Str 10, Dex 15, Con 12, Int 2, Wis 14, Cha 6; \textbf{Special Qualities} low-light vision.
				
\textbf{4th-Level Advancement}: \textbf{Ability Scores} Str +2, Con +2.
				
\subsection{Boar}

				
\textbf{Starting Statistics}:\textbf{ Size} Small; \textbf{Speed} 40 ft.; \textbf{AC} +6 natural armor; \textbf{Attack} gore (1d6); \textbf{Ability Scores }Str 13, Dex 12, Con 15, Int 2, Wis 13, Cha 4; \textbf{Special Qualities} low-light vision, scent.
				
\textbf{4th-Level Advancement}: \textbf{Size }Medium; \textbf{Attack} gore (1d8); \textbf{Ability Scores }Str +4, Dex --2, Con +2; \textbf{Special Attacks} ferocity. 
				
\subsection{Camel}

				
\textbf{Starting Statistics}:\textbf{ Size} Large; \textbf{Speed} 50 ft.; \textbf{AC} +1 natural armor; \textbf{Attack} bite (1d4) or spit (ranged touch attack, target is sickened for 1d4 rounds, range 10 feet); \textbf{Ability Scores }Str 18, Dex 16, Con 14, Int 2, Wis 11, Cha 4; \textbf{Special Qualities} low-light vision, scent.
				
\textbf{4th-Level Advancement}: \textbf{Ability Scores }Str +2, Con +2.
				
\subsection{Cat, Big (Lion, Tiger)}

				
\textbf{Starting Statistics}:\textbf{ Size} Medium; \textbf{Speed} 40 ft.; \textbf{AC} +1 natural armor; \textbf{Attack} bite (1d6), 2 claws (1d4); \textbf{Ability Scores }Str 13, Dex 17, Con 13, Int 2, Wis 15, Cha 10; \textbf{Special Attacks }rake (1d4); \textbf{Special Qualities} low-light vision, scent.
				
\textbf{7th-Level Advancement}: \textbf{Size }Large; \textbf{AC }+2 natural armor; \textbf{Attack} bite (1d8), 2 claws (1d6); \textbf{Ability Scores }Str +8, Dex --2, Con +4; \textbf{Special Attacks} grab, pounce, rake (1d6)  .
				
\subsection{Cat, Small (Cheetah, Leopard)}

				
\textbf{Starting Statistics}:\textbf{ Size} Small; \textbf{Speed} 50 ft.; \textbf{AC} +1 natural armor; \textbf{Attack} bite (1d4 plus trip), 2 claws (1d2); \textbf{Ability Scores }Str 12, Dex 21, Con 13, Int 2, Wis 12, Cha 6; \textbf{Special Qualities} low-light vision, scent.
				
\textbf{4th-Level Advancement}: \textbf{Size }Medium; \textbf{Attack} bite (1d6 plus trip), 2 claws (1d3); \textbf{Ability Scores }Str +4, Dex --2, Con +2; \textbf{Special Qualities} sprint. 
				
\subsection{Crocodile (Alligator)}

				
\textbf{Starting Statistics}:\textbf{ Size} Small; \textbf{Speed} 20 ft., swim 30 ft.; \textbf{AC} +4 natural armor; \textbf{Attack} bite (1d6); \textbf{Ability Scores }Str 15, Dex 14, Con 15, Int 1, Wis 12, Cha 2; \textbf{Special Qualities} hold breath, low-light vision.
				
\textbf{4th-Level Advancement}: \textbf{Size }Medium; \textbf{Attack} bite (1d8) or tail slap (1d12); \textbf{Ability Scores }Str +4, Dex --2, Con +2; \textbf{Special Attacks }death roll, grab, sprint. 
				
\subsection{Dinosaur (Deinonychus, Velociraptor)}

				
\textbf{Starting Statistics}:\textbf{ Size} Small; \textbf{Speed} 60 ft.; \textbf{AC} +1 natural armor; \textbf{Attack} 2 talons (1d6), bite (1d4); \textbf{Ability Scores }Str 11, Dex 17, Con 17, Int 2, Wis 12, Cha 14; \textbf{Special Qualities} low-light vision, scent.
				
\textbf{7th-Level Advancement}: \textbf{Size }Medium; \textbf{AC }+2 natural armor; \textbf{Attack} 2 talons (1d8), bite (1d6), 2 claws (1d4) \textbf{Ability Scores }Str +4, Dex --2, Con +2; \textbf{Special Attacks} pounce. 
				
\subsection{Dog}

				
\textbf{Starting Statistics}:\textbf{ Size} Small; \textbf{Speed} 40 ft.; \textbf{AC} +2 natural armor; \textbf{Attack} bite (1d4); \textbf{Ability Scores }Str 13, Dex 17, Con 15, Int 2, Wis 12, Cha 6; \textbf{Special Qualities} low-light vision, scent.
				
\textbf{4th-Level Advancement}: \textbf{Size }Medium; \textbf{Attack} bite (1d6); \textbf{Ability Scores }Str +4, Dex --2, Con +2. 
				
\subsection{Horse}

				
\textbf{Starting Statistics}:\textbf{ Size} Large; \textbf{Speed} 50 ft.; \textbf{AC} +4 natural armor; \textbf{Attack} bite (1d4), 2 hooves* (1d6); \textbf{Ability Scores }Str 16, Dex 13, Con 15, Int 2, Wis 12, Cha 6; \textbf{Special Qualities} low-light vision, scent. *This is a secondary natural attack, see Combat for more information on how secondary attacks work.
				
\textbf{4th-Level Advancement}: \textbf{Ability Scores }Str +2, Con +2; \textbf{Special Qualities }combat trained (see the Handle Animal skill). 
				
\subsection{Pony}

				
\textbf{Starting Statistics}:\textbf{ Size} Medium; \textbf{Speed} 40 ft.; \textbf{AC} +2 natural armor; \textbf{Attack} 2 hooves (1d3); \textbf{Ability Scores }Str 13, Dex 13, Con 12, Int 2, Wis 11, Cha 4; \textbf{Special Qualities} low-light vision, scent.
				
\textbf{4th-Level Advancement}: \textbf{Ability Scores }Str +2, Con +2; \textbf{Special Qualities }combat trained (see the Handle Animal skill). 
				
\subsection{Shark}

				
\textbf{Starting Statistics}:\textbf{ Size} Small; \textbf{Speed} swim 60 ft.; \textbf{AC} +4 natural armor; \textbf{Attack} bite (1d4); \textbf{Ability Scores }Str 13, Dex 15, Con 15, Int 1, Wis 12, Cha 2; \textbf{Special Qualities} low-light vision, scent.
				
\textbf{4th-Level Advancement}: \textbf{Size }Medium; \textbf{Attack} bite (1d6); \textbf{Ability Scores }Str +4, Dex --2, Con +2; \textbf{Special Qualities} blindsense. 
				
\subsection{Snake, Constrictor}

				
\textbf{Starting Statistics}:\textbf{ Size} Medium; \textbf{Speed} 20 ft., climb 20 ft., swim 20 ft.; \textbf{AC} +2 natural armor; \textbf{Attack} bite (1d3); \textbf{Ability Scores }Str 15, Dex 17, Con 13, Int 1, Wis 12, Cha 2; \textbf{Special Attacks }grab; \textbf{Special Qualities} low-light vision, scent.
				
\textbf{4th-Level Advancement}: \textbf{Size }Large; \textbf{AC} +1 natural armor; \textbf{Attack} bite (1d4); \textbf{Ability Scores }Str +8, Dex --2, Con +4; \textbf{Special Attacks} constrict 1d4. 
				
\subsection{Snake, Viper}

				
\textbf{Starting Statistics}:\textbf{ Size} Small; \textbf{Speed} 20 ft., climb 20 ft., swim 20 ft.; \textbf{AC} +2 natural armor; \textbf{Attack} bite (1d3 plus poison); \textbf{Ability Scores }Str 8, Dex 17, Con 11, Int 1, Wis 12, Cha 2; \textbf{Special Attacks }poison (\textit{Frequency} 1 round (6), \textit{Effect} 1 Con damage, \textit{Cure} 1 save, Con-based DC); \textbf{Special Qualities} low-light vision, scent.
				
\textbf{4th-Level Advancement}: \textbf{Size }Medium; \textbf{Attack} bite (1d4 plus poison); \textbf{Ability Scores }Str +4, Dex --2, Con +2. 
				
\subsection{Wolf}

				
\textbf{Starting Statistics}:\textbf{ Size} Medium; \textbf{Speed} 50 ft.; \textbf{AC} +2 natural armor; \textbf{Attack} bite (1d6 plus trip); \textbf{Ability Scores }Str 13, Dex 15, Con 15, Int 2, Wis 12, Cha 6; \textbf{Special Qualities} low-light vision, scent.
				
\textbf{7th-Level Advancement}: \textbf{Size }Large; \textbf{AC }+2 natural armor; \textbf{Attack} bite (1d8 plus trip); \textbf{Ability Scores }Str +8, Dex --2, Con +4. 
        	

\section{Fighter}

\label{f0}				
Some take up arms for glory, wealth, or revenge. Others do battle to prove themselves, to protect others, or because they know nothing else. Still others learn the ways of weaponcraft to hone their bodies in battle and prove their mettle in the forge of war. Lords of the battlefield, fighters are a disparate lot, training with many weapons or just one, perfecting the uses of armor, learning the fighting techniques of exotic masters, and studying the art of combat, all to shape themselves into living weapons. Far more than mere thugs, these skilled warriors reveal the true deadliness of their weapons, turning hunks of metal into arms capable of taming kingdoms, slaughtering monsters, and rousing the hearts of armies. Soldiers, knights, hunters, and artists of war, fighters are unparalleled champions, and woe to those who dare stand against them.
				
\textbf{Role}: Fighters excel at combat---defeating their enemies, controlling the flow of battle, and surviving such sorties themselves. While their specific weapons and methods grant them a wide variety of tactics, few can match fighters for sheer battle prowess.
				
\textbf{Alignment}: Any.
				
\textbf{Hit Die}: d10.
				
\subsection{Class Skills}

				
The fighter's class skills are Climb (Str), Craft (Int), Handle Animal (Cha), Intimidate (Cha), Knowledge (dungeoneering) (Int), Knowledge (engineering) (Int), Profession (Wis), Ride (Dex), Survival (Wis), and Swim (Str). 
				
\textbf{Skill Ranks per Level}: 2 + Int modifier.

\begin{table}[]
\sffamily
\setlength{\tabcolsep}{1pt}
\caption{Table: Fighter}
\begin{tabular}{llllll}
\textbf{Level} & \textbf{Base Attack Bonus} & \textbf{Fort Save} & \textbf{Ref Save} & \textbf{Will Save} & \textbf{Special}\\
1st & +1 & +2 & +0 & +0 & Bonus feat\\
2nd & +2 & +3 & +0 & +0 & Bonus feat, bravery\\
3rd & +3 & +3 & +1 & +1 & Armor training\\
4th & +4 & +4 & +1 & +1 & Bonus feat\\
5th & +5 & +4 & +1 & +1 & Weapon training\\
6th & +6/+1 & +5 & +2 & +2 & Bonus feat, bravery\\
7th & +7/+2 & +5 & +2 & +2 & Armor training\\
8th & +8/+3 & +6 & +2 & +2 & Bonus feat\\
9th & +9/+4 & +6 & +3 & +3 & Weapon training\\
10th & +10/+5 & +7 & +3 & +3 & Bonus feat, bravery\\
11th & +11/+6/+1 & +7 & +3 & +3 & Armor training\\
12th & +12/+7/+2 & +8 & +4 & +4 & Bonus feat\\
13th & +13/+8/+3 & +8 & +4 & +4 & Weapon training\\
14th & +14/+9/+4 & +9 & +4 & +4 & Bonus feat, bravery\\
15th & +15/+10/+5 & +9 & +5 & +5 & Armor training\\
16th & +16/+11/+6/+1 & +10 & +5 & +5 & Bonus feat\\
17th & +17/+12/+7/+2 & +10 & +5 & +5 & Weapon training\\
18th & +18/+13/+8/+3 & +11 & +6 & +6 & Bonus feat, bravery\\
19th & +19/+14/+9/+4 & +11 & +6 & +6 & Armor mastery\\
20th & +20/+15/+10/+5 & +12 & +6 & +6 & Bonus feat, weapon mastery\\
\end{tabular}
\end{table}
				
\subsection{Class Features}

				
The following are class features of the fighter.
				
\textbf{Weapon and Armor Proficiency}: A fighter is proficient with all simple and martial weapons and with all armor (heavy, light, and medium) and shields (including tower shields).
				
\textbf{Bonus Feats}: At 1st level, and at every even level thereafter, a fighter gains a bonus feat in addition to those gained from normal advancement (meaning that the fighter gains a feat at every level). These bonus feats must be selected from those listed as combat feats, sometimes also called \texttt{{}"{}}fighter bonus feats.\texttt{{}"{}}
				
Upon reaching 4th level, and every four levels thereafter (8th, 12th, and so on), a fighter can choose to learn a new bonus feat in place of a bonus feat he has already learned. In effect, the fighter loses the bonus feat in exchange for the new one. The old feat cannot be one that was used as a prerequisite for another feat, prestige class, or other ability. A fighter can only change one feat at any given level and must choose whether or not to swap the feat at the time he gains a new bonus feat for the level.
				
\textbf{Bravery (Ex)}: Starting at 2nd level, a fighter gains a +1 bonus on Will saves against fear. This bonus increases by +1 for every four levels beyond 2nd.
				
\textbf{Armor Training (Ex)}: Starting at 3rd level, a fighter learns to be more maneuverable while wearing armor. Whenever he is wearing armor, he reduces the armor check penalty by 1 (to a minimum of 0) and increases the maximum Dexterity bonus allowed by his armor by 1. Every four levels thereafter (7th, 11th, and 15th), these bonuses increase by +1 each time, to a maximum --4 reduction of the armor check penalty and a +4 increase of the maximum Dexterity bonus allowed.
				
In addition, a fighter can also move at his normal speed while wearing medium armor. At 7th level, a fighter can move at his normal speed while wearing heavy armor.
				
\textbf{Weapon Training (Ex)}: Starting at 5th level, a fighter can select one group of weapons, as noted below. Whenever he attacks with a weapon from this group, he gains a +1 bonus on attack and damage rolls.
				
Every four levels thereafter (9th, 13th, and 17th), a fighter becomes further trained in another group of weapons. He gains a +1 bonus on attack and damage rolls when using a weapon from this group. In addition, the bonuses granted by previous weapon groups increase by +1 each. For example, when a fighter reaches 9th level, he receives a +1 bonus on attack and damage rolls with one weapon group and a +2 bonus on attack and damage rolls with the weapon group selected at 5th level. Bonuses granted from overlapping groups do not stack. Take the highest bonus granted for a weapon if it resides in two or more groups.
				
A fighter also adds this bonus to any combat maneuver checks made with weapons from this group. This bonus also applies to the fighter's Combat Maneuver Defense when defending against disarm and sunder attempts made against weapons from this group.
				
Weapon groups are defined as follows (GMs may add other weapons to these groups, or add entirely new groups):
				
\textit{Axes}: battleaxe, dwarven waraxe, greataxe, handaxe, heavy pick, light pick, orc double axe, and throwing axe. 
				
\textit{Blades, Heavy}: bastard sword, elven curve blade, falchion, greatsword, longsword, scimitar, scythe, and two-bladed sword.
				
\textit{Blades, Light}: dagger, kama, kukri, rapier, sickle, starknife, and short sword.
				
\textit{Bows}: composite longbow, composite shortbow, longbow, and shortbow.
				
\textit{Close}: gauntlet, heavy shield, light shield, punching dagger, sap, spiked armor, spiked gauntlet, spiked shield, and unarmed strike.
				
\textit{Crossbows}: hand crossbow, heavy crossbow, light crossbow, heavy repeating crossbow, and light repeating crossbow.
				
\textit{Double}: dire flail, dwarven urgrosh, gnome hooked hammer, orc double axe, quarterstaff, and two-bladed sword.
				
\textit{Flails}: dire flail, flail, heavy flail, morningstar, nunchaku, spiked chain, and whip.
				
\textit{Hammers}: club, greatclub, heavy mace, light hammer, light mace, and warhammer.
				
\textit{Monk}: kama, nunchaku, quarterstaff, sai, shuriken, siangham, and unarmed strike.
				
\textit{Natural}: unarmed strike and all natural weapons, such as bite, claw, gore, tail, and wing.
				
\textit{Pole Arms}: glaive, guisarme, halberd, and ranseur.
				
\textit{Spears}: javelin, lance, longspear, shortspear, spear, and trident.
				
\textit{Thrown}: blowgun, bolas, club, dagger, dart, halfling sling staff, javelin, light hammer, net, shortspear, shuriken, sling, spear, starknife, throwing axe, and trident.
				
\textbf{Armor Mastery (Ex)}: At 19th level, a fighter gains DR 5/--- whenever he is wearing armor or using a shield.
				
\textbf{Weapon Mastery (Ex)}: At 20th level, a fighter chooses one weapon, such as the longsword, greataxe, or longbow. Any attacks made with that weapon automatically confirm all critical threats and have their damage multiplier increased by 1 (\mbox{$\times$}2 becomes \mbox{$\times$}3, for example). In addition, he cannot be disarmed while wielding a weapon of this type.
        	

\section{Monk}

\label{f0}				
For the truly exemplary, martial skill transcends the battlefield---it is a lifestyle, a doctrine, a state of mind. These warrior-artists search out methods of battle beyond swords and shields, finding weapons within themselves just as capable of crippling or killing as any blade. These monks (so called since they adhere to ancient philosophies and strict martial disciplines) elevate their bodies to become weapons of war, from battle-minded ascetics to self-taught brawlers. Monks tread the path of discipline, and those with the will to endure that path discover within themselves not what they are, but what they are meant to be.
				
\textbf{Role}: Monks excel at overcoming even the most daunting perils, striking where it's least expected, and taking advantage of enemy vulnerabilities. Fleet of foot and skilled in combat, monks can navigate any battlefield with ease, aiding allies wherever they are needed most.
				
\textbf{Alignment}: Any lawful.
				
\textbf{Hit Die}: d8.
				
\subsection{Class Skills}

				
The monk's class skills are Acrobatics (Dex), Climb (Str), Craft (Int), Escape Artist (Dex), Intimidate (Cha), Knowledge (history) (Int), Knowledge (religion) (Int), Perception (Wis), Perform (Cha), Profession (Wis), Ride (Dex), Sense Motive (Wis), Stealth (Dex), and Swim (Str).
				
\textbf{Skill Ranks per Level}: 4 + Int modifier.
% <div class="table">

Table: Monk
% <
\begin{table}[]
\sffamily
\setlength{\tabcolsep}{1pt}
\caption{Table: Monk}
\begin{tabularx}{\linewidth}{lllllXllll}
      & Base   &      &      &      &         &                 &         &       & \\
      & Attack & Fort & Ref  & Will &         & Flurry of Blows & Unarmed & AC    & Fast\\
Level & Bonus  & Save & Save & Save & Special &  Attack Bonus   &  Damage & Bonus & Movement\\
\hline
1st & +0 & +2 & +2 & +2 & Bonus feat, flurry of blows, stunning fist, unarmed strike & -1/-1 & 1d6 & +0 & +0 ft.\\
2nd & +1 & +3 & +3 & +3 & Bonus feat, evasion & +0/+0 & 1d6 & +0 & +0 ft.\\
3rd & +2 & +3 & +3 & +3 & Fast movement, maneuver training, still mind & +1/+1 & 1d6 & +0 & +10 ft.\\
4th & +3 & +4 & +4 & +4 & Ki,  pool, slow fall & +2/+2 & 1d8 & +1 & +10 ft.\\
5th & +3 & +4 & +4 & +4 & High jump, purity of body & +3/+3 & 1d8 & +1 & +10 ft.\\
6th & +4 & +5 & +5 & +5 & Bonus feat, slow fall & +4/+4 & 1d8 & +1 & +20 ft.\\
7th & +5 & +5 & +5 & +5 & Ki pool (cold iron/silver),  & +5/+5/+0 & 1d8 & +1 & +20 ft.\\
8th & +6/+1 & +6 & +6 & +6 & Slow fall & +6/+6/+1 & 1d10 & +2 & +20 ft.\\
9th & +6/+1 & +6 & +6 & +6 & Improved evasion & +7/+7/+2 & 1d10 & +2 & +30 ft.\\
10th & +7/+2 & +7 & +7 & +7 & Bonus feat, ki,  pool, slow fall & +8/+8/+3 & 1d10 & +2 & +30 ft.\\
11th & +8/+3 & +7 & +7 & +7 & Diamond body & +9/+9/+4/+4/-1 & 1d10 & +2 & +30 ft.\\
12th & +9/+4 & +8 & +8 & +8 & Abundant step, slow fall & +10/+10/+5/+5/+0 & 2d6 & +3 & +40 ft.\\
13th & +9/+4 & +8 & +8 & +8 & Diamond soul & +11/+11/+6/+6/+1 & 2d6 & +3 & +40 ft.\\
14th & +10/+5 & +9 & +9 & +9 & Bonus feat, slow fall & +12/+12/+7/+7/+2 & 2d6 & +3 & +40 ft.\\
15th & +11/+6/+1 & +9 & +9 & +9 & Quivering palm & +13/+13/+8/+8/+3 & 2d6 & +3 & +50 ft.\\
16th & +12/+7/+2 & +10 & +10 & +10 & Ki,  pool, slow fall & +14/+14/+9/+9/+4 & 2d8 & +4 & +50 ft.\\
17th & +12/+7/+2 & +10 & +10 & +10 & Timeless body, tongue of the sun and moon & +15/+15/+10/+10/+5 & 2d8 & +4 & +50 ft.\\
18th & +13/+8/+3 & +11 & +11 & +11 & Bonus feat, slow fall & +16/+16/+11/+11/+6 & 2d8 & +4 & +60 ft.\\
19th & +14/+9/+4 & +11 & +11 & +11 & Empty body & +17/+17/+12/+12/+7 & 2d8 & +4 & +60 ft.\\
20th & +15/+10/+5 & +12 & +12 & +12 & Perfect self, slow fall any distance & +18/+18/+13/+13/+8 & 2d10 & +5 & +60 ft.\\
\end{tabularx}
\end{table}

				
\subsection{Class Features}

				
All of the following are class features of the monk.
				
\textbf{Weapon and Armor Proficiency}: Monks are proficient with the club, crossbow (light or heavy), dagger, handaxe, javelin, kama, nunchaku, quarterstaff, sai, shortspear, short sword, shuriken, siangham, sling, and spear.
				
Monks are not proficient with any armor or shields.
				
When wearing armor, using a shield, or carrying a medium or heavy load, a monk loses his AC bonus, as well as his fast movement and flurry of blows abilities.
				
\textbf{AC Bonus (Ex)}: When unarmored and unencumbered, the monk adds his Wisdom bonus (if any) to his AC and his CMD. In addition, a monk gains a +1 bonus to AC and CMD at 4th level. This bonus increases by 1 for every four monk levels thereafter, up to a maximum of +5 at 20th level.
				
These bonuses to AC apply even against touch attacks or when the monk is flat-footed. He loses these bonuses when he is immobilized or helpless, when he wears any armor, when he carries a shield, or when he carries a medium or heavy load.
				
\textbf{Flurry of Blows (Ex)}: Starting at 1st level, a monk can make a flurry of blows as a full-attack action. When doing so, he may make on additional attack, taking a -2 penalty on all of his attack rolls, as if using the Two-Weapon Fighting feat. These attacks can be any combination of unarmed strikes and attacks with a monk special weapon (he does not need to use two weapons to use this ability). For the purpose of these attacks, the monk's base attack bonus from his monk class levels is equal to his monk level. For all other purposes, such as qualifying for a feat or a prestige class, the monk uses his normal base attack bonus. 
				
At 8th level, the monk can make two additional attacks when he uses flurry of blows, as if using Improved Two-Weapon Fighting (even if the monk does not meet the prerequisites for the feat).
				
At 15th level, the monk can make three additional attacks using flurry of blows, as if using Greater Two-Weapon Fighting (even if the monk does not meet the prerequisites for the feat). 
				
A monk applies his full Strength bonus to his damage rolls for all successful attacks made with flurry of blows, whether the attacks are made with an off-hand or with a weapon wielded in both hands. A monk may substitute disarm, sunder, and trip combat maneuvers for unarmed attacks as part of a flurry of blows. A monk cannot use any weapon other than an unarmed strike or a special monk weapon as part of a flurry of blows. A monk with natural weapons cannot use such weapons as part of a flurry of blows, nor can he make natural attacks in addition to his flurry of blows attacks.
				
\textbf{Unarmed Strike}: At 1st level, a monk gains Improved Unarmed Strike as a bonus feat. A monk's attacks may be with fist, elbows, knees, and feet. This means that a monk may make unarmed strikes with his hands full. There is no such thing as an off-hand attack for a monk striking unarmed. A monk may thus apply his full Strength bonus on damage rolls for all his unarmed strikes.
				
Usually a monk's unarmed strikes deal lethal damage, but he can choose to deal nonlethal damage instead with no penalty on his attack roll. He has the same choice to deal lethal or nonlethal damage while grappling.
				
A monk's unarmed strike is treated as both a manufactured weapon and a natural weapon for the purpose of spells and effects that enhance or improve either manufactured weapons or natural weapons.
				
A monk also deals more damage with his unarmed strikes than a normal person would, as shown above on Table: Monk. The unarmed damage values listed on Table: Monk is for Medium monks. A Small monk deals less damage than the amount given there with his unarmed attacks, while a Large monk deals more damage; see Small or Large Monk Unarmed Damage on the table given below.Small or Large Monk Unarmed Damage
% <
\begin{table}[]
\sffamily
\caption{Small or Large Monk Unarmed Damage}
\begin{tabular}{llllll}
Level & Damage (Small Monk) & Damage (Large Monk)\\
1st-3rd   & 1d4  & 1d8 \\
4th-7th   & 1d6  & 2d6 \\
8th-11th  & 1d8  & 2d8 \\
12th-15th & 1d10 & 3d6 \\
16th-19th & 2d6  & 3d8 \\
20th & 2d8 & 4d8\\
\end{tabular}
\end{table}
		
\textbf{Bonus Feat:} At 1st level, 2nd level, and every 4 levels thereafter, a monk may select a bonus feat. These feats must be taken from the following list: Catch Off-Guard, Combat Reflexes, Deflect Arrows, Dodge, Improved Grapple, Scorpion Style, and Throw Anything. At 6th level, the following feats are added to the list: Gorgon's Fist, Improved Bull Rush, Improved Disarm, Improved Feint, Improved Trip, and Mobility. At 10th level, the following feats are added to the list: Improved Critical, Medusa's Wrath, Snatch Arrows, and Spring Attack. A monk need not have any of the prerequisites normally required for these feats to select them.
				
\textbf{Stunning Fist (Ex)}: At 1st level, the monk gains Stunning Fist as a bonus feat, even if he does not meet the prerequisites. At 4th level, and every 4 levels thereafter, the monk gains the ability to apply a new condition to the target of his Stunning Fist. This condition replaces stunning the target for 1 round, and a successful saving throw still negates the effect. At 4th level, he can choose to make the target fatigued. At 8th level, he can make the target sickened for 1 minute. At 12th level, he can make the target staggered for 1d6+1 rounds. At 16th level, he can permanently blind or deafen the target. At 20th level, he can paralyze the target for 1d6+1 rounds. The monk must choose which condition will apply before the attack roll is made. These effects do not stack with themselves (a creature sickened by Stunning Fist cannot become nauseated if hit by Stunning Fist again), but additional hits do increase the duration.
				
\textbf{Evasion (Ex)}: At 2nd level or higher, a monk can avoid damage from many area-effect attacks. If a monk makes a successful Reflex saving throw against an attack that normally deals half damage on a successful save, he instead takes no damage. Evasion can be used only if a monk is wearing light armor or no armor. A helpless monk does not gain the benefit of evasion.
				
\textbf{Fast Movement (Ex)}: At 3rd level, a monk gains an enhancement bonus to his land speed, as shown on Table: Monk. A monk in armor or carrying a medium or heavy load loses this extra speed.
				
\textbf{Maneuver Training (Ex)}: At 3rd level, a monk uses his monk level in place of his base attack bonus when calculating his Combat Maneuver Bonus. Base attack bonuses granted from other classes are unaffected and are added normally.
				
\textbf{Still Mind (Ex)}: A monk of 3rd level or higher gains a +2 bonus on saving throws against enchantment spells and effects.
				
\textbf{Ki Pool (Su)}: At 4th level, a monk gains a pool of \textit{ki }points, supernatural energy he can use to accomplish amazing feats. The number of points in a monk's ki pool is equal to 1/2 his monk level + his Wisdom modifier. As long as he has at least 1 point in his ki pool, he can make a ki strike. At 4th level, ki strike allows his unarmed attacks to be treated as magic weapons for the purpose of overcoming damage reduction. At 7th level, his unarmed attacks are also treated as cold iron and silver for the purpose of overcoming damage reduction. At 10th level, his unarmed attacks are also treated as lawful weapons for the purpose of overcoming damage reduction. At 16th level, his unarmed attacks are treated as adamantine weapons for the purpose of overcoming damage reduction and bypassing hardness.
				
By spending 1 point from his ki pool, a monk can make one additional attack at his highest attack bonus when making a flurry of blows attack. In addition, he can spend 1 point to increase his speed by 20 feet for 1 round. Finally, a monk can spend 1 point from his ki pool to give himself a +4 dodge bonus to AC for 1 round. Each of these powers is activated as a swift action. A monk gains additional powers that consume points from his ki pool as he gains levels.
				
The ki pool is replenished each morning after 8 hours of rest or meditation; these hours do not need to be consecutive.
				
\textbf{Slow Fall (Ex)}: At 4th level or higher, a monk within arm's reach of a wall can use it to slow his descent. When first gaining this ability, he takes damage as if the fall were 20 feet shorter than it actually is. The monk's ability to slow his fall (that is, to reduce the effective distance of the fall when next to a wall) improves with his monk level until at 20th level he can use a nearby wall to slow his descent and fall any distance without harm.
				
\textbf{High Jump (Ex)}: At 5th level, a monk adds his level to all Acrobatics checks made to jump, both for vertical jumps and horizontal jumps. In addition, he always counts as having a running start when making jump checks using Acrobatics. By spending 1 point from his ki pool as a swift action, a monk gains a +20 bonus on Acrobatics checks made to jump for 1 round.
				
\textbf{Purity of Body (Ex)}: At 5th level, a monk gains immunity to all diseases, including supernatural and magical diseases.
				
\textbf{Wholeness of Body (Su)}: At 7th level or higher, a monk can heal his own wounds as a standard action. He can heal a number of hit points of damage equal to his monk level by using 2 points from his ki pool.
				
\textbf{Improved Evasion (Ex)}: At 9th level, a monk's evasion ability improves. He still takes no damage on a successful Reflex saving throw against attacks, but henceforth he takes only half damage on a failed save. A helpless monk does not gain the benefit of improved evasion.
				
\textbf{Diamond Body (Su)}: At 11th level, a monk gains immunity to poisons of all kinds.
				
\textbf{Abundant Step (Su)}: At 12th level or higher, a monk can slip magically between spaces, as if using the spell \textit{dimension door}. Using this ability is a move action that consumes 2 points from his ki pool. His caster level for this effect is equal to his monk level. He cannot take other creatures with him when he uses this ability.
				
\textbf{Diamond Soul (Ex)}: At 13th level, a monk gains spell resistance equal to his current monk level + 10. In order to affect the monk with a spell, a spellcaster must get a result on a caster level check (1d20 + caster level) that equals or exceeds the monk's spell resistance.
				
\textbf{Quivering Palm (Su)}: Starting at 15th level, a monk can set up vibrations within the body of another creature that can thereafter be fatal if the monk so desires. He can use this quivering palm attack once per day, and he must announce his intent before making his attack roll. Creatures immune to critical hits cannot be affected. Otherwise, if the monk strikes successfully and the target takes damage from the blow, the quivering palm attack succeeds. Thereafter, the monk can try to slay the victim at any later time, as long as the attempt is made within a number of days equal to his monk level. To make such an attempt, the monk merely wills the target to die (a free action), and unless the target makes a Fortitude saving throw (DC 10 + 1/2 the monk's level + the monk's Wis modifier), it dies. If the saving throw is successful, the target is no longer in danger from that particular quivering palm attack, but it may still be affected by another one at a later time. A monk can have no more than 1 quivering palm in effect at one time. If a monk uses quivering palm while another is still in effect, the previous effect is negated.
				
\textbf{Timeless Body (Ex)}: At 17th level, a monk no longer takes penalties to his ability scores for aging and cannot be magically aged. Any such penalties that he has already taken, however, remain in place. Age bonuses still accrue, and the monk still dies of old age when his time is up.
				
\textbf{Tongue of the Sun and Moon (Ex)}: A monk of 17th level or higher can speak with any living creature.
				
\textbf{Empty Body (Su)}: At 19th level, a monk gains the ability to assume an ethereal state for 1 minute as though using the spell \textit{etherealness.} Using this ability is a move action that consumes 3 points from his ki pool. This ability only affects the monk and cannot be used to make other creatures ethereal.
				
\textbf{Perfect Self}: At 20th level, a monk becomes a magical creature. He is forevermore treated as an outsider rather than as a humanoid (or whatever the monk's creature type was) for the purpose of spells and magical effects. Additionally, the monk gains damage reduction 10/chaotic, which allows him to ignore the first 10 points of damage from any attack made by a nonchaotic weapon or by any natural attack made by a creature that doesn't have similar damage reduction. Unlike other outsiders, the monk can still be brought back from the dead as if he were a member of his previous creature type.
				
\subsection{Ex-Monks}

				
A monk who becomes nonlawful cannot gain new levels as a monk but retains all monk abilities.
        	

\section{Paladin}

\label{f0}				
Through a select, worthy few shines the power of the divine. Called paladins, these noble souls dedicate their swords and lives to the battle against evil. Knights, crusaders, and law-bringers, paladins seek not just to spread divine justice but to embody the teachings of the virtuous deities they serve. In pursuit of their lofty goals, they adhere to ironclad laws of morality and discipline. As reward for their righteousness, these holy champions are blessed with boons to aid them in their quests: powers to banish evil, heal the innocent, and inspire the faithful. Although their convictions might lead them into conflict with the very souls they would save, paladins weather endless challenges of faith and dark temptations, risking their lives to do right and fighting to bring about a brighter future. 
				
\textbf{Role}: Paladins serve as beacons for their allies within the chaos of battle. While deadly opponents of evil, they can also empower goodly souls to aid in their crusades. Their magic and martial skills also make them well suited to defending others and blessing the fallen with the strength to continue fighting.
				
\textbf{Alignment}: Lawful good.
				
\textbf{Hit Die}: d10.
				
\subsection{Class Skills}

				
The paladin's class skills are Craft (Int), Diplomacy (Cha), Handle Animal (Cha), Heal (Wis), Knowledge (nobility) (Int), Knowledge (religion) (Int), Profession (Wis), Ride (Dex), Sense Motive (Wis), and Spellcraft (Int).
				
\textbf{Skill Ranks per Level}: 2 + Int modifier.
% <div class="table">

Table: Paladin
% <
\begin{table}[]
\sffamily
\caption{Table: Paladin}
\begin{tabularx}{\linewidth}{lllllXllll}
\multirow{2}{*}{Level} & \multirow{2}{*}{\parbox{5em}{Base Attack Bonus}} & \multirow{2}{*}{\parbox{1.5em}{Fort Save}} & \multirow{2}{*}{\parbox{1.5em}{Ref Save}} & \multirow{2}{*}{\parbox{1.5em}{Will Save}} & \multirow{2}{*}{Special}  & \multicolumn{4}{c}{Spells per day} \\
                       &                                    &                            &                           &                            &                                                                        & 1st  & 2nd & 3rd & 4th \\
\hline
1st & +1 & +2 & +0 & +2 & Aura of good, detect evil, smite evil 1/day & - & - & - & -\\
2nd & +2 & +3 & +0 & +3 & Divine grace, lay on hands & - & - & - & -\\
3rd & +3 & +3 & +1 & +3 & Aura of courage, divine health, mercy & - & - & - & -\\
4th & +4 & +4 & +1 & +4 & Channel positive energy, smite evil 2/day & 0 & - & - & -\\
5th & +5 & +4 & +1 & +4 & Divine bond & 1 & - & - & -\\
6th & +6/+1 & +5 & +2 & +5 & Mercy & 1 & - & - & -\\
7th & +7/+2 & +5 & +2 & +5 & Smite evil 3/day & 1 & 0 & - & -\\
8th & +8/+3 & +6 & +2 & +6 & Aura of resolve & 1 & 1 & - & -\\
9th & +9/+4 & +6 & +3 & +6 & Mercy & 2 & 1 & - & -\\
10th & +10/+5 & +7 & +3 & +7 & Smite evil 4/day & 2 & 1 & 0 & -\\
11th & +11/+6/+1 & +7 & +3 & +7 & Aura of justice & 2 & 1 & 1 & -\\
12th & +12/+7/+2 & +8 & +4 & +8 & Mercy & 2 & 2 & 1 & -\\
13th & +13/+8/+3 & +8 & +4 & +8 & Smite evil 5/day & 3 & 2 & 1 & 0\\
14th & +14/+9/+4 & +9 & +4 & +9 & Aura of faith & 3 & 2 & 1 & 1\\
15th & +15/+10/+5 & +9 & +5 & +9 & Mercy & 3 & 2 & 2 & 1\\
16th & +16/+11/+6/+1 & +10 & +5 & +10 & Smite evil 6/day & 3 & 3 & 2 & 1\\
17th & +17/+12/+7/+2 & +10 & +5 & +10 & Aura of righteousness & 4 & 3 & 2 & 1\\
18th & +18/+13/+8/+3 & +11 & +6 & +11 & Mercy & 4 & 3 & 2 & 2\\
19th & +19/+14/+9/+4 & +11 & +6 & +11 & Smite evil 7/day & 4 & 3 & 3 & 2\\
20th & +20/+15/+10/+5 & +12 & +6 & +12 & Holy champion & 4 & 4 & 3 & 3\\
\end{tabularx}
\end{table}

\subsection{Class Features}

				
All of the following are class features of the paladin.
				
\textbf{Weapon and Armor Proficiency}: Paladins are proficient with all simple and martial weapons, with all types of armor (heavy, medium, and light), and with shields (except tower shields).
				
\textbf{Aura of Good (Ex)}: The power of a paladin's aura of good (see the \textit{detect good} spell) is equal to her paladin level.
				
\textbf{Detect Evil} \textbf{(Sp)}: At will, a paladin can use \textit{detect evil,} as the spell. A paladin can, as a move action, concentrate on a single item or individual within 60 feet and determine if it is evil, learning the strength of its aura as if having studied it for 3 rounds. While focusing on one individual or object, the paladin does not detect evil in any other object or individual within range.
				
\textbf{Smite Evil (Su)}: Once per day, a paladin can call out to the powers of good to aid her in her struggle against evil. As a swift action, the paladin chooses one target within sight to smite. If this target is evil, the paladin adds her Charisma bonus (if any) to her attack rolls and adds her paladin level to all damage rolls made against the target of her smite. If the target of smite evil is an outsider with the evil subtype, an evil-aligned dragon, or an undead creature, the bonus to damage on the first successful attack increases to 2 points of damage per level the paladin possesses. Regardless of the target, smite evil attacks automatically bypass any DR the creature might possess.
				
In addition, while smite evil is in effect, the paladin gains a deflection bonus equal to her Charisma modifier (if any) to her AC against attacks made by the target of the smite. If the paladin targets a creature that is not evil, the smite is wasted with no effect.
				
The smite evil effect remains until the target of the smite is dead or the next time the paladin rests and regains her uses of this ability. At 4th level, and at every three levels thereafter, the paladin may smite evil one additional time per day, as indicated on Table: Paladin, to a maximum of seven times per day at 19th level.
				
\textbf{Divine Grace (Su)}: At 2nd level, a paladin gains a bonus equal to her Charisma bonus (if any) on all saving throws.
				
\textbf{Lay On Hands (Su)}: Beginning at 2nd level, a paladin can heal wounds (her own or those of others) by touch. Each day she can use this ability a number of times equal to 1/2 her paladin level plus her Charisma modifier. With one use of this ability, a paladin can heal 1d6 hit points of damage for every two paladin levels she possesses. Using this ability is a standard action, unless the paladin targets herself, in which case it is a swift action. Despite the name of this ability, a paladin only needs one free hand to use this ability.
				
Alternatively, a paladin can use this healing power to deal damage to undead creatures, dealing 1d6 points of damage for every two levels the paladin possesses. Using lay on hands in this way requires a successful melee touch attack and doesn't provoke an attack of opportunity. Undead do not receive a saving throw against this damage.
				
\textbf{Aura of Courage (Su)}: At 3rd level, a paladin is immune to fear (magical or otherwise). Each ally within 10 feet of her gains a +4 morale bonus on saving throws against fear effects. This ability functions only while the paladin is conscious, not if she is unconscious or dead.
				
\textbf{Divine Health (Ex)}: At 3rd level, a paladin is immune to all diseases, including supernatural and magical diseases, including mummy rot.
				
\textbf{Mercy (Su)}: At 3rd level, and every three levels thereafter, a paladin can select one mercy. Each mercy adds an effect to the paladin's lay on hands ability. Whenever the paladin uses lay on hands to heal damage to one target, the target also receives the additional effects from all of the mercies possessed by the paladin. A mercy can remove a condition caused by a curse, disease, or poison without curing the affliction. Such conditions return after 1 hour unless the mercy actually removes the affliction that causes the condition.
				
At 3rd level, the paladin can select from the following initial mercies. 
				
\textit{Fatigued}: The target is no longer fatigued.
				
\textit{Shaken}: The target is no longer shaken.
				
\textit{Sickened}: The target is no longer sickened. 
				
At 6th level, a paladin adds the following mercies to the list of those that can be selected. 
				
\textit{Dazed}: The target is no longer dazed.
				
\textit{Diseased}: The paladin's lay on hands ability also acts as \textit{remove disease}, using the paladin's level as the caster level.
				
\textit{Staggered}: The target is no longer staggered, unless the target is at exactly 0 hit points.
				
At 9th level, a paladin adds the following mercies to the list of those that can be selected.
				
\textit{Cursed}: The paladin's lay on hands ability also acts as \textit{remove curse,} using the paladin's level as the caster level.
				
\textit{Exhausted}: The target is no longer exhausted. The paladin must have the fatigue mercy before selecting this mercy.
				
\textit{Frightened}: The target is no longer frightened. The paladin must have the shaken mercy before selecting this mercy.
				
\textit{Nauseated}: The target is no longer nauseated. The paladin must have the sickened mercy before selecting this mercy.
				
\textit{Poisoned}: The paladin's lay on hands ability also acts as \textit{neutralize poison}, using the paladin's level as the caster level.
				
At 12th level, a paladin adds the following mercies to the list of those that can be selected.
				
\textit{Blinded}: The target is no longer blinded.
				
\textit{Deafened}: The target is no longer deafened.
				
\textit{Paralyzed}: The target is no longer paralyzed.
				
\textit{Stunned}: The target is no longer stunned. 
				
These abilities are cumulative. For example, a 12th-level paladin's lay on hands ability heals 6d6 points of damage and might also cure fatigued and exhausted conditions as well as removing diseases and neutralizing poisons\textit{. }Once a condition or spell effect is chosen, it can't be changed.
				
\textbf{Channel Positive Energy (Su)}: When a paladin reaches 4th level, she gains the supernatural ability to channel positive energy like a cleric. Using this ability consumes two uses of her lay on hands ability. A paladin uses her level as her effective cleric level when channeling positive energy. This is a Charisma-based ability.
				
\textbf{Spells}: Beginning at 4th level, a paladin gains the ability to cast a small number of divine spells which are drawn from the paladin spell list presented in Spell Lists. A paladin must choose and prepare her spells in advance.
				
To prepare or cast a spell, a paladin must have a Charisma score equal to at least 10 + the spell level. The Difficulty Class for a saving throw against a paladin's spell is 10 + the spell level + the paladin's Charisma modifier.
				
Like other spellcasters, a paladin can cast only a certain number of spells of each spell level per day. Her base daily spell allotment is given on Table: Paladin. In addition, she receives bonus spells per day if she has a high Charisma score (see Table: Ability Modifiers and Bonus Spells). When Table: Paladin indicates that the paladin gets 0 spells per day of a given spell level, she gains only the bonus spells she would be entitled to based on her Charisma score for that spell level.
				
A paladin must spend 1 hour each day in quiet prayer and meditation to regain her daily allotment of spells. A paladin may prepare and cast any spell on the paladin spell list, provided that she can cast spells of that level, but she must choose which spells to prepare during her daily meditation.
				
Through 3rd level, a paladin has no caster level. At 4th level and higher, her caster level is equal to her paladin level -- 3.
				
\textbf{Divine Bond (Sp)}: Upon reaching 5th level, a paladin forms a divine bond with her god. This bond can take one of two forms. Once the form is chosen, it cannot be changed.
				
The first type of bond allows the paladin to enhance her weapon as a standard action by calling upon the aid of a celestial spirit for 1 minute per paladin level. When called, the spirit causes the weapon to shed light as a torch. At 5th level, this spirit grants the weapon a +1 enhancement bonus. For every three levels beyond 5th, the weapon gains another +1 enhancement bonus, to a maximum of +6 at 20th level. These bonuses can be added to the weapon, stacking with existing weapon bonuses to a maximum of +5, or they can be used to add any of the following weapon properties: \textit{axiomatic}, \textit{brilliant energy, defending, disruption, flaming, flaming burst, holy, keen, merciful}, and \textit{speed}. Adding these properties consumes an amount of bonus equal to the property's cost (see Table: Melee Weapon Special Abilities). These bonuses are added to any properties the weapon already has, but duplicate abilities do not stack. If the weapon is not magical, at least a +1 enhancement bonus must be added before any other properties can be added. The bonus and properties granted by the spirit are determined when the spirit is called and cannot be changed until the spirit is called again. The celestial spirit imparts no bonuses if the weapon is held by anyone other than the paladin but resumes giving bonuses if returned to the paladin. These bonuses apply to only one end of a double weapon. A paladin can use this ability once per day at 5th level, and one additional time per day for every four levels beyond 5th, to a total of four times per day at 17th level.
				
If a weapon bonded with a celestial spirit is destroyed, the paladin loses the use of this ability for 30 days, or until she gains a level, whichever comes first. During this 30-day period, the paladin takes a --1 penalty on attack and weapon damage rolls.
				
The second type of bond allows a paladin to gain the service of an unusually intelligent, strong, and loyal steed to serve her in her crusade against evil. This mount is usually a heavy horse (for a Medium paladin) or a pony (for a Small paladin), although more exotic mounts, such as a boar, camel, or dog are also suitable. This mount functions as a druid's animal companion, using the paladin's level as her effective druid level. Bonded mounts have an Intelligence of at least 6. 
				
Once per day, as a full-round action, a paladin may magically call her mount to her side. This ability is the equivalent of a spell of a level equal to one-third the paladin's level. The mount immediately appears adjacent to the paladin. A paladin can use this ability once per day at 5th level, and one additional time per day for every 4 levels thereafter, for a total of four times per day at 17th level.
				
At 11th level, the mount gains the celestial template and becomes a magical beast for the purposes of determining which spells affect it. At 15th level, a paladin's mount gains spell resistance equal to the paladin's level + 11.
				
Should the paladin's mount die, the paladin may not summon another mount for 30 days or until she gains a paladin level, whichever comes first. During this 30-day period, the paladin takes a --1 penalty on attack and weapon damage rolls. 
				
\textbf{Aura of Resolve (Su)}: At 8th level, a paladin is immune to charm spells and spell-like abilities. Each ally within 10 feet of her gains a +4 morale bonus on saving throws against charm effects.
				
This ability functions only while the paladin is conscious, not if she is unconscious or dead.
				
\textbf{Aura of Justice (Su)}: At 11th level, a paladin can expend two uses of her smite evil ability to grant the ability to smite evil to all allies within 10 feet, using her bonuses. Allies must use this smite evil ability by the start of the paladin's next turn and the bonuses last for 1 minute. Using this ability is a free action. Evil creatures gain no benefit from this ability. 
				
\textbf{Aura of Faith (Su)}: At 14th level, a paladin's weapons are treated as good-aligned for the purposes of overcoming damage reduction. Any attack made against an enemy within 10 feet of her is treated as good-aligned for the purposes of overcoming damage reduction.
				
This ability functions only while the paladin is conscious, not if she is unconscious or dead.
				
\textbf{Aura of Righteousness (Su)}: At 17th level, a paladin gains DR 5/evil and immunity to compulsion spells and spell-like abilities. Each ally within 10 feet of her gains a +4 morale bonus on saving throws against compulsion effects. 
				
This ability functions only while the paladin is conscious, not if she is unconscious or dead.
				
\textbf{Holy Champion (Su)}: At 20th level, a paladin becomes a conduit for the power of her god. Her DR increases to 10/evil. Whenever she uses smite evil and successfully strikes an evil outsider, the outsider is also subject to a \textit{banishment,} using her paladin level as the caster level (her weapon and holy symbol automatically count as objects that the subject hates). After the \textit{banishment} effect and the damage from the attack is resolved, the smite immediately ends. In addition, whenever she channels positive energy or uses lay on hands to heal a creature, she heals the maximum possible amount.
				
\textbf{Code of Conduct}: A paladin must be of lawful good alignment and loses all class features except proficiencies if she ever willingly commits an evil act.
				
Additionally, a paladin's code requires that she respect legitimate authority, act with honor (not lying, not cheating, not using poison, and so forth), help those in need (provided they do not use the help for evil or chaotic ends), and punish those who harm or threaten innocents.
				
\textbf{Associates}: While she may adventure with good or neutral allies, a paladin avoids working with evil characters or with anyone who consistently offends her moral code. Under exceptional circumstances, a paladin can ally with evil associates, but only to defeat what she believes to be a greater evil. A paladin should seek an \textit{atonement} spell periodically during such an unusual alliance, and should end the alliance immediately should she feel it is doing more harm than good. A paladin may accept only henchmen, followers, or cohorts who are lawful good.
				
\subsection{Ex-Paladins}

				
A paladin who ceases to be lawful good, who willfully commits an evil act, or who violates the code of conduct loses all paladin spells and class features (including the service of the paladin's mount, but not weapon, armor, and shield proficiencies). She may not progress any further in levels as a paladin. She regains her abilities and advancement potential if she atones for her violations (see the \textit{atonement} spell description in Spell Lists), as appropriate.
        	

\section{Ranger}

\label{f0}				
For those who relish the thrill of the hunt, there are only predators and prey. Be they scouts, trackers, or bounty hunters, rangers share much in common: unique mastery of specialized weapons, skill at stalking even the most elusive game, and the expertise to defeat a wide range of quarries. Knowledgeable, patient, and skilled hunters, these rangers hound man, beast, and monster alike, gaining insight into the way of the predator, skill in varied environments, and ever more lethal martial prowess. While some track man-eating creatures to protect the frontier, others pursue more cunning game---even fugitives among their own people.
				
\textbf{Role}: Rangers are deft skirmishers, either in melee or at range, capable of skillfully dancing in and out of battle. Their abilities allow them to deal significant harm to specific types of foes, but their skills are valuable against all manner of enemies.
				
\textbf{Alignment}: Any.
				
\textbf{Hit Die}: d10.
				
\subsection{Class Skills}

				
The ranger's class skills are Climb (Str), Craft (Int), Handle Animal (Cha), Heal (Wis), Intimidate (Cha), Knowledge (dungeoneering) (Int), Knowledge (geography) (Int), Knowledge (nature) (Int), Perception (Wis), Profession (Wis), Ride (Dex), Spellcraft (Int), Stealth (Dex), Survival (Wis), and Swim (Str).
				
\textbf{Skill Ranks per Level}: 6 + Int modifier.
% <div class="table">

Table: Ranger
% <
\begin{table}[]
\sffamily
\caption{Table: Ranger}
\begin{tabularx}{\linewidth}{lllllXllll}
\multirow{2}{*}{Level} & \multirow{2}{*}{\parbox{5em}{Base Attack Bonus}} & \multirow{2}{*}{\parbox{1.5em}{Fort Save}} & \multirow{2}{*}{\parbox{1.5em}{Ref Save}} & \multirow{2}{*}{\parbox{1.5em}{Will Save}} & \multirow{2}{*}{Special}  & \multicolumn{4}{c}{Spells per day} \\
                       &                                    &                            &                           &                            &                                                                        & 1st  & 2nd & 3rd & 4th \\
\hline
1st & +1 & +2 & +2 & +0 & 1st  & - & - & - & -\\
2nd & +2 & +3 & +3 & +0 & Combat style feat & - & - & - & -\\
3rd & +3 & +3 & +3 & +1 & Endurance, favored terrain & - & - & - & -\\
4th & +4 & +4 & +4 & +1 & Hunter's bond & 0 & - & - & -\\
5th & +5 & +4 & +4 & +1 & 2nd  & 1 & - & - & -\\
6th & +6/+1 & +5 & +5 & +2 & Combat style feat & 1 & - & - & -\\
7th & +7/+2 & +5 & +5 & +2 & Woodland stride & 1 & 0 & - & -\\
8th & +8/+3 & +6 & +6 & +2 & Swift tracker, favored terrain & 1 & 1 & - & -\\
9th & +9/+4 & +6 & +6 & +3 & Evasion & 2 & 1 & - & -\\
10th & +10/+5 & +7 & +7 & +3 & 3rd  & 2 & 1 & 0 & -\\
11th & +11/+6/+1 & +7 & +7 & +3 & Quarry & 2 & 1 & 1 & -\\
12th & +12/+7/+2 & +8 & +8 & +4 & Camouflage & 2 & 2 & 1 & -\\
13th & +13/+8/+3 & +8 & +8 & +4 & 3rd  & 3 & 2 & 1 & 0\\
14th & +14/+9/+4 & +9 & +9 & +4 & Combat style feat & 3 & 2 & 1 & 1\\
15th & +15/+10/+5 & +9 & +9 & +5 & 4th  & 3 & 2 & 2 & 1\\
16th & +16/+11/+6/+1 & +10 & +10 & +5 & Improved evasion & 3 & 3 & 2 & 1\\
17th & +17/+12/+7/+2 & +10 & +10 & +5 & Hide in plain sight & 4 & 3 & 2 & 1\\
18th & +18/+13/+8/+3 & +11 & +11 & +6 & 4th  & 4 & 3 & 2 & 2\\
19th & +19/+14/+9/+4 & +11 & +11 & +6 & Improved quarry & 4 & 3 & 3 & 2\\
20th & +20/+15/+10/+5 & +12 & +12 & +6 & 5th  & 4 & 4 & 3 & 3\\
\end{tabularx}
\end{table}

				
\subsection{Class Features}

				
All of the following are class features of the ranger.
				
\textbf{Weapon and Armor Proficiency}: A ranger is proficient with all simple and martial weapons and with light armor, medium armor, and shields (except tower shields).
\begin{table}[]
\sffamily
\caption{Favored Enemies}
\begin{tabular}{ll}
Type (Subtype)       & Type (Subtype)           \\
Aberration           & Humanoid (other subtype) \\
Animal               & Magical beast            \\
Construct            & Monstrous humanoid       \\
Dragon               & Ooze                     \\
Fey                  & Outsider (air)           \\
Humanoid (aquatic)   & Outsider (chaotic)       \\
Humanoid (dwarf)     & Outsider (earth)         \\
Humanoid (elf)       & Outsider (evil)          \\
Humanoid (giant)     & Outsider (fire)          \\
Humanoid (goblinoid) & Outsider (good)          \\
Humanoid (gnoll)     & Outsider (lawful)        \\
Humanoid (gnome)     & Outsider (native)        \\
Humanoid (halfling)  & Outsider (water)         \\
Humanoid (human)     & Plant                    \\
Humanoid (orc)       & Undead                   \\
Humanoid (reptilian) & Vermin                  
\end{tabular}
\end{table}
\textbf{Favored Enemy (Ex)}: At 1st level, a ranger selects a creature type from the ranger favored enemies table. He gains a +2 bonus on Bluff, Knowledge, Perception, Sense Motive, and Survival checks against creatures of his selected type. Likewise, he gets a +2 bonus on weapon attack and damage rolls against them. A ranger may make Knowledge skill checks untrained when attempting to identify these creatures.
				
At 5th level and every five levels thereafter (10th, 15th, and 20th level), the ranger may select an additional favored enemy. In addition, at each such interval, the bonus against any one favored enemy (including the one just selected, if so desired) increases by +2. 
				
If the ranger chooses humanoids or outsiders as a favored enemy, he must also choose an associated subtype, as indicated on the table below. (Note that there are other types of humanoid to choose from---those called out specifically on the table below are merely the most common.) If a specific creature falls into more than one category of favored enemy, the ranger's bonuses do not stack; he simply uses whichever bonus is higher.
				
\textbf{Track (Ex)}: A ranger adds half his level (minimum 1) to Survival skill checks made to follow tracks.
				
\textbf{Wild Empathy (Ex)}: A ranger can improve the initial attitude of an animal. This ability functions just like a Diplomacy check to improve the attitude of a person (see Using Skills). The ranger rolls 1d20 and adds his ranger level and his Charisma bonus to determine the wild empathy check result. The typical domestic animal has a starting attitude of indifferent, while wild animals are usually unfriendly.
				
To use wild empathy, the ranger and the animal must be within 30 feet of one another under normal visibility conditions. Generally, influencing an animal in this way takes 1 minute, but, as with influencing people, it might take more or less time.
				
The ranger can also use this ability to influence a magical beast with an Intelligence score of 1 or 2, but he takes a --4 penalty on the check.
				
\textbf{Combat Style Feat (Ex)}: At 2nd level, a ranger must select one of two combat styles to pursue: archery or two-weapon combat. The ranger's expertise manifests in the form of bonus feats at 2nd, 6th, 10th, 14th, and 18th level. He can choose feats from his selected combat style, even if he does not have the normal prerequisites.
				
If the ranger selects archery, he can choose from the following list whenever he gains a combat style feat: Far Shot, Point Blank Shot, Precise Shot, and Rapid Shot. At 6th level, he adds Improved Precise Shot and Manyshot to the list. At 10th level, he adds Pinpoint Targeting and Shot on the Run to the list.
				
If the ranger selects two-weapon combat, he can choose from the following list whenever he gains a combat style feat: Double Slice, Improved Shield Bash, Quick Draw, and Two-Weapon Fighting. At 6th level, he adds Improved Two-Weapon Fighting and Two-Weapon Defense to the list. At 10th level, he adds Greater Two-Weapon Fighting and Two-Weapon Rend to the list.
				
The benefits of the ranger's chosen style feats apply only when he wears light, medium, or no armor. He loses all benefits of his combat style feats when wearing heavy armor. Once a ranger selects a combat style, it cannot be changed.
				
\textbf{Endurance}: A ranger gains Endurance as a bonus feat at 3rd level.
Favored Terrains
Cold (ice, glaciers, snow, and tundra)
Desert (sand and wastelands)
Forest (coniferous and deciduous)
Jungle
Mountain (including hills)
Plains
Planes (pick one, other than Material Plane)
Swamp
Underground (caves and dungeons)
Urban (buildings, streets, and sewers)
Water (above and below the surface)

\textbf{Favored Terrain (Ex)}: At 3rd level, a ranger may select a type of terrain from the Favored Terrains table. The ranger gains a +2 bonus on initiative checks and Knowledge (geography), Perception, Stealth, and Survival skill checks when he is in this terrain. A ranger traveling through his favored terrain normally leaves no trail and cannot be tracked (though he may leave a trail if he so chooses).
				
At 8th level and every five levels thereafter, the ranger may select an additional favored terrain. In addition, at each such interval, the skill bonus and initiative bonus in any one favored terrain (including the one just selected, if so desired), increases by +2. 
				
If a specific terrain falls into more than one category of favored terrain, the ranger's bonuses do not stack; he simply uses whichever bonus is higher.
				
\textbf{Hunter's Bond (Ex)}: At 4th level, a ranger forms a bond with his hunting companions. This bond can take one of two forms. Once the form is chosen, it cannot be changed. The first is a bond to his companions. This bond allows him to spend a move action to grant half his favored enemy bonus against a single target of the appropriate type to all allies within 30 feet who can see or hear him. This bonus lasts for a number of rounds equal to the ranger's Wisdom modifier (minimum 1). This bonus does not stack with any favored enemy bonuses possessed by his allies; they use whichever bonus is higher.
				
The second option is to form a close bond with an animal companion. A ranger who selects an animal companion can choose from the following list: badger, bird, camel, cat (small), dire rat, dog, horse, pony, snake (viper or constrictor), or wolf. If the campaign takes place wholly or partly in an aquatic environment, the ranger may choose a shark instead. This animal is a loyal companion that accompanies the ranger on his adventures as appropriate for its kind. A ranger's animal companion shares his favored enemy and favored terrain bonuses.
				
This ability functions like the druid animal companion ability (which is part of the Nature Bond class feature), except that the ranger's effective druid level is equal to his ranger level -- 3.
				
\textbf{Spells}: Beginning at 4th level, a ranger gains the ability to cast a small number of divine spells, which are drawn from the ranger spell list presented in Spell Lists. A ranger must choose and prepare his spells in advance.
				
To prepare or cast a spell, a ranger must have a Wisdom score equal to at least 10 + the spell level. The Difficulty Class for a saving throw against a ranger's spell is 10 + the spell level + the ranger's Wisdom modifier.
				
Like other spellcasters, a ranger can cast only a certain number of spells of each spell level per day. His base daily spell allotment is given on Table: Ranger. In addition, he receives bonus spells per day if he has a high Wisdom score (see Table: Ability Modifiers and Bonus Spells). When Table: Ranger indicates that the ranger gets 0 spells per day of a given spell level, he gains only the bonus spells he would be entitled to based on his Wisdom score for that spell level.
				
A ranger must spend 1 hour per day in quiet meditation to regain his daily allotment of spells. A ranger may prepare and cast any spell on the ranger spell list, provided that he can cast spells of that level, but he must choose which spells to prepare during his daily meditation.
				
Through 3rd level, a ranger has no caster level. At 4th level and higher, his caster level is equal to his ranger level -- 3.
				
\textbf{Woodland Stride (Ex)}: Starting at 7th level, a ranger may move through any sort of undergrowth (such as natural thorns, briars, overgrown areas, and similar terrain) at his normal speed and without taking damage or suffering any other impairment.
				
Thorns, briars, and overgrown areas that are enchanted or magically manipulated to impede motion, however, still affect him.
				
\textbf{Swift Tracker (Ex)}: Beginning at 8th level, a ranger can move at his normal speed while using Survival to follow tracks without taking the normal --5 penalty. He takes only a --10 penalty (instead of the normal --20) when moving at up to twice normal speed while tracking.
				
\textbf{Evasion (Ex)}: When he reaches 9th level, a ranger can avoid even magical and unusual attacks with great agility. If he makes a successful Reflex saving throw against an attack that normally deals half damage on a successful save, he instead takes no damage. Evasion can be used only if the ranger is wearing light armor, medium armor, or no armor. A helpless ranger does not gain the benefit of evasion.
				
\textbf{Quarry (Ex)}: At 11th level, a ranger can, as a standard action, denote one target within his line of sight as his quarry. Whenever he is following the tracks of his quarry, a ranger can take 10 on his Survival skill checks while moving at normal speed, without penalty. In addition, he receives a +2 insight bonus on attack rolls made against his quarry, and all critical threats are automatically confirmed. A ranger can have no more than one quarry at a time and the creature's type must correspond to one of his favored enemy types. He can dismiss this effect at any time as a free action, but he cannot select a new quarry for 24 hours. If the ranger sees proof that his quarry is dead, he can select a new quarry after waiting 1 hour.
				
\textbf{Camouflage (Ex)}: A ranger of 12th level or higher can use the Stealth skill to hide in any of his favored terrains, even if the terrain doesn't grant cover or concealment.
				
\textbf{Improved Evasion (Ex)}: At 16th level, a ranger's evasion improves. This ability works like evasion, except that while the ranger still takes no damage on a successful Reflex saving throw against attacks, he henceforth takes only half damage on a failed save. A helpless ranger does not gain the benefit of improved evasion.
				
\textbf{Hide in Plain Sight (Ex)}: While in any of his favored terrains, a ranger of 17th level or higher can use the Stealth skill even while being observed.
				
\textbf{Improved Quarry (Ex)}: At 19th level, the ranger's ability to hunt his quarry improves. He can now select a quarry as a free action, and can now take 20 while using Survival to track his quarry, while moving at normal speed without penalty. His insight bonus to attack his quarry increases to +4. If his quarry is killed or dismissed, he can select a new one after 10 minutes have passed.
				
\textbf{Master Hunter (Ex)}: A ranger of 20th level becomes a master hunter. He can always move at full speed while using Survival to follow tracks without penalty. He can, as a standard action, make a single attack against a favored enemy at his full attack bonus. If the attack hits, the target takes damage normally and must make a Fortitude save or die. The DC of this save is equal to 10 + 1/2 the ranger's level + the ranger's Wisdom modifier. A ranger can choose instead to deal an amount of nonlethal damage equal to the creature's current hit points. A successful save negates this damage. A ranger can use this ability once per day against each favored enemy type he possesses, but not against the same creature more than once in a 24-hour period.
        	

\section{Rogue}

\label{f0}				
Life is an endless adventure for those who live by their wits. Ever just one step ahead of danger, rogues bank on their cunning, skill, and charm to bend fate to their favor. Never knowing what to expect, they prepare for everything, becoming masters of a wide variety of skills, training themselves to be adept manipulators, agile acrobats, shadowy stalkers, or masters of any of dozens of other professions or talents. Thieves and gamblers, fast talkers and diplomats, bandits and bounty hunters, and explorers and investigators all might be considered rogues, as well as countless other professions that rely upon wits, prowess, or luck. Although many rogues favor cities and the innumerable opportunities of civilization, some embrace lives on the road, journeying far, meeting exotic people, and facing fantastic danger in pursuit of equally fantastic riches. In the end, any who desire to shape their fates and live life on their own terms might come to be called rogues.
				
\textbf{Role}: Rogues excel at moving about unseen and catching foes unaware, and tend to avoid head-to-head combat. Their varied skills and abilities allow them to be highly versatile, with great variations in expertise existing between different rogues. Most, however, excel in overcoming hindrances of all types, from unlocking doors and disarming traps to outwitting magical hazards and conning dull-witted opponents.
				
\textbf{Alignment}: Any.
				
\textbf{Hit Die}: d8.
				
\subsection{Class Skills}

				
The rogue's class skills are Acrobatics (Dex), Appraise (Int), Bluff (Cha), Climb (Str), Craft (Int), Diplomacy (Cha), Disable Device (Dex), Disguise (Cha), Escape Artist (Dex), Intimidate (Cha), Knowledge (dungeoneering) (Int), Knowledge (local) (Int), Linguistics (Int), Perception (Wis), Perform (Cha), Profession (Wis), Sense Motive (Wis), Sleight of Hand (Dex), Stealth (Dex), Swim (Str), and Use Magic Device (Cha). 
				
\textbf{Skill Ranks per Level}: 8 + Int modifier.
\begin{table*}[]
\caption{Table: Rogue}
\sffamily
\begin{tabular}{llllll}
\textbf{Level} & \textbf{Base Attack Bonus} & \textbf{Fort Save} & \textbf{Ref Save} & \textbf{Will Save} & \textbf{Special}\\
1st & +0 & +0 & +2 & +0 & Sneak attack 1d6, trapfinding\\
2nd & +1 & +0 & +3 & +0 & Evasion, rogue talent\\
3rd & +2 & +1 & +3 & +1 & Sneak attack 2d6, trap sense\\
4th & +3 & +1 & +4 & +1 & Rogue talent, uncanny dodge\\
5th & +3 & +1 & +4 & +1 & Sneak attack 3d6\\
6th & +4 & +2 & +5 & +2 & Rogue talent, trap sense\\
7th & +5 & +2 & +5 & +2 & Sneak attack 4d6\\
8th & +6/+1 & +2 & +6 & +2 & Improved uncanny dodge, rogue talent\\
9th & +6/+1 & +3 & +6 & +3 & Sneak attack 5d6, trap sense\\
10th & +7/+2 & +3 & +7 & +3 & Advanced talents, rogue talent\\
11th & +8/+3 & +3 & +7 & +3 & Sneak attack 6d6\\
12th & +9/+4 & +4 & +8 & +4 & Rogue talent, trap sense\\
13th & +9/+4 & +4 & +8 & +4 & Sneak attack 7d6\\
14th & +10/+5 & +4 & +9 & +4 & Rogue talent\\
15th & +11/+6/+1 & +5 & +9 & +5 & Sneak attack 8d6, trap sense\\
16th & +12/+7/+2 & +5 & +10 & +5 & Rogue talent\\
17th & +12/+7/+2 & +5 & +10 & +5 & Sneak attack 9d6\\
18th & +13/+8/+3 & +6 & +11 & +6 & Rogue talent, trap sense\\
19th & +14/+9/+4 & +6 & +11 & +6 & Sneak attack 10d6\\
20th & +15/+10/+5 & +6 & +12 & +6 & Master strike, rogue talent\\
\end{tabular}
\end{table*}
				
\subsection{Class Features}

				
The following are class features of the rogue.
				
\textbf{Weapon and Armor Proficiency}: Rogues are proficient with all simple weapons, plus the hand crossbow, rapier, sap, shortbow, and short sword. They are proficient with light armor, but not with shields.
				
\textbf{Sneak Attack}: If a rogue can catch an opponent when he is unable to defend himself effectively from her attack, she can strike a vital spot for extra damage.
				
The rogue's attack deals extra damage anytime her target would be denied a Dexterity bonus to AC (whether the target actually has a Dexterity bonus or not), or when the rogue flanks her target. This extra damage is 1d6 at 1st level, and increases by 1d6 every two rogue levels thereafter. Should the rogue score a critical hit with a sneak attack, this extra damage is not multiplied. Ranged attacks can count as sneak attacks only if the target is within 30 feet.
				
With a weapon that deals nonlethal damage (like a sap, whip, or an unarmed strike), a rogue can make a sneak attack that deals nonlethal damage instead of lethal damage. She cannot use a weapon that deals lethal damage to deal nonlethal damage in a sneak attack, not even with the usual --4 penalty.
				
The rogue must be able to see the target well enough to pick out a vital spot and must be able to reach such a spot. A rogue cannot sneak attack while striking a creature with concealment.
				
\textbf{Trapfinding}: A rogue adds 1/2 her level to Perception skill checks made to locate traps and to Disable Device skill checks (minimum +1). A rogue can use Disable Device to disarm magic traps.
				
\textbf{Evasion (Ex)}: At 2nd level and higher, a rogue can avoid even magical and unusual attacks with great agility. If she makes a successful Reflex saving throw against an attack that normally deals half damage on a successful save, she instead takes no damage. Evasion can be used only if the rogue is wearing light armor or no armor. A helpless rogue does not gain the benefit of evasion.
				
\textbf{Rogue Talents}: As a rogue gains experience, she learns a number of talents that aid her and confound her foes. Starting at 2nd level, a rogue gains one rogue talent. She gains an additional rogue talent for every 2 levels of rogue attained after 2nd level. A rogue cannot select an individual talent more than once.
				
Talents marked with an asterisk add effects to a rogue's sneak attack. Only one of these talents can be applied to an individual attack and the decision must be made before the attack roll is made. 
				
\textit{Bleeding Attack* (Ex)}: A rogue with this ability can cause living opponents to bleed by hitting them with a sneak attack. This attack causes the target to take 1 additional point of damage each round for each die of the rogue's sneak attack (e.g., 4d6 equals 4 points of bleed). Bleeding creatures take that amount of damage every round at the start of each of their turns. The bleeding can be stopped by a DC 15 Heal check or the application of any effect that heals hit point damage. Bleeding damage from this ability does not stack with itself. Bleeding damage bypasses any damage reduction the creature might possess.
				
\textit{Combat Trick}: A rogue that selects this talent gains a bonus combat feat (see Feats).
				
\textit{Fast Stealth (Ex)}: This ability allows a rogue to move at full speed using the Stealth skill without penalty.
				
\textit{Finesse Rogue}: A rogue that selects this talent gains Weapon Finesse as a bonus feat.
				
\textit{Ledge Walker (Ex)}: This ability allows a rogue to move along narrow surfaces at full speed using the Acrobatics skill without penalty. In addition, a rogue with this talent is not flat-footed when using Acrobatics to move along narrow surfaces.
				
\textit{Major Magic} \textit{(Sp)}: A rogue with this talent gains the ability to cast a 1st-level spell from the sorcerer/wizard spell list two times a day as a spell-like ability. The caster level for this ability is equal to the rogue's level. The save DC for this spell is 11 + the rogue's Intelligence modifier. The rogue must have an Intelligence of at least 11 to select this talent. A rogue must have the minor magic rogue talent before choosing this talent.
				
\textit{Minor Magic} \textit{(Sp)}: A rogue with this talent gains the ability to cast a 0-level spell from the sorcerer/wizard spell list. This spell can be cast three times a day as a spell-like ability. The caster level for this ability is equal to the rogue's level. The save DC for this spell is 10 + the rogue's Intelligence modifier. The rogue must have an Intelligence of at least 10 to select this talent.
				
\textit{Quick Disable (Ex)}: It takes a rogue with this ability half the normal amount of time to disable a trap using the Disable Device skill (minimum 1 round).
				
\textit{Resiliency (Ex)}: Once per day, a rogue with this ability can gain a number of temporary hit points equal to the rogue's level. Activating this ability is an immediate action that can only be performed when she is brought to below 0 hit points. This ability can be used to prevent her from dying. These temporary hit points last for 1 minute. If the rogue's hit points drop below 0 due to the loss of these temporary hit points, she falls unconscious and is dying as normal.
				
\textit{Rogue Crawl (Ex)}: While prone, a rogue with this ability can move at half speed. This movement provokes attacks of opportunity as normal. A rogue with this talent can take a 5-foot step while crawling.
				
\textit{Slow Reactions* (Ex)}: Opponents damaged by the rogue's sneak attack can't make attacks of opportunity for 1 round.
				
\textit{Stand Up (Ex)}: A rogue with this ability can stand up from a prone position as a free action. This still provokes attacks of opportunity for standing up while threatened by a foe.
				
\textit{Surprise Attack (Ex)}: During the surprise round, opponents are always considered flat-footed to a rogue with this ability, even if they have already acted. 
				
\textit{Trap Spotter (Ex)}: Whenever a rogue with this talent comes within 10 feet of a trap, she receives an immediate Perception skill check to notice the trap. This check should be made in secret by the GM.
				
\textit{Weapon Training}: A rogue that selects this talent gains Weapon Focus as a bonus feat.
				
\textbf{Trap Sense (Ex)}: At 3rd level, a rogue gains an intuitive sense that alerts her to danger from traps, giving her a +1 bonus on Reflex saves made to avoid traps and a +1 dodge bonus to AC against attacks made by traps. These bonuses rise to +2 when the rogue reaches 6th level, to +3 when she reaches 9th level, to +4 when she reaches 12th level, to +5 at 15th, and to +6 at 18th level.
				
Trap sense bonuses gained from multiple classes stack.
				
\textbf{Uncanny Dodge (Ex)}: Starting at 4th level, a rogue can react to danger before her senses would normally allow her to do so. She cannot be caught flat-footed, nor does she lose her Dex bonus to AC if the attacker is invisible. She still loses her Dexterity bonus to AC if immobilized. A rogue with this ability can still lose her Dexterity bonus to AC if an opponent successfully uses the feint action (see Combat) against her.
				
If a rogue already has uncanny dodge from a different class, she automatically gains improved uncanny dodge (see below) instead.
				
\textbf{Improved Uncanny Dodge (Ex)}: A rogue of 8th level or higher can no longer be flanked.
				
This defense denies another rogue the ability to sneak attack the character by flanking her, unless the attacker has at least four more rogue levels than the target does.
				
If a character already has uncanny dodge (see above) from another class, the levels from the classes that grant uncanny dodge stack to determine the minimum rogue level required to flank the character.
				
\textbf{Advanced Talents}: At 10th level, and every two levels thereafter, a rogue can choose one of the following advanced talents in place of a rogue talent.
				
\textit{Crippling Strike*} \textit{(Ex)}: A rogue with this ability can sneak attack opponents with such precision that her blows weaken and hamper them. An opponent damaged by one of her sneak attacks also takes 2 points of Strength damage.
				
\textit{Defensive Roll} \textit{(Ex)}: With this advanced talent, the rogue can roll with a potentially lethal blow to take less damage from it than she otherwise would. Once per day, when she would be reduced to 0 or fewer hit points by damage in combat (from a weapon or other blow, not a spell or special ability), the rogue can attempt to roll with the damage. To use this ability, the rogue must attempt a Reflex saving throw (DC = damage dealt). If the save succeeds, she takes only half damage from the blow; if it fails, she takes full damage. She must be aware of the attack and able to react to it in order to execute her defensive roll---if she is denied her Dexterity bonus to AC, she can't use this ability. Since this effect would not normally allow a character to make a Reflex save for half damage, the rogue's evasion ability does not apply to the defensive roll.
				
\textit{Dispelling Attack* (Su)}: Opponents that are dealt sneak attack damage by a rogue with this ability are affected by a targeted \textit{dispel magic}, targeting the lowest-level spell effect active on the target. The caster level for this ability is equal to the rogue's level. A rogue must have the major magic rogue talent before choosing dispelling attack.
				
\textit{Improved Evasion} \textit{(Ex)}: This works like evasion, except that while the rogue still takes no damage on a successful Reflex saving throw against attacks, she henceforth takes only half damage on a failed save. A helpless rogue does not gain the benefit of improved evasion.
				
\textit{Opportunist} \textit{(Ex)}: Once per round, the rogue can make an attack of opportunity against an opponent who has just been struck for damage in melee by another character. This attack counts as an attack of opportunity for that round. Even a rogue with the Combat Reflexes feat can't use the opportunist ability more than once per round.
				
\textit{Skill Mastery}: The rogue becomes so confident in the use of certain skills that she can use them reliably even under adverse conditions.
				
Upon gaining this ability, she selects a number of skills equal to 3 + her Intelligence modifier. When making a skill check with one of these skills, she may take 10 even if stress and distractions would normally prevent her from doing so. A rogue may gain this special ability multiple times, selecting additional skills for skill mastery to apply to each time.
				
\textit{Slippery Mind} \textit{(Ex)}: This ability represents the rogue's ability to wriggle free from magical effects that would otherwise control or compel her. If a rogue with slippery mind is affected by an enchantment spell or effect and fails her saving throw, she can attempt it again 1 round later at the same DC. She gets only this one extra chance to succeed on her saving throw.
				
\textit{Feat}: A rogue may gain any feat that she qualifies for in place of a rogue talent.
				
\textbf{Master Strike }\textit{(Ex)}: Upon reaching 20th level, a rogue becomes incredibly deadly when dealing sneak attack damage. Each time the rogue deals sneak attack damage, she can choose one of the following three effects: the target can be put to sleep for 1d4 hours, paralyzed for 2d6 rounds, or slain. Regardless of the effect chosen, the target receives a Fortitude save to negate the additional effect. The DC of this save is equal to 10 + 1/2 the rogue's level + the rogue's Intelligence modifier. Once a creature has been the target of a master strike, regardless of whether or not the save is made, that creature is immune to that rogue's master strike for 24 hours. Creatures that are immune to sneak attack damage are also immune to this ability.
        	

\section{Sorcerer}

\label{f0}				
Scions of innately magical bloodlines, the chosen of deities, the spawn of monsters, pawns of fate and destiny, or simply flukes of fickle magic, sorcerers look within themselves for arcane prowess and draw forth might few mortals can imagine. Emboldened by lives ever threatening to be consumed by their innate powers, these magic-touched souls endlessly indulge in and refine their mysterious abilities, gradually learning how to harness their birthright and coax forth ever greater arcane feats. Just as varied as these innately powerful spellcasters' abilities and inspirations are the ways in which they choose to utilize their gifts. While some seek to control their abilities through meditation and discipline, becoming masters of their fantastic birthright, others give in to their magic, letting it rule their lives with often explosive results. Regardless, sorcerers live and breathe that which other spellcasters devote their lives to mastering, and for them magic is more than a boon or a field of study; it is life itself.
				
\textbf{Role}: Sorcerers excel at casting a selection of favored spells frequently, making them powerful battle mages. As they become familiar with a specific and ever-widening set of spells, sorcerers often discover new and versatile ways of making use of magics other spellcasters might overlook. Their bloodlines also grant them additional abilities, assuring that no two sorcerers are ever quite alike.
				
\textbf{Alignment}: Any.
				
\textbf{Hit Die}: d6.
				
\subsection{Class Skills}

				
The sorcerer's class skills are Appraise (Int), Bluff (Cha), Craft (Int), Fly (Dex), Intimidate (Cha), Knowledge (arcana) (Int), Profession (Wis), Spellcraft (Int), and Use Magic Device (Cha).
				
\textbf{Skill Ranks per Level}: 2 + Int modifier.
% <div class="table">

Table: Sorcerer
% <
\begin{table}[]
\caption{Table: Sorcerer}
\sffamily
\setlength{\tabcolsep}{1pt}
\begin{tabularx}{\linewidth}{lp{6em}p{2.5em}p{2.5em}p{2.5em}Xlllllllll}
\multirow{2}{*}{Level} & \multirow{2}{*}{\parbox{5em}{Base Attack Bonus}} & \multirow{2}{*}{\parbox{1.5em}{Fort Save}} & \multirow{2}{*}{\parbox{1.5em}{Ref Save}} & \multirow{2}{*}{\parbox{1.5em}{Will Save}} & Special                                                                                              & \multicolumn{9}{c}{Spells per day} \\
                       &                                    &                            &                           &                            &                                                                                                  & 1st & 2nd & 3rd & 4th & 5th & 6th & 7th & 8th & 9th \\
\hline
1st & +0 & +0 & +0 & +2 & Bloodline power, cantrips, eschew materials & 3 & - & - & - & - & - & - & - & -\\
2nd & +1 & +0 & +0 & +3 &  & 4 & - & - & - & - & - & - & - & -\\
3rd & +1 & +1 & +1 & +3 & Bloodline power, bloodline spell & 5 & - & - & - & - & - & - & - & -\\
4th & +2 & +1 & +1 & +4 &  & 6 & 3 & - & - & - & - & - & - & -\\
5th & +2 & +1 & +1 & +4 & Bloodline spell & 6 & 4 & - & - & - & - & - & - & -\\
6th & +3 & +2 & +2 & +5 &  & 6 & 5 & 3 & - & - & - & - & - & -\\
7th & +3 & +2 & +2 & +5 & Bloodline feat, bloodline spell & 6 & 6 & 4 & - & - & - & - & - & -\\
8th & +4 & +2 & +2 & +6 &  & 6 & 6 & 5 & 3 & - & - & - & - & -\\
9th & +4 & +3 & +3 & +6 & Bloodline power, bloodline spell & 6 & 6 & 6 & 4 & - & - & - & - & -\\
10th & +5 & +3 & +3 & +7 &  & 6 & 6 & 6 & 5 & 3 & - & - & - & -\\
11th & +5 & +3 & +3 & +7 & Bloodline spell & 6 & 6 & 6 & 6 & 4 & - & - & - & -\\
12th & +6/+1 & +4 & +4 & +8 &  & 6 & 6 & 6 & 6 & 5 & 3 & - & - & -\\
13th & +6/+1 & +4 & +4 & +8 & Bloodline feat, bloodline spell & 6 & 6 & 6 & 6 & 6 & 4 & - & - & -\\
14th & +7/+2 & +4 & +4 & +9 &  & 6 & 6 & 6 & 6 & 6 & 5 & 3 & - & -\\
15th & +7/+2 & +5 & +5 & +9 & Bloodline power, bloodline spell & 6 & 6 & 6 & 6 & 6 & 6 & 4 & - & -\\
16th & +8/+3 & +5 & +5 & +10 &  & 6 & 6 & 6 & 6 & 6 & 6 & 5 & 3 & -\\
17th & +8/+3 & +5 & +5 & +10 & Bloodline spell & 6 & 6 & 6 & 6 & 6 & 6 & 6 & 4 & -\\
18th & +9/+4 & +6 & +6 & +11 &  & 6 & 6 & 6 & 6 & 6 & 6 & 6 & 5 & 3\\
19th & +9/+4 & +6 & +6 & +11 & Bloodline feat, bloodline spell & 6 & 6 & 6 & 6 & 6 & 6 & 6 & 6 & 4\\
20th & +10/+5 & +6 & +6 & +12 & Bloodline power & 6 & 6 & 6 & 6 & 6 & 6 & 6 & 6 & 6\\
\end{tabularx}
\end{table}

\begin{table}[]
\caption{Table: Sorcerer Spells Known}
\sffamily
\setlength{\tabcolsep}{1pt}
\begin{tabular}{lllllllllll}
Level & \multicolumn{10}{c}{Spells Known}\\
      & 0 & 1st & 2nd & 3rd & 4th & 5th & 6th & 7th & 8th & 9th\\
1st & 4 & 2 & - & - & - & - & - & - & - & -\\
2nd & 5 & 2 & - & - & - & - & - & - & - & -\\
3rd & 5 & 3 & - & - & - & - & - & - & - & -\\
4th & 6 & 3 & 1 & - & - & - & - & - & - & -\\
5th & 6 & 4 & 2 & - & - & - & - & - & - & -\\
6th & 7 & 4 & 2 & 1 & - & - & - & - & - & -\\
7th & 7 & 5 & 3 & 2 & - & - & - & - & - & -\\
8th & 8 & 5 & 3 & 2 & 1 & - & - & - & - & -\\
9th & 8 & 5 & 4 & 3 & 2 & - & - & - & - & -\\
10th & 9 & 5 & 4 & 3 & 2 & 1 & - & - & - & -\\
11th & 9 & 5 & 5 & 4 & 3 & 2 & - & - & - & -\\
12th & 9 & 5 & 5 & 4 & 3 & 2 & 1 & - & - & -\\
13th & 9 & 5 & 5 & 4 & 4 & 3 & 2 & - & - & -\\
14th & 9 & 5 & 5 & 4 & 4 & 3 & 2 & 1 & - & -\\
15th & 9 & 5 & 5 & 4 & 4 & 4 & 3 & 2 & - & -\\
16th & 9 & 5 & 5 & 4 & 4 & 4 & 3 & 2 & 1 & -\\
17th & 9 & 5 & 5 & 4 & 4 & 4 & 3 & 3 & 2 & -\\
18th & 9 & 5 & 5 & 4 & 4 & 4 & 3 & 3 & 2 & 1\\
19th & 9 & 5 & 5 & 4 & 4 & 4 & 3 & 3 & 3 & 2\\
20th & 9 & 5 & 5 & 4 & 4 & 4 & 3 & 3 & 3 & 3\\
\end{tabular}
\end{table}

\subsection{Class Features}

				
All of the following are class features of the sorcerer.
				
\textbf{Weapon and Armor Proficiency}: Sorcerers are proficient with all simple weapons. They are not proficient with any type of armor or shield. Armor interferes with a sorcerer's gestures, which can cause her spells with somatic components to fail (see Arcane Spells and Armor).
				
\textbf{Spells}: A sorcerer casts arcane spells drawn primarily from the sorcerer/wizard spell list presented in Spell Lists. She can cast any spell she knows without preparing it ahead of time. To learn or cast a spell, a sorcerer must have a Charisma score equal to at least 10 + the spell level. The Difficulty Class for a saving throw against a sorcerer's spell is 10 + the spell level + the sorcerer's Charisma modifier.
				
Like other spellcasters, a sorcerer can cast only a certain number of spells of each spell level per day. Her base daily spell allotment is given on Table: Sorcerer. In addition, she receives bonus spells per day if she has a high Charisma score (see Table: Ability Modifiers and Bonus Spells).
				
A sorcerer's selection of spells is extremely limited. A sorcerer begins play knowing four 0-level spells and two 1st-level spells of her choice. At each new sorcerer level, she gains one or more new spells, as indicated on Table: Sorcerer Spells Known. (Unlike spells per day, the number of spells a sorcerer knows is not affected by her Charisma score; the numbers on Table: Sorcerer Spells Known are fixed.) These new spells can be common spells chosen from the sorcerer/wizard spell list, or they can be unusual spells that the sorcerer has gained some understanding of through study. 
				
Upon reaching 4th level, and at every even-numbered sorcerer level after that (6th, 8th, and so on), a sorcerer can choose to learn a new spell in place of one she already knows. In effect, the sorcerer loses the old spell in exchange for the new one. The new spell's level must be the same as that of the spell being exchanged. A sorcerer may swap only a single spell at any given level, and must choose whether or not to swap the spell at the same time that she gains new spells known for the level.
				
Unlike a wizard or a cleric, a sorcerer need not prepare her spells in advance. She can cast any spell she knows at any time, assuming she has not yet used up her spells per day for that spell level.
				
\textbf{Bloodline}: Each sorcerer has a source of magic somewhere in her heritage that grants her spells, bonus feats, an additional class skill, and other special abilities. This source can represent a blood relation or an extreme event involving a creature somewhere in the family's past. For example, a sorcerer might have a dragon as a distant relative or her grandfather might have signed a terrible contract with a devil. Regardless of the source, this influence manifests in a number of ways as the sorcerer gains levels. A sorcerer must pick one bloodline upon taking her first level of sorcerer. Once made, this choice cannot be changed.
				
At 3rd level, and every two levels thereafter, a sorcerer learns an additional spell, derived from her bloodline. These spells are in addition to the number of spells given on Table: Sorcerer Spells Known. These spells cannot be exchanged for different spells at higher levels.
				
At 7th level, and every six levels thereafter, a sorcerer receives one bonus feat, chosen from a list specific to each bloodline. The sorcerer must meet the prerequisites for these bonus feats. 
				
\textbf{Cantrips}: Sorcerers learn a number of cantrips, or 0-level spells, as noted on Table: Sorcerer Spells Known under \texttt{{}"{}}Spells Known.\texttt{{}"{}} These spells are cast like any other spell, but they do not consume any slots and may be used again.
				
\textbf{Eschew Materials}: A sorcerer gains Eschew Materials as a bonus feat at 1st level.
				
\subsection{Sorcerer Bloodlines}

				
The following bloodlines represent only some of the possible sources of power that a sorcerer can draw upon. Unless otherwise noted, most sorcerers are assumed to have the arcane bloodline.
				
\subsection{Aberrant}

				
There is a taint in your blood, one that is alien and bizarre. You tend to think in odd ways, approaching problems from an angle that most would not expect. Over time, this taint manifests itself in your physical form.
				
\textbf{Class Skill}: Knowledge (dungeoneering).
				
\textbf{Bonus Spells}: \textit{enlarge person} (3rd), \textit{see invisibility} (5th), \textit{tongues} (7th), \textit{black tentacles} (9th), \textit{feeblemind} (11th), \textit{veil }(13th), \textit{plane shift} (15th), \textit{mind blank} (17th), \textit{shapechange }(19th).
				
\textbf{Bonus Feats}: Combat Casting, Improved Disarm, Improved Grapple, Improved Initiative, Improved Unarmed Strike, Iron Will, Silent Spell, Skill Focus (Knowledge \mbox{$[$}dungeoneering\mbox{$]$}).
				
\textbf{Bloodline Arcana}: Whenever you cast a spell of the polymorph subschool, increase the duration of the spell by 50\% (minimum 1 round). This bonus does not stack with the increase granted by the Extend Spell feat.
				
\textbf{Bloodline Powers}: Aberrant sorcerers show increasing signs of their tainted heritage as they increase in level, although they are only visible when used.
				
\textit{Acidic Ray (Sp)}: Starting at 1st level, you can fire an acidic ray as a standard action, targeting any foe within 30 feet as a ranged touch attack. The acidic ray deals 1d6 points of acid damage + 1 for every two sorcerer levels you possess. You can use this ability a number of times per day equal to 3 + your Charisma modifier.
				
\textit{Long Limbs (Ex)}: At 3rd level, your reach increases by 5 feet whenever you are making a melee touch attack. This ability does not otherwise increase your threatened area. At 11th level, this bonus to your reach increases to 10 feet. At 17th level, this bonus to your reach increases to 15 feet.
				
\textit{Unusual Anatomy (Ex)}: At 9th level, your anatomy changes, giving you a 25\% chance to ignore any critical hit or sneak attack scored against you. This chance increases to 50\% at 13th level.
				
\textit{Alien Resistance (Su)}: At 15th level, you gain spell resistance equal to your sorcerer level + 10.
				
\textit{Aberrant Form (Ex)}: At 20th level, your body becomes truly unnatural. You are immune to critical hits and sneak attacks. In addition, you gain blindsight with a range of 60 feet and damage reduction 5/---.
				
\subsection{Abyssal}

				
Generations ago, a demon spread its filth into your heritage. While it does not manifest in all of your kin, for you it is particularly strong. You might sometimes have urges to chaos or evil, but your destiny (and alignment) is up to you.
				
\textbf{Class Skill}: Knowledge (planes).
				
\textbf{Bonus Spells}: \textit{cause fear} (3rd), \textit{bull's strength} (5th), \textit{rage} (7th), \textit{stoneskin} (9th), \textit{dismissal} (11th), \textit{transformation} (13th), \textit{greater teleport} (15th), \textit{unholy aura} (17th), \textit{summon monster IX} (19th).
				
\textbf{Bonus Feats}: Augment Summoning, Cleave, Empower Spell, Great Fortitude, Improved Bull Rush, Improved Sunder, Power Attack, Skill Focus (Knowledge \mbox{$[$}planes\mbox{$]$}).
				
\textbf{Bloodline Arcana}: Whenever you cast a spell of the summoning subschool, the creatures summoned gain DR/good equal to 1/2 your sorcerer level (minimum 1). This does not stack with any DR the creature might have.
				
\textbf{Bloodline Powers}: While some would say that you are possessed, you know better. The demonic influence in your blood grows as you gain power.
				
\textit{Claws (Su)}: At 1st level, you can grow claws as a free action. These claws are treated as natural weapons, allowing you to make two claw attacks as a full attack action using your full base attack bonus. These attacks deal 1d4 points of damage each (1d3 if you are Small) plus your Strength modifier. At 5th level, these claws are considered magic weapons for the purpose of overcoming DR. At 7th level, the damage increases by one step to 1d6 points of damage (1d4 if you are Small). At 11th level, these claws become \textit{flaming} \textit{weapons}, each dealing an additional 1d6 points of fire damage on a successful hit. You can use your claws for a number of rounds per day equal to 3 + your Charisma modifier. These rounds do not need to be consecutive.
				
\textit{Demon Resistances (Ex)}: At 3rd level, you gain resist electricity 5 and a +2 bonus on saving throws made against poison. At 9th level, your resistance to electricity increases to 10 and your bonus on poison saving throws increases to +4.
				
\textit{Strength of the Abyss (Ex)}: At 9th level, you gain a +2 inherent bonus to your Strength. This bonus increases to +4 at 13th level, and to +6 at 17th level.
				
\textit{Added Summonings (Su)}: At 15th level, whenever you summon a creature with the demon subtype or the fiendish template using a \textit{summon monster} spell, you summon one additional creature of the same kind.
				
\textit{Demonic Might} \textit{(Su)}: At 20th level, the power of the Abyss flows through you. You gain immunity to electricity and poison. You also gain resistance to acid 10, cold 10, and fire 10, and gain telepathy with a range of 60 feet (allowing you to communicate with any creature that can speak a language).
				
\subsection{Arcane}

				
Your family has always been skilled in the eldritch art of magic. While many of your relatives were accomplished wizards, your powers developed without the need for study and practice.
				
\textbf{Class Skill}: Knowledge (any one).
				
\textbf{Bonus Spells}: \textit{identify} (3rd), \textit{invisibility} (5th), \textit{dispel magic} (7th), \textit{dimension door }(9th), \textit{overland flight }(11th), \textit{true seeing} (13th), \textit{greater teleport} (15th), \textit{power word stun} (17th), \textit{wish} (19th).
				
\textbf{Bonus Feats}: Combat Casting, Improved Counterspell, Improved Initiative, Iron Will, Scribe Scroll, Skill Focus (Knowledge \mbox{$[$}arcana\mbox{$]$}), Spell Focus, Still Spell.
				
\textbf{Bloodline Arcana}: Whenever you apply a metamagic feat to a spell that increases the slot used by at least one level, increase the spell's DC by +1. This bonus does not stack with itself and does not apply to spells modified by the Heighten Spell feat.
				
\textbf{Bloodline Powers}: Magic comes naturally to you, but as you gain levels you must take care to prevent the power from overwhelming you. 
				
\textit{Arcane Bond} \textit{(Su)}: At 1st level, you gain an arcane bond, as a wizard equal to your sorcerer level. Your sorcerer levels stack with any wizard levels you possess when determining the powers of your familiar or bonded object. This ability does not allow you to have both a familiar and a bonded item. Once per day, your bond item allows you to cast any one of our spells known (unlike a wizard's bonded item, which allows him to cast any one spell in his spellbook).
				
\textit{Metamagic Adept (Ex)}: At 3rd level, you can apply any one metamagic feat you know to a spell you are about to cast without increasing the casting time. You must still expend a higher-level spell slot to cast this spell. You can use this ability once per day at 3rd level and one additional time per day for every four sorcerer levels you possess beyond 3rd, up to five times per day at 19th level. At 20th level, this ability is replaced by arcane apotheosis.
				
\textit{New Arcana (Ex)}: At 9th level, you can add any one spell from the sorcerer/wizard spell list to your list of spells known. This spell must be of a level that you are capable of casting. You can also add one additional spell at 13th level and 17th level.
				
\textit{School Power (Ex)}: At 15th level, pick one school of magic. The DC for any spells you cast from that school increases by +2. This bonus stacks with the bonus granted by Spell Focus.
				
\textit{Arcane Apotheosis (Ex)}: At 20th level, your body surges with arcane power. You can add any metamagic feats that you know to your spells without increasing their casting time, although you must still expend higher-level spell slots. Whenever you use magic items that require charges, you can instead expend spell slots to power the item. For every three levels of spell slots that you expend, you consume one less charge when using a magic item that expends charges.
				
\subsection{Celestial}

				
Your bloodline is blessed by a celestial power, either from a celestial ancestor or through divine intervention. Although this power drives you along the path of good, your fate (and alignment) is your own to determine.
				
\textbf{Class Skill}: Heal.
				
\textbf{Bonus Spells}: \textit{bless} (3rd), \textit{resist energy }(5th), \textit{magic circle against evil} (7th), \textit{remove curse} (9th),\textit{ flame strike} (11th), \textit{greater dispel magic} (13th), \textit{banishment} (15th), \textit{sunburst} (17th),\textit{ gate }(19th).
				
\textbf{Bonus Feats}: Dodge, Extend Spell, Iron Will, Mobility, Mounted Combat, Ride-By Attack, Skill Focus (Knowledge \mbox{$[$}religion\mbox{$]$}), Weapon Finesse.
				
\textbf{Bloodline Arcana}: Whenever you cast a spell of the summoning subschool, the creatures summoned gain DR/evil equal to 1/2 your sorcerer level (minimum 1). This does not stack with any DR the creature might have.
				
\textbf{Bloodline Powers}: Your celestial heritage grants you a great many powers, but they come at a price. The lords of the higher planes are watching you and your actions closely.
				
\textit{Heavenly Fire (Sp)}: Starting at 1st level, you can unleash a ray of heavenly fire as a standard action, targeting any foe within 30 feet as a ranged touch attack. Against evil creatures, this ray deals 1d4 points of damage + 1 for every two sorcerer levels you possess. This damage is divine and not subject to energy resistance or immunity. This ray heals good creatures of 1d4 points of damage + 1 for every two sorcerer levels you possess. A good creature cannot benefit from your heavenly fire more than once per day. Neutral creatures are neither harmed nor healed by this effect. You can use this ability a number of times per day equal to 3 + your Charisma modifier.
				
\textit{Celestial Resistances (Ex)}: At 3rd level, you gain resist acid 5 and resist cold 5. At 9th level, your resistances increase to 10.
				
\textit{Wings of Heaven (Su)}: At 9th level, you can sprout feathery wings and fly for a number of minutes per day equal to your sorcerer level, with a speed of 60 feet and good maneuverability. This duration does not need to be consecutive, but it must be used in 1 minute increments. 
				
\textit{Conviction (Su)}: At 15th level, you can reroll any one ability check, attack roll, skill check, or saving throw you just made. You must decide to use this ability after the die is rolled, but before the results are revealed by the GM. You must take the second result, even if it is worse. You can use this ability once per day.
				
\textit{Ascension (Su)}: At 20th level, you become infused with the power of the heavens. You gain immunity to acid, cold, and petrification. You also gain resist electricity 10, resist fire 10, and a +4 racial bonus on saves against poison. Finally, you gain unlimited use of the wings of heaven ability. Finally, you gain the ability to speak with any creature that has a language (as per the \textit{tongues }spell).
				
\subsection{Destined}

				
Your family is destined for greatness in some way. Your birth could have been foretold in prophecy, or perhaps it occurred during an especially auspicious event, such as a solar eclipse. Regardless of your bloodline's origin, you have a great future ahead.
				
\textbf{Class Skill}: Knowledge (history).
				
\textbf{Bonus Spells}:\textit{ alarm} (3rd), \textit{blur} (5th), \textit{protection from energy} (7th), \textit{freedom of movement} (9th), \textit{break enchantment }(11th), \textit{mislead} (13th), \textit{spell turning} (15th), \textit{moment of prescience }(17th), \textit{foresight} (19th).
				
\textbf{Bonus Feats}: Arcane Strike, Diehard, Endurance, Leadership, Lightning Reflexes, Maximize Spell, Skill Focus (Knowledge \mbox{$[$}history\mbox{$]$}), Weapon Focus.
				
\textbf{Bloodline Arcana}: Whenever you cast a spell with a range of \texttt{{}"{}}personal,\texttt{{}"{}} you gain a luck bonus equal to the spell's level on all your saving throws for 1 round.
				
\textbf{Bloodline Powers}: You are destined for great things, and the powers that you gain serve to protect you.
				
\textit{Touch of Destiny} \textit{(Sp)}: At 1st level, you can touch a creature as a standard action, giving it an insight bonus on attack rolls, skill checks, ability checks, and saving throws equal to 1/2 your sorcerer level (minimum 1) for 1 round. You can use this ability a number of times per day equal to 3 + your Charisma modifier.
				
\textit{Fated} \textit{(Su)}: Starting at 3rd level, you gain a +1 luck bonus on all of your saving throws and to your AC during surprise rounds (see Combat) and when you are otherwise unaware of an attack. At 7th level and every four levels thereafter, this bonus increases by +1, to a maximum of +5 at 19th level.
				
\textit{It Was Meant To Be} \textit{(Su)}: At 9th level, you may reroll any one attack roll, critical hit confirmation roll, or level check made to overcome spell resistance. You must decide to use this ability after the first roll is made but before the results are revealed by the GM. You must take the second result, even if it is worse. At 9th level, you can use this ability once per day. At 17th level, you can use this ability twice per day.
				
\textit{Within Reach} \textit{(Su)}: At 15th level, your ultimate destiny is drawing near. Once per day, when an attack or spell that causes damage would result in your death, you may attempt a DC 20 Will save. If successful, you are instead reduced to --1 hit points and are automatically stabilized. The bonus from your fated ability applies to this save.
				
\textit{Destiny Realized} \textit{(Su)}: At 20th level, your moment of destiny is at hand. Any critical threats made against you only confirm if the second roll results in a natural 20 on the die. Any critical threats you score with a spell are automatically confirmed. Once per day, you can automatically succeed at one caster level check made to overcome spell resistance. You must use this ability before making the roll.
				
\subsection{Draconic}

				
At some point in your family's history, a dragon interbred with your bloodline, and now its ancient power flows through your veins. 
				
\textbf{Class Skill}: Perception.
				
\textbf{Bonus Spells}: \textit{mage armor} (3rd), \textit{resist energy} (5th), \textit{fly} (7th), \textit{fear} (9th), \textit{spell resistance} (11th), \textit{form of the dragon I }(13th), \textit{form of the dragon II} (15th), \textit{form of the dragon III }(17th), \textit{wish} (19th).
				
\textbf{Bonus Feats}: Blind-Fight, Great Fortitude, Improved Initiative, Power Attack, Quicken Spell, Skill Focus (Fly), Skill Focus (Knowledge \mbox{$[$}arcana\mbox{$]$}), Toughness.
				
\textbf{Bloodline Arcana}: Whenever you cast a spell with an energy descriptor that matches your draconic bloodline's energy type, that spell deals +1 point of damage per die rolled.
				
\textbf{Bloodline Powers}: The power of dragons flows through you and manifests in a number of ways. At 1st level, you must select one of the chromatic or metallic dragon types. This choice cannot be changed. A number of your abilities grant resistances and deal damage based on your dragon type, as noted on the following table.
% <thead href="../feats.html#toughness">


\begin{table}[]
\caption{Table: Sorcerer Spells Known}
\sffamily
\begin{tabular}{lll}
\textbf{Dragon Type} & \textbf{Energy Type} & \textbf{Breath Shape} \\
Black & Acid & 60-foot line \\
Blue & Electricity & 60-foot line \\
Green & Acid & 30-foot cone \\
Red & Fire & 30-foot cone \\
White & Cold & 30-foot cone \\
Brass & Fire & 60-foot line \\
Bronze & Electricity & 60-foot line\\
Copper & Acid & 60-foot line \\
Gold & Fire & 30-foot cone \\
Silver & Cold & 30-foot cone \\
\end{tabular}
\end{table}

\textit{Claws (Su)}: Starting at 1st level, you can grow claws as a free action. These claws are treated as natural weapons, allowing you to make two claw attacks as a full attack action using your full base attack bonus. Each of these attacks deals 1d4 points of damage plus your Strength modifier (1d3 if you are Small). At 5th level, these claws are considered magic weapons for the purpose of overcoming DR. At 7th level, the damage increases by one step to 1d6 points of damage (1d4 if you are Small). At 11th level, these claws deal an additional 1d6 points of damage of your energy type on a successful hit. You can use your claws for a number of rounds per day equal to 3 + your Charisma modifier. These rounds do not need to be consecutive.
				
\textit{Dragon Resistances (Ex)}: At 3rd level, you gain resist 5 against your energy type and a +1 natural armor bonus. At 9th level, your energy resistance increases to 10 and natural armor bonus increases to +2. At 15th level, your natural armor bonus increases to +4.
				
\textit{Breath Weapon (Su)}: At 9th level, you gain a breath weapon. This breath weapon deals 1d6 points of damage of your energy type per sorcerer level. Those caught in the area of the breath receive a Reflex save for half damage. The DC of this save is equal to 10 + 1/2 your sorcerer level + your Charisma modifier. The shape of the breath weapon depends on your dragon type (as indicated on the above chart). At 9th level, you can use this ability once per day. At 17th level, you can use this ability twice per day. At 20th level, you can use this ability three times per day.
				
\textit{Wings} \textit{(Su)}: At 15th level, leathery dragon wings grow from your back as a standard action, giving you a fly speed of 60 feet with average maneuverability. You can dismiss the wings as a free action.
				
\textit{Power of Wyrms} \textit{(Su)}: At 20th level, your draconic heritage becomes manifest. You gain immunity to paralysis, sleep, and damage of your energy type. You also gain blindsense 60 feet.
				
\subsection{Elemental}

				
The power of the elements resides in you, and at times you can hardly control its fury. This influence comes from an elemental outsider in your family history or a time when you or your relatives were exposed to a powerful elemental force.
				
\textbf{Class Skill}: Knowledge (planes).
				
\textbf{Bonus Spells}: \textit{burning hands}* (3rd), \textit{scorching ray}* (5th), \textit{protection from energy} (7th), \textit{elemental body I} (9th), \textit{elemental body II} (11th), \textit{elemental body III} (13th), \textit{elemental body IV} (15th), \textit{summon monster VIII} (elementals only) (17th), \textit{elemental swarm} (19th).
				
*These spells always deal a type of damage determined by your element. In addition, the subtype of these spells changes to match the energy type of your element.
				
\textbf{Bonus Feats}: Dodge, Empower Spell, Great Fortitude, Improved Initiative, Lightning Reflexes, Power Attack, Skill Focus (Knowledge \mbox{$[$}planes\mbox{$]$}), Weapon Finesse.
				
\textbf{Bloodline Arcana}: Whenever you cast a spell that deals energy damage, you can change the type of damage to match the type of your bloodline. This also changes the spell's type to match the type of your bloodline.
				
\textbf{Bloodline Powers}: One of the four elements infuses your being, and you can draw upon its power in times of need. At first level, you must select one of the four elements: air, earth, fire, or water. This choice cannot be changed. A number of your abilities grant resistances and deal damage based on your element, as noted below.
% <thead href="../feats.html#weapon-finesse">
\begin{table}[]
\sffamily
\begin{tabular}{lll}
\textbf{Element} & \textbf{Energy Type} & \textbf{Elemental Movement} \\
Air & Electricity & Fly 60 feet (average)\\
Earth & Acid & Burrow 30 feet \\
Fire & Fire & +30 feet base speed \\
Water & Cold & Swim 60 feet \\
\end{tabular}
\end{table}
				
\textit{Elemental Ray} \textit{(Sp)}: Starting at 1st level, you can unleash an elemental ray as a standard action, targeting any foe within 30 feet as a ranged touch attack. This ray deals 1d6 points of damage of your energy type + 1 for every two sorcerer levels you possess. You can use this ability a number of times per day equal to 3 + your Charisma modifier.
				
\textit{Elemental Resistance} \textit{(Ex)}: At 3rd level, you gain energy resistance 10 against your energy type. At 9th level, your energy resistance increases to 20.
				
\textit{Elemental Blast} \textit{(Sp)}: At 9th level, you can unleash a blast of elemental power once per day. This 20-foot-radius burst does 1d6 points of damage of your energy type per sorcerer level. Those caught in the area of your blast receive a Reflex save for half damage. Creatures that fail their saves gain vulnerability to your energy type until the end of your next turn. The DC of this save is equal to 10 + 1/2 your sorcerer level + your Charisma modifier. At 9th level, you can use this ability once per day. At 17th level, you can use this ability twice per day. At 20th level, you can use this ability three times per day. This power has a range of 60 feet.
				
\textit{Elemental Movement} \textit{(Su)}: At 15th level, you gain a special movement type or bonus. This ability is based on your chosen element, as indicated on the above chart.
				
\textit{Elemental Body (Su)}: At 20th level, elemental power surges through your body. You gain immunity to sneak attacks, critical hits, and damage from your energy type.
				
\subsection{Fey}

				
The capricious nature of the fey runs in your family due to some intermingling of fey blood or magic. You are more emotional than most, prone to bouts of joy and rage.
				
\textbf{Class Skill}: Knowledge (nature).
				
\textbf{Bonus Spells}: \textit{entangle }(3rd), \textit{hideous laughter} (5th), \textit{deep slumber} (7th), \textit{poison} (9th), \textit{tree stride} (11th), \textit{mislead} (13th), \textit{phase door} (15th), \textit{irresistible dance} (17th), \textit{shapechange }(19th).
				
\textbf{Bonus Feats}: Dodge, Improved Initiative, Lightning Reflexes, Mobility, Point Blank Shot, Precise Shot, Quicken Spell, Skill Focus (Knowledge \mbox{$[$}nature\mbox{$]$}).
				
\textbf{Bloodline Arcana}: Whenever you cast a spell of the compulsion subschool, increase the spell's DC by +2.
				
\textbf{Bloodline Powers}: You have always had a tie to the natural world, and as your power increases, so does the influence of the fey over your magic.
				
\textit{Laughing Touch} \textit{(Sp)}: At 1st level, you can cause a creature to burst out laughing for 1 round as a melee touch attack. A laughing creature can only take a move action but can defend itself normally. Once a creature has been affected by laughing touch, it is immune to its effects for 24 hours. You can use this ability a number of times per day equal to 3 + your Charisma modifier. This is a mind-affecting effect.
				
\textit{Woodland Stride} \textit{(Ex)}: At 3rd level, you can move through any sort of undergrowth (such as natural thorns, briars, overgrown areas, and similar terrain) at your normal speed and without taking damage or suffering any other impairment. Thorns, briars, and overgrown areas that have been magically manipulated to impede motion, however, still affect you.
				
\textit{Fleeting Glance} \textit{(Sp)}: At 9th level, you can turn invisible for a number of rounds per day equal to your sorcerer level. This ability functions as \textit{greater invisibility. }These rounds need not be consecutive.
				
\textit{Fey Magic (Su)}: At 15th level, you may reroll any caster level check made to overcome spell resistance. You must decide to use this ability before the results are revealed by the GM. You must take the second result, even if it is worse. You can use this ability at will.
				
\textit{Soul of the Fey} \textit{(Su)}: At 20th level, your soul becomes one with the world of the fey. You gain immunity to poison and DR 10/cold iron. Creatures of the animal type do not attack you unless compelled to do so through magic. Once per day, you can cast \textit{shadow walk} as a spell-like ability using your sorcerer level as your caster level.
				
\subsection{Infernal}

				
Somewhere in your family's history, a relative made a deal with a devil, and that pact has influenced your family line ever since. In you, it manifests in direct and obvious ways, granting you powers and abilities. While your fate is still your own, you can't help but wonder if your ultimate reward is bound to the Pit.
				
\textbf{Class Skill}: Diplomacy.
				
\textbf{Bonus Spells}: \textit{protection from good} (3rd), \textit{scorching ray} (5th), \textit{suggestion} (7th), \textit{charm monster} (9th), \textit{dominate person }(11th), \textit{planar binding} (devils and creatures with the fiendish template only) (13th), \textit{greater teleport} (15th), \textit{power word stun} (17th), \textit{meteor swarm} (19th).
				
\textbf{Bonus Feats}: Blind-Fight, Combat Expertise, Deceitful, Extend Spell, Improved Disarm, Iron Will, Skill Focus (Knowledge \mbox{$[$}planes\mbox{$]$}), Spell Penetration.
				
\textbf{Bloodline Arcana}: Whenever you cast a spell of the charm subschool, increase the spell's DC by +2.
				
\textbf{Bloodline Powers}: You can draw upon the power of Hell, although you must be wary of its corrupting influence. Such power does not come without a price.
				
\textit{Corrupting Touch} \textit{(Sp)}: At 1st level, you can cause a creature to become shaken as a melee touch attack. This effect persists for a number of rounds equal to 1/2 your sorcerer level (minimum 1). Creatures shaken by this ability radiate an aura of evil, as if they were an evil outsider (see \textit{detect evil}). Multiple touches do not stack, but they do add to the duration. You can use this ability a number of times per day equal to 3 + your Charisma modifier.
				
\textit{Infernal Resistances} \textit{(Ex)}: At 3rd level, you gain resist fire 5 and a +2 bonus on saving throws made against poison. At 9th level, your resistance to fire increases to 10 and your bonus on poison saving throws increases to +4.
				
\textit{Hellfire (Sp)}: At 9th level, you can call down a column of hellfire. This 10-foot-radius burst does 1d6 points of fire damage per sorcerer level. Those caught in the area of your blast receive a Reflex save for half damage. Good creatures that fail their saves are shaken for a number of rounds equal to your sorcerer level. The DC of this save is equal to 10 + 1/2 your sorcerer level + your Charisma modifier. At 9th level, you can use this ability once per day. At 17th level, you can use this ability twice per day. At 20th level, you can use this ability three times per day. This power has a range of 60 feet.
				
\textit{On Dark Wings} \textit{(Su)}: At 15th level, you can grow fearsome bat wings as a standard action, giving you a fly speed of 60 feet with average maneuverability. The wings can be dismissed as a free action.
				
\textit{Power of the Pit} \textit{(Su)}: At 20th level, your form becomes infused with vile power. You gain immunity to fire and poison. You also gain resistance to acid 10 and cold 10, and the ability to see perfectly in darkness of any kind to a range of 60 feet.
				
\subsection{Undead}

				
The taint of the grave runs through your family. Perhaps one of your ancestors became a powerful lich or vampire, or maybe you were born dead before suddenly returning to life. Either way, the forces of death move through you and touch your every action.
				
\textbf{Class Skill}: Knowledge (religion).
				
\textbf{Bonus Spells}: \textit{chill touch} (3rd), \textit{false life} (5th), \textit{vampiric touch} (7th), \textit{animate dead} (9th), \textit{waves of fatigue} (11th), \textit{undeath to death} (13th), \textit{finger of death} (15th), \textit{horrid wilting} (17th), \textit{energy drain} (19th).
				
\textbf{Bonus Feats}: Combat Casting, Diehard, Endurance, Iron Will, Skill Focus (Knowledge \mbox{$[$}religion\mbox{$]$}), Spell Focus, Still Spell, Toughness.
				
\textbf{Bloodline Arcana}: Some undead are susceptible to your mind-affecting spells. Corporeal undead that were once humanoids are treated as humanoids for the purposes of determining which spells affect them.
				
\textbf{Bloodline Powers}: You can call upon the foul powers of the afterlife. Unfortunately, the more you draw upon them, the closer you come to joining them.
				
\textit{Grave Touch (Sp)}: Starting at 1st level, you can make a melee touch attack as a standard action that causes a living creature to become shaken for a number of rounds equal to 1/2 your sorcerer level (minimum 1). If you touch a shaken creature with this ability, it becomes frightened for 1 round if it has fewer Hit Dice than your sorcerer level. You can use this ability a number of times per day equal to 3 + your Charisma modifier.
				
\textit{Death's Gift} \textit{(Su)}: At 3rd level, you gain resist cold 5 and DR 5/--- against nonlethal damage. At 9th level, your resistance to cold increases to 10 and your DR increases to 10/--- against nonlethal damage.
				
\textit{Grasp of the Dead} \textit{(Sp)}: At 9th level, you can cause a swarm of skeletal arms to burst from the ground to rip and tear at your foes. The skeletal arms erupt from the ground in a 20-foot-radius burst. Anyone in this area takes 1d6 points of slashing damage per sorcerer level. Those caught in the area receive a Reflex save for half damage. Those who fail the save are unable to move for 1 round. The DC of this save is equal to 10 + 1/2 your sorcerer level + your Charisma modifier. The skeletal arms disappear after 1 round. The arms must burst up from a solid surface. At 9th level, you can use this ability once per day. At 17th level, you can use this ability twice per day. At 20th level, you can use this ability three times per day. This power has a range of 60 feet.
				
\textit{Incorporeal Form} \textit{(Sp)}: At 15th level, you can become incorporeal for 1 round per sorcerer level. While in this form, you gain the incorporeal subtype. You only take half damage from corporeal sources as long as they are magic (you take no damage from non-magic weapons and objects). Likewise, your spells deal only half damage to corporeal creatures. Spells and other effects that do not deal damage function normally. You can use this ability once per day.
				
\textit{One of Us} \textit{(Ex)}: At 20th level, your form begins to rot (the appearance of this decay is up to you) and undead see you as one of them. You gain immunity to cold, nonlethal damage, paralysis, and sleep. You also gain DR 5/---. Unintelligent undead do not notice you unless you attack them. You receive a +4 morale bonus on saving throws made against spells and spell-like abilities cast by undead.
	

\section{Wizard}

\label{f0}				
Beyond the veil of the mundane hide the secrets of absolute power. The works of beings beyond mortals, the legends of realms where gods and spirits tread, the lore of creations both wondrous and terrible---such mysteries call to those with the ambition and the intellect to rise above the common folk to grasp true might. Such is the path of the wizard. These shrewd magic-users seek, collect, and covet esoteric knowledge, drawing on cultic arts to work wonders beyond the abilities of mere mortals. While some might choose a particular field of magical study and become masters of such powers, others embrace versatility, reveling in the unbounded wonders of all magic. In either case, wizards prove a cunning and potent lot, capable of smiting their foes, empowering their allies, and shaping the world to their every desire.
				
\textbf{Role}: While universalist wizards might study to prepare themselves for any manner of danger, specialist wizards research schools of magic that make them exceptionally skilled within a specific focus. Yet no matter their specialty, all wizards are masters of the impossible and can aid their allies in overcoming any danger.
				
\textbf{Alignment}: Any.
				
\textbf{Hit Die}: d6.
				
\subsection{Class Skills}

				
The wizard's class skills are Appraise (Int), Craft (Int), Fly (Dex), Knowledge (all) (Int), Linguistics (Int), Profession (Wis), and Spellcraft (Int). 
				
\textbf{Skill Ranks per Level}: 2 + Int modifier.

\begin{table*}[]
\caption{Table: Wizard}
\sffamily
\setlength{\tabcolsep}{1pt}
\begin{tabularx}{\linewidth}{lp{6em}p{2.5em}p{2.5em}p{2.5em}Xllllllllll}
\multirow{2}{*}{\textbf{Level}} & \multirow{2}{*}{\parbox{5em}{\textbf{Base Attack Bonus}}} & \multirow{2}{*}{\parbox{1.5em}{\textbf{Fort Save}}} & \multirow{2}{*}{\parbox{1.5em}{\textbf{Ref Save}}} & \multirow{2}{*}{\parbox{1.5em}{\textbf{Will Save}}} & \textbf{Special}     & \multicolumn{10}{c}{\textbf{Spells per day}} \\
                       &                                    &                            &                           &                            &                                                                                                  &  \textbf{0} & \textbf{1st} & \textbf{2nd} & \textbf{3rd} & \textbf{4th} & \textbf{5th} & \textbf{6th} & \textbf{7th} & \textbf{8th} & \textbf{9th} \\
1st & +0 & +0 & +0 & +2 & Arcane bond, arcane school, cantrips, Scribe Scroll & 3 & 1 & - & - & - & - & - & - & - & -\\
2nd & +1 & +0 & +0 & +3 &  & 4 & 2 & - & - & - & - & - & - & - & -\\
3rd & +1 & +1 & +1 & +3 &  & 4 & 2 & 1 & - & - & - & - & - & - & -\\
4th & +2 & +1 & +1 & +4 &  & 4 & 3 & 2 & - & - & - & - & - & - & -\\
5th & +2 & +1 & +1 & +4 & Bonus feat & 4 & 3 & 2 & 1 & - & - & - & - & - & -\\
6th & +3 & +2 & +2 & +5 &  & 4 & 3 & 3 & 2 & - & - & - & - & - & -\\
7th & +3 & +2 & +2 & +5 &  & 4 & 4 & 3 & 2 & 1 & - & - & - & - & -\\
8th & +4 & +2 & +2 & +6 &  & 4 & 4 & 3 & 3 & 2 & - & - & - & - & -\\
9th & +4 & +3 & +3 & +6 &  & 4 & 4 & 4 & 3 & 2 & 1 & - & - & - & -\\
10th & +5 & +3 & +3 & +7 & Bonus feat & 4 & 4 & 4 & 3 & 3 & 2 & - & - & - & -\\
11th & +5 & +3 & +3 & +7 &  & 4 & 4 & 4 & 4 & 3 & 2 & 1 & - & - & -\\
12th & +6/+1 & +4 & +4 & +8 &  & 4 & 4 & 4 & 4 & 3 & 3 & 2 & - & - & -\\
13th & +6/+1 & +4 & +4 & +8 &  & 4 & 4 & 4 & 4 & 4 & 3 & 2 & 1 & - & -\\
14th & +7/+2 & +4 & +4 & +9 &  & 4 & 4 & 4 & 4 & 4 & 3 & 3 & 2 & - & -\\
15th & +7/+2 & +5 & +5 & +9 & Bonus feat & 4 & 4 & 4 & 4 & 4 & 4 & 3 & 2 & 1 & -\\
16th & +8/+3 & +5 & +5 & +10 &  & 4 & 4 & 4 & 4 & 4 & 4 & 3 & 3 & 2 & -\\
17th & +8/+3 & +5 & +5 & +10 &  & 4 & 4 & 4 & 4 & 4 & 4 & 4 & 3 & 2 & 1\\
18th & +9/+4 & +6 & +6 & +11 &  & 4 & 4 & 4 & 4 & 4 & 4 & 4 & 3 & 3 & 2\\
19th & +9/+4 & +6 & +6 & +11 &  & 4 & 4 & 4 & 4 & 4 & 4 & 4 & 4 & 3 & 3\\
20th & +10/+5 & +6 & +6 & +12 & Bonus feat & 4 & 4 & 4 & 4 & 4 & 4 & 4 & 4 & 4 & 4\\
\end{tabularx}
\end{table*}
				
\subsection{Class Features}

				
The following are the class features of the wizard.
				
\textbf{Weapon and Armor Proficiency}: Wizards are proficient with the club, dagger, heavy crossbow, light crossbow, and quarterstaff, but not with any type of armor or shield. Armor interferes with a wizard's movements, which can cause his spells with somatic components to fail.
				
\textbf{Spells}: A wizard casts arcane spells drawn from the sorcerer/wizard spell list presented in Spell Lists. A wizard must choose and prepare his spells ahead of time.
				
To learn, prepare, or cast a spell, the wizard must have an Intelligence score equal to at least 10 + the spell level. The Difficulty Class for a saving throw against a wizard's spell is 10 + the spell level + the wizard's Intelligence modifier.
				
A wizard can cast only a certain number of spells of each spell level per day. His base daily spell allotment is given on Table: Wizard. In addition, he receives bonus spells per day if he has a high Intelligence score (see Table: Ability Modifiers and Bonus Spells).
				
A wizard may know any number of spells. He must choose and prepare his spells ahead of time by getting 8 hours of sleep and spending 1 hour studying his spellbook. While studying, the wizard decides which spells to prepare.
				
\textbf{Bonus Languages}: A wizard may substitute Draconic for one of the bonus languages available to the character because of his race.
				
\textbf{Arcane Bond (Ex or Sp)}: At 1st level, wizards form a powerful bond with an object or a creature. This bond can take one of two forms: a familiar or a bonded object. A familiar is a magical pet that enhances the wizard's skills and senses and can aid him in magic, while a bonded object is an item a wizard can use to cast additional spells or to serve as a magical item. Once a wizard makes this choice, it is permanent and cannot be changed. Rules for bonded items are given below, while rules for familiars are at the end of this section.
				
Wizards who select a bonded object begin play with one at no cost. Objects that are the subject of an arcane bond must fall into one of the following categories: amulet, ring, staff, wand, or weapon. These objects are always masterwork quality. Weapons acquired at 1st level are not made of any special material. If the object is an amulet or ring, it must be worn to have effect, while staves, wands, and weapons must be held in one hand. If a wizard attempts to cast a spell without his bonded object worn or in hand, he must make a concentration check or lose the spell. The DC for this check is equal to 20 + the spell's level. If the object is a ring or amulet, it occupies the ring or neck slot accordingly.
				
A bonded object can be used once per day to cast any one spell that the wizard has in his spellbook and is capable of casting, even if the spell is not prepared. This spell is treated like any other spell cast by the wizard, including casting time, duration, and other effects dependent on the wizard's level. This spell cannot be modified by metamagic feats or other abilities. The bonded object cannot be used to cast spells from the wizard's opposition schools (see arcane school).
				
A wizard can add additional magic abilities to his bonded object as if he has the required item creation feats and if he meets the level prerequisites of the feat. For example, a wizard with a bonded dagger must be at least 5th level to add magic abilities to the dagger (see the Craft Magic Arms and Armor feat in Feats). If the bonded object is a wand, it loses its wand abilities when its last charge is consumed, but it is not destroyed and it retains all of its bonded object properties and can be used to craft a new wand. The magic properties of a bonded object, including any magic abilities added to the object, only function for the wizard who owns it. If a bonded object's owner dies, or the item is replaced, the object reverts to being an ordinary masterwork item of the appropriate type.
				
If a bonded object is damaged, it is restored to full hit points the next time the wizard prepares his spells. If the object of an arcane bond is lost or destroyed, it can be replaced after 1 week in a special ritual that costs 200 gp per wizard level plus the cost of the masterwork item. This ritual takes 8 hours to complete. Items replaced in this way do not possess any of the additional enchantments of the previous bonded item. A wizard can designate an existing magic item as his bonded item. This functions in the same way as replacing a lost or destroyed item except that the new magic item retains its abilities while gaining the benefits and drawbacks of becoming a bonded item.
				
\textbf{Arcane School}: A wizard can choose to specialize in one school of magic, gaining additional spells and powers based on that school. This choice must be made at 1st level, and once made, it cannot be changed. A wizard that does not select a school receives the universalist school instead.
				
A wizard that chooses to specialize in one school of magic must select two other schools as his opposition schools, representing knowledge sacrificed in one area of arcane lore to gain mastery in another. A wizard who prepares spells from his opposition schools must use two spell slots of that level to prepare the spell. For example, a wizard with evocation as an opposition school must expend two of his available 3rd-level spell slots to prepare a \textit{fireball}. In addition, a specialist takes a --4 penalty on any skill checks made when crafting a magic item that has a spell from one of his opposition schools as a prerequisite. A universalist wizard can prepare spells from any school without restriction.
				
Each arcane school gives the wizard a number of school powers. In addition, specialist wizards receive an additional spell slot of each spell level he can cast, from 1st on up. Each day, a wizard can prepare a spell from his specialty school in that slot. This spell must be in the wizard's spellbook. A wizard can select a spell modified by a metamagic feat to prepare in his school slot, but it uses up a higher-level spell slot. Wizards with the universalist school do not receive a school slot.
				
\textbf{Cantrips:} Wizards can prepare a number of cantrips, or 0-level spells, each day, as noted on Table: Wizard under \texttt{{}"{}}Spells per Day.\texttt{{}"{}} These spells are cast like any other spell, but they are not expended when cast and may be used again. A wizard can prepare a cantrip from a prohibited school, but it uses up two of his available slots (see below).
				
\textbf{Scribe Scroll}: At 1st level, a wizard gains Scribe Scroll as a bonus feat. 
				
\textbf{Bonus Feats}: At 5th, 10th, 15th, and 20th level, a wizard gains a bonus feat. At each such opportunity, he can choose a metamagic feat, an item creation feat, or Spell Mastery. The wizard must still meet all prerequisites for a bonus feat, including caster level minimums. These bonus feats are in addition to the feats that a character of any class gets from advancing levels. The wizard is not limited to the categories of item creation feats, metamagic feats, or Spell Mastery when choosing those feats.
				
\textbf{Spellbooks}: A wizard must study his spellbook each day to prepare his spells. He cannot prepare any spell not recorded in his spellbook, except for \textit{read magic}, which all wizards can prepare from memory.
				
A wizard begins play with a spellbook containing all 0-level wizard spells (except those from his prohibited schools, if any; see Arcane Schools) plus three 1st-level spells of his choice. The wizard also selects a number of additional 1st-level spells equal to his Intelligence modifier to add to the spellbook. At each new wizard level, he gains two new spells of any spell level or levels that he can cast (based on his new wizard level) for his spellbook. At any time, a wizard can also add spells found in other wizards' spellbooks to his own (see Magic).
				
\subsection{Arcane Schools}

				
The following descriptions detail each arcane school and its corresponding powers.
				
\subsection{Abjuration School}

				
The abjurer uses magic against itself, and masters the art of defensive and warding magics.
				
\textit{Resistance (Ex)}: You gain resistance 5 to an energy type of your choice, chosen when you prepare spells. This resistance can be changed each day. At 11th level, this resistance increases to 10. At 20th level, this resistance changes to immunity to the chosen energy type.
				
\textit{Protective Ward (Su)}: As a standard action, you can create a 10-foot-radius field of protective magic centered on you that lasts for a number of rounds equal to your Intelligence modifier. All allies in this area (including you) receive a +1 deflection bonus to their AC. This bonus increases by +1 for every five wizard levels you possess. You can use this ability a number of times per day equal to 3 + your Intelligence modifier.
				
\textit{Energy Absorption (Su)}: At 6th level, you gain an amount of energy absorption equal to 3 times your wizard level per day. Whenever you take energy damage, apply immunity, vulnerability (if any), and resistance first and apply the rest to this absorption, reducing your daily total by that amount. Any damage in excess of your absorption is applied to you normally.
				
\subsection{Conjuration School}

				
The conjurer focuses on the study of summoning monsters and magic alike to bend to his will.
				
\textit{Summoner's Charm (Su)}: Whenever you cast a conjuration (summoning) spell, increase the duration by a number of rounds equal to 1/2 your wizard level (minimum 1). This increase is not doubled by Extend Spell. At 20th level, you can change the duration of all \textit{summon monster} spells to permanent. You can have no more than one \textit{summon monster} spell made permanent in this way at one time. If you designate another \textit{summon monster }spell as permanent, the previous spell immediately ends.
				
\textit{Acid Dart (Sp)}: As a standard action you can unleash an acid dart targeting any foe within 30 feet as a ranged touch attack. The acid dart deals 1d6 points of acid damage + 1 for every two wizard levels you possess. You can use this ability a number of times per day equal to 3 + your Intelligence modifier. This attack ignores spell resistance.
				
\textit{Dimensional Steps (Sp)}: At 8th level, you can use this ability to teleport up to 30 feet per wizard level per day as a standard action. This teleportation must be used in 5-foot increments and such movement does not provoke an attack of opportunity. You can bring other willing creatures with you, but you must expend an equal amount of distance for each additional creature brought with you.
				
\subsection{Divination School}

				
Diviners are masters of remote viewing, prophecies, and using magic to explore the world.
				
\textit{Forewarned (Su)}: You can always act in the surprise round even if you fail to make a Perception roll to notice a foe, but you are still considered flat-footed until you take an action. In addition, you receive a bonus on initiative checks equal to 1/2 your wizard level (minimum +1). At 20th level, anytime you roll initiative, assume the roll resulted in a natural 20.
				
\textit{Diviner's Fortune (Sp)}: When you activate this school power, you can touch any creature as a standard action to give it an insight bonus on all of its attack rolls, skill checks, ability checks, and saving throws equal to 1/2 your wizard level (minimum +1) for 1 round. You can use this ability a number of times per day equal to 3 + your Intelligence modifier.
				
\textit{Scrying Adept (Su)}: At 8th level, you are always aware when you are being observed via magic, as if you had a permanent \textit{detect scrying}. In addition, whenever you scry on a subject, treat the subject as one step more familiar to you. Very familiar subjects get a --10 penalty on their save to avoid your scrying attempts.
				
\subsection{Enchantment School}

				
The enchanter uses magic to control and manipulate the minds of his victims.
				
\textit{Enchanting Smile (Su)}: You gain a +2 enhancement bonus on Bluff, Diplomacy, and Intimidate skill checks. This bonus increases by +1 for every five wizard levels you possess, up to a maximum of +6 at 20th level. At 20th level, whenever you succeed at a saving throw against a spell of the enchantment school, that spell is reflected back at its caster, as per \textit{spell turning}.
				
\textit{Dazing Touch (Sp)}: You can cause a living creature to become dazed for 1 round as a melee touch attack. Creatures with more Hit Dice than your wizard level are unaffected. You can use this ability a number of times per day equal to 3 + your Intelligence modifier.
				
\textit{Aura of Despair (Su)}: At 8th level, you can emit a 30-foot aura of despair for a number of rounds per day equal to your wizard level. Enemies within this aura take a --2 penalty on ability checks, attack rolls, damage rolls, saving throws, and skill checks. These rounds do not need to be consecutive. This is a mind-affecting effect.
				
\subsection{Evocation School}

				
Evokers revel in the raw power of magic, and can use it to create and destroy with shocking ease.
				
\textit{Intense Spells (Su)}: Whenever you cast an evocation spell that deals hit point damage, add 1/2 your wizard level to the damage (minimum +1). This bonus only applies once to a spell, not once per missile or ray, and cannot be split between multiple missiles or rays. This bonus damage is not increased by Empower Spell or similar effects. This damage is of the same type as the spell. At 20th level, whenever you cast an evocation spell you can roll twice to penetrate a creature's spell resistance and take the better result.
				
\textit{Force Missile} \textit{(Sp)}: As a standard action you can unleash a force missile that automatically strikes a foe, as \textit{magic missile}. The force missile deals 1d4 points of damage plus the damage from your intense spells evocation power. This is a force effect. You can use this ability a number of times per day equal to 3 + your Intelligence modifier.
				
\textit{Elemental Wall (Sp)}: At 8th level, you can create a wall of energy that lasts for a number of rounds per day equal to your wizard level. These rounds do not need to be consecutive. This wall deals acid, cold, electricity, or fire damage, determined when you create it. The elemental wall otherwise functions like \textit{wall of fire}.
				
\subsection{Illusion School}

				
Illusionists use magic to weave confounding images, figments, and phantoms to baffle and vex their foes.
				
\textit{Extended Illusions (Su)}: Any illusion spell you cast with a duration of \texttt{{}"{}}concentration\texttt{{}"{}} lasts a number of additional rounds equal to 1/2 your wizard level after you stop maintaining concentration (minimum +1 round). At 20th level, you can make one illusion spell with a duration of \texttt{{}"{}}concentration\texttt{{}"{}} become permanent. You can have no more than one illusion made permanent in this way at one time. If you designate another illusion as permanent, the previous permanent illusion ends. 
				
\textit{Blinding Ray (Sp)}: As a standard action you can fire a shimmering ray at any foe within 30 feet as a ranged touch attack. The ray causes creatures to be blinded for 1 round. Creatures with more Hit Dice than your wizard level are dazzled for 1 round instead. You can use this ability a number of times per day equal to 3 + your Intelligence modifier.
				
\textit{Invisibility Field (Sp)}: At 8th level, you can make yourself invisible as a swift action for a number of rounds per day equal to your wizard level. These rounds do not need to be consecutive. This otherwise functions as \textit{greater invisibility.}
				
\subsection{Necromancy School}

				
The dread and feared necromancer commands undead and uses the foul power of unlife against his enemies.
				
\textit{Power over Undead (Su)}: You receive Command Undead or Turn Undead as a bonus feat. You can channel energy a number of times per day equal to 3 + your Intelligence modifier, but only to use the selected feat. You can take other feats to add to this ability, such as Extra Channel and Improved Channel, but not feats that alter this ability, such as Elemental Channel and Alignment Channel. The DC to save against these feats is equal to 10 + 1/2 your wizard level + your Charisma modifier. At 20th level, undead cannot add their channel resistance to the save against this ability.
				
\textit{Grave Touch (Sp)}: As a standard action, you can make a melee touch attack that causes a living creature to become shaken for a number of rounds equal to 1/2 your wizard level (minimum 1). If you touch a shaken creature with this ability, it becomes frightened for 1 round if it has fewer Hit Dice than your wizard level. You can use this ability a number of times per day equal to 3 + your Intelligence modifier.
				
\textit{Life Sight (Su)}: At 8th level, you gain blindsight to a range of 10 feet for a number of rounds per day equal to your wizard level. This ability only allows you to detect living creatures and undead creatures. This sight also tells you whether a creature is living or undead. Constructs and other creatures that are neither living nor undead cannot be seen with this ability. The range of this ability increases by 10 feet at 12th level, and by an additional 10 feet for every four levels beyond 12th. These rounds do not need to be consecutive.
				
\subsection{Transmutation School}

				
Transmuters use magic to change the world around them.
				
\textit{Physical Enhancement (Su)}: You gain a +1 enhancement bonus to one physical ability score (Strength, Dexterity, or Constitution). This bonus increases by +1 for every five wizard levels you possess to a maximum of +5 at 20th level. You can change this bonus to a new ability score when you prepare spells. At 20th level, this bonus applies to two physical ability scores of your choice.
				
\textit{Telekinetic Fist (Sp)}: As a standard action you can strike with a telekinetic fist, targeting any foe within 30 feet as a ranged touch attack. The telekinetic fist deals 1d4 points of bludgeoning damage + 1 for every two wizard levels you possess. You can use this ability a number of times per day equal to 3 + your Intelligence modifier.
				
\textit{Change Shape (Sp)}: At 8th level, you can change your shape for a number of rounds per day equal to your wizard level. These rounds do not need to be consecutive. This ability otherwise functions like \textit{beast shape II} or \textit{elemental body I.} At 12th level, this ability functions like \textit{beast shape III} or \textit{elemental body II}.
				
\subsection{Universalist School}

				
Wizards who do not specialize (known as universalists) have the most diversity of all arcane spellcasters.
				
\textit{Hand of the Apprentice (Su)}: You cause your melee weapon to fly from your grasp and strike a foe before instantly returning to you. As a standard action, you can make a single attack using a melee weapon at a range of 30 feet. This attack is treated as a ranged attack with a thrown weapon, except that you add your Intelligence modifier on the attack roll instead of your Dexterity modifier (damage still relies on Strength). This ability cannot be used to perform a combat maneuver. You can use this ability a number of times per day equal to 3 + your Intelligence modifier.
				
\textit{Metamagic Mastery (Su)}: At 8th level, you can apply any one metamagic feat that you know to a spell you are about to cast. This does not alter the level of the spell or the casting time. You can use this ability once per day at 8th level and one additional time per day for every two wizard levels you possess beyond 8th. Any time you use this ability to apply a metamagic feat that increases the spell level by more than 1, you must use an additional daily usage for each level above 1 that the feat adds to the spell. Even though this ability does not modify the spell's actual level, you cannot use this ability to cast a spell whose modified spell level would be above the level of the highest-level spell that you are capable of casting.
				
\subsection{Familiars}

				
A familiar is an animal chosen by a spellcaster to aid him in his study of magic. It retains the appearance, Hit Dice, base attack bonus, base save bonuses, skills, and feats of the normal animal it once was, but is now a magical beast for the purpose of effects that depend on its type. Only a normal, unmodified animal may become a familiar. An animal companion cannot also function as a familiar.
				
A familiar grants special abilities to its master, as given on the table below. These special abilities apply only when the master and familiar are within 1 mile of each other.
				
Levels of different classes that are entitled to familiars stack for the purpose of determining any familiar abilities that depend on the master's level.
				
If a familiar is dismissed, lost or dies, it can be replaced 1 week later through a specialized ritual that costs 200 gp per wizard level. The ritual takes 8 hours to complete.


\begin{table}
 \sffamily
 \caption{Familiars}
 \begin{tabular}{ll}
  \textbf{Familiar} & \textbf{Special Ability} \\
Bat & Master gains a +3 bonus on Fly checks\\
Cat & Master gains a +3 bonus on Stealth checks\\
Hawk & Master gains a +3 bonus on sight-based and \\
     & opposed Perception checks in bright light\\
Lizard & Master gains a +3 bonus on Climb checks\\
Monkey & Master gains a +3 bonus on Acrobatics checks\\
Owl & Master gains a +3 bonus on sight-based and\\
    & opposed Perception checks in shadows or darkness\\
Rat & Master gains a +2 bonus on Fortitude saves\\
Raven* & Master gains a +3 bonus on Appraise checks\\
Viper & Master gains a +3 bonus on Bluff checks\\
Toad & Master gains +3 hit points\\
Weasel & Master gains a +2 bonus on Reflex saves\\
 \end{tabular}
*A raven familiar can speak one language of its master's choice as a supernatural ability.
\end{table}

				
\textbf{Familiar Basics}: Use the basic statistics for a creature of the familiar's kind, but with the following changes.
				
\textit{Hit Dice}: For the purpose of effects related to number of Hit Dice, use the master's character level or the familiar's normal HD total, whichever is higher.
				
\textit{Hit Points}: The familiar has half the master's total hit points (not including temporary hit points), rounded down, regardless of its actual Hit Dice.
				
\textit{Attacks}: Use the master's base attack bonus, as calculated from all his classes. Use the familiar's Dexterity or Strength modifier, whichever is greater, to calculate the familiar's melee attack bonus with natural weapons. 
				
Damage equals that of a normal creature of the familiar's kind.
				
\textit{Saving Throws}: For each saving throw, use either the familiar's base save bonus (Fortitude +2, Reflex +2, Will +0) or the master's (as calculated from all his classes), whichever is better. The familiar uses its own ability modifiers to saves, and it doesn't share any of the other bonuses that the master might have on saves.
				
\textit{Skills}: For each skill in which either the master or the familiar has ranks, use either the normal skill ranks for an animal of that type or the master's skill ranks, whichever is better. In either case, the familiar uses its own ability modifiers. Regardless of a familiar's total skill modifiers, some skills may remain beyond the familiar's ability to use. Familiars treat Acrobatics, Climb, Fly, Perception, Stealth, and Swim as class skills.
				
\textbf{Familiar Ability Descriptions}: All familiars have special abilities (or impart abilities to their masters) depending on the master's combined level in classes that grant familiars, as shown on the table below. The abilities are cumulative. 

\begin{table}
 \sffamily
 \begin{tabular}{llll}
\textbf{Master}      & \textbf{Natural}\\
\textbf{Class Level} & \textbf{Armor Adj.} & \textbf{Int} & \textbf{Special} \\
1st--2nd & +1 & 6 & Alertness, share spells, \\
         &    &   & improved evasion, \\
         &    &   & empathic link \\
3rd--4th & +2 & 7 & Deliver touch spells\\
5th--6th & +3 & 8 & Speak with master\\
7th--8th & +4 & 9 & Speak with animals \\
         &    &   & of its kind\\
9th--10th & +5 & 10 & - \\
11th--12th & +6 & 11 & Spell resistance \\
13th--14th & +7 & 12 & Scry on familiar\\
15th--16th & +8 & 13 & - \\
17th--18th & +9 & 14 & -\\
19th--20th & +10 & 15 & -\\  
 \end{tabular}

\end{table}
			
\textit{Natural Armor Adj.}: The number noted here is in addition to the familiar's existing natural armor bonus.
				
\textit{Int}: The familiar's Intelligence score.
				
\textit{Alertness (Ex)}: While a familiar is within arm's reach, the master gains the Alertness feat.
				
\textit{Improved Evasion (Ex)}: When subjected to an attack that normally allows a Reflex saving throw for half damage, a familiar takes no damage if it makes a successful saving throw and half damage even if the saving throw fails.
				
\textit{Share Spells}: The wizard may cast a spell with a target of \texttt{{}"{}}You\texttt{{}"{}} on his familiar (as a touch spell) instead of on himself. A wizard may cast spells on his familiar even if the spells do not normally affect creatures of the familiar's type (magical beast).
				
\textit{Empathic Link (Su)}: The master has an empathic link with his familiar to a 1 mile distance. The master can communicate empathically with the familiar, but cannot see through its eyes. Because of the link's limited nature, only general emotions can be shared. The master has the same connection to an item or place that his familiar does.
				
\textit{Deliver Touch Spells (Su)}: If the master is 3rd level or higher, a familiar can deliver touch spells for him. If the master and the familiar are in contact at the time the master casts a touch spell, he can designate his familiar as the \texttt{{}"{}}toucher.\texttt{{}"{}} The familiar can then deliver the touch spell just as the master would. As usual, if the master casts another spell before the touch is delivered, the touch spell dissipates.
				
\textit{Speak with Master (Ex)}: If the master is 5th level or higher, a familiar and the master can communicate verbally as if they were using a common language. Other creatures do not understand the communication without magical help.
				
\textit{Speak with Animals of Its Kind (Ex)}: If the master is 7th level or higher, a familiar can communicate with animals of approximately the same kind as itself (including dire varieties): bats with bats, cats with felines, hawks and owls and ravens with birds, lizards and snakes with reptiles, monkeys with other simians, rats with rodents, toads with amphibians, and weasels with ermines and minks. Such communication is limited by the Intelligence of the conversing creatures.
				
\textit{Spell Resistance (Ex)}: If the master is 11th level or higher, a familiar gains spell resistance equal to the master's level + 5. To affect the familiar with a spell, another spellcaster must get a result on a caster level check (1d20 + caster level) that equals or exceeds the familiar's spell resistance.
				
\textit{Scry on Familiar (Sp)}: If the master is 13th level or higher, he may scry on his familiar (as if casting the \textit{scrying }spell) once per day.
				
\subsection{Arcane Spells and Armor}

				
Armor restricts the complicated gestures required while casting any spell that has a somatic component. The armor and shield descriptions list the arcane spell failure chance for different armors and shields.
				
If a spell doesn't have a somatic component, an arcane spellcaster can cast it with no arcane spell failure chance while wearing armor. Such spells can also be cast even if the caster's hands are bound or he is grappling (although concentration checks still apply normally). The metamagic feat Still Spell allows a spellcaster to prepare or cast a spell without the somatic component at one spell level higher than normal. This also provides a way to cast a spell while wearing armor without risking arcane spell failure.
        	

