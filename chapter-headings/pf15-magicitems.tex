\chapter{Magic Items}
\section{Magic Items and Detect Magic}

When \textit{detect magic} identifies a magic item's school of magic, this information refers to the school of the spell placed within the potion, scroll, or wand, or the prerequisite given for the item. The description of each item provides its aura strength and the school to which it belongs. \\

If more than one spell is given as a prerequisite, use the highest-level spell. If no spells are included in the prerequisites, use the following default guidelines.\\

\begin{tabular}{ll}
\textbf{Item Nature} & \textbf{School} \\
Armor and protection items & Abjuration\\
Weapons or offensive items & Evocation\\
Bonus to ability score, skill check, etc. & Transmutation\\
\end{tabular}

\section{Using Items}

			
To use a magic item, it must be activated, although sometimes activation simply means putting a ring on your finger. Some items, once donned, function constantly. In most cases, though, using an item requires a standard action that does not provoke attacks of opportunity. By contrast, spell completion items are treated like spells in combat and do provoke attacks of opportunity.
				
Activating a magic item is a standard action unless the item description indicates otherwise. However, the casting time of a spell is the time required to activate the same power in an item, regardless of the type of magic item, unless the item description specifically states otherwise.
				
The four ways to activate magic items are described below.
				
\textbf{Spell Completion}: This is the activation method for scrolls. A scroll is a spell that is mostly finished. The preparation is done for the caster, so no preparation time is needed beforehand as with normal spellcasting. All that's left to do is perform the finishing parts of the spellcasting (the final gestures, words, and so on). To use a spell completion item safely, a character must be of high enough level in the right class to cast the spell already. If he can't already cast the spell, there's a chance he'll make a mistake. Activating a spell completion item is a standard action (or the spell's casting time, whichever is longer) and provokes attacks of opportunity exactly as casting a spell does.
				
\textbf{Spell Trigger}: Spell trigger activation is similar to spell completion, but it's even simpler. No gestures or spell finishing is needed, just a special knowledge of spellcasting that an appropriate character would know, and a single word that must be spoken. Spell trigger items can be used by anyone whose class can cast the corresponding spell. This is the case even for a character who can't actually cast spells, such as a 3rd-level paladin. The user must still determine what spell is stored in the item before she can activate it. Activating a spell trigger item is a standard action and does not provoke attacks of opportunity.
				
\textbf{Command Word}: If no activation method is suggested either in the magic item description or by the nature of the item, assume that a command word is needed to activate it. Command word activation means that a character speaks the word and the item activates. No other special knowledge is needed.
				
A command word can be a real word, but when this is the case, the holder of the item runs the risk of activating the item accidentally by speaking the word in normal conversation. More often, the command word is some seemingly nonsensical word, or a word or phrase from an ancient language no longer in common use. Activating a command word magic item is a standard action and does not provoke attacks of opportunity.
				
Sometimes the command word to activate an item is written right on the item. Occasionally, it might be hidden within a pattern or design engraved on, carved into, or built into the item, or the item might bear a clue to the command word.
				
The Knowledge (arcana) and Knowledge (history) skills might be useful in helping to identify command words or deciphering clues regarding them. A successful check against DC 30 is needed to come up with the word itself. If that check is failed, succeeding on a second check (DC 25) might provide some insight into a clue. The spells \textit{detect magic, identify, }and \textit{analyze dweomer }all reveal command words if the properties of the item are successfully identified.
				
\textbf{Use Activated}: This type of item simply has to be used in order to activate it. A character has to drink a potion, swing a sword, interpose a shield to deflect a blow in combat, look through a lens, sprinkle dust, wear a ring, or don a hat. Use activation is generally straightforward and self-explanatory.
				
Many use-activated items are objects that a character wears. Continually functioning items are practically always items that one wears. A few must simply be in the character's possession (meaning on his person). However, some items made for wearing must still be activated. Although this activation sometimes requires a command word (see above), usually it means mentally willing the activation to happen. The description of an item states whether a command word is needed in such a case.
				
Unless stated otherwise, activating a use-activated magic item is either a standard action or not an action at all and does not provoke attacks of opportunity, unless the use involves performing an action that provokes an attack of opportunity in itself. If the use of the item takes time before a magical effect occurs, then use activation is a standard action. If the item's activation is subsumed in its use and takes no extra time use, activation is not an action at all.
				
Use activation doesn't mean that if you use an item, you automatically know what it can do. You must know (or at least guess) what the item can do and then use the item in order to activate it, unless the benefit of the item comes automatically, such as from drinking a potion or swinging a sword.
				
\subsection{Size and Magic Items}

				
When an article of magic clothing or jewelry is discovered, most of the time size shouldn't be an issue. Many magic garments are made to be easily adjustable, or they adjust themselves magically to the wearer. Size should not keep characters of various kinds from using magic items.
				
There may be rare exceptions, especially with race-specific items. 
				
\textbf{Armor and Weapon Sizes}: Armor and weapons that are found at random have a 30\% chance of being Small (01--30), a 60\% chance of being Medium (31--90), and a 10\% chance of being any other size (91--100).
				
\section{Magic Items on the Body}

				
Many magic items need to be donned by a character who wants to employ them or benefit from their abilities. It's possible for a creature with a humanoid-shaped body to wear as many as 15 magic items at the same time. However, each of those items must be worn on (or over) a particular part of the body, known as a \texttt{{}"{}}slot.\texttt{{}"{}}
				
A humanoid-shaped body can be decked out in magic gear consisting of one item from each of the following groups, keyed to which slot on the body the item is worn.
				
\textbf{Armor}: suits of armor.
				
\textbf{Belts}: belts and girdles.
				
\textbf{Body}: robes and vestments.
				
\textbf{Chest}: mantles, shirts, and vests.
				
\textbf{Eyes}: eyes, glasses, and goggles.
				
\textbf{Feet}: boots, shoes, and slippers.
				
\textbf{Hands}: gauntlets and gloves.
				
\textbf{Head}: circlets, crowns, hats, helms, and masks.
				
\textbf{Headband}: headbands and phylacteries.
				
\textbf{Neck}: amulets, brooches, medallions, necklaces, periapts, and scarabs.
				
\textbf{Ring (up to two)}: rings.
				
\textbf{Shield}: shields.
				
\textbf{Shoulders}: capes and cloaks.
				
\textbf{Wrist}: bracelets and bracers. 
				
Of course, a character may carry or possess as many items of the same type as he wishes. However, additional items beyond those in the slots listed above have no effect. 
				
Some items can be worn or carried without taking up a slot on a character's body. The description of an item indicates when an item has this property.
				
\section{Saving Throws Against Magic Item Powers}

				
Magic items produce spells or spell-like effects. For a saving throw against a spell or spell-like effect from a magic item, the DC is 10 + the level of the spell or effect + the ability modifier of the minimum ability score needed to cast that level of spell. 
				
Staves are an exception to the rule. Treat the saving throw as if the wielder cast the spell, including caster level and all modifiers to save DCs. 
				
Most item descriptions give saving throw DCs for various effects, particularly when the effect has no exact spell equivalent (making its level otherwise difficult to determine quickly).
				
\section{Damaging Magic Items}

				
A magic item doesn't need to make a saving throw unless it is unattended, it is specifically targeted by the effect, or its wielder rolls a natural 1 on his save. Magic items should always get a saving throw against spells that might deal damage to them---even against attacks from which a nonmagical item would normally get no chance to save. Magic items use the same saving throw bonus for all saves, no matter what the type (Fortitude, Reflex, or Will). A magic item's saving throw bonus equals 2 + 1/2 its caster level (rounded down). The only exceptions to this are intelligent magic items, which make Will saves based on their own Wisdom scores.
				
Magic items, unless otherwise noted, take damage as nonmagical items of the same sort. A damaged magic item continues to function, but if it is destroyed, all its magical power is lost. Magic items that take damage in excess of half their total hit points, but not more than their total hit points, gain the broken condition, and might not function properly.
				
\subsection{Repairing Magic Items}

				
Repairing a magic item requires material components equal to half the cost to create the item, and requires half the time. The \textit{make whole }spell can also repair a damaged (or even a destroyed) magic items---if the caster is high enough level.
				
\subsection{Charges, Doses, and Multiple Uses}

				
Many items, particularly wands and staves, are limited in power by the number of charges they hold. Normally, charged items have 50 charges at most (10 for staves). If such an item is found as a random part of a treasure, roll d\% and divide by 2 to determine the number of charges left (round down, minimum 1). If the item has a maximum number of charges other than 50, roll randomly to determine how many charges are left. 
				
Prices listed are always for fully charged items. (When an item is created, it is fully charged.) For an item that's worthless when its charges run out (which is the case for almost all charged items), the value of the partially used item is proportional to the number of charges left. For an item that has usefulness in addition to its charges, only part of the item's value is based on the number of charges left.
				
\section{Purchasing Magic Items}

\begin{table}[]
\sffamily
\caption{Table: Available Magic Items}
\setlength{\tabcolsep}{1pt}
\begin{tabular}{lllll}
\textbf{Community} & \textbf{Base}  \\
\textbf{Size}      & \textbf{Value} & \textbf{Minor} & \textbf{Medium} & \textbf{Major}\\
Thorp & 50 gp & 1d4 items & -- & -- \\
 Hamlet & 200 gp & 1d6 items & -- & -- \\
 Village & 500 gp & 2d4 items & 1d4 items & -- \\
 Small town & 1,000 gp & 3d4 items & 1d6 items & -- \\
 Large town & 2,000 gp & 3d4 items & 2d4 items & 1d4 items \\
 Small city & 4,000 gp & 4d4 items & 3d4 items & 1d6 items \\
 Large city & 8,000 gp & 4d4 items & 3d4 items & 2d4 items \\
 Metropolis & 16,000 gp & * & 4d4 items & 3d4 items\\
\end{tabular}\\
* in a metropolis, nearly all minor magic items are available.\\
\end{table}


\begin{table}[]
\sffamily
\caption{Table: Random Magic Item Generation}
\begin{tabular}{llll}
\textbf{Minor} & \textbf{Medium} & \textbf{Major} & \textbf{Item}\\
01--04 & 01--10 & 01--10 & Armor and shields \\
 05--09 & 11--20 & 11--20 & Weapons \\
 10--44 & 21--30 & 21--25 & Potions \\
 45--46 & 31--40 & 26--35 & Rings \\
 --- & 41--50 & 36--45 & Rods \\
 47--81 & 51--65 & 46--55 & Scrolls \\
 --- & 66--68 & 56--75 & Staves \\
 82--91 & 69--83 & 76--80 & Wands \\
 92--100 & 84--100 & 81--100 & Wondrous items\\
\end{tabular}
\end{table}

Magic items are valuable, and most major cities have at least one or two purveyors of magic items, from a simple potion merchant to a weapon smith that specializes in magic swords. Of course, not every item in this book is available in every town. 
				
The following guidelines are presented to help GMs determine what items are available in a given community. These guidelines assume a setting with an average level of magic. Some cities might deviate wildly from these baselines, subject to GM discretion. The GM should keep a list of what items are available from each merchant and should replenish the stocks on occasion to represent new acquisitions.
				
The number and types of magic items available in a community depend upon its size. Each community has a base value associated with it (see Table: Available Magic Items). There is a 75\% chance that any item of that value or lower can be found for sale with little effort in that community. In addition, the community has a number of other items for sale. These items are randomly determined and are broken down by category (minor, medium, or major). After determining the number of items available in each category, refer to Table: Random Magic Item Generation to determine the type of each item (potion, scroll, ring, weapon, etc.) before moving on to the individual charts to determine the exact item. Reroll any items that fall below the community's base value.
				
If you are running a campaign with low magic, reduce the base value and the number of items in each community by half. Campaigns with little or no magic might not have magic items for sale at all. GMs running these sorts of campaigns should make some adjustments to the challenges faced by the characters due to their lack of magic gear.
				
Campaigns with an abundance of magic items might have communities with twice the listed base value and random items available. Alternatively, all communities might count as one size category larger for the purposes of what items are available. In a campaign with very common magic, all magic items might be available for purchase in a metropolis.
				
Nonmagical items and gear are generally available in a community of any size unless the item is particularly expensive, such as full plate, or made of an unusual material, such as an adamantine longsword. These items should follow the base value guidelines to determine their availability, subject to GM discretion. 
				
\section{Magic Item Descriptions}

				
Each general type of magic item gets an overall description, followed by descriptions of specific items.
				
General descriptions include notes on activation, random generation, and other material. The AC, hardness, hit points, and break DC are given for typical examples of some magic items. The AC assumes that the item is unattended and includes a --5 penalty for the item's effective Dexterity of 0. If a creature holds the item, use the creature's Dexterity modifier in place of the --5 penalty.
				
Some individual items, notably those that just store spells, don't get full-blown descriptions. Reference the spell's description for details, modified by the form of the item (potion, scroll, wand, and so on). Assume that the spell is cast at the minimum level required to cast it.
				
Items with full descriptions have their powers detailed, and each of the following topics is covered in notational form as part of its entry.
				
\textbf{Aura}: Most of the time, a \textit{detect magic }spell reveals the school of magic associated with a magic item and the strength of the aura an item emits. This information (when applicable) is given at the beginning of the item's notational entry. See the \textit{detect magic }spell description for details.
				
\textbf{Caster Level (CL)}: The next item in a notational entry gives the caster level of the item, indicating its relative power. The caster level determines the item's saving throw bonus, as well as range or other level-dependent aspects of the powers of the item (if variable). It also determines the level that must be contended with should the item come under the effect of a \textit{dispel magic }spell or similar situation.
				
For potions, scrolls, and wands, the creator can set the caster level of an item at any number high enough to cast the stored spell but not higher than her own caster level. For other magic items, the caster level is determined by the item itself. 
				
\textbf{Slot}: Most magic items can only be utilized if worn or wielded in their proper slots. If the item is stowed or placed elsewhere, it does not function. If the slot lists \texttt{{}"{}}none,\texttt{{}"{}} the item must be held or otherwise carried to function.
				
\textbf{Price}: This is the cost, in gold pieces, to purchase the item, if it is available for sale. Generally speaking, magic items can be sold by PCs for half this value. 
				
\textbf{Weight}: This is the weight of an item. When a weight figure is not given, the item has no weight worth noting (for purposes of determining how much of a load a character can carry).
				
\textbf{Description}: This section of a magic item describes the item's powers and abilities. Potions, scrolls, staves, and wands refer to various spells as part of their descriptions (see Spell Lists for details on these spells).
				
\textbf{Construction}: With the exception of artifacts, most magic items can be built by a spellcaster with the appropriate feats and prerequisites. This section describes those prerequisites. 
				
\textbf{Requirements}: Certain requirements must be met in order for a character to create a magic item. These include feats, spells, and miscellaneous requirements such as level, alignment, and race or kind.
				
A spell prerequisite may be provided by a character who has prepared the spell (or who knows the spell, in the case of a sorcerer or bard), or through the use of a spell completion or spell trigger magic item or a spell-like ability that produces the desired spell effect. For each day that passes in the creation process, the creator must expend one spell completion item or one charge from a spell trigger item if either of those objects is used to supply a prerequisite.
				
It is possible for more than one character to cooperate in the creation of an item, with each participant providing one or more of the prerequisites. In some cases, cooperation may even be necessary.
				
If two or more characters cooperate to create an item, they must agree among themselves who will be considered the creator for the purpose of determinations where the creator's level must be known.
				
\textbf{Cost}: This is the cost in gold pieces to create the item. Generally this cost is equal to half the price of an item, but additional material components might increase this number. The cost to create includes the costs derived from the base cost plus the costs of the components.
			
\section{Armor}

\label{f0}
\begin{table}[]
\sffamily
\caption{Armor and Shields}
\begin{tabular}{lllll}
\textbf{Minor} & \textbf{Medium} & \textbf{Major} & \textbf{Item} & \textbf{Base Price}\\
01-60 & 01-05 & - & +1 shield & 1,000 gp\\
61-80 & 06-10 & - &   & 1,000 gp\\
81-85 & 11-20 & -  & +2 shield  & 4,000 gp\\
86-87 & 21-30 & -  & +2 armor  & 4,000 gp\\
- & 31-40 & 01-08  & +3 shield  & 9,000 gp\\
- & 41-50 & 09-16  & +3 armor  & 9,000 gp\\
- & 51-55 & 17-27  & +4 shield  & 16,000 gp\\
- & 56-57 & 28-38  & +4 armor  & 16,000 gp\\
- & - & 39-49  & +5 shield  & 25,000 gp\\
- & - & 50-57  & +5 armor  & 25,000 gp\\
- & - & -  & +6 armor/shield\(^{1}\) & 36,000 gp\\
- & - & -  & +7 armor/shield\(^{1}\) & 49,000 gp\\
- & - & -  & +8 armor/shield\(^{1}\) & 64,000 gp\\
- & - & -  & +9 armor/shield\(^{1}\) & 81,000 gp\\
- & - & -  & +10 armor/shield\(^{1}\) & 100,000 gp\\
88-89 & 58-60  & 58-60  & Specific armor\(^{2}\) & -\\
90-91 & 61-63 &  61-63  & Specific shield\(^{3}\) & -\\
92-100 & 64-100  & 64-100  & Special ability  and roll again\(^{2,3}\) & -\\
\end{tabular}\\
\textsuperscript{1} Armor and shields can't have enhancement bonuses higher than +5. Use these lines to determine price when special abilities are added in.\\
\textsuperscript{2} Roll on Table: Specific Armors. \\
\textsuperscript{3} Roll on Table: Specific Shields.\\
\end{table}


\begin{table}[]
\sffamily
\caption{Table: Armor Special Abilities}
\setlength{\tabcolsep}{1pt}
\begin{tabularx}{\linewidth}{lllXl}
               &                 &                &                          & \textbf{Base Price} \\
\textbf{Minor} & \textbf{Medium} & \textbf{Major} & \textbf{Special Ability} & \textbf{Modifier}\\
01-25 & 01-05 & 01-03 & Glamered & +2,700 gp\\
26-32 & 06-08 & 04 & Fortification, light & +1 bonus\(^{1}\)\\
33-52 & 09-11 & - & Slick & +3,750 gp\\
53-92 & 12-17 & - & Shadow & +3,750 gp\\
93-96 & 18-19 & - & Spell resistance & +2 bonus\(^{1}\)\\
97 & 20-29 & 05-07 & Slick, improved & +15,000 gp\\
98-99 & 30-49 & 08-13 & Shadow, improved & +15,000 gp\\
- & 50-74 & 14-28 & Energy resistance & +18,000 gp\\
- & 75-79 & 29-33 & Ghost touch & +3 bonus\(^{1}\)\\
- & 80-84 & 34-35 & Invulnerability & +3 bonus\(^{1}\)\\
- & 85-89 & 36-40 & Fortification, & +3 bonus\(^{1}\)\\
  &       &       & Moderate & \\
- & 90-94 & 41-42 & Spell resistance (15) & +3 bonus\(^{1}\)\\
- & 95-99 & 43 & Wild & +3 bonus\(^{1}\)\\
- & - & 44-48 & Slick, greater & +33,750 gp\\
- & - & 49-58 & Shadow, greater & +33,750 gp\\
- & - & 59-83 & Energy resistance & +42,000 gp \\
  &   &       & improved          & \\
- & - & 84-88 & Spell resistance, & +4 bonus\(^{1}\)\\
  &   &       & improved          & \\
- & - & 89 & Etherealness & +49,000 gp\\
- & - & 90 & Undead controlling & +49,000 gp\\
- & - & 91-92 & Fortification, heavy & +5 bonus\(^{1}\)\\
- & - & 93-94 & Spell resistance (19) & +5 bonus\(^{1}\)\\
- & - & 95-99 & Energy resistance, & +66,000 gp \\
  &   &       & greater & \\
100 & 100 & 100 & Roll twice again\(^{2}\) & -\\
\end{tabularx}\\
\textsuperscript{1} Add to enhancement bonus on Table: Armor and Shields to determine total market price\\
\textsuperscript{2} If you roll a special ability twice, only one counts. If you roll two versions of the same special ability, use the better.\\
\end{table}

\begin{table}[]
\sffamily
\caption{Table: Shield Special Abilities}
\setlength{\tabcolsep}{1pt}
\begin{tabular}{lllll}
               &                 &                &                          & \textbf{Base Price} \\
\textbf{Minor} & \textbf{Medium} & \textbf{Major} & \textbf{Special Ability} & \textbf{Modifier}\\
01-20 & 01-10 & 01-05 & Arrow catching & +1 bonus\(^{1}\)\\
21-40 & 11-20 & 06-08 & Bashing & +1 bonus\(^{1}\)\\
41-50 & 21-25 & 09-10 & Blinding & +1 bonus\(^{1}\)\\
51-75 & 26-40 & 11-15 & Fortification & +1 bonus\(^{1}\)\\
76-92 & 41-50 & 16-20 & Arrow deflection & +2 bonus\(^{1}\)\\
93-97 & 51-57 & 21-25 & Animated & +2 bonus\(^{1}\)\\
98-99 & 58-59 & - & Spell resistance (13) & +2 bonus\(^{1}\)\\
- & 60-79 & 26-41 & Energy resistance & +18,000 gp\\
- & 80-85 & 42-46 & Ghost touch & +3 bonus\(^{1}\)\\
- & 86-95 & 47-56 & Fortification & +3 bonus\(^{1}\)\\
- & 96-98 & 57-58 & Spell resistance (15) & +3 bonus\(^{1}\)\\
- & 99 & 59 & Wild & +3 bonus\(^{1}\)\\
- & - & 60-84 & Energy resistance & +42,000 gp \\
  &   &       & improved          & \\
- & - & 85-86 & Spell resistance (17) & +4 bonus\(^{1}\)\\
- & - & 87 & Undead controlling & +49,000 gp\\
- & - & 88-91 & Fortification & +5 bonus\(^{1}\)\\
- & - & 92-93 & Reflecting & +5 bonus\(^{1}\)\\
- & - & 94 & Spell resistance (19) & +5 bonus\(^{1}\)\\
- & - & 95-99 & Energy resistance & +66,000 gp \\
  &   &       & greater & \\
100 & 100 & 100 & Roll twice again & -\\
\end{tabular}\\
\textsuperscript{1} Add to enhancement bonus on Table: Armor and Shields to determine total market price\\
\textsuperscript{2} If you roll a special ability twice, only one counts. If you roll two versions of the same special ability, use the better.\\
\end{table}

\begin{table}[]
\sffamily
\caption{Table: Specific Armors}
\setlength{\tabcolsep}{1pt}
\begin{tabularx}{\linewidth}{lllXl}
\textbf{Minor} & \textbf{Medium} & \textbf{Major} & \textbf{Specific Armor} & \textbf{Market Price}\\
01-50 & 01-25 & - & Mithral shirt  & 1,100 gp\\
51-80 & 26-45 & - & Dragonhide plate  & 3,300 gp\\
81-100 & 46-57 & - & Elven chain  & 5,150 gp\\
- & 58-67 & - & Rhino hide  & 5,165 gp\\
- & 68-82 & 01-10 & Adamantine breastplate & 10,200 gp\\
- & 83-97 & 11-20 & Dwarven plate  & 16,500 gp\\
- & 98-100 & 21-32 & Banded mail of luck & 18,900 gp\\
- & - & 33-50 & Celestial armor  & 22,400 gp\\
- & - & 51-60 & Plate armor of the deep & 24,650 gp\\
- & - & 61-75 & Breastplate of command & 25,400 gp\\
- & - & 76-90 & Mithral full plate of speed & 26,500 gp\\
- & - & 91-100 & Demon armor  & 52,260 gp\\
\end{tabularx}
\end{table}
% <
\begin{table}[]
\sffamily
\setlength{\tabcolsep}{1pt}
\caption{Table: Specific Shields}
\begin{tabularx}{\linewidth}{lllXl}
\textbf{Minor} & \textbf{Medium} & \textbf{Major} & \textbf{Specific Shield} & \textbf{Market Price}\\
01-30 & 01-20 & - & Darkwood buckler & 203 gp\\
31-80 & 21-45 & - & Darkwood shield & 257 gp\\
81-95 & 46-70 & - & Mithral heavy shield & 1,020 gp\\
96-100 & 71-85 & 01-20 & Caster's shield & 3,153 gp\\
- & 86-90 & 21-40 & Spined shield & 5,580 gp\\
- & 91-95 & 41-60 & Lion's shield & 9,170 gp\\
- & 96-100 & 61-90 & Winged shield & 17,257 gp\\
- & - & 91-100 & Absorbing shield & 50,170 gp\\
\end{tabularx}
\end{table}

In general, magic armor protects the wearer to a greater extent than nonmagical armor. Magic armor bonuses are enhancement bonuses, never rise above +5, and stack with regular armor bonuses (and with shield and magic shield enhancement bonuses). All magic armor is also masterwork armor, reducing armor check penalties by 1.
				
In addition to an enhancement bonus, armor may have special abilities. Special abilities usually count as additional bonuses for determining the market value of an item, but do not improve AC. A suit of armor cannot have an effective bonus (enhancement plus special ability bonus equivalents, including those from character abilities and spells) higher than +10. A suit of armor with a special ability must also have at least a +1 enhancement bonus.
				
A suit of armor or a shield may be made of an unusual material. Roll d\%: 01--95 indicates that the item is of a standard sort, and 96--100 indicates that it is made of a special material (see Equipment).
				
Armor is always created so that if the type of armor comes with a pair of boots, a helm, or a set of gauntlets, these pieces can be switched for other magic boots, helms, or gauntlets.
				
\textbf{Caster Level for Armor and Shields}: The caster level of a magic shield or magic armor with a special ability is given in the item description. For an item with only an enhancement bonus, the caster level is three times the enhancement bonus. If an item has both an enhancement bonus and a special ability, the higher of the two caster level requirements must be met.
				
\textbf{Shields}: Shield enhancement bonuses stack with armor enhancement bonuses. Shield enhancement bonuses do not act as attack or damage bonuses when the shield is used in a shield bash. The \textit{bashing} special ability, however, does grant a +1 bonus on attack and damage rolls (see the special ability description).
				
A shield could be built that also acted as a magic weapon, but the cost of the enhancement bonus on attack rolls would need to be added into the cost of the shield and its enhancement bonus to AC.
				
As with armor, special abilities built into the shield add to the market value in the form of additions to the bonus of the shield, although they do not improve AC. A shield cannot have an effective bonus (enhancement plus special ability bonus equivalents) higher than +10. A shield with a special ability must also have at least a +1 enhancement bonus.
				
\textbf{Activation}: Usually a character benefits from magic armor and shields in exactly the way a character benefits from nonmagical armor and shields: by wearing them. If armor or a shield has a special ability that the user needs to activate, then the user usually needs to utter the command word (a standard action).
				
\textbf{Armor for Unusual Creatures}: The cost of armor for nonhumanoid creatures, as well as for creatures who are neither Small nor Medium, varies (see Equipment)\textit{. }The cost of the masterwork quality and any magical enhancement remains the same.
				
\subsection{Magic Armor and Shield Special Ability Descriptions}

				
Most magic armor and shields only have enhancement bonuses. Such items can also have one or more of the special abilities detailed below. Armor or a shield with a special ability must have at least a +1 enhancement bonus.
				
\textbf{Animated}: As a move action, an \textit{animated shield} can be loosed to defend its wielder on its own. For the following 4 rounds, the shield grants its bonus to the one who loosed it and then drops. While animated, the shield provides its shield bonus and the bonuses from all of the other shield special abilities it possesses, but it cannot take actions on its own, such as those provided by the \textit{bashing} and \textit{blinding} abilities. It can, however, use special abilities that do not require an action to function, such as \textit{arrow deflection} and \textit{reflecting}. While animated, a shield shares the same space as the activating character and accompanies the character who activated it, even if the character moves by magical means. A character with an \textit{animated shield} still takes any penalties associated with shield use, such as armor check penalty, arcane spell failure chance, and nonproficiency. If the wielder who loosed it has an unoccupied hand, she can grasp it to end its animation as a free action. Once a shield has been retrieved, it cannot be animated again for at least 4 rounds. This property cannot be added to a tower shield.
				
Strong transmutation\textit{; }CL 12th; Craft Magic Arms and Armor, \textit{animate objects; }Price +2 bonus.
				
\textbf{Arrow Catching}: A shield with this ability attracts ranged weapons to it. It has a deflection bonus of +1 against ranged weapons because projectiles and thrown weapons veer toward it. Additionally, any projectile or thrown weapon aimed at a target within 5 feet of the shield's wearer diverts from its original target and targets the shield's bearer instead. If the wielder has total cover relative to the attacker, the projectile or thrown weapon is not diverted. Additionally, those attacking the wearer with ranged weapons ignore any miss chances that would normally apply. Projectiles and thrown weapons that have an enhancement bonus higher than the shield's base AC bonus are not diverted to the wearer (but the shield's deflection bonus still applies against these weapons). The wielder can activate or deactivate this ability with a command word.
				
Moderate abjuration; CL 8th; Craft Magic Arms and Armor, \textit{entropic shield}; Price +1 bonus.
				
\textbf{Arrow Deflection}: This shield protects the wielder as if he had the Deflect Arrows feat. Once per round when he would normally be struck by a ranged weapon, he can make a DC 20 Reflex save. If the ranged weapon (or piece of ammunition) has an enhancement bonus, the DC increases by that amount. If he succeeds, the shield deflects the weapon. He must be aware of the attack and not flat-footed. Attempting to deflect a ranged weapon doesn't count as an action. Exceptional ranged weapons, such as boulders hurled by giants or \textit{acid arrows}, can't be deflected.
				
Faint abjuration\textit{; }CL 5th\textit{; }Craft Magic Arms and Armor, \textit{shield; }Price +2 bonus.
				
\textbf{Bashing}: A shield with this special ability is designed to perform a shield bash. A \textit{bashing shield} deals damage as if it were a weapon of two size categories larger (a Medium light shield thus deals 1d6 points of damage and a Medium heavy shield deals 1d8 points of damage). The shield acts as a +1 weapon when used to bash. Only light and heavy shields can have this ability.
				
Moderate transmutation\textit{; }CL 8th; Craft Magic Arms and Armor, \textit{bull's strength; }Price +1 bonus.
				
\textbf{Blinding}: A shield with this ability flashes with a brilliant light up to twice per day upon command of the wielder. Except for the wielder, anyone within 20 feet must make a DC 14 Reflex save or be blinded for 1d4 rounds.
				
Moderate evocation\textit{; }CL 7th; Craft Magic Arms and Armor, \textit{searing light; }Price +1 bonus.
				
\textbf{Energy Resistance}: A suit of armor or a shield with this property protects against one type of energy (acid, cold, electricity, fire, or sonic) and is designed with patterns depicting the element it protects against. The armor absorbs the first 10 points of energy damage per attack that the wearer would normally take (similar to the \textit{resist energy }spell).
				
Faint abjuration; CL 3rd; Craft Magic Arms and Armor, \textit{resist energy; }Price +18,000 gp.
				
\textbf{Energy Resistance, Improved}: As \textit{energy resistance}, except it absorbs the first 20 points of energy damage per attack.
				
Moderate abjuration; CL 7th; Craft Magic Arms and Armor, \textit{resist energy; }Price +42,000 gp.
				
\textbf{Energy Resistance, Greater}: As \textit{energy resistance}, except it absorbs the first 30 points of energy damage per attack.
				
Moderate abjuration; CL 11th; Craft Magic Arms and Armor, \textit{resist energy; }Price +66,000 gp.
				
\textbf{Etherealness}: On command, this ability allows the wearer of the armor to become ethereal (as the \textit{ethereal jaunt }spell) once per day. The character can remain ethereal for as long as desired, but once he returns to normal, he cannot become ethereal again that day.
				
Strong transmutation\textit{; }CL 13th; Craft Magic Arms and Armor, \textit{ethereal jaunt; }Price +49,000 gp.
				
\textbf{Fortification}: This suit of armor or shield produces a magical force that protects vital areas of the wearer more effectively. When a critical hit or sneak attack is scored on the wearer, there is a chance that the critical hit or sneak attack is negated and damage is instead rolled normally.

\begin{tabular}{lll}
 &                 \textbf{Chance for Normal} & \textbf{Base Price}\\
\textbf{Fortification Type} & \textbf{Damage} & \textbf{Modifier} \\
Light    &25\%     &+1 bonus \\
Moderate &50\%  &+3 bonus \\
Heavy    &75\%     &+5 bonus  \\
\end{tabular}



				
Strong abjuration\textit{; }CL 13th; Craft Magic Arms and Armor, \textit{limited wish }or \textit{miracle; }Price varies (see above).
				
\textbf{Ghost Touch}: This armor or shield seems almost translucent. Both its enhancement bonus and its armor bonus count against the attacks of corporeal and incorporeal creatures. It can be picked up, moved, and worn by corporeal and incorporeal creatures alike. Incorporeal creatures gain the armor's or shield's enhancement bonus against both corporeal and incorporeal attacks, and they can still pass freely through solid objects.
				
Strong transmutation\textit{; }CL 15th; Craft Magic Arms and Armor, \textit{etherealness; }Price +3 bonus.
				
\textbf{Glamered}: Upon command, a suit of \textit{glamered armor} changes shape and appearance to assume the form of a normal set of clothing. The armor retains all its properties (including weight) when it is so disguised. Only a \textit{true seeing }spell or similar magic reveals the true nature of the armor when disguised.
				
Moderate illusion\textit{; }CL 10th; Craft Magic Arms and Armor, \textit{disguise self; }Price +2,700 gp.
				
\textbf{Invulnerability}: This suit of armor grants the wearer damage reduction 5/magic.
				
Strong abjuration and evocation (if \textit{miracle }is used)\textit{; }CL 18th; Craft Magic Arms and Armor, \textit{stoneskin, wish }or \textit{miracle; }Price +3 bonus.
				
\textbf{Reflecting}: This shield seems like a highly polished mirror. Its surface is completely reflective. Once per day, it can be called on to reflect a spell back at its caster exactly like the \textit{spell turning }spell\textit{.}
				
Strong abjuration\textit{; }CL 14th; Craft Magic Arms and Armor, \textit{spell turning; }Price +5 bonus.
				
\textbf{Shadow}: This armor blurs the wearer whenever she tries to hide, while also dampening the sound around her, granting a +5 competence bonus on Stealth checks. The armor's armor check penalty still applies normally.
				
Faint illusion\textit{; }CL 5th; Craft Magic Arms and Armor, \textit{invisibility, silence; }Price +3,750 gp.
				
\textbf{Shadow, Improved}: As \textit{shadow}, except it grants a +10 competence bonus on Stealth checks.
				
Moderate illusion; CL 10th; Craft Magic Arms and Armor, \textit{invisibility, silence; }Price +15,000 gp.
				
\textbf{Shadow, Greater}: As \textit{shadow}, except it grants a +15 competence bonus on Stealth checks.
				
Strong illusion; CL 15th; Craft Magic Arms and Armor, \textit{invisibility, silence; }Price +33,750 gp.
				
\textbf{Slick}: \textit{Slick} armor seems coated at all times with a slightly greasy oil. It provides a +5 competence bonus on its wearer's Escape Artist checks. The armor's armor check penalty still applies normally.
				
Faint conjuration\textit{; }CL 4th; Craft Magic Arms and Armor, \textit{grease; }Price +3,750 gp.
				
\textbf{Slick, Improved}: As \textit{slick}, except it grants a +10 competence bonus on Escape Artist checks.
				
Moderate conjuration; CL 10th; Craft Magic Arms and Armor, \textit{grease; }Price +15,000 gp.
				
\textbf{Slick, Greater}: As \textit{slick}, except it grants a +15 competence bonus on Escape Artist checks.
				
Strong conjuration; CL 15th; Craft Magic Arms and Armor, \textit{grease; }Price +33,750 gp.
				
\textbf{Spell Resistance}: This property grants the armor's wearer spell resistance while the armor is worn. The spell resistance can be 13, 15, 17, or 19, depending on the armor.
				
Strong abjuration\textit{; }CL 15th; Craft Magic Arms and Armor, \textit{spell resistance; }Price +2 bonus (SR 13), +3 bonus (SR 15), +4 bonus (SR 17), or +5 bonus (SR 19).
				
\textbf{Undead Controlling}: \textit{Undead controlling armor} or \textit{shields} often have skeletal or other grisly decorations or flourishes to their decor. They let the user control up to 26 HD of undead per day, as the \textit{control undead }spell. At dawn each day, the wearer loses control of any undead still under his sway. Armor or a shield with this ability appears to be made of bone; this feature is entirely decorative and has no other effect on the armor.
				
Strong necromancy; CL 13th; Craft Magic Arms and Armor, \textit{control undead}; Price +49,000 gp.
				
\textbf{Wild}: The wearer of a suit of armor or a shield with this ability preserves his armor bonus (and any enhancement bonus) while in a wild shape. Armor and shields with this ability usually appear to be covered in leaf patterns. While the wearer is in a wild shape, the armor cannot be seen.
				
Moderate transmutation; CL 9th; Craft Magic Arms and Armor, \textit{baleful polymorph}; Price +3 bonus.
				
\subsection{Specific Armors}

				
\textbf{Adamantine Breastplate}
				
\textbf{Aura} no aura (nonmagical); \textbf{CL} ---
				
\textbf{Slot} armor; \textbf{Price} 10,200 gp; \textbf{Weight} 30 lbs.
				
Description
				
This nonmagical breastplate is made of adamantine, giving its wearer damage reduction of 2/---. 
				
\textbf{Banded Mail of Luck}
				
\textbf{Aura} strong enchantment; \textbf{CL} 12th
				
\textbf{Slot} armor; \textbf{Price} 18,900 gp; \textbf{Weight} 35 lbs.
				
Description
				
Ten 100-gp gems adorn this \textit{+3 banded mail}. Once per week, the armor allows its wearer to require that an attack roll made against him be rerolled. He must take whatever consequences come from the second roll. The wearer's player must decide whether to have the attack roll rerolled before damage is rolled. 
				
Construction
				
\textbf{Requirements} Craft Magic Arms and Armor, \textit{bless}; \textbf{Cost }9,650 gp
				
\textbf{Breastplate of Command}
				
\textbf{Aura} strong enchantment; \textbf{CL} 15th
				
\textbf{Slot} armor; \textbf{Price} 25,400 gp; \textbf{Weight} 30 lbs.
				
Description
				
This \textit{+2 breastplate} bestows a commanding aura upon its wearer. The wearer gains a +2 competence bonus on all Charisma checks, including Charisma-based skill checks. The wearer also gains a +2 competence bonus to his Leadership score. Friendly troops within 360 feet of the user become braver than normal, gaining a +2 resistance bonus on saving throws against fear. Since the effect arises in great part from the distinctiveness of the armor, it does not function if the wearer hides or conceals herself in any way. 
				
Construction
				
\textbf{Requirements} Craft Magic Arms and Armor, \textit{mass charm monster}; \textbf{Cost }12,875 gp
				
\textbf{Celestial Armor}
				
\textbf{Aura} faint transmutation \mbox{$[$}good\mbox{$]$}; \textbf{CL} 5th
				
\textbf{Slot} armor; \textbf{Price} 22,400 gp; \textbf{Weight }20 lbs.
				
Description
				
This bright silver or gold \textit{+3 chainmail} is so fine and light that it can be worn under normal clothing without betraying its presence. It has a maximum Dexterity bonus of +8, an armor check penalty of --2, and an arcane spell failure chance of 15\%. It is considered light armor and allows the wearer to use \textit{fly} on command (as the spell) once per day. 
				
Construction
				
\textbf{Requirements} Craft Magic Arms and Armor, \textit{fly}, creator must be good; \textbf{Cost }11,350 gp
				
\textbf{Demon Armor}
				
\textbf{Aura} strong necromancy \mbox{$[$}evil\mbox{$]$}; \textbf{CL} 13th
				
\textbf{Slot} armor; \textbf{Price} 52,260 gp; \textbf{Weight }50 lbs.
				
Description
				
This plate armor is fashioned to make the wearer appear to be a demon. The helmet is shaped to look like a horned demon head, and its wearer looks out of the open, tooth-filled mouth. This \textit{+4 full plate} allows the wearer to make claw attacks that deal 1d10 points of damage, strike as +1 weapons, and afflict the target as if she had been struck by a \textit{contagion} spell (Fortitude DC 14 negates). Use of \textit{contagion} requires a normal melee attack with the claws. The \texttt{{}"{}}claws\texttt{{}"{}} are built into the armor's vambraces and gauntlets, and cannot be disarmed. 
				
A suit of \textit{demon armor} is infused with evil, and as a result it bestows one negative level on any nonevil creature wearing it. This negative level persists as long as the armor is worn and disappears when the armor is removed. The negative level cannot be overcome in any way (including \textit{restoration} spells) while the armor is worn. 
				
Construction
				
\textbf{Requirements} Craft Magic Arms and Armor, \textit{contagion}; \textbf{Cost }26,955 gp
				
\textbf{Dwarven Plate}
				
\textbf{Aura} no aura (nonmagical); \textbf{CL} ---
				
\textbf{Slot} armor; \textbf{Price} 16,500 gp; \textbf{Weight }50 lbs.
				
Description
				
This full plate is made of adamantine, giving its wearer damage reduction of 3/---. 
				
\textbf{Dragonhide Plate}
				
\textbf{Aura} no aura (nonmagical); \textbf{CL} ---
				
\textbf{Slot} armor; \textbf{Price} 3,300 gp; \textbf{Weight} 50 lbs.
				
DESCRIPTION
				
This suit of full plate is made of dragonhide, rather than metal, so druids can wear it. It is otherwise identical to masterwork full plate. 
				
\textbf{Elven Chain}
				
\textbf{Aura} no aura (nonmagical); \textbf{CL} ---
				
\textbf{Slot} armor; \textbf{Price} 5,150 gp; \textbf{Weight} 20 lbs.
				
DESCRIPTION
				
This extremely light chainmail is made of very fine mithral links. This armor is treated, in all ways, like light armor, including when determining proficiency. The armor has an arcane spell failure chance of 20\%, a maximum Dexterity bonus of +4, and an armor check penalty of --2. 
				
\textbf{Mithral Full Plate of Speed}
				
\textbf{Aura} faint transmutation; \textbf{CL} 5th
				
\textbf{Slot} armor; \textbf{Price} 26,500 gp; \textbf{Weight }25 lbs.
				
Description
				
As a free action, the wearer of this fine set of \textit{+1 mithral full plate} can activate it, enabling him to act as though affected by a \textit{haste} spell for up to 10 rounds each day. The duration of the \textit{haste} effect need not be consecutive rounds.
				
The armor has an arcane spell failure chance of 25\%, a maximum Dexterity bonus of +3, and an armor check penalty of --3. It is considered medium armor, except that you must be proficient in heavy armor to avoid taking nonproficiency penalties. 
				
Construction
				
\textbf{Requirements} Craft Magic Arms and Armor, \textit{haste}; \textbf{Cost }18,500 gp
				
\textbf{Mithral Shirt}
				
\textbf{Aura} no aura (nonmagical); \textbf{CL} ---
				
\textbf{Slot} armor; \textbf{Price} 1,100 gp; \textbf{Weight} 10 lbs.
				
DESCRIPTION
				
This extremely light chain shirt is made of very fine mithral links. The armor has an arcane spell failure chance of 10\%, a maximum Dexterity bonus of +6, and no armor check penalty. It is considered light armor. 
				
\textbf{Plate Armor of the Deep}
				
\textbf{Aura} moderate abjuration; \textbf{CL} 11th
				
\textbf{Slot} armor; \textbf{Price} 24,650 gp; \textbf{Weight }50 lbs.
				
Description
				
This \textit{+1 full plate} is decorated with a wave and fish motif. Although the armor remains as heavy and bulky as normal full plate, the wearer of \textit{plate armor of the deep} is treated as unarmored for purposes of Swim checks. The wearer can breathe underwater and can converse with any water-breathing creature with a language. 
				
Construction
				
\textbf{Requirements} Craft Magic Arms and Armor, \textit{freedom of movement}, \textit{tongues}, \textit{water breathing}; \textbf{Cost }13,150 gp
				
\textbf{Rhino Hide}
				
\textbf{Aura} moderate transmutation; \textbf{CL} 9th
				
\textbf{Slot} armor; \textbf{Price} 5,165 gp; \textbf{Weight }25 lbs.
				
Description
				
This \textit{+2 hide armor} is made from rhinoceros hide. In addition to granting a +2 enhancement bonus to AC, it has a --1 armor check penalty and deals an additional 2d6 points of damage on any successful charge attack made by the wearer, including a mounted charge. 
				
Construction
				
\textbf{Requirements} Craft Magic Arms and Armor, \textit{bull's strength}; \textbf{Cost }2,665 gp
				
\subsection{Specific Shields}

				
\textbf{Absorbing Shield}
				
\textbf{Aura} strong transmutation; \textbf{CL} 17th
				
\textbf{Slot} shield; \textbf{Price} 50,170 gp; \textbf{Weight }15 lbs.
				
Description
				
This \textit{+1 heavy steel shield} is made of metal, but its color is flat black that seems to absorb light. Once every 2 days, on command, it can \textit{disintegrate} an object that it touches, as the spell but requiring a melee touch attack. This effect only functions as an attack---it can't be activated to target a creature or weapon as it strikes the shield.
				
Construction
				
\textbf{Requirements} Craft Magic Arms and Armor, \textit{disintegrate}; \textbf{Cost }25,170 gp
				
\textbf{Caster's Shield}
				
\textbf{Aura} moderate abjuration; \textbf{CL} 6th
				
\textbf{Slot} shield; \textbf{Price} 3,153 gp (plus the value of the scroll spell if one is currently scribed); \textbf{Weight }5 lbs.
				
Description
				
This \textit{+1 light wooden shield} has a leather strip on the back on which a spellcaster can scribe a single spell as on a scroll. A spell so scribed requires half the normal cost in raw materials. The strip cannot accommodate spells of higher than 3rd level. The strip is reusable.
				
A random \textit{caster's shield} has a 50\% chance of having a single medium scroll spell on it. The spell is divine (01--80 on d\%) or arcane (81--100). A \textit{caster's shield} has a 5\% arcane spell failure chance. 
				
Construction
				
\textbf{Requirements} Craft Magic Arms and Armor, Scribe Scroll, creator must be at least 6th level; \textbf{Cost }1,653 gp
				
\textbf{Darkwood Buckler}
				
\textbf{Aura} no aura (nonmagical); \textbf{CL} ---
				
\textbf{Slot} shield; \textbf{Price} 203 gp; \textbf{Weight }2.5 lbs.
				
Description
				
This nonmagical light wooden shield is made out of darkwood. It has no enhancement bonus, but its construction material makes it lighter than a normal wooden shield. It has no armor check penalty. 
				
\textbf{Darkwood Shield}
				
\textbf{Aura} no aura (nonmagical); \textbf{CL} ---
				
\textbf{Slot} shield; \textbf{Price} 257 gp; \textbf{Weight} 5 lbs.
				
DESCRIPTION
				
This nonmagical heavy wooden shield is made out of darkwood. It has no enhancement bonus, but its construction material makes it lighter than a normal wooden shield. It has no armor check penalty. 
				
\textbf{Lion's Shield}
				
\textbf{Aura} moderate conjuration; \textbf{CL }10th
				
\textbf{Slot} shield; \textbf{Price} 9,170 gp; \textbf{Weight} 15 lbs.
				
Description
				
This \textit{+2 heavy steel shield }is fashioned to appear to be a roaring lion's head. Three times per day as a free action, the lion's head can be commanded to attack (independently of the shield wearer), biting with the wielder's base attack bonus (including multiple attacks, if the wielder has them) and dealing 2d6 points of damage. This attack is in addition to any actions performed by the wielder.
				
Construction
				
\textbf{Requirements} Craft Magic Arms and Armor, \textit{summon nature's ally IV}; \textbf{Cost} 4,670 gp
				
\textbf{Mithral Heavy Shield}
				
\textbf{Aura} no aura (nonmagical); \textbf{CL} ---
				
\textbf{Slot} shield; \textbf{Price} 1,020 gp; \textbf{Weight} 5 lbs.
				
DESCRIPTION
				
This heavy shield is made of mithral and thus is much lighter than a standard steel shield. It has a 5\% arcane spell failure chance and no armor check penalty.
				
\textbf{Spined Shield}
				
\textbf{Aura} moderate evocation; \textbf{CL} 6th
				
\textbf{Slot} shield; \textbf{Price} 5,580 gp; \textbf{Weight }15 lbs.
				
Description
				
This \textit{+1 heavy steel shield} is covered in spines. It acts as a normal spiked shield. On command up to three times per day, the shield's wearer can fire one of the shield's spines. A fired spine has a +1 enhancement bonus, a range increment of 120 feet, and deals 1d10 points of damage (19--20/x2). Fired spines regenerate each day. 
				
Construction
				
\textbf{Requirements} Craft Magic Arms and Armor, \textit{magic missile}; \textbf{Cost }2,875 gp
				
\textbf{Winged Shield}
				
\textbf{Aura} faint transmutation; \textbf{CL} 5th
				
\textbf{Slot} shield; \textbf{Price} 17,257 gp; \textbf{Weight }10 lbs.
				
Description
				
This heavy wooden shield has a +3 enhancement bonus. Arching bird wings are carved into the face of the shield. Once per day, it can be commanded to \textit{fly} (as the spell), carrying the wielder. The shield can carry up to 133 pounds and move at 60 feet per round, or up to 266 pounds and move at 40 feet per round. 
				
Construction
				
\textbf{Requirements} Craft Magic Arms and Armor, \textit{fly}; \textbf{Cost }8,707 gp
        	

\section{Weapons}

\label{f0}		

\begin{table}[]
\sffamily
\caption{Table: Weapons}
\setlength{\tabcolsep}{1pt}
\begin{tabularx}{\linewidth}{lllXl}
               &                 &                &                 & \textbf{Bonus} \\
               &                 &                &                 & \textbf{Base} \\
\textbf{Minor} & \textbf{Medium} & \textbf{Major} & \textbf{Weapon} & \textbf{Price\(^{1}\)}\\
01--70 & 01--10 & --- & +1 & 2,000 gp\\
71--85 & 11--29 & --- & +2 & 8,000 gp\\
--- & 30--58 & 01--20 & +3 & 18,000 gp\\
--- & 59--62 & 21--38 & +4 & 32,000 gp\\
--- & --- & 39--49 & +5 & 50,000 gp\\
--- & --- & --- & +6\(^{2}\) & 72,000 gp\\
--- & --- & --- & +7\(^{2}\) & 98,000 gp\\
--- & --- & --- & +8\(^{2}\) & 128,000 gp\\
--- & --- & --- & +9\(^{2}\) & 162,000 gp\\
--- & --- & --- & +10\(^{2}\) & 200,000 gp\\
86--90 & 63--68 & 50--63 & Specific weapon\(^{3}\) & ---\\
91--100 & 69--100 & 64--100 & Special ability and roll again\(^{4}\) & ---\\
\end{tabularx}
\textsuperscript{1} For ammunition, this price is for 50 arrows, bolts, or bullets. \\
\textsuperscript{2} A weapon can't have an enhancement bonus higher than +5. Use these lines to determine price when special abilities are added in. \\
\textsuperscript{3} See Table: Specific Weapons. \\
\textsuperscript{4} See Table: Melee Weapon Special Abilities for melee weapons and Table: Ranged Weapon Special Abilities for ranged weapons.\\
\end{table}	

\begin{table}[]
\sffamily
\caption{Table: Melee Weapon Special Abilities}
\setlength{\tabcolsep}{1pt}
\begin{tabularx}{\linewidth}{lllXl}
               &                 &                &                  & \textbf{Base} \\
               &                 &                & \textbf{Special} & \textbf{Price} \\
\textbf{Minor} & \textbf{Medium} & \textbf{Major} & \textbf{Ability} & \textbf{Modifier\(^{1}\)}\\
01--10 & 01--06 & 01--03 & Bane & +1 bonus\\
11--17 & 07--12 & --- & Defending & +1 bonus\\
18--27 & 13--19 & 04--06 & Flaming & +1 bonus\\
28--37 & 20--26 & 07--09 & Frost & +1 bonus\\
38--47 & 27--33 & 10--12 & Shock & +1 bonus\\
48--56 & 34--38 & 13--15 & Ghost touch & +1 bonus\\
57--67 & 39--44 & --- & Keen, 2 & +1 bonus\\
68--71 & 45--48 & 16--19 & Ki Focus & +1 bonus\\
72--75 & 49--50 & --- & Merciful & +1 bonus\\
76--82 & 51--54 & 20--21 & Mighty cleaving & +1 bonus\\
83--87 & 55--59 & 22--24 & Spell storing & +1 bonus\\
88--91 & 60--63 & 25--28 & Throwing & +1 bonus\\
92--95 & 64--65 & 29--32 & Thundering & +1 bonus\\
96--99 & 66--69 & 33--36 & Vicious & +1 bonus\\
--- & 70--72 & 37--41 & Anarchic & +2 bonus\\
--- & 73--75 & 42--46 & Axiomatic & +2 bonus\\
--- & 76--78 & 47--49 & Disruption, 3 & +2 bonus\\
--- & 79--81 & 50--54 & Flaming burst & +2 bonus\\
--- & 82--84 & 55--59 & Icy burst & +2 bonus\\
--- & 85--87 & 60--64 & Holy & +2 bonus\\
--- & 88--90 & 65--69 & Shocking burst & +2 bonus\\
--- & 91--93 & 70--74 & Unholy & +2 bonus\\
--- & 94--95 & 75--78 & Wounding & +2 bonus\\
--- & --- & 79--83 & Speed & +3 bonus\\
--- & --- & 84--86 & Brilliant energy & +4 bonus\\
--- & --- & 87--88 & Dancing & +4 bonus\\
--- & --- & 89--90 & Vorpal, 2 & +5 bonus\\
100 & 96--100 & 91--100 & Roll again twice & ---\\
\end{tabularx}\\
\textsuperscript{1} Add to enhancement bonus on Table: Weapons to determine total market price. \\
\textsuperscript{2} Piercing or slashing weapons only (slashing only for vorpal). Reroll if randomly generated for a bludgeoning weapon. \\
\textsuperscript{3} Bludgeoning weapons only. Reroll if randomly generated for a piercing or slashing weapon.\\
\textsuperscript{4} Reroll if you get a duplicate special ability, an ability incompatible with an ability that you've already rolled, or if the extra ability puts you over the +10 limit. A weapon's enhancement bonus and special ability bonus equivalents can't total more than +10.\\
\end{table}
		
\begin{table}[]
\sffamily
\caption{Table: Ranged Weapon Special Abilities}
\setlength{\tabcolsep}{1pt}
\begin{tabular}{lllll}
               &                 &                &                  & \textbf{Base} \\
               &                 &                & \textbf{Special} & \textbf{Price} \\
\textbf{Minor} & \textbf{Medium} & \textbf{Major} & \textbf{Ability} & \textbf{Modifier\(^{1}\)}\\
01--12 & 01--08 & 01--04 & Bane,\(^{2}\) & +1 bonus\\
13--25 & 09--16 & 05--08 & Distance & +1 bonus\\
26--40 & 17--28 & 09--12 & Flaming,\(^{2}\) & +1 bonus\\
41--55 & 29--40 & 13--16 & Frost,\(^{2}\) & +1 bonus\\
56--60 & 41--42 & --- & Merciful,\(^{2}\) & +1 bonus\\
61--68 & 43--47 & 17--21 & Returning & +1 bonus\\
69--83 & 48--59 & 22--25 & Shock,\(^{2}\) & +1 bonus\\
84--93 & 60--64 & 26--27 & Seeking & +1 bonus\\
94--99 & 65--68 & 28--29 & Thundering,\(^{2}\) & +1 bonus\\
--- & 69--71 & 30--34 & Anarchic,\(^{2}\) & +2 bonus\\
--- & 72--74 & 35--39 & Axiomatic,\(^{2}\) & +2 bonus\\
--- & 75--79 & 40--49 & Flaming burst,\(^{2}\) & +2 bonus\\
--- & 80--82 & 50--54 & Holy,\(^{2}\) & +2 bonus\\
--- & 83--87 & 55--64 & Icy burst,\(^{2}\) & +2 bonus\\
--- & 88--92 & 65--74 & Shocking burst,\(^{2}\) & +2 bonus\\
--- & 93--95 & 75--79 & Unholy,\(^{2}\) & +2 bonus\\
--- & --- & 80--84 & Speed & +3 bonus\\
--- & --- & 85--90 & Brilliant energy & +4 bonus\\
100 & 96--100 & 91--100 & Roll again twice & ---\\
\end{tabular}
\textsuperscript{1} Add to enhancement bonus on Table: Weapons to determine total market price.\\
\textsuperscript{2} Bows, crossbows, and slings crafted with this ability bestow this power upon their ammunition.\\
\textsuperscript{3} Reroll if you get a duplicate special ability, an ability incompatible with an ability that you've already rolled, or if the extra ability puts you over the +10 limit. A weapon's enhancement bonus and special ability bonus equivalents can't total more than +10.\\
\end{table}
\begin{table}[]
\sffamily
\caption{Table: Specific Weapons}
\setlength{\tabcolsep}{1pt}
\begin{tabularx}{\linewidth}{lllXl}
               &                 &                & \textbf{Specific} & \textbf{Market} \\
\textbf{Minor} & \textbf{Medium} & \textbf{Major} & \textbf{Weapon} & \textbf{Price}\\
01--15 & --- & --- & Sleep arrow & 132 gp\\
16--25 & --- & --- & Screaming bolt & 267 gp\\
26--45 & --- & --- & Silver dagger, masterwork & 322 gp\\
46--65 & --- & --- & Cold iron longsword, masterwork & 330 gp\\
66--75 & 01--09 & --- & Javelin of lightning & 1,500 gp\\
76--80 & 10--15 & --- & Slaying arrow & 2,282 gp\\
81--90 & 16--24 & --- & Adamantine dagger & 3,002 gp\\
91--100 & 25--33 & --- & Adamantine battleaxe & 3,010 gp\\
--- & 34--37 & --- & Slaying arrow & 4,057 gp\\
--- & 38--40 & --- & Shatterspike & 4,315 gp\\
--- & 41--46 & --- & Dagger of venom & 8,302 gp\\
--- & 47--51 & --- & Trident of warning & 10,115 gp\\
--- & 52--57 & 01--04 & Assassin's dagger & 10,302 gp\\
--- & 58--62 & 05--07 & Shifter's sorrow & 12,780 gp\\
--- & 63--66 & 08--09 & Trident of fish command & 18,650 gp\\
--- & 67--74 & 10--13 & Flame tongue & 20,715 gp\\
--- & 75--79 & 14--17 & Luck blade & 22,060 gp\\
--- & 80--86 & 18--24 & Sword of subtlety & 22,310 gp\\
--- & 87--91 & 25--31 & Sword of the planes & 22,315 gp\\
--- & 92--95 & 32--37 & Nine lives stealer & 23,057 gp\\
--- & 96--98 & 38--42 & Oathbow & 25,600 gp\\
--- & 99--100 & 43--46 & Sword of life stealing & 25,715 gp\\
--- & --- & 47--51 & Mace of terror & 38,552 gp\\
--- & --- & 52--57 & Life-drinker & 40,320 gp\\
--- & --- & 58--62 & Sylvan scimitar & 47,315 gp\\
--- & --- & 63--67 & Rapier of puncturing & 50,320 gp\\
--- & --- & 68--73 & Sun blade & 50,335 gp\\
--- & --- & 74--79 & Frost brand & 54,475 gp\\
--- & --- & 80--84 & Dwarven thrower & 60,312 gp\\
--- & --- & 85--91 & Luck blade, wish & 62,360 gp\\
--- & --- & 92--95 & Mace of smiting & 75,312 gp\\
--- & --- & 96--97 & Luck blade & 102,660 gp\\
--- & --- & 98--99 & Holy avenger & 120,630 gp\\
--- & --- & 100 & Luck blade & 142,960 gp\\
\end{tabularx}
\end{table}
								
A magic weapon is enhanced to strike more truly and deliver more damage. Magic weapons have enhancement bonuses ranging from +1 to +5. They apply these bonuses to both attack and damage rolls when used in combat. All magic weapons are also masterwork weapons, but their masterwork bonuses on attack rolls do not stack with their enhancement bonuses on attack rolls.
				
Weapons come in two basic categories: melee and ranged. Some of the weapons listed as melee weapons can also be used as ranged weapons. In this case, their enhancement bonuses apply to both melee and ranged attacks.
				
Some magic weapons have special abilities. Special abilities count as additional bonuses for determining the market value of the item, but do not modify attack or damage bonuses (except where specifically noted). A single weapon cannot have a modified bonus (enhancement bonus plus special ability bonus equivalents, including those from character abilities and spells) higher than +10. A weapon with a special ability must also have at least a +1 enhancement bonus. Weapons cannot possess the same special ability more than once.
				
Weapons or ammunition can be made of an unusual material. Roll d\%: 01--95 indicates that the item is of a standard sort, and 96--100 indicates that it is made of a special material (see Equipment).
				
\textbf{Caster Level for Weapons}: The caster level of a weapon with a special ability is given in the item description. For an item with only an enhancement bonus and no other abilities, the caster level is three times the enhancement bonus. If an item has both an enhancement bonus and a special ability, the higher of the two caster level requirements must be met.
				
\textbf{Additional Damage Dice}: Some magic weapons deal additional dice of damage. Unlike other modifiers to damage, additional dice of damage are not multiplied when the attacker scores a critical hit.
				
\textbf{Ranged Weapons and Ammunition}: The enhancement bonus from a ranged weapon does not stack with the enhancement bonus from ammunition. Only the higher of the two enhancement bonuses applies.
				
Ammunition fired from a projectile weapon with an enhancement bonus of +1 or higher is treated as a magic weapon for the purpose of overcoming damage reduction. Similarly, ammunition fired from a projectile weapon with an alignment gains the alignment of that projectile weapon.
				
\textbf{Magic Ammunition and Breakage}: When a magic arrow, crossbow bolt, or sling bullet misses its target, there is a 50\% chance it breaks or is otherwise rendered useless. A magic arrow, bolt, or bullet that successfully hits a target is automatically destroyed after it delivers its damage.
				
\textbf{Light Generation}: Fully 30\% of magic weapons shed light equivalent to a \textit{light }spell. These glowing weapons are quite obviously magical. Such a weapon can't be concealed when drawn, nor can its light be shut off. Some of the specific weapons detailed below always or never glow, as defined in their descriptions.
				
\textbf{Hardness and Hit Points}: Each +1 of a magic weapon's enhancement bonus adds +2 to its hardness and +10 to its hit points.
				
\textbf{Activation}: Usually a character benefits from a magic weapon in the same way a character benefits from a mundane weapon---by wielding (attacking with) it. If a weapon has a special ability that the user needs to activate, then the user usually needs to utter a command word (a standard action). A character can activate the special abilities of 50 pieces of ammunition at the same time, assuming each piece has identical abilities.
				
\textbf{Magic Weapons and Critical Hits}: Some weapon special abilities and some specific weapons have an extra effect on a critical hit. This special effect also functions against creatures not normally subject to critical hits. On a successful critical roll, apply the special effect, but do not multiply the weapon's regular damage. 
				
\textbf{Weapons for Unusually Sized Creatures}: The cost of weapons for creatures who are neither Small nor Medium varies (see Equipment). The cost of the masterwork quality and any magical enhancement remains the same.
				
\textbf{Special Qualities}: Roll d\%. A 01--30 result indicates that the item sheds light, 31--45 indicates that something (a design, inscription, or the like) provides a clue to the weapon's function, and 46--100 indicates no special qualities. 
				
\subsection{Magic Weapon Special Ability Descriptions}

				
A weapon with a special ability must also have at least a +1 enhancement bonus.
				
\textbf{Anarchic}: An \textit{anarchic weapon} is infused with the power of chaos. It makes the weapon chaotically aligned and thus bypasses the corresponding damage reduction. It deals an extra 2d6 points of damage against all creatures of lawful alignment. It bestows one permanent negative level on any lawful creature attempting to wield it. The negative level remains as long as the weapon is in hand and disappears when the weapon is no longer wielded. This negative level cannot be overcome in any way (including \textit{restoration }spells) while the weapon is wielded.
				
Moderate evocation \mbox{$[$}chaotic\mbox{$]$}\textit{; }CL 7th; Craft Magic Arms and Armor, \textit{chaos hammer, }creator must be chaotic; Price +2 bonus.
				
\textbf{Axiomatic}: An \textit{axiomatic weapon} is infused with lawful power. It makes the weapon law-aligned and thus bypasses the corresponding damage reduction. It deals an extra 2d6 points of damage against chaotic creatures. It bestows one permanent negative level on any chaotic creature attempting to wield it. The negative level remains as long as the weapon is in hand and disappears when the weapon is no longer wielded. This negative level cannot be overcome in any way (including \textit{restoration }spells) while the weapon is wielded.
				
Moderate evocation \mbox{$[$}lawful\mbox{$]$}\textit{; }CL 7th; Craft Magic Arms and Armor, \textit{order's wrath, }creator must be lawful; Price +2 bonus.
				
\textbf{Bane}: A \textit{bane weapon} excels against certain foes. Against a designated foe, the weapon's enhancement bonus is +2 better than its actual bonus. It also deals an extra 2d6 points of damage against the foe. To randomly determine a weapon's designated foe, roll on the following table.

\begin{tabular}{ll}
\textbf{d\%} & \textbf{Designated Foe}      \\
01--05        & Aberrations                  \\
06--09        & Animals                      \\
10--16        & Constructs                   \\
17--22        & Dragons                      \\
23--27        & Fey                          \\
28--60        & Humanoids (pick one subtype) \\
61--65        & Magical beasts               \\
66--70        & Monstrous humanoids          \\
71--72        & Oozes                        \\
73--88        & Outsiders (pick one subtype) \\
89--90        & Plants                       \\
91--98        & Undead                       \\
99--100       & Vermin                      \\
\end{tabular}
				
Moderate conjuration; CL 8th; Craft Magic Arms and Armor, \textit{summon monster I}; Price +1 bonus.
				
\textbf{Brilliant Energy}: A \textit{brilliant energy weapon} has its significant portion transformed into light, although this does not modify the item's weight. It always gives off light as a torch (20-foot radius). A \textit{brilliant energy weapon} ignores nonliving matter. Armor and shield bonuses to AC (including any enhancement bonuses to that armor) do not count against it because the weapon passes through armor. (Dexterity, deflection, dodge, natural armor, and other such bonuses still apply.) A \textit{brilliant energy weapon} cannot harm undead, constructs, and objects. This property can only be applied to melee weapons, thrown weapons, and ammunition.
				
Strong transmutation\textit{; }CL 16th; Craft Magic Arms and Armor, \textit{gaseous form, continual flame; }Price +4 bonus.
				
\textbf{Dancing}: As a standard action, a \textit{dancing weapon }can be loosed to attack on its own. It fights for 4 rounds using the base attack bonus of the one who loosed it and then drops. While dancing, it cannot make attacks of opportunity, and the person who activated it is not considered armed with the weapon. The weapon is considered wielded or attended by the creature for all maneuvers and effects that target items. While dancing, the weapon shares the same space as the activating character and can attack adjacent foes (weapons with reach can attack opponents up to 10 feet away). The dancing weapon accompanies the person who activated it everywhere, whether she moves by physical or magical means. If the wielder who loosed it has an unoccupied hand, she can grasp it while it is attacking on its own as a free action; when so retrieved, the weapon can't dance (attack on its own) again for 4 rounds.
				
Strong transmutation\textit{; }CL 15th; Craft Magic Arms and Armor, \textit{animate objects; }Price +4 bonus.
				
\textbf{Defending}: A \textit{defending weapon} allows the wielder to transfer some or all of the weapon's enhancement bonus to his AC as a bonus that stacks with all others. As a free action, the wielder chooses how to allocate the weapon's enhancement bonus at the start of his turn before using the weapon, and the bonus to AC lasts until his next turn.
				
Moderate abjuration\textit{; }CL 8th; Craft Magic Arms and Armor, \textit{shield }or \textit{shield of faith; }Price +1 bonus.
				
\textbf{Disruption}: A \textit{disruption weapon} is the bane of all undead. Any undead creature struck in combat must succeed on a DC 14 Will save or be destroyed. A \textit{disruption weapon} must be a bludgeoning melee weapon.
				
Strong conjuration; CL 14th; Craft Magic Arms and Armor, \textit{heal; }Price +2 bonus.
				
\textbf{Distance}: This special ability can only be placed on a ranged weapon. A \textit{distance weapon} has double the range increment of other weapons of its kind.
				
Moderate divination; CL 6th; Craft Magic Arms and Armor, \textit{clairaudience/clairvoyance; }Price +1 bonus.
				
\textbf{Flaming}: Upon command, a \textit{flaming weapon} is sheathed in fire that deals an extra 1d6 points of fire damage on a successful hit. The fire does not harm the wielder. The effect remains until another command is given. 
				
Moderate evocation\textit{; }CL 10th; Craft Magic Arms and Armor and \textit{flame blade, flame strike, }or \textit{fireball; }Price +1 bonus.
				
\textbf{Flaming Burst}: A \textit{flaming burst weapon} functions as a \textit{flaming weapon} that also explodes with flame upon striking a successful critical hit. The fire does not harm the wielder. In addition to the extra fire damage from the \textit{flaming} ability (see above), a \textit{flaming burst weapon} deals an extra 1d10 points of fire damage on a successful critical hit. If the weapon's critical multiplier is \mbox{$\times$}3, add an extra 2d10 points of fire damage instead, and if the multiplier is \mbox{$\times$}4, add an extra 3d10 points of fire damage. 
				
Even if the \textit{flaming} ability is not active, the weapon still deals its extra fire damage on a successful critical hit.
				
Strong evocation\textit{; }CL 12th; Craft Magic Arms and Armor and \textit{flame blade, flame strike, }or \textit{fireball; }Price +2 bonus.
				
\textbf{Frost}: Upon command, a \textit{frost} \textit{weapon} is sheathed in icy cold that deals an extra 1d6 points of cold damage on a successful hit. The cold does not harm the wielder. The effect remains until another command is given.
				
Moderate evocation\textit{; }CL 8th; Craft Magic Arms and Armor, \textit{chill metal }or \textit{ice storm; }Price +1 bonus.
				
\textbf{Ghost Touch}: A \textit{ghost touch weapon} deals damage normally against incorporeal creatures, regardless of its bonus. An incorporeal creature's 50\% reduction in damage from corporeal sources does not apply to attacks made against it with \textit{ghost touch weapons}. The weapon can be picked up and moved by an incorporeal creature at any time. A manifesting ghost can wield the weapon against corporeal foes. Essentially, a \textit{ghost touch weapon} counts as both corporeal or incorporeal.
				
Moderate conjuration; CL 9th; Craft Magic Arms and Armor, \textit{plane shift; }Price +1 bonus.
				
\textbf{Holy}: A \textit{holy weapon} is imbued with holy power. This power makes the weapon good-aligned and thus bypasses the corresponding damage reduction. It deals an extra 2d6 points of damage against all creatures of evil alignment. It bestows one permanent negative level on any evil creature attempting to wield it. The negative level remains as long as the weapon is in hand and disappears when the weapon is no longer wielded. This negative level cannot be overcome in any way (including by \textit{restoration }spells) while the weapon is wielded.
				
Moderate evocation \mbox{$[$}good\mbox{$]$}; CL 7th; Craft Magic Arms and Armor, \textit{holy smite, }creator must be good; Price +2 bonus.
				
\textbf{Icy Burst}: An \textit{icy burst weapon} functions as a \textit{frost weapon} that also explodes with frost upon striking a successful critical hit. The frost does not harm the wielder. In addition to the extra damage from the \textit{frost} ability, an \textit{icy burst weapon} deals an extra 1d10 points of cold damage on a successful critical hit. If the weapon's critical multiplier is \mbox{$\times$}3, add an extra 2d10 points of cold damage instead, and if the multiplier is \mbox{$\times$}4, add an extra 3d10 points. 
				
Even if the \textit{frost} ability is not active, the weapon still deals its extra cold damage on a successful critical hit.
				
Moderate evocation\textit{; }CL 10th; Craft Magic Arms and Armor, \textit{chill metal }or \textit{ice storm; }Price +2 bonus.
				
\textbf{Keen}: This ability doubles the threat range of a weapon. Only piercing or slashing melee weapons can be \textit{keen}. If you roll this property randomly for an inappropriate weapon, reroll. This benefit doesn't stack with any other effect that expands the threat range of a weapon (such as the \textit{keen edge }spell or the Improved Critical feat).
				
Moderate transmutation\textit{; }CL 10th; Craft Magic Arms and Armor, \textit{keen edge; }Price +1 bonus.
				
\textbf{Ki Focus}: The magic weapon serves as a channel for the wielder's ki, allowing her to use her special \textit{ki }attacks through the weapon as if they were unarmed attacks. These attacks include the monk's \textit{ki }strike, quivering palm, and the Stunning Fist feat (including any condition that the monk can apply using this feat). Only melee weapons can have the \textit{ki focus} ability.
				
Moderate transmutation; CL 8th; Craft Magic Arms and Armor, creator must be a monk; Price +1 bonus.
				
\textbf{Merciful}: The weapon deals an extra 1d6 points of damage, and all damage it deals is nonlethal damage. On command, the weapon suppresses this ability until told to resume it (allowing it to deal lethal damage, but without any bonus damage from this ability).
				
Faint conjuration; CL 5th; Craft Magic Arms and Armor, \textit{cure light wounds}; Price +1 bonus.
				
\textbf{Mighty Cleaving}: A \textit{mighty cleaving weapon} allows a wielder using the Cleave feat to make one additional attack if the first attack hits, as long as the next foe is adjacent to the first and also within reach. This additional attack cannot be against the first foe. Only melee weapons can be \textit{mighty cleaving weapons}.
				
Moderate evocation\textit{; }CL 8th; Craft Magic Arms and Armor, \textit{divine power; }Price +1 bonus.
				
\textbf{Returning}: This special ability can only be placed on a weapon that can be thrown. A \textit{returning weapon} flies through the air back to the creature that threw it. It returns to the thrower just before the creature's next turn (and is therefore ready to use again in that turn). Catching a \textit{returning weapon} when it comes back is a free action. If the character can't catch it, or if the character has moved since throwing it, the weapon drops to the ground in the square from which it was thrown.
				
Moderate transmutation\textit{; }CL 7th; Craft Magic Arms and Armor, \textit{telekinesis; }Price +1 bonus.
				
\textbf{Seeking}: Only ranged weapons can have the \textit{seeking }ability. The weapon veers toward its target, negating any miss chances that would otherwise apply, such as from concealment. The wielder still has to aim the weapon at the right square. Arrows mistakenly shot into an empty space, for example, do not veer and hit invisible enemies, even if they are nearby.
				
Strong divination; CL 12th; Craft Magic Arms and Armor, \textit{true seeing}; Price +1 bonus.
				
\textbf{Shock}: Upon command, a \textit{shock weapon} is sheathed in crackling electricity that deals an extra 1d6 points of electricity damage on a successful hit. The electricity does not harm the wielder. The effect remains until another command is given. 
				
Moderate evocation\textit{; }CL 8th; Craft Magic Arms and Armor, \textit{call lightning }or \textit{lightning bolt; }Price +1 bonus.
				
\textbf{Shocking Burst}: A \textit{shocking burst weapon} functions as a \textit{shock weapon} that explodes with electricity upon striking a successful critical hit. The electricity does not harm the wielder. In addition to the extra electricity damage from the \textit{shock} ability, a \textit{shocking burst weapon} deals an extra 1d10 points of electricity damage on a successful critical hit. If the weapon's critical multiplier is \mbox{$\times$}3, add an extra 2d10 points of electricity damage instead, and if the multiplier is \mbox{$\times$}4, add an extra 3d10 points.
				
Even if the \textit{shock} ability is not active, the weapon still deals its extra electricity damage on a successful critical hit.
				
Moderate evocation\textit{; }CL 10th; Craft Magic Arms and Armor, \textit{call lightning }or \textit{lightning bolt; }Price +2 bonus.
				
\textbf{Speed}: When making a full-attack action, the wielder of a \textit{speed weapon} may make one extra attack with it. The attack uses the wielder's full base attack bonus, plus any modifiers appropriate to the situation. (This benefit is not cumulative with similar effects, such as a \textit{haste }spell.)
				
Moderate transmutation\textit{; }CL 7th; Craft Magic Arms and Armor, \textit{haste; }Price +3 bonus.
				
\textbf{Spell Storing}: A \textit{spell storing weapon} allows a spellcaster to store a single targeted spell of up to 3rd level in the weapon. (The spell must have a casting time of 1 standard action.) Anytime the weapon strikes a creature and the creature takes damage from it, the weapon can immediately cast the spell on that creature as a free action if the wielder desires. (This special ability is an exception to the general rule that casting a spell from an item takes at least as long as casting that spell normally.) Once the spell has been cast from the weapon, a spellcaster can cast any other targeted spell of up to 3rd level into it. The weapon magically imparts to the wielder the name of the spell currently stored within it. A randomly rolled \textit{spell storing weapon} has a 50\% chance to have a spell stored in it already.
				
Strong evocation (plus aura of stored spell)\textit{; }CL 12th; Craft Magic Arms and Armor, creator must be a caster of at least 12th level; Price +1 bonus.
				
\textbf{Throwing}: This ability can only be placed on a melee weapon. A melee weapon crafted with this ability gains a range increment of 10 feet and can be thrown by a wielder proficient in its normal use.
				
Faint transmutation\textit{; }CL 5th; Craft Magic Arms and Armor, \textit{magic stone; }Price +1 bonus.
				
\textbf{Thundering}: A \textit{thundering weapon} creates a cacophonous roar like thunder upon striking a successful critical hit. The sonic energy does not harm the wielder. A \textit{thundering weapon} deals an extra 1d8 points of sonic damage on a successful critical hit. If the weapon's critical multiplier is \mbox{$\times$}3, add an extra 2d8 points of sonic damage instead, and if the multiplier is \mbox{$\times$}4, add an extra 3d8 points of sonic damage. Subjects dealt critical hits by a \textit{thundering weapon} must make a DC 14 Fortitude save or be deafened permanently.
				
Faint necromancy\textit{; }CL 5th; Craft Magic Arms and Armor, \textit{blindness/deafness; }Price +1 bonus.
				
\textbf{Unholy}: An \textit{unholy weapon} is imbued with unholy power. This power makes the weapon evil-aligned and thus bypasses the corresponding damage reduction. It deals an extra 2d6 points of damage against all creatures of good alignment. It bestows one permanent negative level on any good creature attempting to wield it. The negative level remains as long as the weapon is in hand and disappears when the weapon is no longer wielded. This negative level cannot be overcome in any way (including \textit{restoration }spells) while the weapon is wielded.
				
Moderate evocation \mbox{$[$}evil\mbox{$]$}\textit{; }CL 7th; Craft Magic Arms and Armor, \textit{unholy blight, }creator must be evil; Price +2 bonus.
				
\textbf{Vicious}: When a \textit{vicious weapon} strikes an opponent, it creates a flash of disruptive energy that resonates between the opponent and the wielder. This energy deals an extra 2d6 points of damage to the opponent and 1d6 points of damage to the wielder. Only melee weapons can be \textit{vicious}.
				
Moderate necromancy; CL 9th; Craft Magic Arms and Armor, \textit{enervation}; Price +1 bonus.
				
\textbf{Vorpal}: This potent and feared ability allows the weapon to sever the heads of those it strikes. Upon a roll of natural 20 (followed by a successful roll to confirm the critical hit), the weapon severs the opponent's head (if it has one) from its body. Some creatures, such as many aberrations and all oozes, have no heads. Others, such as golems and undead creatures other than vampires, are not affected by the loss of their heads. Most other creatures, however, die when their heads are cut off. A \textit{vorpal weapon} must be a slashing melee weapon. If you roll this property randomly for an inappropriate weapon, reroll.
				
Strong necromancy and transmutation\textit{; }CL 18th; Craft Magic Arms and Armor\textit{, circle of death}, \textit{keen edge; }Price +5 bonus.
				
\textbf{Wounding}: A \textit{wounding weapon} deals 1 point of bleed damage when it hits a creature. Multiple hits from a wounding weapon increase the bleed damage. Bleeding creatures take the bleed damage at the start of their turns. Bleeding can be stopped by a DC 15 Heal check or through the application of any spell that cures hit point damage. A critical hit does not multiply the bleed damage. Creatures immune to critical hits are immune to the bleed damage dealt by this weapon.
				
Moderate evocation; CL 10th; Craft Magic Arms and Armor, \textit{bleed; }Price +2 bonus.
				
\subsection{Specific Weapons}

				
\textbf{Adamantine Battleaxe}
				
\textbf{Aura} no aura (nonmagical);\textbf{ CL }---
				
\textbf{Slot} none; \textbf{Price} 3,010 gp; \textbf{Weight} 6 lbs.
				
DESCRIPTION
				
This nonmagical axe is made out of adamantine. As a masterwork weapon, it has a +1 enhancement bonus on attack rolls. 
				
\textbf{Adamantine Dagger}
				
\textbf{Aura} no aura (nonmagical); \textbf{CL} ---
				
\textbf{Slot} none; \textbf{Price} 3,002 gp; \textbf{Weight} 1 lb.
				
DESCRIPTION
				
This nonmagical dagger is made out of adamantine. As a masterwork weapon, it has a +1 enhancement bonus on attack rolls. 
				
\textbf{Assassin's Dagger}
				
\textbf{Aura} moderate necromancy;\textbf{ CL }9th
				
\textbf{Slot} none; \textbf{Price} 10,302 gp; \textbf{Weight} 1 lb.
				
DESCRIPTION
				
This wicked-looking, curved \textit{+2 dagger} provides a +1 bonus to the DC of a Fortitude save forced by the death attack of an assassin. 
				
Construction
				
\textbf{Requirements} Craft Magic Arms and Armor, \textit{slay living}; \textbf{Cost }5,302 gp
				
\textbf{Dagger of Venom}
				
\textbf{Aura} faint necromancy; \textbf{CL} 5th
				
\textbf{Slot} none; \textbf{Price} 8,302 gp; \textbf{Weight} 1 lb.
				
DESCRIPTION
				
This black \textit{+1 dagger} has a serrated edge. It allows the wielder to use a \textit{poison} effect (as the spell, save DC 14) upon a creature struck by the blade once per day. The wielder can decide to use the power after he has struck. Doing so is a free action, but the \textit{poison} effect must be invoked in the same round that the dagger strikes. 
				
Construction
				
\textbf{Requirements} Craft Magic Arms and Armor, \textit{poison}; \textbf{Cost }4,302 gp
				
\textbf{Dwarven Thrower}
				
\textbf{Aura} moderate evocation; \textbf{CL} 10th
				
\textbf{Slot} none; \textbf{Price} 60,312 gp; \textbf{Weight} 5 lbs.
				
DESCRIPTION
				
This weapon functions as a \textit{+2 warhammer} in the hands of most users. Yet in the hands of a dwarf, the warhammer gains an additional +1 enhancement bonus (for a total enhancement bonus of +3) and gains the \textit{returning} special ability. It can be hurled with a 30-foot range increment. When hurled, a \textit{dwarven thrower} deals an extra 2d8 points of damage against creatures of the giant subtype or an extra 1d8 points of damage against any other target. 
				
Construction
				
\textbf{Requirements} Craft Magic Arms and Armor, creator must be a dwarf of at least 10th level; \textbf{Cost }30,312 gp
				
\textbf{Flame Tongue}
				
\textbf{Aura} strong evocation; \textbf{CL} 12th
				
\textbf{Slot} none; \textbf{Price} 20,715 gp; \textbf{Weight} 4 lbs.
				
DESCRIPTION
				
This is a \textit{+1 flaming burst longsword}. Once per day, the sword can blast forth a fiery ray at any target within 30 feet as a ranged touch attack. The ray deals 4d6 points of fire damage on a successful hit. 
				
Construction
				
\textbf{Requirements} Craft Magic Arms and Armor, \textit{scorching ray }and \textit{fireball}, \textit{flame blade}, or \textit{flame strike}; \textbf{Cost }10,515 gp
				
\textbf{Frost Brand}
				
\textbf{Aura} strong evocation; \textbf{CL} 14th
				
\textbf{Slot} none; \textbf{Price} 54,475 gp; \textbf{Weight} 8 lbs.
				
DESCRIPTION
				
This \textit{+3 frost greatsword} sheds light as a torch when the temperature drops below 0\^A\mbox{${}^\circ$} F. At such times it cannot be concealed when drawn, nor can its light be shut off. Its wielder is protected from fire; the sword absorbs the first 10 points of fire damage each round that the wielder would otherwise take.
				
A \textit{frost brand} extinguishes all nonmagical fires in a 20-foot radius. As a standard action, it can also dispel lasting fire spells, but not instantaneous effects. You must succeed on a dispel check (1d20 +14) against each spell to dispel it. The DC to dispel such spells is 11 + the caster level of the fire spell. 
				
Construction
				
\textbf{Requirements} Craft Magic Arms and Armor, \textit{ice storm}, \textit{dispel magic}, \textit{protection from energy}; \textbf{Cost }27,375 gp and 5 sp
				
\textbf{Holy Avenger}
				
\textbf{Aura} strong abjuration; \textbf{CL} 18th
				
\textbf{Slot} none; \textbf{Price} 120,630 gp; \textbf{Weight} 4 lbs.
				
DESCRIPTION
				
This \textit{+2 cold iron longsword} becomes a \textit{+5 holy cold iron longsword} in the hands of a paladin.
				
This sacred weapon provides spell resistance of 5 + the paladin's level to the wielder and anyone adjacent to her. It also enables the paladin to use \textit{greater dispel magic} (once per round as a standard action) at the class level of the paladin. Only the area dispel is possible, not the targeted dispel or counterspell versions of \textit{greater dispel magic}.
				
Construction
				
\textbf{Requirements} Craft Magic Arms and Armor, \textit{holy aura}, creator must be good; \textbf{Cost }60,630 gp
				
\textbf{Javelin of Lightning}
				
\textbf{Aura} faint evocation; \textbf{CL} 5th
				
\textbf{Slot} none; \textbf{Price} 1,500 gp; \textbf{Weight} 2 lbs.
				
DESCRIPTION
				
This javelin becomes a 5d6 \textit{lightning bolt} when thrown (Reflex DC 14 half). It is consumed in the attack. 
				
Construction
				
\textbf{Requirements} Craft Magic Arms and Armor, \textit{lightning bolt}; \textbf{Cost }750 gp
				
\textbf{Life-Drinker}
				
\textbf{Aura} strong necromancy;\textbf{ CL }13th
				
\textbf{Slot} none; \textbf{Price} 40,320 gp; \textbf{Weight} 12 lbs.
				
DESCRIPTION
				
This \textit{+1 greataxe} is favored by undead and constructs, who do not suffer its drawback. A life-drinker bestows two negative levels on its target whenever it deals damage, just as if its target had been struck by an undead creature. One day after being struck, subjects must make a DC 16 Fortitude save for each negative level or the negative levels become permanent.
				
Each time a \textit{life-drinker} deals damage to a foe, it also bestows one negative level on the wielder. Any negative levels gained by the wielder in this fashion lasts for 1 hour. 
				
Construction
				
\textbf{Requirements} Craft Magic Arms and Armor, \textit{enervation}; \textbf{Cost }20,320 gp
				
\textbf{Longsword, Cold Iron Masterwork }
				
\textbf{Aura} no aura (nonmagical); \textbf{CL} ---
				
\textbf{Slot} none; \textbf{Price} 330 gp; \textbf{Weight} 4 lbs.
				
DESCRIPTION
				
This nonmagical longsword is crafted out of cold iron. As a masterwork weapon, it has a +1 enhancement bonus on attack rolls. 
				
\textbf{Luck Blade}
				
\textbf{Aura} strong evocation; \textbf{CL} 17th
				
\textbf{Slot} none; \textbf{Price} 22,060 gp (0 \textit{wishes}), 62,360 gp (1 \textit{wish}), 102,660 gp (2 \textit{wishes}), 142,960 gp (3 \textit{wishes}); \textbf{Weight} 2 lbs.
				
DESCRIPTION
				
This \textit{+2 short sword} gives its possessor a +1 luck bonus on all saving throws. Its possessor also gains the power of good fortune, usable once per day. This extraordinary ability allows its possessor to reroll one roll that she just made, before the results are revealed. She must take the result of the reroll, even if it's worse than the original roll. In addition, a luck blade may contain up to three \textit{wishes} (when randomly rolled, a luck blade holds 1d4--1 \textit{wishes}, minimum 0). When the last \textit{wish} is used, the sword remains a \textit{+2 short sword}, still grants the +1 luck bonus, and still grants its reroll power. 
				
Construction
				
\textbf{Requirements} Craft Magic Arms and Armor, \textit{wish} or \textit{miracle}; \textbf{Cost }11,185 gp (0 wishes), 43,835 gp (1 wish), 76,485 gp (2 wishes), 109,135 gp (3 wishes).
				
\textbf{Mace of Smiting}
				
\textbf{Aura} moderate transmutation;\textbf{ CL }11th
				
\textbf{Slot} none; \textbf{Price} 75,312 gp; \textbf{Weight} 8 lbs.
				
DESCRIPTION
				
This \textit{+3 adamantine heavy mace} has a +5 enhancement bonus against constructs, and a successful critical hit dealt to a construct completely destroys the construct (no saving throw). A critical hit dealt to an outsider deals \mbox{$\times$}4 damage rather than \mbox{$\times$}2. 
				
Construction
				
\textbf{Requirements} Craft Magic Arms and Armor, \textit{disintegrate}; \textbf{Cost }39,312 gp
				
\textbf{Mace of Terror}
				
\textbf{Aura} strong necromancy;\textbf{ CL }13th
				
\textbf{Slot} none; \textbf{Price} 38,552 gp; \textbf{Weight} 8 lbs.
				
DESCRIPTION
				
This weapon usually appears to be a particularly frightening-looking iron or steel mace. On command, this \textit{+2 heavy mace} causes the wielder's clothes and appearance to transform into an illusion of darkest horror such that living creatures in a 30-foot cone become panicked as if by a \textit{fear} spell (Will DC 16 partial). Those who fail take a --2 morale penalty on saving throws, and they flee from the wielder. The wielder may use this ability up to three times per day. 
				
Construction
				
\textbf{Requirements} Craft Magic Arms and Armor, \textit{fear}; \textbf{Cost }19,432 gp
				
\textbf{Nine Lives Stealer}
				
\textbf{Aura} strong necromancy \mbox{$[$}evil\mbox{$]$}; \textbf{CL} 13th
				
\textbf{Slot} none; \textbf{Price} 23,057 gp; \textbf{Weight} 4 lbs.
				
DESCRIPTION
				
This longsword always performs as a \textit{+2 longsword}, but it also has the power to draw the life force from an opponent. It can do this nine times before the ability is lost. At that point, the sword becomes a simple \textit{+2 longsword} (with a faint evil aura). A critical hit must be dealt for the sword's death-dealing ability to function, and this weapon has no effect on creatures not subject to critical hits. The victim is entitled to a DC 20 Fortitude save to avoid death. If the save is successful, the sword's death-dealing ability does not function, no use of the ability is expended, and normal critical damage is determined. This sword is evil, and any good character attempting to wield it gains two negative levels. These negative levels remain as long as the sword is in hand and disappear when the sword is no longer wielded. These negative levels never result in actual level loss, but they cannot be overcome in any way (including \textit{restoration} spells) while the sword is wielded. 
				
Construction
				
\textbf{Requirements} Craft Magic Arms and Armor, \textit{finger of death}; \textbf{Cost }11,528 gp 5 sp.
				
\textbf{Oathbow}
				
\textbf{Aura} strong evocation;\textbf{ CL }15th
				
\textbf{Slot} none; \textbf{Price} 25,600 gp; \textbf{Weight} 3 lbs.
				
DESCRIPTION
				
Of elven make, this white \textit{+2 composite longbow} (+2 Str bonus) whispers, \texttt{{}"{}}Swift defeat to my enemies\texttt{{}"{}} in Elven when nocked and pulled. Once per day, if the archer swears aloud to slay her target (a free action), the bow's whisper becomes the shout \texttt{{}"{}}Death to those who have wronged me!\texttt{{}"{}} Against such a sworn enemy, the bow has a +5 enhancement bonus, and arrows launched from it deal an additional 2d6 points of damage (and \mbox{$\times$}4 on a critical hit instead of the normal \mbox{$\times$}3). After an enemy has been sworn, the bow is treated as only a masterwork weapon against all foes other than the sworn enemy, and the archer takes a --1 penalty on attack rolls with any weapon other than the \textit{oathbow}. These bonuses and penalties last for 7 days or until the sworn enemy is slain or destroyed by the wielder of the \textit{oathbow}, whichever comes first.
				
The \textit{oathbow} may only have one sworn enemy at a time. Once the wielder swears to slay a target, he cannot make a new oath until he has slain that target or 7 days have passed. Even if the wielder slays the sworn enemy on the same day that he makes the oath, he cannot activate the \textit{oathbow's} special power again until 24 hours have passed from the time he made the oath. 
				
Construction
				
\textbf{Requirements} Craft Magic Arms and Armor, creator must be an elf; \textbf{Cost }13,100 gp
				
\textbf{Rapier of Puncturing}
				
\textbf{Aura} strong necromancy;\textbf{ CL }13th
				
\textbf{Slot} none; \textbf{Price} 50,320 gp; \textbf{Weight} 2 lbs.
				
DESCRIPTION
				
Three times per day, this \textit{+2 wounding rapier} allows the wielder to make a touch attack with the weapon that deals 1d6 points of Constitution damage by draining blood. Creatures immune to critical hits are immune to the Constitution damage dealt by this weapon. 
				
Construction
				
\textbf{Requirements} Craft Magic Arms and Armor, \textit{harm}; \textbf{Cost }25,320 gp
				
\textbf{Screaming Bolt}
				
\textbf{Aura} faint enchantment;\textbf{ CL }5th
				
\textbf{Slot} none; \textbf{Price} 267 gp; \textbf{Weight} 1/10 lb.
				
DESCRIPTION
				
These \textit{+2 bolts} scream when fired, forcing all enemies of the wielder within 20 feet of the path of the bolt to succeed on a DC 14 Will save or become shaken. This is a mind-affecting fear effect. 
				
Construction
				
\textbf{Requirements} Craft Magic Arms and Armor, \textit{doom}; \textbf{Cost }137 gp
				
\textbf{Shatterspike}
				
\textbf{Aura} strong evocation;\textbf{ CL }13th
				
\textbf{Slot} none; \textbf{Price} 4,315 gp; \textbf{Weight} 4 lbs.
				
DESCRIPTION
				
This intimidating weapon appears to be a longsword with multiple hooks, barbs, and serrations along the blade, excellent for catching and sundering a foe's weapon. Wielders without the Improved Sunder feat use a \textit{shatterspike} as a \textit{+1 longsword} only. Wielders with the Improved Sunder feat instead use \textit{shatterspike} as a \textit{+4 longsword} when attempting to sunder an opponent's weapon. \textit{Shatterspike} can damage weapons with an enhancement bonus of +4 or lower.
				
Construction
				
\textbf{Requirements} Str 13, Craft Magic Arms and Armor, Improved Sunder, Power Attack, \textit{shatter}; \textbf{Cost }2,315 gp
				
\textbf{Shifter's Sorrow}
				
\textbf{Aura} strong transmutation;\textbf{ CL }15th
				
\textbf{Slot} none; \textbf{Price} 12,780 gp; \textbf{Weight} 10 lbs.
				
DESCRIPTION
				
This \textit{+1/+1 two-bladed sword} has blades of alchemical silver. The weapon deals an extra 2d6 points of damage against any creature with the shapechanger subtype. When a shapechanger or a creature in an alternate form (such as a druid using wild shape) is struck by the weapon, it must make a DC 15 Will save or return to its natural form. 
				
Construction
				
\textbf{Requirements} Craft Magic Arms and Armor, \textit{baleful polymorph}; \textbf{Cost }6,780 gp
				
\textbf{Silver Dagger, Masterwork}
				
\textbf{Aura} no aura (nonmagical); \textbf{CL} ---
				
\textbf{Slot} none; \textbf{Price} 322 gp; \textbf{Weight} 1 lb.
				
DESCRIPTION
				
As a masterwork weapon, this alchemical silver dagger has a +1 enhancement bonus on attack rolls (but not to damage rolls).
				
\textbf{Slaying Arrow}
				
\textbf{Aura} strong necromancy;\textbf{ CL }13th
				
\textbf{Slot} none; \textbf{Price} 2,282 gp (\textit{slaying arrow}) or 4,057 gp (\textit{greater slaying arrow}); \textbf{Weight} 1/10 lb.
				
DESCRIPTION
				
This \textit{+1 arrow} is keyed to a particular type or subtype of creature. If it strikes such a creature, the target must make a DC 20 Fortitude save or take 50 points of damage. Note that even creatures normally exempt from Fortitude saves (undead and constructs) are subject to this attack. When keyed to a living creature, this is a death effect (and thus \textit{death ward} protects a target). To determine the type or subtype of creature the arrow is keyed to, roll on the table below.
				
A \textit{greater slaying arrow} functions just like a normal \textit{slaying arrow}, but the DC to avoid the death effect is 23 and the arrow deals 100 points of damage if the saving throw is failed. 
				
Construction
				
\textbf{Requirements} Craft Magic Arms and Armor, \textit{finger of death} (\textit{slaying arrow}) or heightened\textit{ finger of death} (\textit{greater slaying arrow}); \textbf{Cost }1,144 gp 5 sp (\textit{slaying arrow}) or 2,032 gp (\textit{greater slaying arrow})
% <thead href="../spells/fingerOfDeath.html#finger-of-death">
\begin{tabular}{ll}
\textbf{d\%} & \textbf{Designated Type or Subtype} \\
01--05        & Aberrations                         \\
06--09        & Animals                             \\
10--16        & Constructs                          \\
17--27        & Dragons                             \\
28--32        & Fey                                 \\
33           & Humanoids, aquatic                  \\
34--35        & Humanoids, dwarf                    \\
36--37        & Humanoids, elf                      \\
38--44        & Humanoids, giant                    \\
45           & Humanoids, gnoll                    \\
46           & Humanoids, gnome                    \\
47--49        & Humanoids, goblinoid                \\
50           & Humanoids, halfling                 \\
51--54        & Humanoids, human                    \\
55--57        & Humanoids, reptilian                \\
58--60        & Humanoids, orc                      \\
61--65        & Magical beasts                      \\
66--70        & Monstrous humanoids                 \\
71--72        & Oozes                               \\
73           & Outsiders, air                      \\
74--76        & Outsiders, chaotic                  \\
77           & Outsiders, earth                    \\
78--80        & Outsiders, evil                     \\
81           & Outsiders, fire                     \\
82--84        & Outsiders, good                     \\
85--87        & Outsiders, lawful                   \\
88           & Outsiders, water                    \\
89--90        & Plants                              \\
91--98        & Undead                              \\
99--100       & Vermin                             
\end{tabular}

\textbf{Sleep Arrow}
				
\textbf{Aura} faint enchantment;\textbf{ CL }5th
				
\textbf{Slot} none; \textbf{Price} 132 gp; \textbf{Weight} 1/10 lb.
				
DESCRIPTION
				
This \textit{+1 arrow} is painted white and has white fletching. If it strikes a foe so that it would normally deal damage, it instead bursts into magical energy that deals nonlethal damage (in the same amount as would lethal damage) and forces the target to make a DC 11 Will save or fall asleep. 
				
Construction
				
\textbf{Requirements} Craft Magic Arms and Armor, \textit{sleep}; \textbf{Cost }70 gp
				
\textbf{Sun Blade}
				
\textbf{Aura} moderate evocation;\textbf{ CL }10th
				
\textbf{Slot} none; \textbf{Price} 50,335 gp; \textbf{Weight} 2 lbs.
				
DESCRIPTION
				
This sword is the size of a bastard sword. However, a \textit{sun blade} is wielded as if it were a short sword with respect to weight and ease of use. In other words, the weapon appears to all viewers to be a bastard sword, and deals bastard sword damage, but the wielder feels and reacts as if the weapon were a short sword. Any individual able to use either a bastard sword or a short sword with proficiency is proficient in the use of a \textit{sun blade}. Likewise, Weapon Focus and Weapon Specialization in short sword and bastard sword apply equally, but the benefits of those feats do not stack.
				
In normal combat, the glowing golden blade of the weapon is equal to a \textit{+2 bastard sword}. Against evil creatures, its enhancement bonus is +4. Against Negative Energy Plane creatures or undead creatures, the sword deals double damage (and \mbox{$\times$}3 on a critical hit instead of the usual \mbox{$\times$}2).
				
The blade also has a special sunlight power. Once per day, the wielder can swing the blade vigorously above his head while speaking a command word. The \textit{sun blade} then sheds a bright yellow radiance that acts like bright light and affects creatures susceptible to light as if it were natural sunlight. The radiance begins shining in a 10-foot radius around the sword wielder and extends outward at 5 feet per round for 10 rounds thereafter, to create a globe of light with a 60-foot radius. When the wielder stops swinging, the radiance fades to a dim glow that persists for another minute before disappearing entirely. All \textit{sun blades} are of good alignment, and any evil creature attempting to wield one gains one negative level. The negative level remains as long as the sword is in hand and disappears when the sword is no longer wielded. This negative level cannot be overcome in any way (including by \textit{restoration} spells) while the sword is wielded. 
				
Construction
				
\textbf{Requirements} Craft Magic Arms and Armor, \textit{daylight}, creator must be good; \textbf{Cost }25,335 gp
				
\textbf{Sword of Life Stealing}
				
\textbf{Aura} strong necromancy;\textbf{ CL }17th
				
\textbf{Slot} none; \textbf{Price} 25,715 gp; \textbf{Weight} 4 lbs.
				
DESCRIPTION
				
This black iron \textit{+2 longsword} bestows a negative level when it deals a critical hit. The sword wielder gains 1d6 temporary hit points each time a negative level is bestowed on another. These temporary hit points last for 24 hours. One day after being struck, subjects must make a DC 16 Fortitude save for each negative level gained or they become permanent. 
				
Construction
				
\textbf{Requirements} Craft Magic Arms and Armor, \textit{enervation}; \textbf{Cost }13,015 gp
				
\textbf{Sword of the Planes}
				
\textbf{Aura} strong evocation;\textbf{ CL }15th
				
\textbf{Slot} none; \textbf{Price} 22,315 gp; \textbf{Weight} 4 lbs.
				
DESCRIPTION
				
This longsword has an enhancement bonus of +1 on the Material Plane, but on any Elemental Plane its enhancement bonus increases to +2. The +2 enhancement bonus also applies whenever the weapon is used against creatures native to the Elemental Plane. It operates as a \textit{+3 longsword} on the Astral Plane and the Ethereal Plane, or when used against opponents native to either of those planes. On any other plane, or against any outsider, it functions as a \textit{+4 longsword}. 
				
Construction
				
\textbf{Requirements} Craft Magic Arms and Armor, \textit{plane shift}; \textbf{Cost }11,315 gp
				
\textbf{Sword of Subtlety}
				
\textbf{Aura} moderate illusion;\textbf{ CL }7th
				
\textbf{Slot} none; \textbf{Price} 22,310 gp; \textbf{Weight} 2 lbs.
				
DESCRIPTION
				
A \textit{+1 short sword} with a thin, dull gray blade, this weapon provides a +4 bonus on its wielder's attack and damage rolls when he is making a sneak attack with it. 
				
Construction
				
\textbf{Requirements} Craft Magic Arms and Armor, \textit{blur}; \textbf{Cost }11,310 gp
				
\textbf{Sylvan Scimitar}
				
\textbf{Aura} moderate evocation;\textbf{ CL }11th
				
\textbf{Slot} none; \textbf{Price} 47,315 gp; \textbf{Weight} 4 lbs.
				
DESCRIPTION
				
This \textit{+3 scimitar}, when used outdoors in a temperate climate, grants its wielder the use of the Cleave feat and deals an extra 1d6 points of damage. 
				
Construction
				
\textbf{Requirements} Craft Magic Arms and Armor, \textit{divine power} or creator must be a 7th-level druid; \textbf{Cost }23,815 gp
				
\textbf{Trident of Fish Command}
				
\textbf{Aura} moderate enchantment;\textbf{ CL }7th
				
\textbf{Slot} none; \textbf{Price} 18,650 gp; \textbf{Weight} 4 lbs.
				
DESCRIPTION
				
The magical properties of this \textit{+1 trident} with a 6-foot-long haft enable its wielder to charm up to 14 HD of aquatic animals as per the spell \textit{charm animals} (Will DC 16 negates, animals get a +5 bonus if currently under attack by the wielder or his allies), no two of which can be more than 30 feet apart. The wielder can use this effect up to three times per day. The wielder can communicate with the animals as if using a \textit{speak with animals} spell. Animals making their saving throws are free of control, but they will not approach within 10 feet of the trident. 
				
Construction
				
\textbf{Requirements} Craft Magic Arms and Armor, \textit{charm animals, speak with animals}; \textbf{Cost }9,482 gp and 5 sp
				
\textbf{Trident of Warning}
				
\textbf{Aura} moderate divination;\textbf{ CL }7th
				
\textbf{Slot} none; \textbf{Price} 10,115 gp; \textbf{Weight} 4 lbs.
				
DESCRIPTION
				
A weapon of this type enables its wielder to determine the location, depth, kind, and number of aquatic predators within 680 feet. A \textit{trident of warning} must be grasped and pointed in order for the character using it to gain such information, and it requires 1 round to scan a hemisphere with a radius of 680 feet. The weapon is otherwise a \textit{+2 trident}. 
				
Construction
				
\textbf{Requirements} Craft Magic Arms and Armor, \textit{locate creature}; \textbf{Cost }5,215 gp	

\section{Potions}

\label{f0}
\begin{table}[]
\sffamily
\caption{Table: Potions}
\begin{tabular}{lllll}
               &                 &                & \textbf{Spell} & \textbf{Caster} \\
\textbf{Minor} & \textbf{Medium} & \textbf{Major} & \textbf{Level} & \textbf{Level}\\
01--20 & --- & --- & 0 & 1st\\
21--60 & 01--20 & --- & 1st & 1st\\
 61--100 & 21--60 & 01--20 & 2nd & 3rd\\
 --- & 61--100 & 21--100 & 3rd & 5th\\
\end{tabular}
\end{table}

\begin{table}
 \sffamily
 \caption{Potion Costs}
 \begin{tabular}{lllll}
               & \textbf{Cleric} & \\
\textbf{Spell} & \textbf{Druid}  &                   &               & \textbf{Paladin} \\
\textbf{Level} & \textbf{Wizard} & \textbf{Sorcerer} & \textbf{Bard} & \textbf{Ranger} \\
0 & 25 gp & 25 gp & 25 gp & ---\\
1st & 50 gp & 50 gp & 50 gp & 50 gp\\
2nd & 300 gp & 400 gp & 400 gp & 400 gp\\
3rd & 750 gp & 900 gp & 1,050 gp & 1,050 gp\\
 \end{tabular}

\end{table}

				
A potion is a magic liquid that produces its effect when imbibed. Potions vary incredibly in appearance. Magic oils are similar to potions, except that oils are applied externally rather than imbibed. A potion or oil can be used only once. It can duplicate the effect of a spell of up to 3rd level that has a casting time of less than 1 minute and targets one or more creatures or objects. The price of a potion is equal to the level of the spell \mbox{$\times$} the creator's caster level \mbox{$\times$} 50 gp. If the potion has a material component cost, it is added to the base price and cost to create. Table: Potions gives sample prices for potions created at the lowest possible caster level for each spellcasting class. Note that some spells appear at different levels for different casters. The level of such spells depends on the caster brewing the potion.
				
Potions are like spells cast upon the imbiber. The character taking the potion doesn't get to make any decisions about the effect---the caster who brewed the potion has already done so. The drinker of a potion is both the effective target and the caster of the effect (though the potion indicates the caster level, the drinker still controls the effect).
				
The person applying an oil is the effective caster, but the object is the target.
				
\textbf{Physical Description}: A typical potion or oil consists of 1 ounce of liquid held in a ceramic or glass vial fitted with a tight stopper. The stoppered container is usually no more than 1 inch wide and 2 inches high. The vial has AC 13, 1 hit point, hardness 1, and a break DC of 12. 
				
\textbf{Identifying Potions}: In addition to the standard methods of identification, PCs can sample from each container they find to attempt to determine the nature of the liquid inside with a Perception check. The DC of this check is equal to 15 + the spell level of the potion (although this DC might be higher for rare or unusual potions). 
				
\textbf{Activation}: Drinking a potion or applying an oil requires no special skill. The user merely removes the stopper and swallows the potion or smears on the oil. The following rules govern potion and oil use.
				
Drinking a potion or using an oil is a standard action. The potion or oil takes effect immediately. Using a potion or oil provokes attacks of opportunity. An enemy may direct an attack of opportunity against the potion or oil container rather than against the character. A successful attack of this sort can destroy the container, preventing the character from drinking the potion or applying the oil. 
				
A creature must be able to swallow a potion or smear on an oil. Because of this, incorporeal creatures cannot use potions or oils. Any corporeal creature can imbibe a potion or use an oil.
				
A character can carefully administer a potion to an unconscious creature as a full-round action, trickling the liquid down the creature's throat. Likewise, it takes a full-round action to apply an oil to an unconscious creature.
        	

\section{Rings}

\label{f0}				
Rings bestow magical powers upon their wearers. Only a rare few have charges---most magic rings are permanent and potent magic items. Anyone can use a ring.
				
A character can only effectively wear two magic rings. A third magic ring doesn't work if the wearer is already wearing two magic rings.
				
\textbf{Physical Description}: Rings have no appreciable weight. Although exceptions exist that are crafted from glass or bone, the vast majority of rings are forged from metal---usually precious metals such as gold, silver, and platinum. A ring has AC 13, 2 hit points, hardness 10, and a break DC of 25.
				
\textbf{Activation}: A ring's ability is usually activated by a spoken command word (a standard action that does not provoke attacks of opportunity) or its effects work continually. Some rings have unusual activations, as mentioned in the ring's specific description.
				
\textbf{Special Qualities}: Roll d\%. A result of 01 indicates the ring is intelligent, 02--31 indicates that something (a design, inscription, or the like) provides a clue to its function, and 32--100 indicates no special qualities. Intelligent items have extra abilities and sometimes extraordinary powers and special purposes (see Intelligent Items). Rings with charges can never be intelligent.

\begin{table}[]
\sffamily
\caption{Table: Rings}
\setlength{\tabcolsep}{1pt}
\begin{tabularx}{\linewidth}{lllXl}
\textbf{Minor} & \textbf{Medium} & \textbf{Major} & \textbf{Ring} & \textbf{Market Price}\\
01--18 & --- & --- & Protection & 2,000 gp \\
 19--28 & --- & --- & Feather falling & 2,200 gp \\
 29--36 & --- & --- & Sustenance & 2,500 gp \\
 37--44 & --- & --- & Climbing & 2,500 gp \\
 45--52 & --- & --- & Jumping & 2,500 gp \\
 53--60 & --- & --- & Swimming & 2,500 gp \\
 61--70 & 01--05 & --- & Counterspells & 4,000 gp \\
 71--75 & 06--08 & --- & Mind shielding & 8,000 gp \\
 76--80 & 09--18 & --- & Protection & 8,000 gp \\
 81--85 & 19--23 & --- & Force shield & 8,500 gp \\
 86--90 & 24--28 & --- & Ram, the & 8,600 gp \\
 --- & 29--34 & --- & Climbing, improved & 10,000 gp \\
 --- & 35--40 & --- & Jumping, improved & 10,000 gp \\
 --- & 41--46 & --- & Swimming, improved & 10,000 gp \\
 91--93 & 47--50 & --- & Animal friendship & 10,800 gp \\
 94--96 & 51--56 & 01--02 & Energy resistance & 12,000 gp \\
 97--98 & 57--61 & --- & Chameleon power & 12,700 gp \\
 99--100 & 62--66 & --- & Water walking & 15,000 gp \\
 --- & 67--71 & 03--07 & Protection & 18,000 gp \\
 --- & 72--76 & 08--10 & Spell storing, minor & 18,000 gp \\
 --- & 77--81 & 11--15 & Invisibility & 20,000 gp \\
 --- & 82--85 & 16--19 & Wizardry & 20,000 gp \\
 --- & 86--90 & 20--25 & Evasion & 25,000 gp \\
 --- & 91--93 & 26--28 & X-ray vision & 25,000 gp \\
 --- & 94--97 & 29--32 & Blinking & 27,000 gp \\
 --- & 98--100 & 33--39 & Energy resistance & 28,000 gp \\
 --- & --- & 40--49 & Protection & 32,000 gp \\
 --- & --- & 50--55 & Wizardry & 40,000 gp \\
 --- & --- & 56--60 & Freedom of movement & 40,000 gp \\
 --- & --- & 61--63 & Energy resistance & 44,000 gp \\
 --- & --- & 64--65 & Friend shield & 50,000 gp \\
 --- & --- & 66--70 & Protection & 50,000 gp \\
 --- & --- & 71--74 & Shooting stars & 50,000 gp \\
 --- & --- & 75--79 & Spell storing & 50,000 gp \\
 --- & --- & 80--83 & Wizardry & 70,000 gp \\
 --- & --- & 84--86 & Telekinesis & 75,000 gp \\
 --- & --- & 87--88 & Regeneration & 90,000 gp \\
 --- & --- & 89--91 & Spell turning & 100,000 gp \\
 --- & --- & 92--93 & Wizardry & 100,000 gp \\
 --- & --- & 94 & Three wishes & 120,000 gp \\
 --- & --- & 95 & Djinni calling & 125,000 gp \\
 --- & --- & 96 & Elemental command (air) & 200,000 gp  \\
 --- & --- & 97 & Elemental command (earth) & 200,000 gp \\
 --- & --- & 98 & Elemental command (fire) & 200,000 gp \\
 --- & --- & 99 & Elemental command (water) & 200,000 gp \\
 --- & --- & 100 & Spell storing, major & 200,000 gp\\
\end{tabularx}
\end{table}

\textbf{Ring of Animal Friendship}
				
\textbf{Aura} faint enchantment;\textbf{ CL }3rd
				
\textbf{Slot} ring; \textbf{Price} 10,800 gp; \textbf{Weight} ---
				
Description
				
A \textit{ring of animal friendship} always bears some sort of animal-like design in its craftsmanship. On command, this ring affects an animal as if the wearer had cast \textit{charm animal}. 
				
Construction
				
\textbf{Requirements} Forge Ring, \textit{charm animal}; \textbf{Cost }5,400 gp
				
\textbf{Ring of Blinking}
				
\textbf{Aura} moderate transmutation;\textbf{ CL }7th
				
\textbf{Slot} ring; \textbf{Price} 27,000 gp; \textbf{Weight} ---
				
Description
				
On command, this ring makes the wearer blink, as the \textit{blink }spell. 
				
Construction
				
\textbf{Requirements} Forge Ring, \textit{blink}; \textbf{Cost }13,500 gp
				
\textbf{Ring of Chameleon Power}
				
\textbf{Aura} faint illusion;\textbf{ CL }3rd
				
\textbf{Slot} ring; \textbf{Price} 12,700 gp; \textbf{Weight} ---
				
Description
				
As a free action, the wearer of this ring can gain the ability 
% <span class="grame">
to magically blend
% </span class="grame">
 in with the surroundings. This provides a +
% <span class="grame">
10 competence
% </span class="grame">
 bonus on her Stealth checks. As a standard action, she can also use the spell \textit{disguise self }as often as she wants. 
				
Construction
				
\textbf{Requirements} Forge Ring, \textit{disguise self, invisibility}; \textbf{Cost }6,350 gp
				
\textbf{Ring of Climbing}
				
\textbf{Aura} faint transmutation;\textbf{ CL }5th
				
\textbf{Slot} ring; \textbf{Price} 2,500 gp; \textbf{Weight} ---
				
Description
				
This ring is actually a magic leather cord that ties around a finger. It continually grants the wearer a +
% <span class="grame">
5 competence
% </span class="grame">
 bonus on Climb checks. 
				
Construction
				
\textbf{Requirements} Forge Ring, creator must have 5 ranks in the Climb skill; \textbf{Cost }1,250 gp
				
\textbf{Ring of Climbing, Improved}
				
\textbf{Aura} faint transmutation;\textbf{ CL }5th
				
\textbf{Slot} ring; \textbf{Price} 10,000 gp; \textbf{Weight} ---
				
Description
				
As a \textit{ring of climbing}, except it grants a +
% <span class="grame">
10 competence
% </span class="grame">
 bonus on its wearer's Climb checks. 
				
Construction
				
\textbf{Requirements} Forge Ring, creator must have 10 ranks in the Climb skill; \textbf{Cost }5,000 gp
				
\textbf{Ring of Counterspells}
				
\textbf{Aura} moderate evocation;\textbf{ CL }11th
				
\textbf{Slot} ring; \textbf{Price} 4,000 gp; \textbf{Weight} ---
				
Description
				
This ring might seem to be a \textit{ring of spell storing }upon first examination. However, while it allows a single spell of 
% <span class="grame">
1st
% </span class="grame">
 through 6th level to be cast into it, that spell cannot be cast out of the ring again. Instead, should that spell ever be cast upon the wearer, the spell is immediately countered, as a counterspell action, requiring no action (or even knowledge) on the wearer's part. Once so used, the spell cast within the ring is gone. A new spell (or the same one as before) 
% <span class="grame">
may be placed
% </span class="grame">
 into it again. 
				
Construction
				
\textbf{Requirements} Forge Ring, \textit{imbue with spell ability}; \textbf{Cost }2,000 gp
				
\textbf{Ring of Djinni Calling}
				
\textbf{Aura} strong conjuration;\textbf{ CL }17th
				
\textbf{Slot} ring; \textbf{Price} 125,000 gp; \textbf{Weight} ---
				
Description
				
One of the many rings of fable, this \texttt{{}"{}}genie\texttt{{}"{}} ring is useful indeed. It serves as a special \textit{gate }by means of which a specific djinni 
% <span class="grame">
can be called
% </span class="grame">
 from the Plane of Air. 
% <span class="grame">
When the ring is rubbed (a standard action), the call goes out, and the djinni appears on the next round.
% </span class="grame">
 The djinni faithfully obeys and serves the wearer of the ring, but never for more than 1 hour per day. If the djinni of the ring 
% <span class="grame">
is ever killed
% </span class="grame">
, the ring becomes nonmagical and worthless. 
				
Construction
				
\textbf{Requirements} Forge Ring, \textit{gate}; \textbf{Cost }62,500 gp
				
\textbf{Ring of Elemental Command}
				
\textbf{Aura} strong conjuration; \textbf{CL} 15th
				
\textbf{Slot} ring; \textbf{Price} 200,000 gp; \textbf{Weight} ---
				
Description
				
All four kinds of \textit{elemental command }rings are very powerful. Each appears to be nothing more than a lesser magic ring until fully activated (by meeting a special condition, such as single-handedly slaying an elemental of the appropriate type or exposure to a sacred material of the appropriate element), but each has certain other powers as well as the following common properties.
				
Elementals of the plane to which the ring is attuned can't attack the wearer, or even approach within 5 feet of him. If the wearer desires, he may forego this protection and instead attempt to charm the elemental (as \textit{charm monster, }Will DC 17 negates). If the charm attempt fails, however, absolute protection is lost and no further attempt at charming can be made.
				
Creatures from the plane to which the ring is attuned who attack the wearer take a --1 penalty on their attack rolls. The ring wearer makes applicable saving throws against the extraplanar creature's attacks with a +2 resistance bonus. He gains a +4 morale bonus on all attack rolls against such creatures. Any weapon he uses bypasses the damage reduction of such creatures, regardless of any qualities the weapon may or may not have.
				
The wearer of the ring is able to converse with creatures from the plane to which his ring is attuned. These creatures recognize that he wears the ring, and show a healthy respect for the wearer if alignments are similar. If alignments are opposed, creatures fear the wearer if he is strong. If he is weak, they hate and desire to slay him.
				
The possessor of a \textit{ring of elemental command }takes a saving throw penalty as follows:
% <thead href="../glossary.html#damage-reduction">

ElementSaving Throw Penalty
% </thead href="../glossary.html#damage-reduction">

% <tbody href="../glossary.html#damage-reduction">

Air--2 against earth-based effects
Earth--2 against air- or electricity-based effects
Fire--2 against water- or cold-based effects
Water--2 against fire-based effects
% </tbody href="../glossary.html#damage-reduction">

				
In addition to the powers described above, each specific ring gives its wearer the following abilities according to its kind.
				
\textit{Ring of Elemental Command (Air)}
				\begin{itemize}\item  \textit{Feather fall }(unlimited use, wearer only)
				\item  \textit{Resist energy (electricity) }(unlimited use, wearer only)
				\item  \textit{Gust of wind }(twice per day)
				\item  \textit{Wind wall }(unlimited use)
				\item  \textit{Air walk }(once per day, wearer only)
				\item  \textit{Chain lightning }(once per week)
\end{itemize}
				
The ring appears to be a \textit{ring of feather falling }until a certain condition is met to activate its full potential. It must be reactivated each time a new wearer acquires it.
				
\textit{Ring of Elemental Command (Earth)}
				\begin{itemize}\item  \textit{Meld into stone }(unlimited use, wearer only) 
				\item  \textit{Soften earth and stone }(unlimited use)
				\item  \textit{Stone shape }(twice per day)
				\item  \textit{Stoneskin }(once per week, wearer only)
				\item  \textit{Passwall }(twice per week)
				\item  \textit{Wall of stone }(once per day)
\end{itemize}
				
The ring appears to be a \textit{ring of meld into stone }(allowing the wearer to cast \textit{meld into stone }at will) until the established condition is met.
				
\textit{Ring of Elemental Command (Fire)}
				\begin{itemize}\item  \textit{Resist energy (fire) }(as a \textit{major ring of energy resistance \mbox{$[$}fire\mbox{$]$}})
				\item  \textit{Burning hands }(unlimited use)
				\item  \textit{Flaming sphere }(twice per day)
				\item  \textit{Pyrotechnics }(twice per day)
				\item  \textit{Wall of fire }(once per day)
				\item  \textit{Flame strike }(twice per week)
\end{itemize}
				
The ring appears to be a \textit{major ring of energy resistance (fire) }until the established condition is met.
				
\textit{Ring of Elemental Command (Water)}
				\begin{itemize}\item  \textit{Water walk }(unlimited use)
				\item  \textit{Create water }(unlimited use)
				\item  \textit{Water breathing }(unlimited use)
				\item  \textit{Wall of ice }(once per day)
				\item  \textit{Ice storm }(twice per week)
				\item  \textit{Control water} (twice per week)
\end{itemize}
				
The ring appears to be a \textit{ring of water walking }until the established condition is met. 
				
Construction
				
\textbf{Requirements} Forge Ring, \textit{summon monster VI}, all appropriate spells; \textbf{Cost }100,000 gp
				
\textbf{Ring of Energy Resistance}
				
\textbf{Aura} faint (minor) or moderate (major or greater) abjuration;\textbf{ CL }3rd (minor), 7th (major), or 11th (greater)
				
\textbf{Slot} ring; \textbf{Price} 12,000 gp (minor), 28,000 gp (major), 44,000 gp (greater); \textbf{Weight} ---
				
Description
				
This ring continually protects the wearer from damage from one type of energy---acid, cold, electricity, fire, or sonic (chosen by the creator of the item; determine randomly if found as part of a treasure hoard). Each time the wearer would normally take such damage, subtract the ring's resistance value from the damage dealt.
				
A \textit{minor ring of energy resistance} grants 10 points of resistance. A \textit{major ring of energy resistance} grants 20 points of resistance. A \textit{greater ring of energy resistance} grants 30 points of resistance. 
				
Construction
				
\textbf{Requirements} Forge Ring, \textit{resist energy}; \textbf{Cost }6,000 gp (minor), 14,000 gp (major), 22,000 gp (greater)
				
\textbf{Ring of Evasion}
				
\textbf{Aura} moderate transmutation;\textbf{ CL }7th
				
\textbf{Slot} ring; \textbf{Price} 25,000 gp; \textbf{Weight} ---
				
Description
				
This ring continually grants the wearer the ability to avoid damage as if she had evasion. Whenever she makes a Reflex saving throw to determine whether she takes half damage, a successful save results in no damage. 
				
Construction
				
\textbf{Requirements} Forge Ring, \textit{jump}; \textbf{Cost }12,500 gp
				
\textbf{Ring of Feather Falling}
				
\textbf{Aura} faint transmutation;\textbf{ CL }1st
				
\textbf{Slot} ring; \textbf{Price} 2,200 gp; \textbf{Weight} ---
				
Description
				
This ring 
% <span class="grame">
is crafted
% </span class="grame">
 with a feather pattern all around its edge. It acts exactly like a \textit{feather fall }spell, activated immediately if the wearer falls more than 5 feet. 
				
Construction
				
\textbf{Requirements} Forge Ring, \textit{feather fall}; \textbf{Cost }1,100 gp
				
\textbf{Ring of Force Shield}
				
\textbf{Aura} moderate evocation;\textbf{ CL }9th
				
\textbf{Slot} ring; \textbf{Price} 8,500 gp; \textbf{Weight} ---
				
Description
				
An iron band, this simple ring generates a shield-sized (and shield-shaped) \textit{wall of force }that stays with the ring and can be wielded by the wearer as if it were a heavy shield (+
% <span class="grame">
2
% </span class="grame">
 AC). This special creation has no armor check penalty or arcane spell failure chance since it is weightless and encumbrance-free. It 
% <span class="grame">
can be activated and deactivated at will as a free action
% </span class="grame">
. 
				
Construction
				
\textbf{Requirements} Forge Ring, \textit{wall of force}; \textbf{Cost }4,250 gp
				
\textbf{Ring of Freedom of Movement}
				
\textbf{Aura} moderate abjuration;\textbf{ CL }7th
				
\textbf{Slot} ring; \textbf{Price} 40,000 gp; \textbf{Weight} ---
				
Description
				
This gold ring allows the wearer to act as if continually under the effect of a \textit{freedom of movement }spell. 
				
Construction
				
\textbf{Requirements} Forge Ring, \textit{freedom of movement}; \textbf{Cost }20,000 gp
				
\textbf{Ring of Friend Shield}
				
\textbf{Aura} moderate abjuration;\textbf{ CL }10th
				
\textbf{Slot} ring; \textbf{Price} 50,000 gp (for a pair); \textbf{Weight} ---
				
Description
				
These curious rings always come in pairs. A \textit{friend shield }ring without its mate is useless. Either wearer of one of a pair of the rings 
% <span class="grame">
can, at any time, command
% </span class="grame">
 his ring to cast a \textit{shield other }spell with the wearer of the mated ring as the recipient. This effect has no range limitation. 
				
Construction
				
\textbf{Requirements} Forge Ring, \textit{shield other}; \textbf{Cost }25,000 gp
				
\textbf{Ring of Invisibility}
				
\textbf{Aura} faint illusion;\textbf{ CL }3rd
				
\textbf{Slot} ring; \textbf{Price} 20,000 gp; \textbf{Weight} ---
				
Description
				
By activating this simple silver ring, the wearer can benefit from \textit{invisibility}, as the spell. 
				
Construction
				
\textbf{Requirements} Forge Ring, \textit{invisibility}; \textbf{Cost }10,000 gp
				
\textbf{Ring of Jumping}
				
\textbf{Aura} faint transmutation;\textbf{ CL }2nd
				
\textbf{Slot} ring; \textbf{Price} 2,500 gp; \textbf{Weight} ---
				
Description
				
This ring continually allows the wearer to leap about, providing a +
% <span class="grame">
5 competence
% </span class="grame">
 bonus on all his Acrobatics checks made to make high or long jumps. 
				
Construction
				
\textbf{Requirements} Forge Ring, creator must have 5 ranks in the Acrobatics skill; \textbf{Cost }1,250 gp
				
\textbf{Ring of Jumping, Improved}
				
\textbf{Aura} moderate transmutation;\textbf{ CL }7th
				
\textbf{Slot} ring; \textbf{Price} 10,000 gp; \textbf{Weight} ---
				
Description
				
As a \textit{ring of jumping}, except it grants a +
% <span class="grame">
10 competence
% </span class="grame">
 bonus on its wearer's Acrobatics checks made to make high or long jumps. 
				
Construction
				
\textbf{Requirements} Forge Ring, creator must have 10 ranks in the Acrobatics skill; \textbf{Cost }5,000 gp
				
\textbf{Ring of Mind Shielding}
				
\textbf{Aura} faint abjuration;\textbf{ CL }3rd
				
\textbf{Slot} ring; \textbf{Price} 8,000 gp; \textbf{Weight} ---
				
Description
				
This ring is usually of fine workmanship and wrought from heavy gold. The wearer is continually immune to \textit{detect thoughts, discern lies, }and any attempt to magically discern her alignment\textit{.} 
				
Construction
				
\textbf{Requirements} Forge Ring, \textit{nondetection}; \textbf{Cost }4,000 gp
				
\textbf{Ring of Protection}
				
\textbf{Aura} faint abjuration;\textbf{ CL }5th
				
\textbf{Slot} ring; \textbf{Price} 2,000 gp (+1), 8,000 gp (+2), 18,000 gp (+3), 32,000 gp (+4), 50,000 gp (+5); \textbf{Weight} ---
				
Description
				
This ring offers continual magical protection in the form of a deflection bonus of 
% <span class="grame">
+1
% </span class="grame">
 to +5 to AC. 
				
Construction
				
\textbf{Requirements} Forge Ring, \textit{shield of faith,} caster must be of a level at least three times the bonus of the ring; \textbf{Cost }1,000 gp (+1), 4,000 gp (+2), 9,000 gp (+3), 16,000 gp (+4), 25,000 gp (+5)
				
\textbf{Ring of the Ram}
				
\textbf{Aura} moderate transmutation; \textbf{CL} 9th
				
\textbf{Slot }ring; \textbf{Price} 8,600 gp; \textbf{Weight} ---
				
DESCRIPTION
				
The \textit{ring of the ram }is an ornate ring forged of hard metal, usually iron or an iron alloy. It has the head of a ram as its device. The wearer can command the ring to give forth a ram-like force, manifested by a vaguely discernible shape that resembles the head of a ram or a goat. This force strikes a single target, dealing 1d6 points of damage if 1 charge is expended, 2d6 points if 2 charges are used, or 3d6 points if 3 charges (the maximum) are used. Treat this as a ranged attack with a 50-foot maximum range and no penalties for distance. 
				
The force of the blow is considerable, and those struck by the ring are subject to a bull rush if within 30 feet of the ring-wearer. The ram is Large and uses the ring's caster level as its base attack bonus with a Strength of 25. This gives the ram a Combat Maneuver Bonus of +17. The ram gains a +1 bonus on the bull rush attempt if 2 charges are expended, or +2 if 3 charges are expended.
				
In addition to its attack mode, the \textit{ring of the ram }also has the power to open doors as if it were a character with Strength 25. This expends 1 charge. If 2 charges are expended, the effect is equivalent to a character with Strength 27. If 3 charges are expended, the effect is that of a character with Strength 29.
				
A newly created ring has 50 charges. When all the charges are expended, the ring becomes a nonmagical item.
				
CONSTRUCTION
				
\textbf{Requirements} Forge Ring, \textit{bull's strength, telekinesis}; \textbf{Cost} 4,300 gp
				
\textbf{Ring of Regeneration}
				
\textbf{Aura} strong conjuration;\textbf{ CL }15th
				
\textbf{Slot} ring; \textbf{Price} 90,000 gp; \textbf{Weight} ---
				
Description
				
This white gold ring is generally set with a large green sapphire. When worn, the ring continually allows a living wearer to heal 1 point of damage per round and an equal amount of nonlethal damage. In addition, he is immune to bleed damage while wearing a \textit{ring of regeneration}. If the wearer loses a limb, an organ, or any other body part while wearing this ring, the ring \textit{regenerates }it as the spell. In either case, only damage taken while wearing the ring 
% <span class="grame">
is regenerated
% </span class="grame">
. 
				
Construction
				
\textbf{Requirements} Forge Ring, \textit{regenerate}; \textbf{Cost }45,000 gp
				
\textbf{Ring of Shooting Stars}
				
\textbf{Aura} strong evocation; \textbf{CL} 12th
				
\textbf{Slot} ring; \textbf{Price} 50,000 gp; \textbf{Weight} ---
				
Description
				
This ring has two modes of operation: one for being in dim light or outdoors at night, and a second one when the wearer is underground or indoors at night.
				
During the night, under the open sky or in areas of shadow or darkness, the \textit{ring of shooting stars }can perform the following functions on command.
				\begin{itemize}\item  \textit{Dancing lights} (once per hour)
				\item  \textit{Light} (twice per night)
				\item  Ball lightning (special, once per night)
				\item  Shooting stars (special, three per week)
\end{itemize}
				
The first special function, ball lightning\textit{, }releases one to four balls of lightning (ring wearer's choice). These glowing globes resemble \textit{dancing lights}, and the ring wearer controls them similarly (see the \textit{dancing lights }spell description). The spheres have a 120-foot range and a duration of 4 rounds. They can be moved at 120 feet per round. Each sphere is about 3 feet in diameter, and any creature who comes within 5 feet of one causes its charge to dissipate, taking electricity damage in the process according to the number of balls created.
% <thead href="../spells/dancingLights.html#dancing-lights">

Number of BallsDamage per Ball
% </thead href="../spells/dancingLights.html#dancing-lights">

% <tbody href="../spells/dancingLights.html#dancing-lights">

1 lightning ball4d6 points of electricity damage
2 lightning balls3d6 points of electricity damage each
3 lightning balls2d6 points of electricity damage each
4 lightning balls1d6 points of electricity damage each
% </tbody href="../spells/dancingLights.html#dancing-lights">

				
Once the ball lightning function is activated, the balls can be released at any time before the sun rises. Multiple balls can be released in the same round.
				
The second special function produces three shooting stars that can be released from the ring each week, simultaneously or one at a time. They impact for 12 points of damage and spread (as a \textit{fireball}) in a 5-foot-radius sphere for 24 points of fire damage.
				
Any creature struck by a shooting star takes full damage from impact plus full fire damage from the spread unless it makes a DC 13 Reflex save. Creatures not struck but within the spread ignore the impact damage and take only half damage from the fire spread on a successful DC 13 Reflex save. Range is 70 feet, at the end of which the shooting star explodes unless it strikes a creature or object before that. A shooting star always follows a straight line, and any creature in its path must make a save or be hit by the projectile.
				
Indoors at night, or underground, the \textit{ring of shooting stars }has the following properties.
				\begin{itemize}\item  \textit{Faerie fire} (twice per day)
				\item  Spark shower (special, once per day)
\end{itemize}
				
The spark shower is a flying cloud of sizzling purple sparks that fan out from the ring for a distance of 20 feet in an arc 10 feet wide. Creatures within this area take 2d8 points of damage each if not wearing metal armor or carrying a metal weapon. Those wearing metal armor and/or carrying a metal weapon take 4d8 points of damage. 
				
Construction
				
\textbf{Requirements} Forge Ring\textit{, faerie fire, fireball}, \textit{light, lightning bolt}; \textbf{Cost }25,000 gp
				
\textbf{Ring of Spell Storing, Minor}
				
\textbf{Aura} faint evocation;\textbf{ CL }5th
				
\textbf{Slot} ring; \textbf{Price} 18,000 gp; \textbf{Weight} ---
				
Description
				
A \textit{minor ring of spell storing }contains up to three levels of spells (either divine or arcane, or even a mix of both spell types) that the wearer can cast. Each spell has a caster level equal to the minimum level needed to cast that spell. The user need not provide any material components or focus to cast the spell, and there is no arcane spell failure chance for wearing armor (because the ring wearer need not gesture). The activation time for the ring is the same as the casting time for the relevant spell, with a minimum of 1 standard action.
				
For a randomly generated ring, treat it as a scroll to determine what spells are stored in it. If you roll a spell that would put the ring over the three-level limit, ignore that roll; the ring has no more spells in it.
				
A spellcaster can cast any spells into the ring, so long as the total spell levels do not add up to more than three. Metamagic versions of spells take up storage space equal to their spell level modified by the metamagic feat. A spellcaster can use a scroll to put a spell into the \textit{minor ring of spell storing}.
				
The ring magically imparts to the wearer the names of all spells currently stored within it. 
				
Construction
				
\textbf{Requirements} Forge Ring, \textit{imbue with spell ability}; \textbf{Cost }9,000 gp
				
\textbf{Ring of Spell Storing}
				
\textbf{Aura} moderate evocation;\textbf{ CL }9th
				
\textbf{Slot} ring; \textbf{Price} 50,000 gp; \textbf{Weight} ---
				
Description
				
As the \textit{minor ring of spell storing, }except it holds up to 5 levels of spells. 
				
Construction
				
\textbf{Requirements} Forge Ring, \textit{imbue with spell ability}; \textbf{Cost }25,000 gp
				
\textbf{Ring of Spell Storing, Major}
				
\textbf{Aura} strong evocation;\textbf{ CL }17th
				
\textbf{Slot} ring; \textbf{Price} 200,000 gp; \textbf{Weight} ---
				
Description
				
As the \textit{minor ring of spell storing, }except it holds up to 10 levels of spells. 
				
Construction
				
\textbf{Requirements} Forge Ring, \textit{imbue with spell ability}; \textbf{Cost }100,000 gp
				
\textbf{Ring of Spell Turning}
				
\textbf{Aura} strong abjuration;\textbf{ CL }13th
				
\textbf{Slot} ring; \textbf{Price} 100,000 gp; \textbf{Weight} ---
				
Description
				
Up to three times per day on command, this simple platinum band automatically reflects the next nine levels of spells cast at the wearer, exactly as if \textit{spell turning }
% <span class="grame">
had been cast
% </span class="grame">
 upon him. 
				
Construction
				
\textbf{Requirements} Forge Ring, \textit{spell turning}; \textbf{Cost }50,000 gp
				
\textbf{Ring of Sustenance}
				
\textbf{Aura} faint conjuration;\textbf{ CL }5th
				
\textbf{Slot} ring; \textbf{Price} 2,500 gp; \textbf{Weight} ---
				
Description
				
This ring continually provides its wearer with life-sustaining nourishment. The ring also refreshes the body and mind, so that its wearer needs only sleep 2 hours per day to gain the benefit of 8 hours of sleep. This allows a spellcaster that requires rest to prepare spells to do so after only 2 hours, but this does not allow a spellcaster to prepare spells more than once per day. The ring 
% <span class="grame">
must be worn
% </span class="grame">
 for a full week before it begins to work. If it is removed, the owner must wear it for another week to reattune it to 
% <span class="grame">
himself
% </span class="grame">
. 
				
Construction
				
\textbf{Requirements} Forge Ring, \textit{create food and water}; \textbf{Cost }1,250 gp
				
\textbf{Ring of Swimming}
				
\textbf{Aura} faint transmutation;\textbf{ CL }2nd
				
\textbf{Slot} ring; \textbf{Price} 2,500 gp; \textbf{Weight} ---
				
Description
				
This silver ring usually has fish-like designs and motifs etched into the band. It continually grants the wearer a +
% <span class="grame">
5 competence
% </span class="grame">
 bonus on Swim checks. 
				
Construction
				
\textbf{Requirements} Forge Ring, creator must have 5 ranks in the Swim skill; \textbf{Cost }1,250 gp
				
\textbf{Ring of Swimming, Improved}
				
\textbf{Aura} moderate transmutation;\textbf{ CL }7th
				
\textbf{Slot} ring; \textbf{Price} 10,000 gp; \textbf{Weight} ---
				
Description
				
As a \textit{ring of swimming}, except it grants a +
% <span class="grame">
10 competence
% </span class="grame">
 bonus on its wearer's Swim checks. 
				
Construction
				
\textbf{Requirements} Forge Ring, creator must have 10 ranks in the Swim skill; \textbf{Cost }5,000 gp
				
\textbf{Ring of Telekinesis}
				
\textbf{Aura} moderate transmutation;\textbf{ CL }9th
				
\textbf{Slot} ring; \textbf{Price} 75,000 gp; \textbf{Weight} ---
				
Description
				
This ring allows the caster to use the spell \textit{telekinesis }on command. 
				
Construction
				
\textbf{Requirements} Forge Ring, \textit{telekinesis}; \textbf{Cost }37,500 gp
				
\textbf{Ring of Three Wishes}
				
\textbf{Aura} strong universal or evocation (if \textit{miracle }is used);\textbf{ CL }20th
				
\textbf{Slot} ring; \textbf{Price} 120,000 gp; \textbf{Weight} ---
				
Description
				
This ring is set with three rubies. Each ruby stores a \textit{wish }spell, activated by the ring. When a \textit{wish }is used, that ruby disappears. For a randomly generated ring, roll 1d3 to determine the remaining number of rubies. When all the \textit{wishes }are used, the ring becomes a nonmagical item. 
				
Construction
				
\textbf{Requirements} Forge Ring, \textit{wish} or \textit{miracle}; \textbf{Cost }97,500 gp
				
\textbf{Ring of Water Walking}
				
\textbf{Aura} moderate transmutation;\textbf{ CL }9th
				
\textbf{Slot} ring; \textbf{Price} 15,000 gp; \textbf{Weight} ---
				
Description
				
This ring is often made of coral or bluish metal decorated with wave motifs. It allows the wearer 
% <span class="grame">
to continually utilize
% </span class="grame">
 the effects of the spell \textit{water walk}. 
				
Construction
				
\textbf{Requirements} Forge Ring, \textit{water walk}; \textbf{Cost }7,500 gp
				
\textbf{Ring of Wizardry}
				
\textbf{Aura} moderate (\textit{wizardry I}) or strong (\textit{wizardry II--IV}) (no school);\textbf{ CL }11th (\textit{I}), 14th (\textit{II}), 17th (\textit{III}), 20th (\textit{IV})
				
\textbf{Slot} ring; \textbf{Price} 20,000 gp (I), 40,000 gp (II), 70,000 gp (III), 100,000 gp (IV); \textbf{Weight} ---
				
Description
				
This special ring comes in four kinds (\textit{ring of wizardry I, ring of wizardry II, ring of wizardry III, }and \textit{ring of wizardry IV}), all of them useful only to arcane spellcasters. The wearer's arcane spells per day 
% <span class="grame">
are doubled
% </span class="grame">
 for one specific spell level. A \textit{ring of wizardry I }doubles 1st-level spells, a \textit{ring of wizardry II }doubles 2nd-level spells, a \textit{ring of wizardry III }doubles 3rd-level spells, and a \textit{ring of wizardry IV }doubles 4th-level spells. Bonus spells from high ability scores or school specialization 
% <span class="grame">
are not doubled
% </span class="grame">
. 
				
Construction
				
\textbf{Requirements} Forge Ring, \textit{limited wish}; \textbf{Cost }10,000 gp (I), 20,000 gp (II), 35,000 gp (III), 50,000 gp (IV)
				
\textbf{Ring of X-Ray Vision}
				
\textbf{Aura} moderate divination;\textbf{ CL }6th
				
\textbf{Slot} ring; \textbf{Price} 25,000 gp; \textbf{Weight} ---
				
Description
				
On command, this ring gives its wearer the ability to see into and through solid matter. Vision range is 20 feet, with the viewer seeing as if he were looking at something in normal light even if there is no illumination. X-ray vision can penetrate 1 foot of stone, 1 inch of common metal, or up to 3 feet of wood or dirt. Thicker substances or a thin sheet of lead blocks the vision.
				
Using the ring is exhausting, causing the wearer 1 point of Constitution damage per minute after the first 10 minutes of use in a single day. The ring must be used in 1-minute increments.
				
Construction
				
\textbf{Requirements} Forge Ring, \textit{true seeing}; \textbf{Cost }12,500 gp
        	

\section{Rods}

\label{f0}
\begin{table}[]
\sffamily
\caption{Table: Rods}
\setlength{\tabcolsep}{1pt}
\begin{tabularx}{\linewidth}{llXl}
\textbf{Medium} & \textbf{Major} & \textbf{Rod} & \textbf{Market Price}\\
01--07 & --- & \textit{Metamagic, Enlarge} & 3,000 gp \\
 08--14 & --- & \textit{Metamagic, Extend} & 3,000 gp \\
 15--21 & --- & \textit{Metamagic, Silent} & 3,000 gp \\
 22--28 & --- & \textit{Immovable} & 5,000 gp \\
 29--35 & --- & \textit{Metamagic, Empower} & 9,000 gp \\
 36--42 & --- & \textit{Metal and mineral detection} & 10,500 gp \\
 43--53 & 01--04 & \textit{Cancellation} & 11,000 gp \\
 54--57 & 05--06 & \textit{Metamagic, Enlarge} & 11,000 gp \\
 58--61 & 07--08 & \textit{Metamagic, Extend} & 11,000 gp \\
 62--65 & 09--10 & \textit{Metamagic, Silent} & 11,000 gp \\
 66--71 & 11--14 & \textit{Wonder} & 12,000 gp \\
 72--79 & 15--19 & \textit{Python} & 13,000 gp \\
 80--83 & --- & \textit{Metamagic, Maximize} & 14,000 gp \\
 84--89 & 20--21 & \textit{Flame extinguishing} & 15,000 gp \\
 90--97 & 22--25 & \textit{Viper} & 19,000 gp \\
 --- & 26--30 & \textit{Enemy detection} & 23,500 gp \\
 --- & 31--36 & \textit{Metamagic, Enlarge} & 24,500 gp \\
 --- & 37--42 & \textit{Metamagic, Extend} & 24,500 gp \\
 --- & 43--48 & \textit{Metamagic, Silent} & 24,500 gp \\
 --- & 49--53 & \textit{Splendor} & 25,000 gp \\
 --- & 54--58 & \textit{Withering} & 25,000 gp \\
 98--99 & 59--64 & \textit{Metamagic, Empower} & 32,500 gp \\
 --- & 65--69 & \textit{Thunder and lightning} & 33,000 gp \\
 100 & 70--73 & \textit{Metamagic, Quicken} & 35,000 gp \\
 --- & 74--77 & \textit{Negation} & 37,000 gp \\
 --- & 78--80 & \textit{Absorption} & 50,000 gp \\
 --- & 81--84 & \textit{Flailing} & 50,000 gp \\
 --- & 85--86 & \textit{Metamagic, Maximize} & 54,000 gp \\
 --- & 87--88 & \textit{Rulership} & 60,000 gp \\
 --- & 89--90 & \textit{Security} & 61,000 gp \\
 --- & 91--92 & \textit{Lordly might} & 70,000 gp \\
 --- & 93--94 & \textit{Metamagic, Empower} & 73,000 gp \\
 --- & 95--96 & \textit{Metamagic, Quicken} & 75,500 gp \\
 --- & 97--98 & \textit{Alertness} & 85,000 gp \\
 --- & 99 & \textit{Metamagic, Maximize} & 121,500 gp \\
 --- & 100 & \textit{Metamagic, Quicken} & 170,000 gp\\
\end{tabularx}
\end{table}
				
Rods are scepter-like devices that have unique magical powers and do not usually have charges. Anyone can use a rod.
				
\textbf{Physical Description}: Rods weigh approximately 5 pounds. They range from 2 feet to 3 feet long and are usually made of iron or some other metal. (Many, as noted in their descriptions, can function as light maces or clubs due to their hardy construction.) These sturdy items have AC 9, 10 hit points, hardness 10, and a break DC of 27.
				
\textbf{Activation}: Details relating to rod use vary from item to item. Unless noted otherwise, you must be holding a rod to use its abilities. See the individual descriptions for specifics.
				
\textbf{Special Qualities}: Roll d\%. A 01 result indicates the rod is intelligent, 02--31 indicates that something (a design, inscription, or the like) provides a clue to its function, and 32--100 indicates no special qualities. Intelligent items have extra abilities and sometimes extraordinary powers and special purposes (see Intelligent Items).
				
Rods with charges can never be intelligent.
				
\textbf{Immovable Rod}
				
\textbf{Aura} moderate transmutation;\textbf{ CL }10th
				
\textbf{Slot} none; \textbf{Price} 5,000 gp; \textbf{Weight} 5 lbs.
				
Description
				
This rod looks like a flat iron bar with a small button on one end. When the button is pushed (a move action), the rod does not move from where it is, even if staying in place defies gravity. Thus, the owner can lift or place the rod wherever he wishes, push the button, and let go. Several \textit{immovable rods }can even make a ladder when used together (although only two 
% <span class="grame">
are needed

). An \textit{immovable rod }can support up to 8,000 pounds before falling to the ground. If a creature pushes against an \textit{immovable rod, }it must make a DC 30 Strength check to move the rod up to 10 feet in a single round. 
				
Construction
				
\textbf{Requirements} Craft Rod, \textit{levitate}; \textbf{Cost }2,500 gp
				
\textbf{Metamagic Rods}
				
Metamagic rods hold the essence of a metamagic feat, allowing the user to apply metamagic effects to spells (but not spell-like abilities) as they are cast. This does not change the spell slot of the altered spell. All the rods described here are use-activated (but casting spells in a threatened area still draws an attack of opportunity). A caster may only use one metamagic rod on any given spell, but it is permissible to combine a rod with metamagic feats possessed by the rod's wielder. In this case, only the feats possessed by the wielder adjust the spell slot of the spell 

being cast

.
				
Possession of a metamagic rod does not confer the associated feat on the owner, only the ability to use the given feat a specified number of times per day. A sorcerer still must take a full-round action when using a metamagic rod, just as if using a metamagic feat he possesses (except for \textit{quicken metamagic rods,} which can be used as a swift action).
				
\textbf{Lesser and Greater Metamagic Rods}: Normal metamagic rods can be used with spells of 6th level or lower. Lesser rods can be used with spells of 3rd level or lower, while greater rods can be used with spells of 9th level or lower. 
				
\textbf{Metamagic, Empower}
				
\textbf{Aura} strong (no school);\textbf{ CL }17th
				
\textbf{Slot} none; \textbf{Price} 9,000 gp (lesser), 32,500 gp (normal), 73,000 gp (greater); \textbf{Weight} 5 lbs.
				
Description
				
The wielder can cast up to three spells per day that are empowered as though using the Empower Spell feat. 
				
Construction
				
\textbf{Requirements} Craft Rod, Empower Spell; \textbf{Cost }4,500 gp (lesser), 16,250 gp (normal), 36,500 gp (greater)
				
\textbf{Metamagic, Enlarge}
				
\textbf{Aura} strong (no school);\textbf{ CL }17th
				
\textbf{Slot} none; \textbf{Price} 3,000 gp (lesser), 11,000 gp (normal), 24,500 gp (greater); \textbf{Weight} 5 lbs.
				
Description
				
The wielder can cast up to three spells per day that 
% <span class="grame">
are enlarged
% </span class="grame">
 as though using the Enlarge Spell feat. 
				
Construction
				
\textbf{Requirements} Craft Rod, Enlarge Spell; \textbf{Cost }1,500 gp (lesser), 5,500 gp (normal), 12,250 gp (greater)
				
\textbf{Metamagic, Extend}
				
\textbf{Aura} strong (no school);\textbf{ CL }17th
				
\textbf{Slot} none; \textbf{Price} 3,000 gp (lesser), 11,000 gp (normal), 24,500 gp (greater); \textbf{Weight} 5 lbs.
				
Description
				
The wielder can cast up to three spells per day that 
% <span class="grame">
are extended
% </span class="grame">
 as though using the Extend Spell feat. 
				
Construction
				
\textbf{Requirements} Craft Rod, Extend Spell; \textbf{Cost }1,500 gp (lesser), 5,500 gp (normal), 12,250 gp (greater)
				
\textbf{Metamagic, Maximize}
				
\textbf{Aura} strong (no school);\textbf{ CL }17th
				
\textbf{Slot} none; \textbf{Price} 14,000 gp (lesser), 54,000 gp (normal), 121,500 gp (greater); \textbf{Weight} 5 lbs.
				
Description
				
The wielder can cast up to three spells per day that 
% <span class="grame">
are maximized
% </span class="grame">
 as though using the Maximize Spell feat. 
				
Construction
				
\textbf{Requirements} Craft Rod, Maximize Spell feat; \textbf{Cost }7,000 gp (lesser), 27,000 gp (normal), 60,750 gp (greater)
				
\textbf{Metamagic, Quicken}
				
\textbf{Aura} strong (no school);\textbf{ CL }17th
				
\textbf{Slot} none; \textbf{Price} 35,000 gp (lesser), 75,500 gp (normal), 170,000 gp (greater); \textbf{Weight} 5 lbs.
				
Description
				
The wielder can cast up to three spells per day that 
% <span class="grame">
are quickened
% </span class="grame">
 as though using the Quicken Spell feat. 
				
Construction
				
\textbf{Requirements} Craft Rod, Quicken Spell; \textbf{Cost }17,500 gp (lesser), 37,750 gp (normal), 85,000 gp (greater)
				
\textbf{Metamagic, Silent}
				
\textbf{Aura} strong (no school);\textbf{ CL }17th
				
\textbf{Slot} none; \textbf{Price} 3,000 gp (lesser), 11,000 gp (normal), 24,500 gp (greater); \textbf{Weight} 5 lbs.
				
Description
				
The wielder can cast up to three spells per day without verbal components as though using the Silent Spell feat. 
				
Construction
				
\textbf{Requirements} Craft Rod, Silent Spell; \textbf{Cost }1,500 gp (lesser), 5,500 gp (normal), 12,250 gp (greater)
				
\textbf{Rod of Absorption}
				
\textbf{Aura} strong abjuration;\textbf{ CL }15th
				
\textbf{Slot} none; \textbf{Price} 50,000 gp; \textbf{Weight} 5 lbs.
				
Description
				
This rod absorbs spells or spell-like abilities into itself. The magic absorbed must be a single-target spell or a ray directed at either the character holding the rod or her gear. The rod then nullifies the spell's effect and stores its potential until the wielder releases this energy in the form of spells of her own. She can instantly detect a spell's level as the rod absorbs that spell's energy. Absorption requires no action on the part of the user if the rod is in hand at the time.
				
A running total of absorbed (and used) spell levels should be kept. The wielder of the rod can use captured spell energy to cast any spell she has prepared, without expending the preparation itself. The only restrictions are that the levels of spell energy stored in the rod must be equal to or greater than the level of the spell the wielder wants to cast, that any material components required for the spell be present, and that the rod be in hand when casting. For casters such as bards or sorcerers who do not prepare spells, the rod's energy can be used to cast any spell of the appropriate level or levels that they know.
				
A \textit{rod of absorption} absorbs a maximum of 50 spell levels and can thereafter only discharge any remaining potential it might have. The rod cannot be recharged. The wielder knows the rod's remaining absorbing potential and current amount of stored energy.
				
To determine the absorption potential remaining in a newly found rod, roll d\% and divide the result by 2. Then roll d\% again: on a result of 71--100, half the levels already absorbed by the rod are still stored within. 
				
Construction
				
\textbf{Requirements} Craft Rod, \textit{spell turning}; \textbf{Cost }25,000 gp
				
\textbf{Rod of Alertness}
				
\textbf{Aura} moderate abjuration, divination, enchantment, and evocation;\textbf{ CL }11th
				
\textbf{Slot} none; \textbf{Price} 85,000 gp; \textbf{Weight} 4 lbs.
				
Description
				
This rod is indistinguishable from a \textit{+1 light mace. }It has eight flanges on its mace-like head. The rod bestows a +1 insight bonus on initiative checks. If grasped firmly, the rod enables the holder to use \textit{detect evil, detect good, detect chaos, detect law}, \textit{detect magic, discern lies, light, }or \textit{see invisibility. }Each different use is a standard action.
				
If the head of a \textit{rod of alertness} is planted in the ground and the possessor wills it to alertness (a standard action), the rod senses any creatures within 120 feet who intend to harm the possessor. At the same time, the rod creates the effect of a \textit{prayer} spell upon all creatures friendly to the possessor in a 20-foot radius. Immediately thereafter, the rod sends forth a mental alert to these friendly creatures, warning them of any unfriendly creatures within the 120-foot radius. These effects last for 10 minutes, and the rod can perform this function once per day. Last, the rod can be used to simulate the casting of an \textit{animate objects} spell, utilizing any 11 (or fewer) Small objects located roughly around the perimeter of a 5-foot-radius circle centered on the rod when planted in the ground. Objects remain animated for 11 rounds. The rod can perform this function once per day. 
				
Construction
				
\textbf{Requirements} Craft Rod, \textit{alarm, animate objects, detect chaos, detect evil, detect good, detect law, detect magic, discern lies, light, prayer, see invisibility}; \textbf{Cost }42,500 gp
				
\textbf{Rod of Cancellation}
				
\textbf{Aura} strong abjuration;\textbf{ CL }17th
				
\textbf{Slot} none; \textbf{Price} 11,000 gp; \textbf{Weight} 5 lbs.
				
Description
				
This dreaded rod is a bane to magic items, for its touch drains an item of all magical properties. The item touched must make a DC 23 Will save to prevent the rod from draining it. If a creature is holding it at the time, then the item can use the holder's Will save bonus in place of its own if the holder's is better. In such cases, contact 
% <span class="grame">
is made
% </span class="grame">
 by making a melee touch attack roll. Upon draining an item, the rod itself becomes brittle and 
% <span class="grame">
cannot be used
% </span class="grame">
 again. Drained items are only restorable by \textit{wis}h or \textit{miracle. }If a \textit{sphere of annihilation }and a \textit{rod of cancellation }negate each other, nothing can restore either of them.
				
Construction
				
\textbf{Requirements} Craft Rod, \textit{mage's disjunction}; \textbf{Cost }5,500 gp
				
\textbf{Rod of Enemy Detection}
				
\textbf{Aura} moderate divination;\textbf{ CL }10th
				
\textbf{Slot} none; \textbf{Price} 23,500 gp; \textbf{Weight} 5 lbs.
				
Description
				
This device pulses in the wielder's hand and points in the direction of any creature or creatures hostile to the bearer of the device (nearest ones first). These creatures can be invisible, ethereal, hidden, disguised, or in plain sight. Detection range is 60 feet. If the bearer of the rod concentrates for a full round, the rod pinpoints the location of the nearest enemy and indicates how many enemies are within range. The rod 
% <span class="grame">
can be used
% </span class="grame">
 to pinpoint three times each day, each use lasting up to 10 minutes. Activating the rod is a standard action. 
				
Construction
				
\textbf{Requirements} Craft Rod, \textit{true seeing}; \textbf{Cost }11,750 gp
				
\textbf{Rod of Flailing}
				
\textbf{Aura} moderate enchantment;\textbf{ CL }9th
				
\textbf{Slot} none; \textbf{Price} 50,000 gp; \textbf{Weight} 5 lbs.
				
Description
				
Upon the command of its possessor, the rod activates, changing from a normal-seeming rod to a \textit{+3/+3 dire flail}. The dire flail is a double weapon, which means that each of the weapon's heads 
% <span class="grame">
can be used
% </span class="grame">
 to attack. The wielder can gain an extra attack (with the second head) at the cost of making all attacks at a --2 penalty (as if she had the Two-Weapon Fighting feat).
				
Once per day, the wielder can use a free action to cause the rod to grant her a +4 deflection bonus to Armor Class and a +4 resistance bonus on saving throws for 10 minutes. The rod need not be in weapon form to grant this benefit.
				
Transforming it into a weapon or back into a rod is a move action. 
				
Construction
				
\textbf{Requirements} Craft Rod, Craft Magic Arms and Armor, \textit{bless}; \textbf{Cost }25,000 gp
				
\textbf{Rod of Flame Extinguishing}
				
\textbf{Aura} strong transmutation;\textbf{ CL }12th
				
\textbf{Slot} none; \textbf{Price} 15,000 gp; \textbf{Weight} 5 lbs.
				
Description
				
This rod can extinguish Medium or smaller nonmagical fires with simply a touch (a standard action). For the rod to be effective against other sorts of fires, the wielder must expend 1 or more of the rod's charges.
				
Extinguishing a Large or larger nonmagical fire, or a magic fire of Medium or smaller (such as that of a \textit{flaming} weapon or a \textit{burning hands} spell), expends 1 charge. Continual magic flames, such as those of a weapon or a fire creature, are suppressed for 6 rounds and flare up again after that time. To extinguish an instantaneous fire spell, the rod must be within the area of the effect and the wielder must have used a ready action, effectively countering the entire spell.
				
When applied to Large or larger magic fires, such as those caused by \textit{fireball, flame strike, }or \textit{wall of fire, }extinguishing the flames expends 2 charges from the rod.
				
If a \textit{rod of flame extinguishing} is touched to a creature with the fire subtype by making a successful melee touch attack, the rod deals 6d6 points of damage to the creature. This use requires 3 charges.
				
A \textit{rod of flame extinguishing} has 10 charges when found. Spent charges are renewed every day, so that a wielder can expend up to 10 charges in any 24-hour period. 
				
Construction
				
\textbf{Requirements} Craft Rod, \textit{pyrotechnics}; \textbf{Cost }7,500 gp
				
\textbf{Rod of Lordly Might}
				
\textbf{Aura} strong enchantment, evocation, necromancy, and transmutation;\textbf{ CL }19th
				
\textbf{Slot} none; \textbf{Price} 70,000 gp; \textbf{Weight} 10 lbs.
				
Description
				
This rod has functions that are spell-like, and it can also be used as a magic weapon of various sorts. In addition, it has several more mundane uses. The \textit{rod of lordly might }is metal, thicker than other rods, with a flanged ball at one end and six stud-like buttons along its length. Pushing any of the rod's buttons is an action equivalent to drawing a weapon, and the rod weighs 10 pounds.
				
The following spell-like functions of the rod can each be used once per day.
				\begin{itemize}\item  \textit{Hold person} upon a touched creature, if the wielder so commands (Will DC 14 negates). The wielder must choose to use this power (a free action) and then succeed on a melee touch attack to activate the power. If the attack fails, the effect is lost.
				\item  \textit{Fear} upon all enemies viewing it, if the wielder so desires (10-foot maximum range, Will DC 16 partial). Invoking this power is a standard action.
				\item  Deal 2d4 hit points of damage to an opponent on a successful touch attack (Will DC 17 half) and cure the wielder of the same amount of damage. The wielder must choose to use this power before attacking, as with \textit{hold person}.
\end{itemize}
				
The following functions of the rod have no limit on the number of times they can be employed.
				\begin{itemize}\item  In its normal form, the rod can be used as a \textit{+2 light mace.}
				\item  When button 1 is pushed, the rod becomes a \textit{+1 flaming longsword.} A blade springs from the ball, with the ball itself becoming the sword's hilt. The weapon stretches to an overall length of 4 feet.
				\item  When button 2 is pushed, the rod becomes a \textit{+4 battleaxe.} A wide blade springs forth at the ball, and the whole lengthens to 4 feet.
				\item  When button 3 is pushed, the rod becomes a \textit{+3 shortspear} or \textit{+3 longspear.} The spear blade springs forth, and the handle can be lengthened up to 12 feet (wielder's choice) for an overall length ranging from 6 feet to 15 feet. At its 15-foot length, the rod is suitable for use as a lance.
\end{itemize}
				
The following other functions of the rod also have no limit on the number of times they can be employed.
				\begin{itemize}\item  Climbing pole/ladder. When button 4 is pushed, a spike that can anchor in stone is extruded from the ball, while the other end sprouts three sharp hooks. The rod lengthens to anywhere between 5 and 50 feet in a single round, stopping when button 4 is pushed again. Horizontal bars 3 inches long fold out from the sides, 1 foot apart, in staggered progression. The rod is firmly held by the spike and hooks and can bear up to 4,000 pounds. The wielder can retract the pole by pushing button 5.
				\item  The ladder function can be used to force open doors. The wielder plants the rod's base 30 feet or less from the portal to be forced and in line with it, then pushes button 4. The force exerted has a Strength modifier of +12.
				\item  When button 6 is pushed, the rod indicates magnetic north and gives the wielder knowledge of his approximate depth beneath the surface or height above it. 
\end{itemize}
				
Construction
				
\textbf{Requirements} Craft Magic Arms and Armor, Craft Rod\textit{, bull's strength, fear, flame blade, hold person}, \textit{inflict light wounds}; \textbf{Cost }35,000 gp
				
\textbf{Rod of Metal and Mineral Detection}
				
\textbf{Aura} moderate divination;\textbf{ CL }9th
				
\textbf{Slot} none; \textbf{Price} 10,500 gp; \textbf{Weight} 5 lbs.
				
Description
				
This rod is valued by treasure hunters and miners alike, for it pulses and hums in the wielder's hand in the proximity of metal. As the wearer aims the rod, the pulsations grow more noticeable as it points to the largest mass of metal within 30 feet. However, the wielder can concentrate on a specific metal or mineral. If the specific mineral is within 30 feet, the rod points to any places it is located, and the rod wielder knows the approximate quantity as well. If more than one deposit of the specified metal or mineral is within range, the rod points to the largest cache first. Each operation requires a full-round action. 
				
Construction
				
\textbf{Requirements} Craft Rod, \textit{locate object}; \textbf{Cost }5,250 gp
				
\textbf{Rod of Negation}
				
\textbf{Aura} strong varied;\textbf{ CL }15th
				
\textbf{Slot} none; \textbf{Price} 37,000 gp; \textbf{Weight} 5 lbs.
				
Description
				
This device negates the spell or spell-like function or functions of magic items. The wielder points the rod at the magic item, and a pale gray beam shoots forth to touch the target device, attacking as a ray (a ranged touch attack). The ray functions as a \textit{greater dispel magic }spell, except it only affects magic items. To negate instantaneous effects from an item, the rod wielder needs to have a readied action. The dispel check uses the rod's caster level (15th). The target item gets no saving throw, although the rod 
% <span class="grame">
can't
% </span class="grame">
 negate artifacts (even minor artifacts). The rod can function three times per day. 
				
Construction
				
\textbf{Requirements} Craft Rod, \textit{dispel magic,} and \textit{limited wish} or \textit{miracle}; \textbf{Cost }18,500 gp
				
\textbf{Rod of the Python}
				
\textbf{Aura} moderate transmutation;\textbf{ CL }10th
				
\textbf{Slot} none; \textbf{Price} 13,000 gp; \textbf{Weight} 10w lbs.
				
Description
				
Unlike most rods, one end of this rod curls and twists back on itself in a crook---the tip of this crook sometimes looks like the head of a snake. The rod itself is about 4 feet long and weighs 10 pounds. It strikes as a \textit{+1/+1 quarterstaff}. If the user throws the rod to the ground (a standard action), it grows to become a constrictor snake by the end of the round. The python obeys all commands of the owner. (In animal form, it retains the +1 enhancement bonus on attacks and damage possessed by the rod form.) The serpent returns to rod form (a full-round action) whenever the wielder desires, or whenever it moves farther than 100 feet from the owner. If the snake form 
% <span class="grame">
is slain
% </span class="grame">
, it returns to rod form and cannot be activated again for three days. A \textit{rod of the} \textit{python }only functions if the possessor is good. 
				
Construction
				
\textbf{Requirements} Craft Rod, Craft Magic Arms and Armor, \textit{baleful polymorph,} creator must be good; \textbf{Cost }6,500 gp
				
\textbf{Rod of Rulership}
				
\textbf{Aura} strong enchantment;\textbf{ CL }20th
				
\textbf{Slot} none; \textbf{Price} 60,000 gp; \textbf{Weight} 8 lbs.
				
Description
				
This rod looks like a royal scepter worth at least 5,000 gp in materials and workmanship alone. The wielder can command the obedience and fealty of creatures within 120 feet when she activates the device (a standard action). Creatures totaling 300 Hit Dice 
% <span class="grame">
can be ruled
% </span class="grame">
, but creatures with Intelligence scores of 12 or higher are each entitled to a DC 16 Will save to negate the effect. Ruled creatures obey the wielder as if she were their absolute sovereign. Still, if the wielder gives a command that is contrary to the nature of the creatures commanded, the magic is broken. The rod 
% <span class="grame">
can be used
% </span class="grame">
 for 500 total minutes before crumbling to dust. This duration need not be continuous. 
				
Construction
				
\textbf{Requirements} Craft Rod, \textit{mass charm monster}; \textbf{Cost }32,500 gp
				
\textbf{Rod of Security}
				
\textbf{Aura} strong conjuration;\textbf{ CL }20th
				
\textbf{Slot} none; \textbf{Price} 61,000 gp; \textbf{Weight} 5 lbs.
				
Description
				
This item creates a nondimensional space, a pocket paradise. There the rod's possessor and as many as 199 other creatures can stay in complete safety for a period of time, up to 200 days divided by the number of creatures affected. All fractions are rounded down. In this pocket paradise, creatures don't age, and natural healing takes place at twice the normal rate. Fresh water and food (fruits and vegetables only) are in abundance. The climate is comfortable for all creatures involved.
				
Activating the rod (a standard action) causes the wielder and all creatures touching the rod to be transported instantaneously to the paradise. Members of large groups can hold hands or otherwise maintain physical contact, allowing all connected creatures in a circle or a chain to be affected by the rod. Unwilling creatures get a DC 17 Will save to negate the effect. If such a creature succeeds on its save, other creatures beyond that point in a chain can still be affected by the rod.
				
When the rod's effect expires, is dismissed, or is dispelled, all the affected creatures instantly reappear in the location they occupied when the rod was activated. If something else occupies the space that a traveler would be returning to, then his body is displaced a sufficient distance to provide the space required for reentry. The rod's possessor can dismiss the effect whenever he wishes before the maximum time period expires, but the rod can only be activated once per week. 
				
Construction
				
\textbf{Requirements} Craft Rod, \textit{gate}; \textbf{Cost }30,500 gp
				
\textbf{Rod of Splendor}
				
\textbf{Aura} strong conjuration and transmutation;\textbf{ CL }12th
				
\textbf{Slot} none; \textbf{Price} 25,000 gp; \textbf{Weight} 5 lbs.
				
Description
				
The possessor of this fantastically bejeweled rod gains a +4 enhancement bonus to her Charisma score for as long as she holds or carries the item. Once per day, the rod garbs her in magically created clothing of the finest fabrics, plus adornments of furs and jewels. 
				
Apparel created by the magic of the rod remains in existence for 12 hours. However, if the possessor attempts to sell or give away any part of it, use it for a spell component, or the like, all the apparel immediately disappears. The same applies if any of it is forcibly taken from her.
				
The value of noble garb created by the rod ranges from 7,000 to 10,000 gp (1d4+6 \mbox{$\times$} 1,000 gp)---1,000 gp for the fabric alone, 5,000 gp for the furs, and the rest for the jewel trim (maximum of twenty gems, maximum value 200 gp each).
				
In addition, the rod has a second special power, usable once per week. Upon command, it creates a palatial tent---a huge pavilion of silk 60 feet across. Inside the tent are temporary furnishings and food suitable to the splendor of the pavilion and sufficient to entertain as many as 100 people. The tent and its trappings last for 1 day. At the end of that time, the tent and all objects associated with it (including any items that were taken out of the tent) disappear. 
				
Construction
				
\textbf{Requirements} Craft Rod, \textit{eagle's splendor, fabricate, major creation}; \textbf{Cost }12,500 gp
				
\textbf{Rod of Thunder and Lightning}
				
\textbf{Aura} moderate evocation;\textbf{ CL }9th
				
\textbf{Slot} none; \textbf{Price} 33,000 gp; \textbf{Weight} 5 lbs.
				
Description
				
Constructed of iron set with silver rivets, this rod has the properties of a \textit{+2 light mace}. Its other powers are as follows.
				\begin{itemize}\item  \textbf{Thunder}: Once per day, the rod can strike as a \textit{+3 light mace,} and the opponent struck is stunned from the noise of the rod's impact (Fortitude DC 16 negates). Activating this sonic power counts as a free action, and it works if the wielder strikes an opponent within 1 round.
				\item  \textbf{Lightning}: Once per day, when the wielder desires, a short spark of electricity can leap forth when the rod strikes an opponent to deal the normal damage for a \textit{+2 light mace} (1d6+2) and an extra 2d6 points of electricity damage. Even when the rod might not score a normal hit in combat, if the roll was good enough to count as a successful melee touch attack, then the 2d6 points of electricity damage still apply. The wielder activates this power as a free action, and it works if he strikes an opponent within 1 round.
				\item  \textbf{Thunderclap}: Once per day as a standard action, the wielder can cause the rod to give out a deafening noise, just as a \textit{shout} spell (Fortitude DC 16 partial, 2d6 points of sonic damage, target deafened for 2d6 rounds).
				\item  \textbf{Lightning Stroke}: Once per day as a standard action, the wielder can cause the rod to shoot out a 5-foot-wide \textit{lightning bolt} (9d6 points of electricity damage, Reflex DC 16 half) to a range of 200 feet.
				\item  \textbf{Thunder and Lightning}: Once per week as a standard action, the wielder of the rod can combine the thunderclap described above with a \textit{lightning bolt,} as in the lightning stroke. The thunderclap affects all within 10 feet of the bolt. The lightning stroke deals 9d6 points of electricity damage (count rolls of 1 or 2 as rolls of 3, for a range of 27 to 54 points), and the thunderclap deals 2d6 points of sonic damage. A single DC 16 Reflex save applies for both effects. 
\end{itemize}
				
Construction
				
\textbf{Requirements} Craft Magic Arms and Armor, Craft Rod, \textit{lightning bolt}, \textit{shout}; \textbf{Cost }16,500 gp
				
\textbf{Rod of the Viper}
				
\textbf{Aura} moderate necromancy;\textbf{ CL }10th
				
\textbf{Slot} none; \textbf{Price} 19,000 gp; \textbf{Weight} 5 lbs.
				
Description
				
This rod strikes as a \textit{+2 heavy mace}. Once per day, upon command, the head of the rod becomes that of an actual serpent for 10 minutes. During this period, any successful strike with the rod deals its usual damage 
% <span class="grame">
and also
% </span class="grame">
 poisons the creature hit. This poison deals 1d3 Constitution damage per round for 6 rounds. Poisoned creatures can make a DC 16 Fortitude save each round to negate the damage and end the affliction. Multiple hits extend the duration by 3 rounds and increase the DC by +2 for each hit. The rod only functions if its possessor is evil. 
				
Construction
				
\textbf{Requirements} Craft Rod, Craft Magic Arms and Armor, \textit{poison}, creator must be evil; \textbf{Cost }9,500 gp
				
\textbf{Rod of Withering}
				
\textbf{Aura} strong necromancy;\textbf{ CL }13th
				
\textbf{Slot} none; \textbf{Price} 25,000 gp; \textbf{Weight} 5 lbs.
				
Description
				
A \textit{rod of withering }acts as a \textit{+1 light mace }that deals no hit point damage. Instead, the wielder deals 1d4 points of Strength damage and 1d4 points of Constitution damage to any creature she touches with the rod (by making a melee touch attack). If she scores a critical hit, the damage from that hit is permanent ability drain. In either case, the defender negates the effect with a DC 17 Fortitude save. 
				
Construction
				
\textbf{Requirements} Craft Rod, Craft Magic Arms and Armor, \textit{contagion}; \textbf{Cost }12,500 gp
				
\textbf{Rod of Wonder}
				
\textbf{Aura} moderate enchantment;\textbf{ CL }10th
				
\textbf{Slot} none; \textbf{Price} 12,000 gp; \textbf{Weight} 5 lbs.
				
Description
				
A \textit{rod of wonder }is a strange and unpredictable device that randomly generates any number of weird effects each time it is used. Activating the rod is a standard action. Typical powers of the rod include the following.
% <thead class="stat-block-breaker">

d\%Wondrous Effect
% </thead class="stat-block-breaker">

% <tbody class="stat-block-breaker">

01--05\textit{Slow }target for 10 rounds (Will DC 15 negates).
06--10\textit{Faerie fire }surrounds the target.
11--15
% <span class="grame">
Deludes wielder for 1 round into believing the rod functions as indicated by a second die roll (no save).
% </span class="grame">

16--20\textit{Gust of wind, }but at windstorm force (Fortitude DC 14 negates).
21--25Wielder learns target's surface thoughts (as with \textit{detect thoughts) }for 1d4 rounds (no save).
26--30\textit{Stinking cloud} appears at 30-ft. range (Fortitude DC 15 negates).
31--33Heavy rain falls for 1 round in 60-ft. radius centered on rod wielder.
34--36Summon an animal---a rhino (01--25 on 
% <span class="grame">
d\%
% </span class="grame">
), elephant (26--50), or mouse (51--100).
37--46\textit{Lightning bolt}
% <span class="grame">
 (70 ft. long, 5 ft. wide), 6d6 damage (
% </span class="grame">
Reflex
% <span class="grame">
 DC 15 half).
% </span class="grame">

47--49A stream of 600 large butterflies pours forth and flutters around for 2 rounds, blinding everyone within 25 ft. (Reflex DC 14 negates).
50--53\textit{Enlarge person} on target if within 60 ft. of rod (Fortitude DC 13 negates).
54--58\textit{Darkness}, 30-ft.-diameter hemisphere, centered 30 ft. away from rod.
59--62Grass grows in 160-square-ft. area before the rod, or grass existing there grows to 10 
% <span class="grame">
times
% </span class="grame">
 normal size.
63--65Turn ethereal any nonliving object of up to 1,000 lbs. mass and up to 30 cubic ft. in size.
66--69Reduce wielder two size categories (no save) for 1 day.
70--79\textit{Fireball}
% <span class="grame">
 at target or 100 ft. straight ahead, 6d6 damage (
% </span class="grame">
Reflex
% <span class="grame">
 DC 15 half).
% </span class="grame">

80--84\textit{Invisibility }covers rod wielder.
85--87Leaves grow from target if within 60 ft. of rod. 
% <span class="grame">
These last 24 hours.
% </span class="grame">

88--9010--40 gems, value 1 gp each, shoot forth in a 30-ft.-long stream. Each gem deals 1 point of damage to any creature in its path: roll 5d4 for the number of hits and divide them among the available targets.
91--95Shimmering colors dance and play over a 40-ft.-by-30-ft. area in front of rod. Creatures therein 
% <span class="grame">
are 
% </span class="grame">
blinded for 1d6 rounds (Fortitude DC 15 negates).
96--97Wielder (50\% chance) or target (50\% chance) turns permanently blue, green, or purple (no save).
98--100\textit{Flesh to stone }(or \textit{stone to flesh }if target is stone already) if target is within 60 ft. (Fortitude DC 18 negates). 
% </tbody href="../combat.html#fortitude">

				
Construction
				
\textbf{Requirements} Craft Rod, \textit{confusion}, creator must be chaotic; \textbf{Cost }6,000 gp
        	

\section{Staves}

\label{f0}
\begin{table}[]
\sffamily
\caption{Table: Staves}
\begin{tabular}{llll}
\textbf{Medium} & \textbf{Major} & \textbf{Staff} & \textbf{Market Price}\\
01–15 & 01–03 & Charming & 17,600 gp \\
 16–30 & 04–09 & Fire & 18,950 gp \\
 31–40 & 10–11 & Swarming insects & 22,800 gp \\
 41–55 & 12–13 & Size alteration & 26,150 gp \\
 56–75 & 14–19 & Healing & 29,600 gp \\
 76–90 & 20–24 & Frost & 41,400 gp \\
 91–95 & 25–31 & Illumination & 51,500 gp \\
 96–100 & 32–38 & Defense & 62,000 gp \\
 — & 39–45 & Abjuration & 82,000 gp \\
 — & 46–50 & Conjuration & 82,000 gp \\
 — & 51–55 & Divination & 82,000 gp \\
 — & 56–60 & Enchantment & 82,000 gp \\
 — & 61–65 & Evocation & 82,000 gp \\
 — & 66–70 & Illusion & 82,000 gp \\
 — & 71–75 & Necromancy & 82,000 gp \\
 — & 76–80 & Transmutation & 82,000 gp \\
 — & 81–85 & Earth and stone & 85,800 gp \\
 — & 86–90 & Woodlands & 100,400 gp \\
 — & 91–95 & Life & 109,400 gp \\
 — & 96–98 & Passage & 206,900 gp \\
 — & 99–100 & Power & 235,000 gp\\
\end{tabular}
\end{table}
				
A staff is a long shaft that stores several spells. Unlike wands, which can contain a wide variety of spells, each staff is of a certain kind and holds specific spells. A staff has 10 charges when created.
				
\textbf{Physical Description}: A typical staff measures anywhere from 4 feet to 7 feet long and is 2 inches to 3 inches thick, weighing about 5 pounds. Most staves are wood, but an exotic few are bone, metal, or even glass. A staff often has a gem or some device at its tip or is shod in metal at one or both ends. Staves are often decorated with carvings or runes. A typical staff is like a walking stick, quarterstaff, or cudgel. It has AC 7, 10 hit points, hardness 5, and a break DC of 24.
				
\textbf{Activation}: Staves use the spell trigger activation method, so casting a spell from a staff is usually a standard action that doesn't provoke attacks of opportunity. (If the spell being cast has a longer casting time than 1 standard action, however, it takes that long to cast the spell from a staff.) To activate a staff, a character must hold it forth in at least one hand (or whatever passes for a hand, for nonhumanoid creatures).
				
\textbf{Special Qualities}: Roll d\%. A 01--30 result indicates that something (a design, inscription, or the like) provides some clue to the staff 's function, and 31--100 indicates no special qualities.
				
\textbf{Using Staves}: Staves use the wielder's ability score and relevant feats to set the DC for saves against their spells. Unlike with other sorts of magic items, the wielder can use his caster level when activating the power of a staff if it's higher than the caster level of the staff. 
				
This means that staves are far more potent in the hands of a powerful spellcaster. Because they use the wielder's ability score to set the save DC for the spell, spells from a staff are often harder to resist than those from other magic items, which use the minimum ability score required to cast the spell. Not only are aspects of the spell dependent on caster level (range, duration, and so on) potentially higher, but spells from a staff are also harder to dispel and have a better chance of overcoming a target's spell resistance.
				
Staves hold a maximum of 10 charges. Each spell cast from a staff consumes one or more charges. When a staff runs out of charges, it cannot be used until it is recharged. Each morning, when a spellcaster prepares spells or regains spell slots, he can also imbue one staff with a portion of his power so long as one or more of the spells cast by the staff is on his spell list and he is capable of casting at least one of the spells. Imbuing a staff with this power restores one charge to the staff, but the caster must forgo one prepared spell or spell slot of a level equal to the highest-level spell cast by the staff. For example, a 9th-level wizard with a \textit{staff of fire} could imbue the staff with one charge per day by using up one of his 4th-level spells. A staff cannot gain more than one charge per day and a caster cannot imbue more than one staff per day.
				
Furthermore, a staff can hold a spell of any level, unlike a wand, which is limited to spells of 4th level or lower. The minimum caster level of a staff is 8th.
				
\textbf{Staff of Abjuration}
				
\textbf{Aura} strong abjuration; \textbf{CL} 13th
				
\textbf{Slot} none; \textbf{Price} 82,000 gp; \textbf{Weight} 5 lbs.
				
Description
				
Usually carved from the heartwood of an ancient oak or other large tree, this staff allows use of the following spells:
				\begin{itemize}\item  \textit{Dispel magic} (1 charge)
				\item  \textit{Resist energy} (1 charge)
				\item  \textit{Shield} (1 charge)
				\item  \textit{Dismissal} (2 charges)
				\item  \textit{Lesser globe of invulnerability} (2 charges)
				\item  \textit{Repulsion} (3 charges) 
\end{itemize}
				
Construction
				
\textbf{Requirements} Craft Staff, \textit{dismissal, dispel magic, lesser globe of invulnerability, repulsion, resist energy, shield}; \textbf{Cost }41,000 gp
				
\textbf{Staff of Charming}
				
\textbf{Aura} moderate enchantment;\textbf{ CL }8th
				
\textbf{Slot} none; \textbf{Price} 17,600 gp; \textbf{Weight} 5 lbs.
				
Description
				
Made of twisting wood ornately shaped and carved, this staff allows use of the following spells:
				\begin{itemize}\item  \textit{Charm person} (1 charge)
				\item  \textit{Charm monster} (2 charges) 
\end{itemize}
				
Construction
				
\textbf{Requirements} Craft Staff, \textit{charm person, charm monster}; \textbf{Cost }8,800 gp
				
\textbf{Staff of Conjuration}
				
\textbf{Aura} strong conjuration; \textbf{CL} 13th
				
\textbf{Slot} none; \textbf{Price} 82,000 gp; \textbf{Weight} 5 lbs.
				
Description
				
This staff is usually made of ash or walnut and bears ornate carvings of many different kinds of creatures. It allows use of the following spells:
				\begin{itemize}\item  \textit{Stinking cloud} (1 charge)
				\item  \textit{Summon swarm} (1 charge)
				\item  \textit{Unseen servant} (1 charge)
				\item  \textit{Cloudkill} (2 charges)
				\item  \textit{Minor creation} (2 charges)
				\item  \textit{Summon monster VI} (3 charges) 
\end{itemize}
				
Construction
				
\textbf{Requirements} Craft Staff, \textit{cloudkill, minor creation, stinking cloud, summon monster VI, summon swarm, unseen servant}; \textbf{Cost }41,000 gp
				
\textbf{Staff of Defense}
				
\textbf{Aura} strong abjuration;\textbf{ CL }15th
				
\textbf{Slot} none; \textbf{Price} 62,000 gp; \textbf{Weight} 5 lbs.
				
Description
				
The \textit{staff of defense }is a simple-looking polished wooden staff that throbs with power when held defensively. It allows use of the following spells:
				\begin{itemize}\item  \textit{Shield} (1 charge)
				\item  S\textit{hield of Faith} (1 charge)
				\item  \textit{Shield other} (1 charge)
				\item  \textit{Shield of law} (3 charges) 
\end{itemize}
				
Construction
				
\textbf{Requirements} Craft Staff, \textit{shield, shield of faith, shield of law, shield other}, creator must be lawful; \textbf{Cost }31,000 gp
				
\textbf{Staff of Earth and Stone}
				
\textbf{Aura} moderate transmutation;\textbf{ CL }11th
				
\textbf{Slot} none; \textbf{Price} 85,800 gp; \textbf{Weight} 5 lbs.
				
Description
				
This staff is topped with a fist-sized emerald that gleams with smoldering power. It allows the use of the following spells: 
				\begin{itemize}\item  \textit{Move earth} (1 charge) 
				\item  \textit{Passwall} (1 charge)
\end{itemize}
				
Construction
				
\textbf{Requirements} Craft Staff, \textit{move earth, passwall}; \textbf{Cost }42,900 gp
				
\textbf{Staff of Divination}
				
\textbf{Aura} strong divination; \textbf{CL} 13th
				
\textbf{Slot} none; \textbf{Price} 82,000 gp; \textbf{Weight} 5 lbs.
				
Description
				
Made from a supple length of willow, often with a forked tip, this staff allows use of the following spells:
				\begin{itemize}\item  \textit{Detect secret doors} (1 charge)
				\item  \textit{Locate object} (1 charge)
				\item  \textit{Tongues} (1 charge)
				\item  \textit{Locate creature} (2 charges)
				\item  \textit{Prying eyes} (2 charges)
				\item  \textit{True seeing} (3 charges) 
\end{itemize}
				
Construction
				
\textbf{Requirements} Craft Staff, \textit{detect secret doors}, \textit{locate creature, locate object, prying eyes, tongues}, \textit{true seeing}; \textbf{Cost }41,000 gp
				
\textbf{Staff of Enchantment}
				
\textbf{Aura} strong enchantment; \textbf{CL} 13th
				
\textbf{Slot} none; \textbf{Price} 82,000 gp; \textbf{Weight} 5 lbs.
				
Description
				
Often made from applewood and topped with a clear crystal, this staff allows use of the following spells:
				\begin{itemize}\item  \textit{Hideous laughter} (1 charge)
				\item  \textit{Sleep} (1 charge)
				\item  \textit{Suggestion} (1 charge)
				\item  \textit{Crushing despair} (2 charges)
				\item  \textit{Mind fog} (2 charges)
				\item  \textit{Mass suggestion} (3 charges) 
\end{itemize}
				
Construction
				
\textbf{Requirements} Craft Staff, c\textit{rushing despair, hideous laughter, mass suggestion, mind fog, sleep, suggestion}; \textbf{Cost }41,000 gp
				
\textbf{Staff of Evocation}
				
\textbf{Aura} strong evocation; \textbf{CL} 13th
				
\textbf{Slot} none; \textbf{Price} 82,000 gp; \textbf{Weight} 5 lbs.
				
Description
				
This smooth hickory or yew staff allows use of the following spells:
				\begin{itemize}\item  \textit{Fireball} (1 charge)
				\item  \textit{Magic missile} (1 charge)
				\item  \textit{Shatter} (1 charge)
				\item  \textit{Ice storm} (2 charges)
				\item  \textit{Wall of force} (2 charges)
				\item  \textit{Chain lightning} (3 charges) 
\end{itemize}
				
Construction
				
\textbf{Requirements} Craft Staff\textit{, chain lightning, fireball, ice storm, magic missile, shatter, wall of force};\textbf{ Cost }41,000 gp
				
\textbf{Staff of Fire}
				
\textbf{Aura} moderate evocation; \textbf{CL} 8th
				
\textbf{Slot }none; \textbf{Price} 18,950 gp; \textbf{Weight} 5 lbs.
				
Description
				
Crafted from bronzewood with brass bindings, this staff allows use of the following spells:
				\begin{itemize}\item  \textit{Burning hands }(1 charge)
				\item  \textit{Fireball }(2 charges)
				\item  \textit{Wall of fire }(3 charges)
\end{itemize}
				
Construction
				
\textbf{Requirements }Craft Staff, \textit{burning hands, fireball, wall of fire}; \textbf{Cost} 9,475 gp
				
\textbf{Staff of Frost}
				
\textbf{Aura} moderate evocation; \textbf{CL} 10th
				
\textbf{Slot }none; \textbf{Price} 41,400 gp; \textbf{Weight} 5 lbs.
				
Description
				
Tipped on either end with a glistening diamond, this rune-covered staff allows use of the following spells:
				\begin{itemize}\item  \textit{Ice storm }(1 charge)
				\item  \textit{Wall of ice }(2 charges)
				\item  \textit{Cone of cold }(3 charges)
\end{itemize}
				
Construction
				
\textbf{Requirements }Craft Staff, \textit{cone of cold}, \textit{ice storm, wall of ice}; \textbf{Cost} 20,700 gp
				
\textbf{Staff of Healing}
				
\textbf{Aura} moderate conjuration;\textbf{ CL }8th
				
\textbf{Slot} none; \textbf{Price} 29,600 gp; \textbf{Weight} 5 lbs.
				
Description
				
This white ash staff is decorated with inlaid silver runes. It allows use of the following spells:
				\begin{itemize}\item  \textit{Cure serious wounds} (1 charge)
				\item  \textit{Lesser restoration} (1 charge)
				\item  \textit{Remove blindness/deafness} (2 charges)
				\item  \textit{Remove disease} (3 charges) 
\end{itemize}
				
Construction
				
\textbf{Requirements} Craft Staff, \textit{cure serious wounds, lesser restoration, remove blindness/deafness, remove disease};\textbf{ Cost }14,800 gp
				
\textbf{Staff of Illumination}
				
\textbf{Aura} strong evocation;\textbf{ CL }15th
				
\textbf{Slot} none; \textbf{Price} 51,500 gp; \textbf{Weight} 5 lbs.
				
Description
				
This staff is usually sheathed in silver and decorated with sunbursts. It allows use of the following spells:
				\begin{itemize}\item  \textit{Dancing lights} (1 charge)
				\item  \textit{Flare} (1 charge)
				\item  \textit{Daylight} (2 charges)
				\item  \textit{Sunburst} (3 charges) 
\end{itemize}
				
Construction
				
\textbf{Requirements} Craft Staff, \textit{dancing lights, daylight, flare, sunburst}; \textbf{Cost }20,750 gp
				
\textbf{Staff of Illusion}
				
\textbf{Aura} strong illusion; \textbf{CL} 13th
				
\textbf{Slot} none; \textbf{Price} 82,000 gp; \textbf{Weight} 5 lbs.
				
Description
				
This staff is made from ebony or other dark wood and carved into an intricately twisted, fluted, or spiral shape. It allows use of the following spells:
				\begin{itemize}\item  \textit{Disguise self} (1 charge)
				\item  \textit{Major image} (1 charge)
				\item  \textit{Mirror image} (1 charge)
				\item  \textit{Persistent image} (2 charges)
				\item  \textit{Rainbow pattern} (2 charges)
				\item  \textit{Mislead} (3 charges) 
\end{itemize}
				
Construction
				
\textbf{Requirements} Craft Staff, \textit{disguise self, major image, mirror image, persistent image, mislead, rainbow pattern}; \textbf{Cost }41,000 gp
				
\textbf{Staff of Life}
				
\textbf{Aura} moderate conjuration;\textbf{ CL }11th
				
\textbf{Slot} none; \textbf{Price} 109,400 gp; \textbf{Weight} 5 lbs.
				
Description
				
A \textit{staff of life is} made of thick polished oak shod in gold and decorated with sinuous runes. This staff allows use of the following spells:
				\begin{itemize}\item  \textit{Heal} (1 charge)
				\item  \textit{Raise dead} (5 charges) 
\end{itemize}
				
Construction
				
\textbf{Requirements} Craft Staff, \textit{heal, raise dead}; \textbf{Cost }79,700 gp
				
\textbf{Staff of Necromancy}
				
\textbf{Aura} strong necromancy; \textbf{CL} 13th
				
\textbf{Slot} none; \textbf{Price} 82,000 gp; \textbf{Weight} 5 lbs.
				
Description
				
This staff is made from ebony or other dark wood and carved with images of bones and skulls mingled with strange spidery runes. It allows use of the following spells:
				\begin{itemize}\item  \textit{Cause fear} (1 charge)
				\item  \textit{Ghoul touch} (1 charge)
				\item  \textit{Halt undead} (1 charge)
				\item  \textit{Enervation} (2 charges)
				\item  \textit{Waves of fatigue} (2 charges)
				\item  \textit{Circle of death} (3 charges) 
\end{itemize}
				
Construction
				
\textbf{Requirements} Craft Staff, \textit{cause fear, circle of death, enervation, ghoul touch, halt undead, waves of fatigue}; \textbf{Cost }41,000 gp
				
\textbf{Staff of Passage}
				
\textbf{Aura} strong varied;\textbf{ CL }17th
				
\textbf{Slot} none; \textbf{Price} 206,900 gp; \textbf{Weight} 5 lbs.
				
Description
				
This potent item allows use of the following spells: 
				\begin{itemize}\item  \textit{Dimension door} (1 charge)
				\item  \textit{Passwall} (1 charge)
				\item  \textit{Greater teleport} (2 charges)
				\item  \textit{Phase door} (2 charges)
				\item  \textit{Astral projection} (2 charges) 
\end{itemize}
				
Construction
				
\textbf{Requirements} Craft Staff, \textit{astral projection, dimension door, greater teleport, passwall, phase door}; \textbf{Cost }115,950 gp
				
\textbf{Staff of Power}
				
\textbf{Aura} strong varied; \textbf{CL} 15th
				
\textbf{Slot }none; \textbf{Price} 235,000 gp; \textbf{Weight} 5 lbs.
				
Description
				
The \textit{staff of power }is a very potent magic item with offensive and defensive abilities. It is usually topped with a glistening gem that often burns from within with a flickering red light. The staff allows the use of the following spells:
				\begin{itemize}\item  \textit{Continual flame} (1 charge)
				\item  \textit{Fireball} (heightened to 5th level) (1 charge)
				\item  \textit{Levitate} (1 charge)
				\item  \textit{Lightning bolt} (heightened to 5th level) (1 charge)
				\item  \textit{Magic missile} (1 charge)
				\item  \textit{Ray of enfeeblement} (heightened to 5th level) (1 charge)
				\item  \textit{Cone of cold} (2 charges)
				\item  \textit{Globe of invulnerability} (2 charges)
				\item  \textit{Hold monster} (2 charges)
				\item  \textit{Wall of force} (in a 10-ft.-diameter hemisphere around the caster only) (2 charges)
\end{itemize}
				
The wielder of a \textit{staff of power }gains a +2 luck bonus to AC and on saving throws. The staff is also a \textit{+2 quarterstaff, }and its wielder may use it to smite opponents. If 1 charge is expended (as a free action), the staff causes double damage (\mbox{$\times$}3 on a critical hit) for 1 round.
				
A \textit{staff of power }can be used for a retributive strike, requiring it to be broken by its wielder. (If this breaking of the staff is purposeful and declared by the wielder, it can be performed as a standard action that does not require the wielder to make a Strength check.) All charges currently in the staff are instantly released in a 30-foot spread. All within 2 squares of the broken staff take points of damage equal to 20 \mbox{$\times$} the number of charges in the staff, those 3 or 4 squares away take 15 \mbox{$\times$} the number of charges in damage, and those 5 or 6 squares distant take 10 \mbox{$\times$} the number of charges in damage. All those affected can make DC 17 Reflex saves to reduce the damage by half.
				
The character breaking the staff has a 50\% chance of traveling to another plane of existence, but if he does not, the explosive release of spell energy destroys him. Only certain items, including the \textit{staff of the magi} and the \textit{staff of power, }are capable of being used for a retributive strike.
				
Construction
				
\textbf{Requirements }Craft Staff, Craft Magic Arms and Armor, \textit{cone of cold, continual flame, }heightened \textit{fireball, globe of invulnerability, hold monster, levitate, }heightened \textit{lightning bolt, magic missile, }heightened \textit{ray of enfeeblement, wall of force}; \textbf{Cost} 117,500 gp
				
\textbf{Staff of Size Alteration}
				
\textbf{Aura} moderate transmutation; \textbf{CL} 8th
				
\textbf{Slot }none; \textbf{Price} 26,150 gp; \textbf{Weight} 5 lbs.
				
Description
				
This staff of dark wood is relatively more stout and sturdy than most magical staves, with a gnarled and twisted knot of wood at the top end. It allows use of the following spells:
				\begin{itemize}\item  \textit{Enlarge person} (1 charge)
				\item  \textit{Reduce person} (1 charge)
				\item  \textit{Shrink item} (2 charges)
				\item  \textit{Mass enlarge person} (3 charges)
				\item  \textit{Mass reduce person} (3 charges)
\end{itemize}
				
Construction
				
\textbf{Requirements }Craft Staff, \textit{enlarge person, mass enlarge person, mass reduce person, reduce person, shrink item}; \textbf{Cost} 13,075 gp
				
\textbf{Staff of Swarming Insects}
				
\textbf{Aura} moderate conjuration;\textbf{ CL }9th
				
\textbf{Slot} none; \textbf{Price} 22,800 gp; \textbf{Weight} 5 lbs.
				
Description
				
Made of twisted darkwood covered with knots and nodules resembling crawling insects (which occasionally seem to move), this staff allows use of the following spells:
				\begin{itemize}\item  \textit{Summon swarm} (1 charge)
				\item  \textit{Insect plague} (3 charges) 
\end{itemize}
				
Construction
				
\textbf{Requirements} Craft Staff, \textit{insect plague, summon swarm}; \textbf{Cost }11,400 gp
				
\textbf{Staff of Transmutation}
				
\textbf{Aura} strong transmutation; \textbf{CL} 13th
				
\textbf{Slot} none; \textbf{Price} 82,000 gp; \textbf{Weight} 5 lbs.
				
Description
				
This staff is generally carved from or decorated with petrified wood or fossilized bone, each etched with tiny but complex runes. It allows use of the following spells:
				\begin{itemize}\item  \textit{Alter self} (1 charge)
				\item  \textit{Blink} (1 charge)
				\item  \textit{Expeditious retreat} (1 charge)
				\item  \textit{Baleful polymorph} (2 charges)
				\item  \textit{Polymorph} (2 charges)
				\item  \textit{Disintegrate} (3 charges) 
\end{itemize}
				
Construction
				
\textbf{Requirements} Craft Staff, \textit{alter self, baleful polymorph, blink, disintegrate, expeditious retreat, polymorph}; \textbf{Cost }41,000 gp
				
\textbf{Staff of the Woodlands}
				
\textbf{Aura} strong varied; \textbf{CL} 13th
				
\textbf{Slot} none; \textbf{Price} 100,400 gp; \textbf{Weight} 5 lbs.
				
Description
				
Appearing to have grown naturally into its shape, this oak, ash, or yew staff allows use of the following spells:
				\begin{itemize}\item  \textit{Charm animal} (1 charge)
				\item  \textit{Speak with animals} (1 charge)
				\item  \textit{Barkskin} (2 charges)
				\item  \textit{Summon nature's ally VI} (3 charges)
				\item  \textit{Wall of thorns} (3 charges)
				\item  \textit{Animate plants} (4 charges)
\end{itemize}
				
The staff may be used as a weapon, functioning as a \textit{+2 quarterstaff}. The \textit{staff of the woodlands} also allows its wielder to cast \textit{pass without trace} at will, with no charge cost. These two attributes continue to function after all the charges are expended. 
				
Construction
				
\textbf{Requirements} Craft Magic Arms and Armor, Craft Staff, \textit{animate plants, barkskin, charm animal, pass without trace, speak with animals, summon nature's ally VI, wall of thorns}; \textbf{Cost }50,500 gp
        	

\section{Wands}

\label{f0}
\begin{table}[]
\sffamily
\caption{Table: Wands}
\begin{tabular}{lllll}
               &                 &                & \textbf{Spell} & \textbf{Caster}\\
\textbf{Minor} & \textbf{Medium} & \textbf{Major} & \textbf{Level} & \textbf{Level}\\
 01--05 & --- & --- & 0 & 1st\\
 06--60 & --- & --- & 1st & 1st\\
 61--100 & 01--60 & --- & 2nd & 3rd\\
 --- & 61--100 & 01--60 & 3rd & 5th\\
 --- & --- & 61--100 & 4th & 7th\\
\end{tabular}
\end{table}

				
A wand is a thin baton that contains a single spell of 4th level or lower. A wand has 50 charges when created---each charge allows the use of the wand's spell one time. A wand that runs out of charges is just a stick. The price of a wand is equal to the level of the spell \mbox{$\times$} the creator's caster level \mbox{$\times$} 750 gp. If the wand has a material component cost, it is added to the base price and cost to create once for each charge (50 \mbox{$\times$} material component cost). Table: Wands gives sample prices for wands created at the lowest possible caster level for each spellcasting class. Note that some spells appear at different levels for different casters. The level of such spells depends on the caster crafting the wand.

\begin{table}
 \sffamily
 \caption{Wand Costs}
 \begin{tabular}{lllll}
               & \textbf{Cleric} & \\
\textbf{Spell} & \textbf{Druid}  &                   &               & \textbf{Paladin} \\
\textbf{Level} & \textbf{Wizard} & \textbf{Sorcerer} & \textbf{Bard} & \textbf{Ranger} \\
0 & 375 gp & 375 gp & 375 gp & ---\\
1st & 750 gp & 750 gp & 750 gp & 750 gp\\
2nd & 4,500 gp & 6,000 gp & 6,000 gp & 6,000 gp\\
3rd & 11,250 gp & 13,500 gp & 15,750 gp & 15,750 gp\\
4th & 21,000 gp & 24,000 gp & 30,000 gp & 30,000 gp\\  
 \end{tabular}

\end{table}

% </div class="table">

				
\textbf{Physical Description}: A wand is 6 to 12 inches long, 1/4 inch thick, and weighs no more than 1 ounce. Most wands are wood, but some are bone, metal, or even crystal. A typical wand has AC 7, 5 hit points, hardness 5, and a break DC of 16.
				
\textbf{Activation}: Wands use the spell trigger activation method, so casting a spell from a wand is usually a standard action that doesn't provoke attacks of opportunity. (If the spell being cast has a longer casting time than 1 action, however, it takes that long to cast the spell from a wand.) To activate a wand, a character must hold it in hand (or whatever passes for a hand, for nonhumanoid creatures) and point it in the general direction of the target or area. A wand may be used while grappling or while swallowed whole.
				
\textbf{Special Qualities}: Roll d\%. A 01--30 result indicates that something (a design, inscription, or the like) provides some clue to the wand's function, and 31--100 indicates no special qualities.        	

\section{Wondrous Items}

% Please add the following required packages to your document preamble:
% \usepackage[normalem]{ulem}
% \useunder{\uline}{\ul}{}

\captionof{table}{Minor Wondrous Items}
\setlength{\tabcolsep}{1pt}
\begin{xtabular}{lll}
\textbf{d\%} & \textbf{Item}                      & \textbf{Market Price} \\
01  & Feather token, anchor                       & 50 gp        \\
02  & Universal solvent                           & 50 gp        \\
03  & Elixir of love                              & 150 gp       \\
04  & Unguent of timelessness                     & 150 gp       \\
05  & Feather token, fan                          & 200 gp       \\
06  & Dust of tracelessness                       & 250 gp       \\
07  & Elixir of hiding                            & 250 gp       \\
08  & Elixir of tumbling                          & 250 gp       \\
09  & Elixir of swimming                          & 250 gp       \\
10  & Elixir of vision                            & 250 gp       \\
11  & Silversheen                                 & 250 gp       \\
12  & Feather token, bird                         & 300 gp       \\
13  & Feather token, tree                         & 400 gp       \\
14  & Feather token, swan boat                    & 450 gp       \\
15  & Elixir of truth                             & 500 gp       \\
16  & Feather token, whip                         & 500 gp       \\
17  & Dust of dryness                             & 850 gp       \\
18  & Hand of the mage                            & 900 gp       \\
19  & Bracers of armor +1                         & 1,000 gp     \\
20  & Cloak of resistance +1                      & 1,000 gp     \\
21  & Pearl of power, 1st-level spell             & 1,000 gp     \\
22  & Phylactery of faithfulness                  & 1,000 gp     \\
23  & Salve of slipperiness                       & 1,000 gp     \\
24  & Elixir of fire breath                       & 1,100 gp     \\
25  & Pipes of the sewers                         & 1,150 gp     \\
26  & Dust of illusion                            & 1,200 gp     \\
27  & Brooch of shielding                         & 1,500 gp     \\
28  & Necklace of fireballs type I                & 1,650 gp     \\
29  & Dust of appearance                          & 1,800 gp     \\
30  & Hat of disguise                             & 1,800 gp     \\
31  & Pipes of sounding                           & 1,800 gp     \\
32  & Efficient quiver                            & 1,800 gp     \\
33  & Amulet of natural armor +1                  & 2,000 gp     \\
34  & Handy haversack                             & 2,000 gp     \\
35  & Horn of fog                                 & 2,000 gp     \\
36  & Elemental gem                               & 2,250 gp     \\
37  & Robe of bones                               & 2,400 gp     \\
38  & Sovereign glue                              & 2,400 gp     \\
39  & Bag of holding type I                       & 2,500 gp     \\
40  & Boots of elvenkind                          & 2,500 gp     \\
41  & Boots of the winterlands                    & 2,500 gp     \\
42  & Candle of truth                             & 2,500 gp     \\
43  & Cloak of elvenkind                          & 2,500 gp     \\
44  & Eyes of the eagle                           & 2,500 gp     \\
45  & Goggles of minute seeing                    & 2,500 gp     \\
46  & Scarab, golembane                           & 2,500 gp     \\
47  & Necklace of fireballs type II               & 2,700 gp     \\
48  & Stone of alarm                              & 2,700 gp     \\
49  & Bead of force                               & 3,000 gp     \\
50  & Chime of opening                            & 3,000 gp     \\
51  & Horseshoes of speed                         & 3,000 gp     \\
52  & Rope of climbing                            & 3,000 gp     \\
53  & Bag of tricks, gray                         & 3,400 gp     \\
54  & Dust of disappearance                       & 3,500 gp     \\
55  & Lens of detection                           & 3,500 gp     \\
56  & Vestment, druid's                           & 3,750 gp     \\
57  & Figurine of wondrous power, silver raven    & 3,800 gp     \\
58  & Amulet of mighty fists +1                   & 4,000 gp     \\
59  & Belt of giant strength +2                   & 4,000 gp     \\
60  & Belt of incredible dexterity +2             & 4,000 gp     \\
61  & Belt of mighty constitution +2              & 4,000 gp     \\
62  & Bracers of armor +2                         & 4,000 gp     \\
63  & Cloak of resistance +2                      & 4,000 gp     \\
64  & Gloves of arrow snaring                     & 4,000 gp     \\
65  & Headband of alluring charisma +2            & 4,000 gp     \\
66  & Headband of inspired wisdom +2              & 4,000 gp     \\
67  & Headband of vast intelligence +2            & 4,000 gp     \\
68  & Ioun stone, clear spindle                   & 4,000 gp     \\
69  & Restorative ointment                        & 4,000 gp     \\
70  & Marvelous pigments                          & 4,000 gp     \\
71  & Pearl of power, 2nd-level spell             & 4,000 gp     \\
72  & Stone salve                                 & 4,000 gp     \\
73  & Necklace of fireballs type III              & 4,350 gp     \\
74  & Circlet of persuasion                       & 4,500 gp     \\
75  & Slippers of spider climbing                 & 4,800 gp     \\
76  & Incense of meditation                       & 4,900 gp     \\
77  & Bag of holding type II                      & 5,000 gp     \\
78  & Bracers of archery, lesser                  & 5,000 gp     \\
79  & Ioun stone, dusty rose prism                & 5,000 gp     \\
80  & Helm of comprehend languages and read magic & 5,200 gp     \\
81  & Vest of escape                              & 5,200 gp     \\
82  & Eversmoking bottle                          & 5,400 gp     \\
83  & Sustaining spoon                            & 5,400 gp     \\
84  & Necklace of fireballs type IV               & 5,400 gp     \\
85  & Boots of striding and springing             & 5,500 gp     \\
86  & Wind fan                                    & 5,500 gp     \\
87  & Necklace of fireballs type V                & 5,850 gp     \\
88  & Horseshoes of a zephyr                      & 6,000 gp     \\
89  & Pipes of haunting                           & 6,000 gp     \\
90  & Gloves of swimming and climbing             & 6,250 gp     \\
91  & Crown of blasting, minor                    & 6,480 gp     \\
92  & Horn of goodness/evil                       & 6,500 gp     \\
93  & Robe of useful items                        & 7,000 gp     \\
94  & Boat, folding                               & 7,200 gp     \\
95  & Cloak of the manta ray                      & 7,200 gp     \\
96  & Bottle of air                               & 7,250 gp     \\
97  & Bag of holding type III                     & 7,400 gp     \\
98  & Periapt of health                           & 7,400 gp     \\
99  & Boots of levitation                         & 7,500 gp     \\
100 & Harp of charming                            & 7,500 gp    
\end{xtabular}

\captionof{table}{Medium Wondrous Items}
\setlength{\tabcolsep}{1pt}
\begin{xtabular}{lll}
d\% & Item                                        & Market Price \\
01  & Amulet of natural armor +2                  & 8,000 gp     \\
02  & Golem manual, flesh                         & 8,000 gp     \\
03  & Hand of glory                               & 8,000 gp     \\
04  & Ioun stone, deep red sphere                 & 8,000 gp     \\
05  & Ioun stone, incandescent blue sphere        & 8,000 gp     \\
06  & Ioun stone, pale blue rhomboid              & 8,000 gp     \\
07  & Ioun stone, pink and green sphere           & 8,000 gp     \\
08  & Ioun stone, pink rhomboid                   & 8,000 gp     \\
09  & Ioun stone, scarlet and blue sphere         & 8,000 gp     \\
10  & Deck of illusions                           & 8,100 gp     \\
11  & Necklace of fireballs type VI               & 8,100 gp     \\
12  & Candle of invocation                        & 8,400 gp     \\
13  & Robe of blending                            & 8,400 gp     \\
14  & Bag of tricks, rust                         & 8,500 gp     \\
15  & Necklace of fireballs type VII              & 8,700 gp     \\
16  & Bracers of armor +3                         & 9,000 gp     \\
17  & Cloak of resistance +3                      & 9,000 gp     \\
18  & Decanter of endless water                   & 9,000 gp     \\
19  & Necklace of adaptation                      & 9,000 gp     \\
20  & Pearl of power, 3rd-level spell             & 9,000 gp     \\
21  & Figurine of wondrous power, serpentine owl  & 9,100 gp     \\
22  & Strand of prayer beads, lesser              & 9,600 gp     \\
23  & Bag of holding type IV                      & 10,000 gp    \\
24  & Belt of physical might +2                   & 10,000 gp    \\
25  & Figurine of wondrous power, bronze griffon  & 10,000 gp    \\
26  & Figurine of wondrous power, ebony fly       & 10,000 gp    \\
27  & Glove of storing                            & 10,000 gp    \\
28  & Headband of mental prowess +2               & 10,000 gp    \\
29  & Ioun stone, dark blue rhomboid              & 10,000 gp    \\
30  & Cape of the mountebank                      & 10,800 gp    \\
31  & Phylactery of negative channeling           & 11,000 gp    \\
32  & Phylactery of positive channeling           & 11,000 gp    \\
33  & Gauntlet of rust                            & 11,500 gp    \\
34  & Boots of speed                              & 12,000 gp    \\
35  & Goggles of night                            & 12,000 gp    \\
36  & Golem manual, clay                          & 12,000 gp    \\
37  & Medallion of thoughts                       & 12,000 gp    \\
38  & Blessed book                                & 12,500 gp    \\
39  & Gem of brightness                           & 13,000 gp    \\
40  & Lyre of building                            & 13,000 gp    \\
41  & Robe, Monk's                                & 13,000 gp    \\
42  & Cloak of arachnida                          & 14,000 gp    \\
43  & Belt of dwarvenkind                         & 14,900 gp    \\
44  & Periapt of wound closure                    & 15,000 gp    \\
45  & Pearl of the sirines                        & 15,300 gp    \\
46  & Figurine of wondrous power, onyx dog        & 15,500 gp    \\
47  & Amulet of mighty fists +2                   & 16,000 gp    \\
48  & Bag of tricks, tan                          & 16,000 gp    \\
49  & Belt of giant strength +4                   & 16,000 gp    \\
50  & Belt of incredible dexterity +4             & 16,000 gp    \\
51  & Belt of mighty constitution +4              & 16,000 gp    \\
52  & Belt of physical perfection +2              & 16,000 gp    \\
53  & Boots, winged                               & 16,000 gp    \\
54  & Bracers of armor +4                         & 16,000 gp    \\
55  & Cloak of resistance +4                      & 16,000 gp    \\
56  & Headband of alluring charisma +4            & 16,000 gp    \\
57  & Headband of inspired wisdom +4              & 16,000 gp    \\
58  & Headband of mental superiority +2           & 16,000 gp    \\
59  & Headband of vast intelligence +4            & 16,000 gp    \\
60  & Pearl of power, 4th-level spell             & 16,000 gp    \\
61  & Scabbard of keen edges                      & 16,000 gp    \\
62  & Figurine of wondrous power, golden lions    & 16,500 gp    \\
63  & Chime of interruption                       & 16,800 gp    \\
64  & Broom of flying                             & 17,000 gp    \\
65  & Figurine of wondrous power, marble elephant & 17,000 gp    \\
66  & Amulet of natural armor +3                  & 18,000 gp    \\
67  & Ioun stone, iridescent spindle              & 18,000 gp    \\
68  & Bracelet of friends                         & 19,000 gp    \\
69  & Carpet of flying, 5 ft. by 5 ft.            & 20,000 gp    \\
70  & Horn of blasting                            & 20,000 gp    \\
71  & Ioun stone, pale lavender ellipsoid         & 20,000 gp    \\
72  & Ioun stone, pearly white spindle            & 20,000 gp    \\
73  & Portable hole                               & 20,000 gp    \\
74  & Stone of good luck (luckstone)              & 20,000 gp    \\
75  & Figurine of wondrous power, ivory goats     & 21,000 gp    \\
76  & Rope of entanglement                        & 21,000 gp    \\
77  & Golem manual, stone                         & 22,000 gp    \\
78  & Mask of the skull                           & 22,000 gp    \\
79  & Mattock of the titans                       & 23,348 gp    \\
80  & Crown of blasting, major                    & 23,760 gp    \\
81  & Cloak of displacement, minor                & 24,000 gp    \\
82  & Helm of underwater action                   & 24,000 gp    \\
83  & Bracers of archery, greater                 & 25,000 gp    \\
84  & Bracers of armor +5                         & 25,000 gp    \\
85  & Cloak of resistance +5                      & 25,000 gp    \\
86  & Eyes of doom                                & 25,000 gp    \\
87  & Pearl of power, 5th-level spell             & 25,000 gp    \\
88  & Maul of the titans                          & 25,305 gp    \\
89  & Cloak of the bat                            & 26,000 gp    \\
90  & Iron bands of binding                       & 26,000 gp    \\
91  & Cube of frost resistance                    & 27,000 gp    \\
92  & Helm of telepathy                           & 27,000 gp    \\
93  & Periapt of proof against poison             & 27,000 gp    \\
94  & Robe of scintillating colors                & 27,000 gp    \\
95  & Manual of bodily health +1                  & 27,500 gp    \\
96  & Manual of gainful exercise +1               & 27,500 gp    \\
97  & Manual of quickness in action +1            & 27,500 gp    \\
98  & Tome of clear thought +1                    & 27,500 gp    \\
99  & Tome of leadership and influence +1         & 27,500 gp    \\
100 & Tome of understanding +1                    & 27,500 gp   
\end{xtabular}

\captionof{table}{Major Wondrous Items}
\setlength{\tabcolsep}{1pt}
\begin{xtabular}{lll}
d\% & Item                                           & Market Price \\
01  & Dimensional shackles                           & 28,000 gp    \\
02  & Figurine of wondrous power, obsidian steed     & 28,500 gp    \\
03  & Drums of panic                                 & 30,000 gp    \\
04  & Ioun stone, orange prism                       & 30,000 gp    \\
05  & Ioun stone, pale green prism                   & 30,000 gp    \\
06  & Lantern of revealing                           & 30,000 gp    \\
07  & Amulet of natural armor +4                     & 32,000 gp    \\
08  & Amulet of proof against detection and location & 35,000 gp    \\
09  & Carpet of flying, 5 ft. by 10 ft.              & 35,000 gp    \\
10  & Golem manual, iron                             & 35,000 gp    \\
11  & Amulet of mighty fists +3                      & 36,000 gp    \\
12  & Belt of giant strength +6                      & 36,000 gp    \\
13  & Belt of incredible dexterity +6                & 36,000 gp    \\
14  & Belt of mighty constitution +6                 & 36,000 gp    \\
15  & Bracers of armor +6                            & 36,000 gp    \\
16  & Headband of alluring charisma +6               & 36,000 gp    \\
17  & Headband of inspired wisdom +6                 & 36,000 gp    \\
18  & Headband of vast intelligence +6               & 36,000 gp    \\
19  & Ioun stone, vibrant purple prism               & 36,000 gp    \\
20  & Pearl of power, 6th-level spell                & 36,000 gp    \\
21  & Scarab of protection                           & 38,000 gp    \\
22  & Belt of physical might +4                      & 40,000 gp    \\
23  & Headband of mental prowess +4                  & 40,000 gp    \\
24  & Ioun stone, lavender and green ellipsoid       & 40,000 gp    \\
25  & Ring gates                                     & 40,000 gp    \\
26  & Crystal ball                                   & 42,000 gp    \\
27  & Golem manual, stone guardian                   & 44,000 gp    \\
28  & Strand of prayer beads                         & 45,800 gp    \\
29  & Orb of storms                                  & 48,000 gp    \\
30  & Boots of teleportation                         & 49,000 gp    \\
31  & Bracers of armor +7                            & 49,000 gp    \\
32  & Pearl of power, 7th-level spell                & 49,000 gp    \\
33  & Amulet of natural armor +5                     & 50,000 gp    \\
34  & Cloak of displacement, major                   & 50,000 gp    \\
35  & Crystal ball with see invisibility             & 50,000 gp    \\
36  & Horn of Valhalla                               & 50,000 gp    \\
37  & Crystal ball with detect thoughts              & 51,000 gp    \\
38  & Wings of flying                                & 54,000 gp    \\
39  & Cloak of etherealness                          & 55,000 gp    \\
40  & Instant fortress                               & 55,000 gp    \\
41  & Manual of bodily health +2                     & 55,000 gp    \\
42  & Manual of gainful exercise +2                  & 55,000 gp    \\
43  & Manual of quickness in action +2               & 55,000 gp    \\
44  & Tome of clear thought +2                       & 55,000 gp    \\
45  & Tome of leadership and influence +2            & 55,000 gp    \\
46  & Tome of understanding +2                       & 55,000 gp    \\
47  & Eyes of charming                               & 56,000 gp    \\
48  & Robe of stars                                  & 58,000 gp    \\
49  & Carpet of flying, 10 ft. by 10 ft.             & 60,000 gp    \\
50  & Darkskull                                      & 60,000 gp    \\
51  & Cube of force                                  & 62,000 gp    \\
52  & Amulet of mighty fists +4                      & 64,000 gp    \\
53  & Belt of physical perfection +4                 & 64,000 gp    \\
54  & Bracers of armor +8                            & 64,000 gp    \\
55  & Headband of mental superiority +4              & 64,000 gp    \\
56  & Pearl of power, 8th-level spell                & 64,000 gp    \\
57  & Crystal ball with telepathy                    & 70,000 gp    \\
58  & Horn of blasting, greater                      & 70,000 gp    \\
59  & Pearl of power, two spells                     & 70,000 gp    \\
60  & Helm of teleportation                          & 73,500 gp    \\
61  & Gem of seeing                                  & 75,000 gp    \\
62  & Robe of the archmagi                           & 75,000 gp    \\
63  & Mantle of faith                                & 76,000 gp    \\
64  & Crystal ball with true seeing                  & 80,000 gp    \\
65  & Pearl of power, 9th-level spell                & 81,000 gp    \\
66  & Well of many worlds                            & 82,000 gp    \\
67  & Manual of bodily health +3                     & 82,500 gp    \\
68  & Manual of gainful exercise +3                  & 82,500 gp    \\
69  & Manual of quickness in action +3               & 82,500 gp    \\
70  & Tome of clear thought +3                       & 82,500 gp    \\
71  & Tome of leadership and influence +3            & 82,500 gp    \\
72  & Tome of understanding +3                       & 82,500 gp    \\
73  & Apparatus of the crab                          & 90,000 gp    \\
74  & Belt of physical might +6                      & 90,000 gp    \\
75  & Headband of mental prowess +6                  & 90,000 gp    \\
76  & Mantle of spell resistance                     & 90,000 gp    \\
77  & Mirror of opposition                           & 92,000 gp    \\
78  & Strand of prayer beads, greater                & 95,800 gp    \\
79  & Amulet of mighty fists +5                      & 100,000 gp   \\
80  & Manual of bodily health +4                     & 110,000 gp   \\
81  & Manual of gainful exercise +4                  & 110,000 gp   \\
82  & Manual of quickness in action +4               & 110,000 gp   \\
83  & Tome of clear thought +4                       & 110,000 gp   \\
84  & Tome of leadership and influence +4            & 110,000 gp   \\
85  & Tome of understanding +4                       & 110,000 gp   \\
86  & Amulet of the planes                           & 120,000 gp   \\
87  & Robe of eyes                                   & 120,000 gp   \\
88  & Helm of Brilliance                             & 125,000 gp   \\
89  & Manual of bodily health +5                     & 137,500 gp   \\
90  & Manual of gainful exercise +5                  & 137,500 gp   \\
91  & Manual of quickness in action +5               & 137,500 gp   \\
92  & Tome of clear thought +5                       & 137,500 gp   \\
93  & Tome of leadership and influence +5            & 137,500 gp   \\
94  & Tome of understanding +5                       & 137,500 gp   \\
95  & Belt of physical perfection +6                 & 144,000 gp   \\
96  & Headband of mental superiority +6              & 144,000 gp   \\
97  & Efreeti bottle                                 & 145,000 gp   \\
98  & Cubic gate                                     & 164,000 gp   \\
99  & Iron flask                                     & 170,000 gp   \\
100 & Mirror of life trapping                        & 200,000 gp  
\end{xtabular}

\label{f0}				
This is a catch all category for anything that doesn't fall into the other groups. Anyone can use a wondrous item (unless specified otherwise in the description).
				
\textbf{Physical Description}: Varies.
				
\textbf{Activation}: Usually use-activated or command word, but details vary from item to item.
				
\textbf{Special Qualities}: Roll d\%. A 01 result indicates the wondrous item is intelligent, 02--31 indicates that something (a design, inscription, or the like) provides a clue to its function, and 32--100 indicates no special qualities. Intelligent items have extra abilities and sometimes extraordinary powers and special purposes (see Intelligent Items).
				
Wondrous items with charges can never be intelligent.
				
\subsection{Extradimensional Spaces}

			
A number of spells and magic items utilize extradimensional spaces, such as \textit{rope trick}, a \textit{bag of holding}, a \textit{handy haversack}, and a \textit{portable hole}. These spells and magic items create a tiny pocket space that does not exist in any dimension. Such items do not function, however, inside another extradimensional space. If placed inside such a space, they cease to function until removed from the extradimensional space. For example, if a \textit{bag of holding} is brought into a\textit{ rope trick}, the contents of the \textit{bag of holding} become inaccessible until the \textit{bag of holding} is taken outside the \textit{rope trick}. The only exception to this is when a\textit{ bag of holding} and a \textit{portable hole} interact, forming a rift to the Astral Plane, as noted in their descriptions.
				
\textbf{Amulet of Mighty Fists}
				
\textbf{Aura} faint evocation;\textbf{ CL }5th
				
\textbf{Slot} neck; \textbf{Price} 4,000 gp (+1), 16,000 gp (+2), 36,000 gp (+3), 64,000 (+4), 100,000 gp (+5); \textbf{Weight} ---
				
Description
				
This amulet grants an enhancement bonus of +1 to +5 on attack and damage rolls with unarmed attacks and natural weapons. 
				
Alternatively, this amulet can grant melee weapon special abilities, so long as they can be applied to unarmed attacks. See Table: Melee Weapon Special Abilities for a list of abilities. Special abilities count as additional bonuses for determining the market value of the item, but do not modify attack or damage bonuses. An \textit{amulet of mighty fists} cannot have a modified bonus (enhancement bonus plus special ability bonus equivalents) higher than +5. An \textit{amulet of mighty fists} does not need to have a +1 enhancement bonus to grant a melee weapon special ability.
				
Construction
				
\textbf{Requirements} Craft Wondrous Item, \textit{greater magic fang}, creator's caster level must be at least three times the amulet's bonus, plus any requirements of the melee weapon special abilities; \textbf{Cost} 2,000 gp (+1), 8,000 gp (+2), 18,000 gp (+3), 32,000 (+4), 50,000 (+5)
				
\textbf{Amulet of Natural Armor}
				
\textbf{Aura} faint transmutation;\textbf{ CL }5th
				
\textbf{Slot} neck; \textbf{Price} 2,000 gp (+1), 8,000 gp (+2), 18,000 gp (+3), 32,000 gp (+4), or 50,000 gp (+5); \textbf{Weight} ---
				
Description
				
This amulet, usually crafted from bone or beast scales, toughens the wearer's body and flesh, giving him an enhancement bonus to his natural armor from +1 to +5, depending on the kind of amulet. 
				
Construction
				
\textbf{Requirements} Craft Wondrous Item, \textit{barkskin,} creator's caster level must be at least three times the amulet's bonus; \textbf{Cost }1,000 gp (+1), 4,000 gp (+2), 9,000 gp (+3), 16,000 gp (+4), 25,000 gp (+5)
				
\textbf{Amulet of the Planes}
				
\textbf{Aura} strong conjuration;\textbf{ CL }15th
				
\textbf{Slot} neck; \textbf{Price} 120,000 gp; \textbf{Weight} ---
				
Description
				
This device usually appears to be a black circular amulet, although any character looking closely at it sees a dark, moving swirl of color. The amulet allows its wearer to utilize \textit{plane shift}. However, this is a difficult item to master. The user must make a DC 15 Intelligence check in order to get the amulet to take her to the plane (and the specific location on that plane) that she wants. If she fails, the amulet transports her and all those traveling with her to a random location on that plane (01--60 on d\%) or to a random plane (61--100). 
				
Construction
				
\textbf{Requirements} Craft Wondrous Item, \textit{plane shift}; \textbf{Cost }60,000 gp
				
\textbf{Amulet of Proof against Detection and Location}
				
\textbf{Aura} moderate abjuration;\textbf{ CL }8th
				
\textbf{Slot} neck; \textbf{Price} 35,000 gp; \textbf{Weight} ---
				
Description
				
This silver amulet protects the wearer from scrying and magical location just as a \textit{nondetection }spell does. If a divination spell is attempted against the wearer, the caster of the divination must succeed on a caster level check (1d20 + caster level) against a DC of 19 (as if the wearer had cast \textit{nondetection }on herself). 
				
Construction
				
\textbf{Requirements} Craft Wondrous Item, \textit{nondetection}; \textbf{Cost }17,500 gp
				
\textbf{Apparatus of the Crab}
				
\textbf{Aura} strong evocation and transmutation; \textbf{CL} 19th
				
\textbf{Slot} none; \textbf{Price} 90,000 gp; \textbf{Weight} 500 lbs.
				
Description
				
At first glance, an inactive \textit{apparatus of the crab} appears to be a large, sealed iron barrel big enough to hold two Medium creatures. Close examination, and a DC 20 Perception check, reveals a secret catch that opens a hatch at one end. Anyone who crawls inside finds 10 (unlabeled) levers and seating for two Medium or Small occupants. These levers allow those inside to activate and control the apparatus's movements and actions.

\begin{tabular}{ll}
\textbf{Lever (1d10)} & \textbf{Lever Function}\\
1 & Extend/retract legs and tail\\
2 & Uncover/cover forward porthole\\
3 & Uncover/cover side portholes\\
4 & Extend/retract pincers and feelers\\
5 & Snap pincers\\
6 & Move forward/backward\\
7 & Turn left/right\\
8 & Open/close \texttt{{}"{}}eyes\texttt{{}"{}} with \textit{continual flame }inside\\
9 & Rise/sink in water\\
10 & Open/close hatch\\
\end{tabular}

				
Operating a lever is a full-round action, and no lever may be operated more than once per round. However, since two characters can fit inside, the apparatus can move and attack in the same round. The device can function in water up to 900 feet deep. It holds enough air for a crew of two to survive 1d4+1 hours (twice as long for a single occupant). When activated, the apparatus looks something like a giant lobster.
				
When active, an \textit{apparatus of the crab} has the following characteristics: \textbf{hp} 200; \textbf{hardness} 15; \textbf{Spd} 20 ft., swim 20 ft.; \textbf{AC} 20 (--1 size, +11 natural); \textbf{Attack} 2 pincers +12 melee (2d8); \textbf{CMB }+14; \textbf{CMD} 24.
				
Construction
				
\textbf{Requirements} Craft Wondrous Item, \textit{animate objects, continual flame, }creator must have 8 ranks in Knowledge (engineering); \textbf{Cost }45,000 gp
				
\textbf{Bag of Holding}
				
\textbf{Aura} moderate conjuration;\textbf{ CL }9th
				
\textbf{Slot} none; \textbf{Price} see below; \textbf{Weight }see below
				
Description
				
This appears to be a common cloth sack about 2 feet by 4 feet in size. The \textit{bag of holding }opens into a nondimensional space: its inside is larger than its outside dimensions. Regardless of what is put into the bag, it weighs a fixed amount. This weight, and the limits in weight and volume of the bag's contents, depend on the bag's type, as shown on the table below.
				
% <div class="table">
\setlength{\tabcolsep}{1pt}
\begin{tabular}{lllll}
             & \textbf{Bag}    & \textbf{Contents}& \textbf{Contents}     & \textbf{Market} \\
\textbf{Bag} & \textbf{Weight} & \textbf{Limit}   & \textbf{Volume Limit} & \textbf{Price}\\
Type I & 15 lbs. & 250 lbs. & 30 cubic ft. & 2,500 gp\\
Type II & 25 lbs. & 500 lbs. & 70 cubic ft. & 5,000 gp\\
Type III & 35 lbs. & 1,000 lbs. & 150 cubic ft. & 7,400 gp\\
Type IV & 60 lbs. & 1,500 lbs. & 250 cubic ft. & 10,000 gp\\
\end{tabular}
				
If a \textit{bag of holding} is overloaded, or if sharp objects pierce it (from inside or outside), the bag immediately ruptures and is ruined, and all contents are lost forever. If a \textit{bag of holding }is turned inside out, all of its contents spill out, unharmed, but the bag must be put right before it can be used again. If living creatures are placed within the bag, they can survive for up to 10 minutes, after which time they suffocate. Retrieving a specific item from a \textit{bag of holding }is a move action, unless the bag contains more than an ordinary backpack would hold, in which case retrieving a specific item is a full-round action. Magic items placed inside the bag do not offer any benefit to the character carrying the bag.
				
If a \textit{bag of holding }is placed within a \textit{portable hole,} a rift to the Astral Plane is torn in the space: bag and hole alike are sucked into the void and forever lost. If a \textit{portable hole }is placed within a \textit{bag of holding, }it opens a gate to the Astral Plane: the hole, the bag, and any creatures within a 10-foot radius are drawn there, destroying the \textit{portable hole }and \textit{bag of holding }in the process. 
				
Construction
				
\textbf{Requirements} Craft Wondrous Item, \textit{secret chest}; \textbf{Cost }1,250 gp (type I), 2,500 gp (type II), 3,700 gp (type III), 5,000 gp (type IV)
				
\textbf{Bag of Tricks}
				
\textbf{Aura} faint (gray or rust) or moderate (tan) conjuration; \textbf{CL} 3rd (gray), 5th (rust), 9th (tan)
				
\textbf{Slot} none; \textbf{Price} 3,400 gp (gray); 8,500 gp (rust); 16,000 gp (tan)
				
Description
				
This small sack appears empty. Anyone reaching into the bag feels a small, fuzzy ball. If the ball is removed and tossed up to 20 feet away, it turns into an animal. The animal serves the character who drew it from the bag for 10 minutes (or until slain or ordered back into the bag), at which point it disappears. It can follow any of the commands described in the Handle Animal skill. Each of the three kinds of \textit{bags of tricks }produces a different set of animals. Use the following tables to determine what animals can be drawn out of each.
				
\setlength{\tabcolsep}{1pt}
\begin{tabular}{cccccc}
\multicolumn{2}{c}{\textbf{Gray Bag}} & \multicolumn{2}{c}{\textbf{Rust Bag}} & \multicolumn{2}{c}{\textbf{Tan Bag}}\\
\textbf{d\%} & \textbf{Animal} & \textbf{d\%} & \textbf{Animal} & \textbf{d\%} & \textbf{Animal}\\
 01--30 & Bat & 01--30 & Wolverine & 01--30 & Grizzly bear\\
 31--60 & Rat & 31--60 & Wolf & 31--60 & Lion\\
 61--75 & Cat & 61--85 & Boar & 61--80 & Heavy horse\\
 76--90 & Weasel & 86--100 & Leopard & 81--90 & Tiger\\
 91--100 & Riding dog & & --- & 91--100 & Rhinoceros\\
\end{tabular}
				
The heavy horse appears with harness and tack and accepts the character who drew it from the bag as a rider.
				
Animals produced are always random, and only one may exist at a time. Up to 10 animals can be drawn from the bag each week, but no more than two per day.
				
Construction
				
\textbf{Requirements }Craft Wondrous Item, \textit{summon nature's ally II }(gray), \textit{summon nature's ally III }(rust), or \textit{summon nature's ally V} (tan); \textbf{Cost} 1,700 gp (gray); 4,250 gp (rust); 8,000 gp (tan)
				
\textbf{Bead of Force}
				
\textbf{Aura} moderate evocation;\textbf{ CL }10th
				
\textbf{Slot} none; \textbf{Price} 3,000 gp; \textbf{Weight} ---
				
Description
				
This small black sphere appears to be a lusterless pearl. A \textit{bead of force} can be thrown up to 60 feet with no range penalties. Upon sharp impact, the bead explodes, sending forth a burst that deals 5d6 points of force damage to all creatures within a 10-foot radius.
				
Once thrown, a \textit{bead of force} functions like a\textit{ resilient sphere }spell (Reflex DC 16 negates) with a radius of 10 feet and a duration of 10 minutes. A globe of shimmering force encloses a creature, provided the latter is small enough to fit within the diameter of the sphere. The sphere contains its subject for the spell's duration. The sphere is not subject to damage of any sort except from a \textit{rod of cancellation, }a \textit{rod of negation, disintegrate, }or a targeted \textit{dispel magic }spell. These effects destroy the sphere without harm to the subject. Nothing can pass through the sphere, inside or out, though the subject can breathe normally. The subject may struggle, but the globe cannot be physically moved either by people outside it or by the struggles of those within. The explosion completely consumes the bead, making this a one-use item. 
				
Construction
				
\textbf{Requirements} Craft Wondrous Item, \textit{resilient sphere}; \textbf{Cost }1,500 gp
				
\textbf{Belt of Dwarvenkind}
				
\textbf{Aura} strong divination;\textbf{ CL }12th
				
\textbf{Slot} belt; \textbf{Price} 14,900 gp; \textbf{Weight} 1 lb.
				
Description
				
This belt gives the wearer a +4 competence bonus on Charisma checks and Charisma-based skill checks as they relate to dealing with dwarves, a +2 competence bonus on similar checks when dealing with gnomes and halflings, and a --2 competence penalty on similar checks when dealing with anyone else. The wearer can understand, speak, and read Dwarven. If the wearer is not a dwarf, he gains 60-foot darkvision, dwarven stonecunning, a +2 enhancement bonus to Constitution, and a +2 resistance bonus on saves against poison, spells, and spell-like effects. 
				
Construction
				
\textbf{Requirements} Craft Wondrous Item, \textit{tongues}, creator must be a dwarf; \textbf{Cost }7,450 gp
				
\textbf{Belt of Giant Strength}
				
\textbf{Aura} moderate transmutation; \textbf{CL} 8th
				
\textbf{Slot }belt; \textbf{Weight} 1 lb.; \textbf{Price} 4,000 gp (+2), 16,000 gp (+4), 36,000 gp (+6)
				
Description
				
This belt is a thick leather affair, often decorated with huge metal buckles. The belt grants the wearer an enhancement bonus to Strength of +2, +4, or +6. Treat this as a temporary ability bonus for the first 24 hours the belt is worn.
				
Construction
				
\textbf{Requirements }Craft Wondrous Item, \textit{bull's strength}; \textbf{Cost} 2,000 gp (+2), 8,000 gp (+4), 18,000 gp (+6)
				
\textbf{Belt of Incredible Dexterity}
				
\textbf{Aura} moderate transmutation; \textbf{CL} 8th
				
\textbf{Slot }belt; \textbf{Weight} 1 lb.; \textbf{Price} 4,000 gp (+2), 16,000 gp (+4), 36,000 gp (+6)
				
Description
				
This belt has a large silver buckle, usually depicting the image of a tiger. The belt grants the wearer an enhancement bonus to Dexterity of +2, +4, or +6. Treat this as a temporary ability bonus for the first 24 hours the belt is worn.
				
Construction
				
\textbf{Requirements }Craft Wondrous Item, \textit{cat's grace}; \textbf{Cost} 2,000 gp (+2), 8,000 gp (+4), 18,000 gp (+6)
				
\textbf{Belt of Mighty Constitution}
				
\textbf{Aura} moderate transmutation; \textbf{CL} 8th
				
\textbf{Slot }belt; \textbf{Weight} 1 lb.; \textbf{Price} 4,000 gp (+2), 16,000 gp (+4), 36,000 gp (+6)
				
Description
				
This belt's golden buckle depicts a bear. The belt grants the wearer an enhancement bonus to Constitution of +2, +4, or +6.Treat this as a temporary ability bonus for the first 24 hours the belt is worn.
				
Construction
				
\textbf{Requirements }Craft Wondrous Item, \textit{bear's endurance}; \textbf{Cost} 2,000 gp (+2), 8,000 gp (+4), 18,000 gp (+6)
				
\textbf{Belt of Physical Might}
				
\textbf{Aura} strong transmutation; \textbf{CL} 12th
				
\textbf{Slot }belt; \textbf{Weight} 1 lb.; \textbf{Price} 10,000 gp (+2), 40,000 gp (+4), 90,000 gp (+6)
				
Description
				
This belt has a large steel buckle, usually depicting the image of a giant. The belt grants the wearer an enhancement bonus to two physical ability scores (Strength, Dexterity, or Constitution) of +2, +4, or +6. Treat this as a temporary ability bonus for the first 24 hours the belt is worn. These bonuses are chosen when the belt is created and cannot be changed.
				
Construction
				
\textbf{Requirements }Craft Wondrous Item, \textit{bear's endurance, bull's strength, }and/or\textit{ cat's grace}; \textbf{Cost} 5,000 gp (+2), 20,000 gp (+4), 45,000 gp (+6)
				
\textbf{Belt of Physical Perfection}
				
\textbf{Aura} strong transmutation; \textbf{CL} 16th
				
\textbf{Slot }belt; \textbf{Weight} 1 lb.; \textbf{Price} 16,000 gp (+2), 64,000 gp (+4), 144,000 gp (+6)
				
Description
				
This belt has a large platinum buckle, usually depicting the image of a titan. The belt grants the wearer an enhancement bonus to all physical ability scores (Strength, Dexterity, and Constitution) of +2, +4, or +6. Treat this as a temporary ability bonus for the first 24 hours the belt is worn.
				
Construction
				
\textbf{Requirements }Craft Wondrous Item, \textit{bear's endurance, bull's strength, cat's grace}; \textbf{Cost} 8,000 gp (+2), 32,000 gp (+4), 77,000 gp (+6)
				
\textbf{Blessed Book}
				
\textbf{Aura} moderate transmutation;\textbf{ CL }7th
				
\textbf{Slot} none; \textbf{Price} 12,500 gp; \textbf{Weight} 1 lb.
				
Description
				
This well-made tome is always of small size, typically no more than 12 inches tall, 8 inches wide, and 1 inch thick. All such books are durable, waterproof, bound with iron overlaid with silver, and locked.
				
A wizard can fill the 1,000 pages of a \textit{blessed book} with spells without paying the material cost. This book is never found as randomly generated treasure with spells already inscribed in it. 
				
Construction
				
\textbf{Requirements} Craft Wondrous Item, \textit{secret page}; \textbf{Cost }6,250 gp
				
\textbf{Boat, Folding}
				
\textbf{Aura} moderate transmutation;\textbf{ CL }6th
				
\textbf{Slot} none; \textbf{Price} 7,200 gp; \textbf{Weight} 4 lbs.
				
Description
				
A \textit{folding boat} looks like a small wooden box about 12 inches long, 6 inches wide, and 6 inches deep when it is inactive. In this mode, it can be used to store items just like any other box. Yet when the proper command word is given, the box unfolds itself rapidly in the space of a single round to form a boat 10 feet long, 4 feet wide, and 2 feet in depth. A second command word causes it to unfold even further into a ship 24 feet long, 8 feet wide, and 6 feet deep. The \textit{folding boat} cannot unfold if there isn't enough open space for it to occupy once unfolded. Any objects formerly stored in the box now rest inside the boat or ship.
				
In its smaller form, the boat has one pair of oars, an anchor, a mast, and a lateen sail. In its larger form, the boat has a deck, single rowing seats, five sets of oars, a rudder, an anchor, a deck cabin, and a mast with a square sail. The boat can hold 4 people comfortably, while the ship carries 15 with ease.
				
A third word of command causes the boat or ship to fold itself into a box once again, but only when it is unoccupied. 
				
Construction
				
\textbf{Requirements} Craft Wondrous Item, \textit{fabricate}, creator must have 2 ranks in the Craft (ships) skill; \textbf{Cost }3,600 gp
				
\textbf{Boots of Elvenkind}
				
\textbf{Aura} faint transmutation;\textbf{ CL }5th
				
\textbf{Slot} feet; \textbf{Price} 2,500 gp; \textbf{Weight} 1 lb.
				
Description
				
These soft boots enable the wearer to move nimbly about in virtually any surroundings, granting a +5 competence bonus on Acrobatics checks. 
				
Construction
				
\textbf{Requirements} Craft Wondrous Item, creator must be an elf; \textbf{Cost }1,250 gp
				
\textbf{Boots of Levitation}
				
\textbf{Aura} faint transmutation;\textbf{ CL }3rd
				
\textbf{Slot} feet; \textbf{Price} 7,500 gp; \textbf{Weight} 1 lb.
				
Description
				
These soft leather boots are incredibly light and comfortable, with thin soles reinforced by strips of tough hide that provide an unexpected amount of support and protection to the foot. On command, these boots allow the wearer to levitate as if she had cast \textit{levitate }on herself. 
				
Construction
				
\textbf{Requirements} Craft Wondrous Item, \textit{levitate}; \textbf{Cost }3,750 gp
				
\textbf{Boots of Speed}
				
\textbf{Aura} moderate transmutation;\textbf{ CL }10th
				
\textbf{Slot} feet; \textbf{Price} 12,000 gp; \textbf{Weight} 1 lb.
				
Description
				
As a free action, the wearer can click her heels together, letting her act as though affected by a \textit{haste }spell for up to 10 rounds each day. The \textit{haste }effect's duration need not be consecutive rounds. 
				
Construction
				
\textbf{Requirements} Craft Wondrous Item, \textit{haste}; \textbf{Cost }6,000 gp
				
\textbf{Boots of Striding and Springing}
				
\textbf{Aura} faint transmutation;\textbf{ CL }3rd
				
\textbf{Slot} feet; \textbf{Price} 5,500 gp; \textbf{Weight} 1 lb.
				
Description
				
These boots increase the wearer's base land speed by 10 feet. In addition to this striding ability (considered an enhancement bonus), these boots allow the wearer to make great leaps. She can jump with a +5 competence bonus on Acrobatics checks. 
				
Construction
				
\textbf{Requirements} Craft Wondrous Item, \textit{longstrider}, creator must have 5 ranks in the Acrobatics skill; \textbf{Cost }2,750 gp
				
\textbf{Boots of Teleportation}
				
\textbf{Aura} moderate conjuration;\textbf{ CL }9th
				
\textbf{Slot} feet; \textbf{Price} 49,000 gp; \textbf{Weight} 3 lbs.
				
Description
				
Any character wearing this footwear may \textit{teleport }three times per day, exactly as if he had cast the spell of the same name. 
				
Construction
				
\textbf{Requirements} Craft Wondrous Item, \textit{teleport}; \textbf{Cost }24,500 gp
				
\textbf{Boots of the Winterlands}
				
\textbf{Aura} faint abjuration and transmutation;\textbf{ CL }5th
				
\textbf{Slot} feet; \textbf{Price} 2,500 gp; \textbf{Weight} 1 lb.
				
Description
				
This footgear bestows many powers upon the wearer. First, he is able to travel across snow at his normal speed, leaving no tracks. Second, the boots also enable him to travel at normal speed across the most slippery ice (horizontal surfaces only, not vertical or sharply slanted ones) without falling or slipping. Finally, \textit{boots of the winterlands }warm the wearer, as if he were affected by an \textit{endure elements }spell. 
				
Construction
				
\textbf{Requirements} Craft Wondrous Item\textit{, cat's grace, endure elements, pass without trace}; \textbf{Cost }1,250 gp
				
\textbf{Boots, Winged}
				
\textbf{Aura} moderate transmutation;\textbf{ CL }8th
				
\textbf{Slot} feet; \textbf{Price} 16,000 gp; \textbf{Weight} 1 lb.
				
Description
				
These boots appear to be ordinary footgear. On command, they sprout wings at the heel and let the wearer fly, without having to maintain concentration, as if affected by a \textit{fly }spell (including a +4 bonus on Fly skill checks). He can fly three per times day for up to 5 minutes per flight.
				
Construction
				
\textbf{Requirements} Craft Wondrous Item, \textit{fly}; \textbf{Cost }8,000 gp
				
\textbf{Bottle of Air}
				
\textbf{Aura} moderate transmutation;\textbf{ CL }7th
				
\textbf{Slot} none; \textbf{Price} 7,250 gp; \textbf{Weight} 2 lbs.
				
Description
				
This item appears to be a normal glass bottle with a cork. When taken to any airless environment, it retains air within it at all times, continually renewing its contents. This means that a character can draw air out of the bottle to breathe. The bottle can even be shared by multiple characters who pass it around. Breathing out of the bottle is a standard action, but a character so doing can then act for as long as she can hold her breath. 
				
Construction
				
\textbf{Requirements} Craft Wondrous Item, \textit{water breathing}; \textbf{Cost }3,625 gp
				
\textbf{Bracelet of Friends}
				
\textbf{Aura} strong conjuration;\textbf{ CL }15th
				
\textbf{Slot} wrists; \textbf{Price} 19,000 gp; \textbf{Weight }---
				
Description
				
This silver charm bracelet has four charms upon it when created. The owner may designate one person known to him to be keyed to each charm. (This designation takes a standard action, but once done it lasts forever or until changed.) When a charm is grasped and the name of the keyed individual is spoken, that person is called to the spot (a standard action) along with his gear, as long as the owner and the called person are on the same plane. The keyed individual knows who is calling, and the \textit{bracelet of friends }only functions on willing travelers. Once a charm is activated, it disappears. Charms separated from the bracelet are worthless. A bracelet found with fewer than four charms is worth 25\% less for each missing charm. 
				
Construction
				
\textbf{Requirements} Craft Wondrous Item, \textit{refuge}; \textbf{Cost }9,500 gp
				
\textbf{Bracers of Archery, Greater}
				
\textbf{Aura} moderate transmutation;\textbf{ CL }8th
				
\textbf{Slot} wrists; \textbf{Price} 25,000 gp; \textbf{Weight} 1 lb.
				
Description
				
These wristbands look like normal protective wear. The bracers empower the wearer to use any bow (not including crossbows) as if she were proficient in its use. If she already has proficiency with any type of bow, she gains a +2 competence bonus on attack rolls and a +1 competence bonus on damage rolls whenever using that type of bow. Both bracers must be worn for the magic to be effective. 
				
Construction
				
\textbf{Requirements} Craft Wondrous Item, Craft Magic Arms and Armor, crafter must be proficient with a longbow or shortbow; \textbf{Cost }12,500 gp
				
\textbf{Bracers of Archery, Lesser}
				
\textbf{Aura} faint transmutation;\textbf{ CL }4th
				
\textbf{Slot} wrists; \textbf{Price} 5,000 gp; \textbf{Weight} 1 lb.
				
Description
				
These wristbands function as \textit{greater bracers of archery}, except that they grant a +1 competence bonus on attack rolls and no bonus on damage rolls. 
				
Construction
				
\textbf{Requirements} Craft Wondrous Item, Craft Magic Arms and Armor, crafter must be proficient with a longbow or shortbow; \textbf{Cost }2,500 gp
				
\textbf{Bracers of Armor}
				
\textbf{Aura} moderate conjuration;\textbf{ CL }7th
				
\textbf{Slot} wrists; \textbf{Price} 1,000 gp (+1), 4,000 gp (+2), 9,000 gp (+3), 16,000 gp (+4), 25,000 gp (+5), 36,000 gp (+6), 49,000 gp (+7), 64,000 gp (+8); \textbf{Weight} 1 lb.
				
Description
				
These items appear to be wrist or arm guards. They surround the wearer with an invisible but tangible field of force, granting him an armor bonus of +1 to +8, just as though he were wearing armor. Both bracers must be worn for the magic to be effective. 
				
Alternatively, \textit{bracers of armor} can be enchanted with armor special abilities. See Table: Armor Special Qualities for a list of abilities. Special abilities usually count as additional bonuses for determining the market value of an item, but do not improve AC. \textit{Bracers of armor} cannot have a modified bonus (armor bonus plus armor special ability bonus equivalents) higher than +8. \textit{Bracers of armor} must have at least a +1 armor bonus to grant an armor special ability. \textit{Bracers of armor }cannot have any armor special abilities that add a flat gp amount to their cost. \textit{Bracers of armor} and ordinary armor do not stack. If a creature receives a larger armor bonus from another source, the \textit{bracers of armor} cease functioning and do not grant their armor bonus or their armor special abilities. If the \textit{bracers of armor} grant a larger armor bonus, the other source of armor ceases functioning.
				
Construction
				
\textbf{Requirements} Craft Wondrous Item, \textit{mage armor}, creator's caster level must be at least two times that of the bonus placed in the bracers, plus any requirements of the armor special abilities; \textbf{Cost }500 gp (+1), 2,000 gp (+2), 4,500 gp (+3), 8,000 gp (+4), 12,500 gp (+5), 18,000 gp (+6), 24,500 gp (+7), 32,000 gp (+8)
				
\textbf{Brooch of Shielding}
				
\textbf{Aura} faint abjuration;\textbf{ CL }1st
				
\textbf{Slot} neck; \textbf{Price} 1,500 gp; \textbf{Weight }---
				
Description
				
This appears to be a piece of silver or gold jewelry used to fasten a cloak or cape. In addition to this mundane task, it can absorb \textit{magic missiles }of the sort generated by the spell or spell-like ability. A brooch can absorb up to 101 points of damage from \textit{magic missiles }before it melts and becomes useless. 
				
Construction
				
\textbf{Requirements} Craft Wondrous Item, \textit{shield}; \textbf{Cost }750 gp
				
\textbf{Broom of Flying}
				
\textbf{Aura} moderate transmutation;\textbf{ CL }9th
				
\textbf{Slot} none; \textbf{Price} 17,000 gp; \textbf{Weight} 3 lbs.
				
Description
				
This broom is able to fly through the air as if affected by an\textit{ overland flight }spell (+4 on Fly skill checks) for up to 9 hours per day (split up as its owner desires). The broom can carry 200 pounds and fly at a speed of 40 feet, or up to 400 pounds at a speed at 30 feet. In addition, the broom can travel alone to any destination named by the owner as long as she has a good idea of the location and layout of that destination. It flies to its owner from as far away as 300 yards when she speaks the command word. The \textit{broom of flying }has a speed of 40 feet when it has no rider. 
				
Construction
				
\textbf{Requirements} Craft Wondrous Item, \textit{overland flight, permanency}; \textbf{Cost }8,500 gp
				
\textbf{Candle of Invocation}
				
\textbf{Aura} strong conjuration;\textbf{ CL }17th
				
\textbf{Slot} none; \textbf{Price} 8,400 gp; \textbf{Weight} 1/2 lb.
				
Description
				
Each of these special tapers is dedicated to one of the nine alignments. Simply burning the candle generates a favorable aura for the individual if the candle's alignment matches that of the character. Characters of the same alignment as the burning candle add a +2 morale bonus on attack rolls, saving throws, and skill checks while within 30 feet of the flame.
				
A cleric whose alignment matches the candle's operates as if two levels higher for purposes of determining spells per day if he burns the candle during or just prior to his spell preparation time. He can even cast spells normally unavailable to him as if he were of that higher level, but only so long as the candle continues to burn. Except in special cases (see below), a candle burns for 4 hours. It is possible to extinguish the candle simply by blowing it out, so users often place it in a lantern to protect it from drafts and the like. Doing this doesn't interfere with its magical properties.
				
In addition, burning a candle also allows the owner to cast a \textit{gate }spell, the respondent being of the same alignment as the candle, but the taper is immediately consumed in the process. 
				
Construction
				
\textbf{Requirements} Craft Wondrous Item, \textit{gate,} creator must be same alignment as candle created; \textbf{Cost }4,200 gp
				
\textbf{Candle of Truth}
				
\textbf{Aura} faint enchantment;\textbf{ CL }3rd
				
\textbf{Slot} none; \textbf{Price} 2,500 gp; \textbf{Weight} 1/2 lb.
				
Description
				
This white tallow candle, when burned, calls into place a \textit{zone of truth }spell (Will DC 13 negates) in a 5-foot radius centered on the candle. The zone lasts for 1 hour, while the candle burns. If the candle is snuffed before that time, the effect is canceled and the candle ruined. 
				
Construction
				
\textbf{Requirements} Craft Wondrous Item, \textit{zone of truth}; \textbf{Cost }1,250 gp
				
\textbf{Cape of the Mountebank}
				
\textbf{Aura} moderate conjuration;\textbf{ CL }9th
				
\textbf{Slot} shoulders; \textbf{Price} 10,800 gp; \textbf{Weight} 1 lb.
				
Description
				
On command, this bright red and gold cape allows the wearer to use the magic of the \textit{dimension door }spell once per day. When he disappears, he leaves behind a cloud of smoke, appearing in a similar fashion at his destination. 
				
Construction
				
\textbf{Requirements} Craft Wondrous Item, \textit{dimension door}; \textbf{Cost }5,400 gp
				
\textbf{Carpet of Flying}
				
\textbf{Aura} moderate transmutation;\textbf{ CL }10th
				
\textbf{Slot} none; \textbf{Price} varies; \textbf{Weight }---
				
Description
				
This rug is able to fly through the air as if affected by an \textit{overland flight }spell of unlimited duration. The size, carrying capacity, and speed of the different \textit{carpets of flying }are shown on the table below. Beautifully and intricately made, each carpet has its own command word to activate it---if the device is within voice range, the command word activates it, whether the speaker is on the rug or not. The carpet is then controlled by spoken directions.
				
\begin{tabular}{lllll}
\textbf{Size} & \textbf{Capacity} & \textbf{Speed} & \textbf{Weight} & \textbf{Market Price}\\
5 ft. by 5 ft. & 200 lbs. & 40 ft. & 8 lbs. & 20,000 gp\\
5 ft. by 10 ft. & 400 lbs. & 40 ft. & 10 lbs. & 35,000 gp\\
10 ft. by 10 ft. & 800 lbs. & 40 ft. & 15 lbs. & 60,000 gp\\
\end{tabular}

				
A \textit{carpet of flying }can carry up to double its capacity, but doing so reduces its speed to 30 feet. A \textit{carpet of flying }can hover without making a Fly skill check and gives a +5 bonus to other Fly checks. 
				
Construction
				
\textbf{Requirements} Craft Wondrous Item, \textit{overland flight}; \textbf{Cost }10,000 gp (5 ft. by 5 ft.), 17,500 gp (5 ft. by 10 ft.), 30,000 gp (10 ft. by 10 ft.)
				
\textbf{Chime of Interruption}
				
\textbf{Aura} moderate evocation;\textbf{ CL }7th
				
\textbf{Slot} none; \textbf{Price} 16,800 gp; \textbf{Weight} 1 lb.
				
Description
				
This instrument can be struck once every 10 minutes, and its resonant tone lasts for 3 full minutes.
				
While the chime is resonating, no spell requiring a verbal component can be cast within a 30-foot radius of it unless the caster can make a concentration check (DC 15 + the spell's level). 
				
Construction
				
\textbf{Requirements} Craft Wondrous Item, \textit{shout}; \textbf{Cost }8,400 gp
				
\textbf{Chime of Opening}
				
\textbf{Aura} moderate transmutation;\textbf{ CL }11th
				
\textbf{Slot} none; \textbf{Price} 3,000 gp; \textbf{Weight} 1 lb.
				
Description
				
A \textit{chime of opening }is a hollow mithral tube about 1 foot long. When struck, it sends forth magical vibrations that cause locks, lids, doors, valves, and portals to open. The device functions against normal bars, shackles, chains, bolts, and so on. A \textit{chime of opening }also automatically dispels a \textit{hold portal }spell or even an \textit{arcane lock }cast by a wizard of lower than 15th level.
				
The chime must be pointed at the item or gate to be loosed or opened (which must be visible and known to the user). The chime is then struck and a clear tone rings forth. The wielder can make a caster level check against the lock or binding, using the chime's caster level of 11th. The DC of this check is equal to the Disable Device DC to open the lock or binding. Each sounding only opens one form of locking, so if a chest is chained, padlocked, locked, and \textit{arcane locked, }it takes four successful uses of a \textit{chime of opening }to get it open. A \textit{silence }spell negates the power of the device. A brand-new chime can be used a total of 10 times before it cracks and becomes useless. 
				
Construction
				
\textbf{Requirements} Craft Wondrous Item, \textit{knock}; \textbf{Cost }1,500 gp
				
\textbf{Circlet of Persuasion}
				
\textbf{Aura} faint transmutation;\textbf{ CL }5th
				
\textbf{Slot} head; \textbf{Price} 4,500 gp; \textbf{Weight }---
				
Description
				
This silver headband grants a +3 competence bonus on the wearer's Charisma-based checks. 
				
Construction
				
\textbf{Requirements} Craft Wondrous Item, \textit{eagle's splendor}; \textbf{Cost }2,250 gp
				
\textbf{Cloak of Arachnida}
				
\textbf{Aura} moderate conjuration and transmutation;\textbf{ CL }6th
				
\textbf{Slot} shoulders; \textbf{Price} 14,000 gp; \textbf{Weight} 1 lb.
				
Description
				
This black garment, embroidered with a web-like pattern in silk, gives the wearer the ability to climb as if a \textit{spider climb }spell had been placed upon her. In addition, the cloak grants her immunity to entrapment by \textit{web }spells or webs of any sort; she can actually move in webs at half her normal speed. Once per day, the wearer of this cloak can cast \textit{web}. She also gains a +2 luck bonus on all Fortitude saves against poison from spiders. 
				
Construction
				
\textbf{Requirements} Craft Wondrous Item, \textit{spider climb, web}; \textbf{Cost }7,000 gp
				
\textbf{Cloak of the Bat}
				
\textbf{Aura} moderate transmutation;\textbf{ CL }7th
				
\textbf{Slot} shoulders; \textbf{Price} 26,000 gp; \textbf{Weight} 1 lb.
				
Description
				
Fashioned of dark brown or black cloth, this cloak bestows a +5 competence bonus on Stealth checks. The wearer is also able to hang upside down from the ceiling like a bat.
				
By holding the edges of the garment, the wearer is able to \textit{fly }as per the spell (including a +7 bonus on Fly skill checks). If he desires, the wearer can actually polymorph himself into an ordinary bat and fly accordingly (as \textit{beast shape III}). All possessions worn or carried are part of the transformation. Flying, either with the cloak or in bat form, can be accomplished only in darkness (either under the night sky or in a lightless or near-lightless environment underground). Either of the flying powers is usable for up to 7 minutes at a time, but after a flight of any duration the cloak cannot bestow any flying power for a like period of time. 
				
Construction
				
\textbf{Requirements} Craft Wondrous Item, \textit{beast shape III, fly}; \textbf{Cost }13,000 gp
				
\textbf{Cloak of Displacement, Major}
				
\textbf{Aura} moderate illusion;\textbf{ CL }7th
				
\textbf{Slot} shoulders; \textbf{Price} 50,000 gp; \textbf{Weight} 1 lb.
				
Description
				
This item appears to be a normal cloak, but on command its magical properties distort and warp light waves. This displacement works just like the \textit{displacement }spell and lasts for a total of 15 rounds per day, which the wearer can divide up as she sees fit. 
				
Construction
				
\textbf{Requirements} Craft Wondrous Item, Extend Spell, \textit{displacement}; \textbf{Cost }25,000 gp
				
\textbf{Cloak of Displacement, Minor}
				
\textbf{Aura} faint illusion;\textbf{ CL }3rd
				
\textbf{Slot} shoulders; \textbf{Price} 24,000 gp; \textbf{Weight} 1 lb.
				
Description
				
This item appears to be a normal cloak, but when worn by a character, its magical properties distort and warp light waves. This displacement works similar to the \textit{blur }spell, granting a 20\% miss chance on attacks against the wearer. It functions continually. 
				
Construction
				
\textbf{Requirements} Craft Wondrous Item, \textit{blur}; \textbf{Cost }12,000 gp
				
\textbf{Cloak of Elvenkind}
				
\textbf{Aura} faint illusion;\textbf{ CL }3rd
				
\textbf{Slot} shoulders; \textbf{Price} 2,500 gp; \textbf{Weight} 1 lb.
				
Description
				
When this plain gray cloak is worn with the hood drawn up around the head, the wearer gains a +5 competence bonus on Stealth checks. 
				
Construction
				
\textbf{Requirements} Craft Wondrous Item, \textit{invisibility,} creator must be an elf; \textbf{Cost }1,250 gp
				
\textbf{Cloak of Etherealness}
				
\textbf{Aura} strong transmutation;\textbf{ CL }15th
				
\textbf{Slot} shoulders; \textbf{Price} 55,000 gp; \textbf{Weight} 1 lb.
				
Description
				
This silvery gray cloak seems to absorb light rather than be illuminated by it. On command, the cloak makes its wearer ethereal (as the \textit{ethereal jaunt }spell). The effect is dismissible. The cloak works for a total of up to 10 minutes per day. This duration need not be continuous, but it must be used in 1 minute increments. 
				
Construction
				
\textbf{Requirements} Craft Wondrous Item, \textit{ethereal jaunt}; \textbf{Cost }27,500 gp
				
\textbf{Cloak of the Manta Ray}
				
\textbf{Aura} moderate transmutation;\textbf{ CL }9th
				
\textbf{Slot} shoulders; \textbf{Price} 7,200 gp; \textbf{Weight} 1 lb.
				
Description
				
This cloak appears to be made of leather until the wearer enters salt water. At that time, the \textit{cloak of the manta ray }adheres to the individual, and he appears nearly identical to a manta ray (as the \textit{beast shape II }spell, except that it allows only manta ray form). He gains a +3 natural armor bonus, the ability to breathe underwater, and a swim speed of 60 feet, like a real manta ray.
				
The cloak does allow the wearer to attack with a manta ray's tail spine, dealing 1d6 points of damage. This attack can be used in addition to any other attack the character has, using his highest melee attack bonus. The wearer can release his arms from the cloak without sacrificing underwater movement if so desired. 
				
Construction
				
\textbf{Requirements} Craft Wondrous Item, \textit{beast shape II}, \textit{water breathing}; \textbf{Cost }3,600 gp
				
\textbf{Cloak of Resistance}
				
\textbf{Aura} faint abjuration;\textbf{ CL }5th
				
\textbf{Slot} shoulders; \textbf{Price} 1,000 gp (+1), 4,000 gp (+2), 9,000 gp (+3), 16,000 gp (+4), 25,000 gp (+5); \textbf{Weight} 1 lb.
				
Description
				
These garments offer magic protection in the form of a +1 to +5 resistance bonus on all saving throws (Fortitude, Reflex, and Will). 
				
Construction
				
\textbf{Requirements} Craft Wondrous Item, resistance\textit{,} creator's caster level must be at least three times the cloak's bonus; \textbf{Cost }500 gp (+1), 2,000 gp (+2), 4,500 gp (+3), 8,000 gp (+4), 12,500 gp (+5)
				
\textbf{Crown of Blasting, Minor}
				
\textbf{Aura} moderate evocation; \textbf{CL} 6th
				
\textbf{Slot }head; \textbf{Price} 6,480 gp; \textbf{Weight} 1 lb.
				
Description
				
On command, this simple golden crown projects a blast of \textit{searing light }(3d8 points of damage) once per day.
				
Construction
				
\textbf{Requirements }Craft Wondrous Item, \textit{searing light}; \textbf{Cost} 3,240 gp
				
\textbf{Crown of Blasting, Major}
				
\textbf{Aura} strong evocation; \textbf{CL} 17th
				
\textbf{Slot }head; \textbf{Price} 23,760 gp; \textbf{Weight} 1 lb.
				
Description
				
On command, this elaborate golden crown projects a blast of \textit{searing light }(5d8 maximized for 40 points of damage) once per day.
				
Construction
				
\textbf{Requirements }Craft Wondrous Item, Maximize Spell, \textit{searing light}; \textbf{Cost} 11,880 gp
				
\textbf{Crystal Ball}
				
\textbf{Aura} moderate divination;\textbf{ CL }10th
				
\textbf{Slot} none; \textbf{Price} varies; \textbf{Weight} 7 lbs.
				
Description
				
This is the most common form of scrying device, a crystal sphere about 6 inches in diameter. So well-known are these items that many so-called oracles or fortune-tellers use similar appearing (but completely non-magical) replicas of these items to ply their trades. A character can use a magical \textit{crystal ball} to see over virtually any distance or into other planes of existence, as with the spell \textit{scrying }(Will DC 16 negates). A \textit{crystal ball} can be used multiple times per day, but the DC to resist its power decreases by 1 for each additional use.
				
Certain \textit{crystal balls }have additional powers that can be used through the \textit{crystal ball }on the target viewed.
\begin{tabularx}{\linewidth}{Xl}
\textbf{Crystal Ball Type} & \textbf{Market Price}\\
\textit{Crystal ball} & 42,000 gp\\
\textit{Crystal ball} with \textit{see invisibility} & 50,000 gp\\
\textit{Crystal ball} with \textit{telepathy}* & 70,000 gp\\
\textit{Crystal ball} with \textit{true seeing} & 80,000 gp\\
\textit{Crystal ball} with \textit{detect thoughts} (Will DC 13 negates) & 51,000 gp\\
\end{tabularx}
* The viewer is able to send and receive silent mental messages with the person appearing in the crystal ball. Once per day, the character may attempt to implant a \textit{suggestion} (as the spell, Will DC 14 negates) as well.

Construction
				
\textbf{Requirements} Craft Wondrous Item, \textit{scrying} (plus any additional spells put into item); \textbf{Cost }21,000 gp (standard), 25,000 (with \textit{see invisibility}), 25,500 gp (with \textit{detect thoughts}), 35,000 gp (with \textit{telepathy}), 40,000 gp (with \textit{true seeing})
				
\textbf{Cube of Force}
				
\textbf{Aura} moderate evocation;\textbf{ CL }10th
				
\textbf{Slot} none; \textbf{Price} 62,000 gp; \textbf{Weight }1/2 lb.
				
Description
				
This device is just under an inch across and can be made of ivory, bone, or any hard mineral. Typically, each of the cube's faces are polished smooth, but sometimes they are etched with runes. The device enables its possessor to put up a special cube made up of 6 individual \textit{wall of force }spells, 10 feet on a side around her person. This cubic screen moves with the character and is impervious to the attack forms mentioned on the table below. The cube has 36 charges when fully charged---charges used are automatically renewed each day. The possessor presses one face of the cube to activate a particular type of screen or to deactivate the device. Each effect costs a certain number of charges to maintain for every minute (or portion of a minute) it is in operation. Also, when an effect is active, the possessor's speed is limited to the maximum value given on the table.
				
When the \textit{cube of force }is active, attacks dealing more than 30 points of damage drain 1 charge for every 10 points of damage beyond 30 that they deal. The charge cost to maintain each of the cube's six walls is summarized below.\\

\begin{tabularx}{\linewidth}{lllX}
\textbf{Cube} & \textbf{Charge Cost} & \textbf{Max.} & \\
\textbf{Face} & \textbf{per Minute} & \textbf{Speed} & \textbf{Effect}\\
1 & 1 & 30 ft. & Keeps out gases, wind, etc.\\
2 & 2 & 20 ft. & Keeps out nonliving matter\\
3 & 3 & 15 ft. & Keeps out living matter\\
4 & 4 & 10 ft. & Keeps out magic\\
5 & 6 & 10 ft. & Keeps out all things\\
6 & 0 & As normal & Deactivates\\
\end{tabularx}
				
Spells that affect the integrity of the screen also drain extra charges. These spells cannot be cast into or out of the cube. \\

\begin{tabular}{lc}
\textbf{Attack Form} & \textbf{Extra Charges}\\
Disintegrate & 6\\
Horn of blasting & 6\\
Passwall & 3\\
Phase door & 5\\
Prismatic spray & 7 \\
Wall of fire & 2\\
\end{tabular}
				
Construction
				
\textbf{Requirements} Craft Wondrous Item, \textit{wall of force}; \textbf{Cost }31,000 gp
				
\textbf{Cube of Frost Resistance}
				
\textbf{Aura} faint abjuration;\textbf{ CL }5th
				
\textbf{Slot} none; \textbf{Price} 27,000 gp; \textbf{Weight }2 lbs.
				
Description
				
This cube is activated or deactivated by pressing one side. When activated, it creates a cube-shaped area 10 feet on a side centered on the possessor (or on the cube itself, if the item is later placed on a surface)\textit{. }The temperature within this area is always at least 65\^A\mbox{${}^\circ$} F. The field absorbs all cold-based attacks. However, if the field is subjected to more than 50 points of cold damage in 1 round (from one or multiple attacks), it collapses and cannot be reactivated for 1 hour. If the field absorbs more than 100 points of cold damage in a 10-round period, the cube is destroyed. 
				
Construction
				
\textbf{Requirements} Craft Wondrous Item, \textit{protection from energy}; \textbf{Cost }13,500 gp
				
\textbf{Cubic Gate}
				
\textbf{Aura} strong conjuration;\textbf{ CL }13th
				
\textbf{Slot} none; \textbf{Price} 164,000 gp; \textbf{Weight }2 lbs.
				
Description
				
This potent magical item is a small cube fashioned from carnelian. Each of the six sides of the cube is keyed to a different plane of existence or dimension, one of which is the Material Plane. The character creating the item chooses the planes to which the other five sides are keyed.
				
If a side of the \textit{cubic gate }is pressed once, it opens a \textit{gate }to a random point on the plane keyed to that side. There is a 10\% chance per minute that an outsider from that plane (determine randomly) comes through it looking for food, fun, or trouble. Pressing the side a second time closes the \textit{gate}. It is impossible to open more than one \textit{gate }at a time.
				
If a side is pressed twice in quick succession, the character so doing is transported to a random point on the other plane, along with all creatures in adjacent squares. The other creatures may avoid this fate by succeeding on DC 23 Will saves. 
				
Construction
				
\textbf{Requirements} Craft Wondrous Item, \textit{plane shift}; \textbf{Cost }82,000 gp
				
\textbf{Darkskull}
				
\textbf{Aura} moderate evocation \mbox{$[$}evil\mbox{$]$};\textbf{ CL }9th
				
\textbf{Slot} none; \textbf{Price} 60,000 gp; \textbf{Weight} 5 lbs.
				
Description
				
This skull, carved from ebony, is wholly evil. Wherever the skull goes, the area around it is treated as though an \textit{unhallow }spell had been cast with the skull as the touched point of origin. Each \textit{darkskull }has a single spell effect tied to it. This spell is from the standard list given in the \textit{unhallow }spell description, and it cannot be changed. 
				
Construction
				
\textbf{Requirements} Craft Wondrous Item, \textit{unhallow,} creator must be evil; \textbf{Cost }30,000 gp
				
\textbf{Decanter of Endless Water}
				
\textbf{Aura} moderate transmutation;\textbf{ CL }9th
				
\textbf{Slot} none; \textbf{Price} 9,000 gp; \textbf{Weight} 2 lbs.
				
Description
				
If the stopper is removed from this ordinary-looking flask and a command word spoken, an amount of fresh or salt water pours out. Separate command words determine the type of water as well as the volume and velocity.
\begin{itemize}\item  \texttt{{}"{}}Stream\texttt{{}"{}} pours out 1 gallon per round.
\item  \texttt{{}"{}}Fountain\texttt{{}"{}} produces a 5-foot-long stream at 5 gallons per round.
\item  \texttt{{}"{}}Geyser\texttt{{}"{}} produces a 20-foot-long, 1-foot-wide stream at 30 gallons per round.
\end{itemize}
				
The geyser effect exerts considerable pressure, requiring the holder to make a DC 12 Strength check to avoid being knocked down each round the effect is maintained. In addition, the powerful force of the geyser deals 1d4 points of damage per round to a creature that is subjected to it. The geyser can only affect one target per round, but the user can direct the beam of water without needing to make an attack role to strike the target since the geyser's constant flow allows for ample opportunity to aim. Creatures with the fire subtype take 2d4 points of damage per round from the geyser rather than 1d4. The command word must be spoken to stop it. 
				
Construction
				
\textbf{Requirements} Craft Wondrous Item, \textit{control water};\textbf{ Cost }4,500 gp
				
\textbf{Deck of Illusions}
				
\textbf{Aura} moderate illusion;\textbf{ CL }6th
				
\textbf{Slot} none; \textbf{Price} 8,100 gp; \textbf{Weight} 1/2 lb.
				
Description
				
This set of parchment cards is usually found in an ivory, leather, or wooden box. A full deck consists of 34 cards. When a card is drawn at random and thrown to the ground, a \textit{major image }of a creature is formed. The figment lasts until dispelled. The illusory creature cannot move more than 30 feet away from where the card landed, but otherwise moves and acts as if it were real. At all times it obeys the desires of the character who drew the card. When the illusion is dispelled, the card becomes blank and cannot be used again. If the card is picked up, the illusion is automatically and instantly dispelled. The cards in a deck and the illusions they bring forth are summarized on the following table. (Use one of the first two columns to simulate the contents of a full deck using either ordinary playing cards or tarot cards.)
\setlength{\tabcolsep}{1pt}
\begin{tabularx}{\linewidth}{lll}
\textbf{Playing Card}  & \textbf{Tarot Card}    & \textbf{Creature}                       \\
Ace of hearts          & IV. The Emperor        & Red dragon                              \\
King of hearts         & Knight of swords       & Male human fighter \\
                       &                        & and four guards      \\
Queen of hearts        & Queen of staves        & Female human wizard                     \\
Jack of hearts         & King of staves         & Male human druid                        \\
Ten of hearts          & VII. The Chariot       & Cloud giant                             \\
Nine of hearts         & Page of staves         & Ettin                                   \\
Eight of hearts        & Ace of cups            & Bugbear                                 \\
Two of hearts          & Five of staves         & Goblin                                  \\
\textbf{Playing Card}  & \textbf{Tarot Card}    & \textbf{Creature}                       \\
Ace of diamonds        & III. The Empress       & Glabrezu (demon)                        \\
King of diamonds       & Two of cups            & Male elf wizard \\
                       &                        & and female apprentice   \\
Queen of diamonds      & Queen of swords        & Half-elf ranger                         \\
Jack of diamonds       & XIV. Temperance        & Harpy                                   \\
Ten of diamonds        & Seven of staves        & Male half-orc \\
                       &                        & barbarian                 \\
Nine of diamonds       & Four of pentacles      & Ogre mage                               \\
Eight of diamonds      & Ace of pentacles       & Gnoll                                   \\
Two of diamonds        & Six of pentacles       & Kobold                                  \\
\textbf{Playing Card}  & \textbf{Tarot Card}    & \textbf{Creature}                       \\
Ace of spades          & II. The High           & Lich                                    \\
                       & Priestess              &                                         \\
King of spades         & Three of staves        & Three human clerics                     \\
Queen of spades        & Four of cups           & Medusa                                  \\
Jack of spades         & Knight of pentacles    & Male dwarf paladin                      \\
Ten of spades          & Seven of swords        & Frost giant                             \\
Nine of spades         & Three of swords        & Troll                                   \\
Eight of spades        & Ace of swords          & Hobgoblin                               \\
Two of spades          & Five of cups           & Goblin                                  \\
\textbf{Playing Card}  & \textbf{Tarot Card}    & \textbf{Creature}                       \\
Ace of clubs           & VIII. Strength         & Iron golem                              \\
King of clubs          & Page of pentacles      & Three halfling rogues                   \\
Queen of clubs         & Ten of cups            & Pixies                                  \\
Jack of clubs          & Nine of pentacles      & Half-elf bard                           \\
Ten of clubs           & Nine of staves         & Hill giant                              \\
Nine of clubs          & King of swords         & Ogre                                    \\
Eight of clubs         & Ace of staves          & Orc                                     \\
Two of clubs           & Five of cups           & Kobold                                  \\
\textbf{Playing Card}  & \textbf{Tarot Card}    & \textbf{Creature}                       \\
Joker                  & Two of pentacles       & Illusion of deck's \\
                       &                        & owner                \\
Joker                  & Two of staves          & Illusion of deck's \\
 (with trademark)      &                        & owner (sex reversed)
\end{tabularx}				
A randomly generated deck is usually complete (11--100 on d\%), but may be discovered (01--10) with 1d20 of its cards missing. If cards are missing, reduce the price by a corresponding amount. 
				
Construction
				
\textbf{Requirements} Craft Wondrous Item, \textit{major image}; \textbf{Cost }4,050 gp
				
\textbf{Dimensional Shackles}
				
\textbf{Aura} moderate abjuration;\textbf{ CL }11th
				
\textbf{Slot} wrists; \textbf{Price} 28,000 gp; \textbf{Weight} 5 lbs.
				
Description
				
These shackles have golden runes traced across their cold iron links. Any creature bound within them is affected as if a \textit{dimensional anchor }spell were cast upon it (no save). They fit any Small to Large creature. The DC to break or slip out of the shackles is 30. 
				
Construction
				
\textbf{Requirements} Craft Wondrous Item, \textit{dimensional anchor}; \textbf{Cost }14,000 gp
				
\textbf{Drums of Panic}
				
\textbf{Aura} moderate necromancy;\textbf{ CL }7th
				
\textbf{Slot} none; \textbf{Price} 30,000 gp; \textbf{Weight} 10 lbs. for the pair.
				
Description
				
These drums are kettle drums (hemispheres about 1-1/2 feet in diameter on stands). They come in pairs and are unremarkable in appearance. If both of the pair are sounded, all creatures within 120 feet (with the exception of those within a 20-foot-radius safe zone around the drums) are affected as by a \textit{fear} spell (Will DC 16 partial). \textit{Drums of panic} can be used once per day. 
				
Construction
				
\textbf{Requirements} Craft Wondrous Item\textit{, fear; }\textbf{Cost }15,000 gp
				
\textbf{Dust of Appearance}
				
\textbf{Aura} faint conjuration;\textbf{ CL }5th
				
\textbf{Slot} none; \textbf{Price} 1,800 gp; \textbf{Weight }---
				
Description
				
This powder appears to be a very fine, very light metallic dust. A single handful of this substance flung into the air coats objects within a 10-foot radius, making them visible even if they are invisible. It likewise negates the effects of \textit{blur} and \textit{displacement}. In this, it works just like the \textit{faerie fire} spell. The dust also reveals figments, mirror images, and projected images for what they are. A creature coated with the dust takes a --30 penalty on its Stealth checks. The dust's effect lasts for 5 minutes.
				
\textit{Dust of appearance} is typically stored in small silk packets or hollow bone tubes. 
				
Construction
				
\textbf{Requirements} Craft Wondrous Item\textit{, glitterdust; }\textbf{Cost }900 gp
				
\textbf{Dust of Disappearance}
				
\textbf{Aura} moderate illusion;\textbf{ CL }7th
				
\textbf{Slot} none; \textbf{Price} 3,500 gp; \textbf{Weight }---
				
Description
				
This dust looks like \textit{dust of appearance} and is typically stored in the same manner. A creature or object touched by it becomes invisible (as \textit{greater invisibility}). Normal vision can't see dusted creatures or objects, nor can they be detected by magical means, including \textit{see invisibility} or \textit{invisibility purge}. \textit{Dust of appearance}, however, does reveal people and objects made invisible by \textit{dust of disappearance}. Other factors, such as sound and smell, also allow possible detection. The \textit{greater invisibility} bestowed by the dust lasts for 2d6 rounds. The invisible creature doesn't know when the duration will end. 
				
Construction
				
\textbf{Requirements} Craft Wondrous Item\textit{, greater invisibility;}\textbf{ Cost }1,750 gp
				
\textbf{Dust of Dryness}
				
\textbf{Aura} moderate transmutation;\textbf{ CL }11th
				
\textbf{Slot} none; \textbf{Price} 850 gp; \textbf{Weight }---
				
Description
				
This special dust has many uses. If it is thrown into water, a volume of as much as 100 gallons is instantly transformed into nothingness, and the dust becomes a marble-sized pellet, floating or resting where it was thrown. If this pellet is hurled, it breaks and releases the same volume of water. The dust affects only water (fresh, salt, alkaline), not other liquids.
				
If the dust is employed against an outsider with the elemental and water subtypes, the creature must make a DC 18 Fortitude save or be destroyed. The dust deals 5d6 points of damage to the creature even if its saving throw succeeds. 
				
Construction
				
\textbf{Requirements} Craft Wondrous Item\textit{, control water;}\textbf{ Cost }425 gp
				
\textbf{Dust of Illusion}
				
\textbf{Aura} moderate illusion; \textbf{CL} 6th
				
\textbf{Slot}\textit{ ---; }\textbf{Price} 1,200 gp; \textbf{Weight }\textit{---}
				
Description
				
This unremarkable powder resembles chalk dust or powdered graphite. Stare at it, however, and the dust changes color and form. Put \textit{dust of illusion }on a creature, and that creature is affected as if by a \textit{disguise self} glamer, with the individual who sprinkles the dust envisioning the illusion desired. An unwilling target is allowed a DC 11 Reflex save to avoid the dust. The glamer lasts for 2 hours. 
				
Construction
				
\textbf{Requirements} Craft Wondrous Item, \textit{disguise self;}\textbf{ Cost }600 gp
				
\textbf{Dust of Tracelessness}
				
\textbf{Aura} faint transmutation;\textbf{ CL }3rd
				
\textbf{Slot} none; \textbf{Price} 250 gp; \textbf{Weight }---
				
Description
				
This normal-seeming dust is actually a magic powder that can conceal the passage of its possessor and his companions. Tossing a handful of this dust into the air causes a chamber of up to 100 square feet of floor space to become as dusty, dirty, and cobweb-laden as if it had been abandoned and disused for a decade.
				
A handful of dust sprinkled along a trail causes evidence of the passage of as many as a dozen men and horses to be obliterated for 250 feet back into the distance. The results of the dust are instantaneous, and no magical aura lingers afterward from this use of the dust. Survival checks made to track a quarry across an area affected by this dust have a DC 20 higher than normal. 
				
Construction
				
\textbf{Requirements} Craft Wondrous Item,\textit{ pass without trace;}\textbf{\textit{ Cost }}125 gp
				
\textbf{Efficient Quiver}
				
\textbf{Aura} moderate conjuration;\textbf{ CL }9th
				
\textbf{Slot} none; \textbf{Price} 1,800 gp; \textbf{Weight} 2 lbs.
				
Description
				
This appears to be a typical arrow container capable of holding about 20 arrows. It has three distinct portions, each with a nondimensional space allowing it to store far more than would normally be possible. The first and smallest one can contain up to 60 objects of the same general size and shape as an arrow. The second slightly longer compartment holds up to 18 objects of the same general size and shape as a javelin. The third and longest portion of the case contains as many as 6 objects of the same general size and shape as a bow (spears, staves, or the like). Once the owner has filled it, the quiver can quickly produce any item she wishes that is within the quiver, as if from a regular quiver or scabbard. The efficient quiver weighs the same no matter what's placed inside it. 
				
Construction
				
\textbf{Requirements} Craft Wondrous Item\textit{, secret chest;} \textbf{Cost} 900 gp
				
\textbf{Efreeti Bottle}
				
\textbf{Aura} strong conjuration;\textbf{ CL }14th
				
\textbf{Slot} none; \textbf{Price} 145,000 gp; \textbf{Weight} 1 lb.
				
Description
				
This item is typically fashioned of brass or bronze, with a lead stopper bearing special seals. Periodically, a thin stream of bitter-smelling smoke issues from the bottle's top. The bottle can be opened once per day. When opened, the efreeti imprisoned within issues from the bottle instantly amid a cloud of noxious smoke. There is a 10\% chance (01--10 on d\%) that the efreeti is insane and attacks immediately upon being released. There is also a 10\% chance (91--100) that the efreeti of the bottle grants three \textit{wishes.} In either case, afterward the efreeti disappears forever, and the bottle becomes nonmagical. The other 80\% of the time (11--90), the inhabitant of the bottle loyally serves the character for up to 10 minutes per day (or until the efreeti's death), doing as she commands. Roll each day the bottle is opened for that day's effect. 
				
Construction
				
\textbf{Requirements} Craft Wondrous Item\textit{, planar binding;}\textbf{\textit{ Cost }}72,500 gp
				
\textbf{Elemental Gem}
				
\textbf{Aura} moderate conjuration;\textbf{ CL }11th
				
\textbf{Slot} none; \textbf{Price} 2,250 gp; \textbf{Weight }---
				
Description
				
An elemental gem comes in one of four different varieties. Each contains a conjuration spell attuned to a specific elemental plane (Air, Earth, Fire, or Water).
				
When the gem is crushed, smashed, or broken (a standard action), a Large elemental appears as if summoned by a\textit{ summon nature's ally} spell. The elemental is under the control of the creature that broke the gem.
				
The coloration of the gem varies with the type of elemental it summons. Air elemental gems are transparent, earth elemental gems are light brown, fire elemental gems are reddish orange, and water elemental gems are blue-green. 
				
Construction
				
\textbf{Requirements} Craft Wondrous Item\textit{, summon monster V} or \textit{summon nature's ally V;}\textbf{ Cost }1,125 gp
				
\textbf{Elixir of Fire Breath}
				
\textbf{Aura} moderate evocation;\textbf{ CL }11th
				
\textbf{Slot} none; \textbf{Price} 1,100 gp; \textbf{Weight }---
				
Description
				
This strange bubbling elixir bestows upon the drinker the ability to spit gouts of flame. He can breathe fire up to three times, each time dealing 4d6 points of fire damage to a single target up to 25 feet away. The victim can attempt a DC 13 Reflex save for half damage. Unused blasts of fire dissipate 1 hour after the liquid is consumed. 
				
Construction
				
\textbf{Requirements} Craft Wondrous Item, \textit{scorching ray;}\textbf{ Cost }550 gp
				
\textbf{Elixir of Hiding}
				
\textbf{Aura} faint illusion;\textbf{ CL }5th
				
\textbf{Slot} none; \textbf{Price} 250 gp; \textbf{Weight }---
				
Description
				
A character drinking this liquid gains an intuitive ability to sneak and hide (+10 competence bonus on Stealth checks for 1 hour). 
				
Construction
				
\textbf{Requirements} Craft Wondrous Item\textit{, invisibility;}\textbf{ Cost }125 gp
				
\textbf{Elixir of Love}
				
\textbf{Aura} faint enchantment;\textbf{ CL }4th
				
\textbf{Slot} none; \textbf{Price} 150 gp; \textbf{Weight }---
				
Description
				
This sweet-tasting liquid causes the character drinking it to become enraptured with the first creature she sees after consuming the draft (as \textit{charm person}---the drinker must be a humanoid of Medium or smaller size, Will DC 14 negates). The charm effect wears off in 1d3 hours. 
				
Construction
				
\textbf{Requirements} Craft Wondrous Item\textit{, charm person;}\textbf{ Cost }75 gp
				
\textbf{Elixir of Swimming}
				
\textbf{Aura} faint transmutation;\textbf{ CL }2nd
				
\textbf{Slot} none; \textbf{Price} 250 gp; \textbf{Weight }---
				
Description
				
This elixir bestows swimming ability. An almost imperceptible magic sheath surrounds the drinker, allowing him to glide through the water easily (+10 competence bonus on Swim checks for 1 hour). 
				
Construction
				
\textbf{Requirements} Craft Wondrous Item\textit{, }creator must have 5 ranks in the Swim skill;\textbf{ Cost }125 gp
				
\textbf{Elixir of Truth}
				
\textbf{Aura} faint enchantment;\textbf{ CL }5th
				
\textbf{Slot} none; \textbf{Price} 500 gp; \textbf{Weight }---
				
Description
				
This elixir forces the drinker it to say nothing but the truth for 10 minutes (Will DC 13 negates). She must answer any questions put to her in that time, but with each question she can make a separate DC 13 Will save. If one of these secondary saves is successful, she doesn't break free of the truth-compelling enchantment but also doesn't have to answer that particular question (if she does answer, she must tell the truth). No more than one question can be asked each round. This is a mind-affecting compulsion enchantment. 
				
Construction
				
\textbf{Requirements} Craft Wondrous Item,\textit{ zone of truth;}\textbf{ Cost }250 gp
				
\textbf{Elixir of Tumbling}
				
\textbf{Aura} faint transmutation;\textbf{ CL }5th
				
\textbf{Slot} none; \textbf{Price} 250 gp; \textbf{Weight }---
				
Description
				
This draught of liquid grants the drinker the ability to tumble about, avoiding attacks and moving carefully across nearly any surface, granting a +10 competence bonus on Acrobatics checks for 1 hour. 
				
Construction
				
\textbf{Requirements} Craft Wondrous Item\textit{, cat's grace;}\textbf{ Cost }125 gp
				
\textbf{Elixir of Vision}
				
\textbf{Aura} faint divination;\textbf{ CL }2nd
				
\textbf{Slot} none; \textbf{Price} 250 gp; \textbf{Weight }---
				
Description
				
Drinking this elixir grants the imbiber the ability to notice acute details with great accuracy (+10 competence bonus on Perception checks for 1 hour). 
				
Construction
				
\textbf{Requirements} Craft Wondrous Item,\textit{ true seeing;}\textbf{ Cost }125 gp
				
\textbf{Eversmoking Bottle}
				
\textbf{Aura} faint transmutation;\textbf{ CL }3rd
				
\textbf{Slot} none; \textbf{Price} 5,400 gp; \textbf{Weight} 1 lb.
				
Description
				
This metal urn is identical in appearance to an \textit{efreeti bottle}, except that it does nothing but smoke. The amount of smoke is great if the stopper is pulled out, pouring from the bottle and totally obscuring vision across a 50-foot spread in 1 round. If the bottle is left unstoppered, the smoke billows out another 10 feet per round until it has covered a 100-foot radius. This area remains smoke-filled until the \textit{eversmoking bottle} is stoppered.
				
The bottle must be resealed by a command word, after which the smoke dissipates normally. A moderate wind (11+ mph) disperses the smoke in 4 rounds; a strong wind (21+ mph) disperses the smoke in 1 round. 
				
Construction
				
\textbf{Requirements} Craft Wondrous Item,\textit{ pyrotechnics;}\textbf{ Cost }2,700 gp
				
\textbf{Eyes of Charming}
				
\textbf{Aura} moderate enchantment;\textbf{ CL }7th
				
\textbf{Slot} eyes; \textbf{Price} 56,000 gp for a pair; \textbf{Weight }---
				
Description
				
These two crystal lenses fit over the user's eyes. The wearer is able to use \textit{charm person} (one target per round) merely by meeting a target's gaze. Those failing a DC 16 Will save are charmed as per the spell. Both lenses must be worn for the magic item to take effect. 
				
Construction
				
\textbf{Requirements} Craft Wondrous Item, Heighten Spell\textit{, charm person;}\textbf{ Cost }28,000 gp
				
\textbf{Eyes of Doom}
				
\textbf{Aura} moderate necromancy;\textbf{ CL }11th
				
\textbf{Slot} eyes; \textbf{Price} 25,000 gp; \textbf{Weight }---
				
Description
				
These crystal lenses fit over the user's eyes, enabling him to cast \textit{doom} upon those around him (one target per round) as a gaze attack, except that the wearer must take a standard action, and those merely looking at the wearer are not affected. Those failing a DC 11 Will save are affected as by the \textit{doom} spell. The wearer also gains the additional power of a continual \textit{deathwatch} effect and can use \textit{fear }(Will DC 16 partial) as a normal gaze attack once per week. Both lenses must be worn for the magic item to take effect. 
				
Construction
				
\textbf{Requirements} Craft Wondrous Item,\textit{ doom, deathwatch, fear;}\textbf{ Cost }12,500 gp
				
\textbf{Eyes of the Eagle}
				
\textbf{Aura} faint divination;\textbf{ CL }3rd
				
\textbf{Slot} eyes; \textbf{Price} 2,500 gp; \textbf{Weight }---
				
Description
				
These items are made of special crystal and fit over the eyes of the wearer. These lenses grant a +5 competence bonus on Perception checks. Wearing only one of the pair causes a character to become dizzy and stunned for 1 round. Both lenses must be worn for the magic item to take effect. 
				
Construction
				
\textbf{Requirements} Craft Wondrous Item,\textit{ clairaudience/clairvoyance;}\textbf{ Cost }1,250 gp
				
\textbf{Feather Token}
				
\textbf{Aura} strong conjuration;\textbf{ CL }12th
				
\textbf{Slot} none; \textbf{Price} 50 gp (anchor), 300 gp (bird), 200 gp (fan), 450 gp (swan boat), 400 gp (tree), 500 gp (whip); \textbf{Weight }---
				
Description
				
Each of these items is a small feather that has a power to suit a special need. The kinds of tokens are described below. Each token is usable once. A particular feather token has no specific features to identify it unless its magic aura is viewed---even tokens with identical powers can be wildly different in appearance.
				
\textit{Anchor}: A token that creates an anchor that moors a craft in water so as to render it immobile for up to 1 day.
				
\textit{Bird}: A token that creates a small bird that can be used to deliver a small written message unerringly to a designated target. The token lasts as long as it takes to carry the message.
				
\textit{Fan}: A token that forms a huge flapping fan, causing a breeze of sufficient strength to propel one ship (about 25 mph). This wind is not cumulative with existing wind speed. The token can, however, be used to lessen existing winds, creating an area of relative calm or lighter winds (but wave size in a storm is not affected). The fan can be used for up to 8 hours. It does not function on land.
				
\textit{Swan Boat}: A token that forms a swan-like boat capable of moving on water at a speed of 60 feet. It can carry eight horses and gear, 32 Medium characters, or any equivalent combination. The boat lasts for 1 day.
				
\textit{Tree}: A token that causes a great oak to spring into being (5-foot-diameter trunk, 60-foot height, 40-foot top diameter). This is an instantaneous effect.
				
\textit{Whip}: A token that forms into a huge leather whip and wields itself against any opponent desired just like a \textit{dancing weapon}. The weapon has a +10 base attack bonus, does 1d6+1 points of nonlethal damage, has a +1 enhancement bonus on attack and damage rolls, and a makes a free grapple attack (with a +15 bonus on its combat maneuver checks) if it hits. The whip lasts no longer than 1 hour. 
				
Construction
				
\textbf{Requirements} Craft Wondrous Item, \textit{major creation;}\textbf{ Cost }25 gp (anchor), 150 gp (bird), 100 gp (fan), 225 gp (swan boat), 200 gp (tree), 250 gp (whip)
				
\textbf{Figurines of Wondrous Power}
				
\textbf{Aura} varies;\textbf{ CL }varies
				
\textbf{Slot} none; \textbf{Price} 10,000 gp (bronze griffon), 10,000 gp (ebony fly), 16,500 gp (golden lions), 21,000 gp (ivory goats), 17,000 gp (marble elephant), 28,500 gp (obsidian steed), 15,500 gp (onyx dog), 9,100 gp (serpentine owl), 3,800 gp (silver raven); \textbf{Weight }1 lb.
				
Description
				
Each of the several kinds of \textit{figurines of wondrous power} appears to be a miniature statuette of a creature an inch or so high (with one exception). When the figurine is tossed down and the correct command word spoken, it becomes a living creature of normal size (except when noted otherwise below). The creature obeys and serves its owner. Unless stated otherwise, the creature understands Common but does not speak.
				
If a \textit{figurine of wondrous power }is broken or destroyed in its statuette form, it is forever ruined. All magic is lost, its power departed. If slain in animal form, the figurine simply reverts to a statuette that can be used again at a later time.
				
\textbf{Bronze Griffon}: When animated, a bronze griffon acts in all ways like a normal griffon under the command of its possessor. The item can be used twice per week for up to 6 hours per use. When 6 hours have passed or when the command word is spoken, the bronze griffon once again becomes a tiny statuette. Moderate transmutation; CL 11th; Craft Wondrous Item,\textit{ animate objects}.
				
\textbf{Ebony Fly}: When animated, an ebony fly is the size of a pony and has all the statistics of a pegasus but can make no attacks. The item can be used three times per week for up to 12 hours per use. When 12 hours have passed or when the command word is spoken, the ebony fly again becomes a tiny statuette. Moderate transmutation; CL 11th; Craft Wondrous Item,\textit{ animate objects}.
				
\textbf{Golden Lions}: These figurines come in pairs. They become normal adult male lions. If slain in combat, the lions cannot be brought back from statuette form for 1 full week. Otherwise, they can be used once per day for up to 1 hour. They enlarge and shrink upon speaking the command word. Moderate transmutation; CL 11th; Craft Wondrous Item,\textit{ animate objects}.
				
\textbf{Ivory Goats}: These figurines come in threes. Each goat of this trio looks slightly different from the others, and each has a different function:
				\begin{itemize}\item  \textit{The Goat of Traveling}: This statuette provides a speedy and enduring mount equal to that of a heavy horse in every way except appearance. The goat can travel for a maximum of 1 day each week---continuously or in any combination of periods totaling 24 hours. At this point, or when the command word is uttered, it returns to its statuette form for no less than 1 day before it can again be used.
				\item  \textit{The Goat of Travail}: This statuette becomes an enormous creature, larger than a bull, with the statistics of a nightmare except for the addition of a pair of wicked horns of exceptional size (damage 1d8+4 for each horn). If it is charging to attack, it may only use its horns (but add 6 points of damage to each successful attack in that round). It can be called to life just once per month for up to 12 hours at a time.
				\item  \textit{The Goat of Terror}: When called upon with the proper command word, this statuette becomes a destrier-like mount with the statistics of a light horse. However, its rider can employ the goat's horns as weapons (one horn as a \textit{+3 heavy lance,} the other as a \textit{+5 longsword}). When ridden in an attack against an opponent, the goat of terror radiates \textit{fear} as the spell in a 30-foot radius (Will DC 16 partial). It can be used once every 2 weeks for up to 3 hours per use. Moderate transmutation; CL 11th; Craft Wondrous Item\textit{, animate objects}.
\end{itemize}
				
\textbf{Marble Elephant}: This is the largest of the figurines, the statuette being about the size of a human hand. Upon utterance of the command word, a marble elephant grows to the size and specifications of a true elephant. The animal created from the statuette is fully obedient to the figurine's owner, serving as a beast of burden, a mount, or a combatant. The statuette can be used four times per month for up to 24 hours at a time. Moderate transmutation; CL 11th; Craft Wondrous Item,\textit{ animate objects}.
				
\textbf{Obsidian Steed}: This figurine appears to be a small, shapeless lump of black stone. Only careful inspection reveals that it vaguely resembles some form of quadruped. On command, the near-formless piece of obsidian becomes a fantastic mount. Treat it as a heavy horse with the following additional powers usable once per round at will: \textit{overland flight, plane shift,} and \textit{ethereal jaunt.} The steed allows itself to be ridden, but if the rider is of good alignment, the steed is 10\% likely per use to carry him to the lower planes and then return to its statuette form. The statuette can be used once per week for one continuous period of up to 24 hours. Note that when an obsidian steed becomes ethereal or plane shifts, its rider and his gear follow suit. Thus, the user can travel to other planes via this means. Strong conjuration and transmutation; CL 15th; Craft Wondrous Item,\textit{ animate objects, etherealness, fly, plane shift}.
				
\textbf{Onyx Dog}: When commanded, this statuette changes into a creature with the same properties as a riding dog except that it is endowed with an Intelligence of 8, can communicate in Common, and has exceptional olfactory and visual abilities. It has the scent ability and adds +4 on its Perception checks. It has 60-foot darkvision, and it can \textit{see invisibility}. An onyx dog can be used once per week for up to 6 hours. It obeys only its owner. Moderate transmutation; CL 11th; Craft Wondrous Item,\textit{ animate objects}.
				
\textbf{Serpentine Owl}: This figurine becomes either a normal-sized horned owl or a giant owl (use the stats for the giant eagle) according to the command word used. The transformation can take place once per day, with a maximum duration of 8 continuous hours. However, after three transformations into giant owl form, the statuette loses all its magical properties. The owl communicates with its owner by telepathic means, informing her of all it sees and hears. Moderate transmutation; CL 11th; Craft Wondrous Item,\textit{ animate objects}.
				
\textbf{Silver Raven}: This silver figurine turns into a raven on command (but it retains its metallic consistency, which gives it hardness 10). Another command sends it off into the air, bearing a message just like a creature affected by an \textit{animal messenger} spell. If not commanded to carry a message, the raven obeys the commands of its owner, although it has no special powers or telepathic abilities. It can maintain its nonfigurine status for only 24 hours per week, but the duration need not be continuous. Moderate enchantment and transmutation; CL 6th; Craft Wondrous Item, \textit{animal messenger, animate objects}.
				
Construction
				
\textbf{Requirements} Craft Wondrous Item, \textit{animate objects, }additional spells, see text\textit{;}\textbf{ Cost }5,000 gp (bronze griffon), 5,000 gp (ebony fly), 8,250 gp (golden lions), 10,500 gp (ivory goats), 8,500 gp (marble elephant), 14,250 gp (obsidian steed), 7,750 gp (onyx dog), 4,550 gp (serpentine owl), 1,900 gp (silver raven)
				
\textbf{Gauntlet of Rust}
				
\textbf{Aura} moderate transmutation;\textbf{ CL }7th
				
\textbf{Slot} hands; \textbf{Price} 11,500 gp; \textbf{Weight} 2 lbs.
				
Description
				
This single metal gauntlet looks rusted and pitted but is actually quite powerful. Once per day, it can affect an object as with the \textit{rusting grasp} spell. It also completely protects the wearer and her gear from rust (magical or otherwise), including the attack of a rust monster. 
				
Construction
				
\textbf{Requirements} Craft Wondrous Item\textit{, rusting grasp;}\textbf{ Cost }5,750 gp
				
\textbf{Gem of Brightness}
				
\textbf{Aura} moderate evocation;\textbf{ CL }6th
				
\textbf{Slot} none; \textbf{Price} 13,000 gp; \textbf{Weight }---
				
Description
				
This crystal appears to be a long, rough prism. Upon utterance of a command word, though, the gem's facets suddenly grow highly polished as the crystal emits bright light of one of three sorts.
				\begin{itemize}\item  One command word causes the gem to shed light as a hooded lantern. This use of the gem does not expend any charges, and it continues to emit light until this command word is spoken a second time to extinguish the illumination.
				\item  Another command word causes the \textit{gem of brightness} to send out a bright ray 1 foot in diameter and 50 feet long. This strikes as a ranged touch attack, and any creature struck by this beam is blinded for 1d4 rounds unless it makes a DC 14 Fortitude save. This use of the gem expends 1 charge.
				\item  The third command word causes the gem to flare in a blinding flash of light that fills a 30-foot cone. Although this glare lasts but a moment, any creature within the cone must make a DC 14 Fortitude save or be blinded for 1d4 rounds. This use expends 5 charges.
\end{itemize}
				
A newly created \textit{gem of brightness} has 50 charges. When all its charges are expended, the gem becomes nonmagical and its facets grow cloudy with a fine network of cracks. 
				
Construction
				
\textbf{Requirements} Craft Wondrous Item,\textit{ daylight;}\textbf{ Cost }6,500 gp
				
\textbf{Gem of Seeing}
				
\textbf{Aura} moderate divination;\textbf{ CL }10th
				
\textbf{Slot} none; \textbf{Price} 75,000 gp; \textbf{Weight }---
				
Description
				
This finely cut and polished stone is indistinguishable from an ordinary jewel in appearance. When it is gazed through, a \textit{gem of seeing} enables the user to see as though she were affected by the \textit{true seeing} spell. A \textit{gem of seeing} can be used for as many as 30 minutes a day, in increments of 5 minutes. These increments do not need to be consecutive.
				
Construction
				
\textbf{Requirements} Craft Wondrous Item\textit{, true seeing;}\textbf{ Cost }37,500 gp
				
\textbf{Gloves of Arrow Snaring}
				
\textbf{Aura} faint abjuration; \textbf{CL} 3rd
				
\textbf{Slot} hands; \textbf{Price} 4,000 gp; \textbf{Weight }---
				
Description
				
Once worn, these snug gloves seem to meld with the hands, becoming almost invisible to casual observation. Twice per day, the wearer can act as if he had the Snatch Arrows feat (see Feats for details), even if he does not meet the prerequisites for the feat. Both gloves must be worn for the magic to be effective, and at least one hand must be free to take advantage of the magic. 
				
Construction
				
\textbf{Requirements} Craft Wondrous Item,\textit{ shield;}\textbf{ Cost }2,000 gp
				
\textbf{Glove of Storing}
				
\textbf{Aura} moderate transmutation;\textbf{ CL }6th
				
\textbf{Slot} hands; \textbf{Price} 10,000 gp (one glove); \textbf{Weight }---
				
Description
				
This device is a single leather glove. On command, one item held in the hand wearing the glove disappears. The item can weigh no more than 20 pounds and must be able to be held in one hand. While stored, the item has negligible weight. With a snap of the fingers wearing the glove, the item reappears. A glove can only store one item at a time. Storing or retrieving the item is a free action. The item is shrunk down so small within the palm of the glove that it cannot be seen. Spell durations are not suppressed, but continue to expire. If the glove's effect is suppressed or dispelled, the stored item appears instantly. A \textit{glove of storing }uses up your entire hands slot. You may not use another item (even another \textit{glove of storing}) that also uses the hands slot.
				
Construction
				
\textbf{Requirements} Craft Wondrous Item,\textit{ shrink item;}\textbf{ Cost }5,000 gp
				
\textbf{Gloves of Swimming and Climbing}
				
\textbf{Aura} faint transmutation;\textbf{ CL }5th
				
\textbf{Slot} hands; \textbf{Price} 6,250 gp; \textbf{Weight }---
				
Description
				
These apparently normal lightweight gloves grant a +5 competence bonus on Swim checks and Climb checks. Both gloves must be worn for the magic to be effective. 
				
Construction
				
\textbf{Requirements} Craft Wondrous Item\textit{, bull's strength, cat's grace;}\textbf{ Cost }3,125 gp
				
\textbf{Goggles of Minute Seeing}
				
\textbf{Aura} faint divination;\textbf{ CL }3rd
				
\textbf{Slot} eyes; \textbf{Price }2,500 gp; \textbf{Weight }---
				
Description
				
The lenses of this item are made of special crystal. When placed over the eyes of the wearer, the lenses enable her to see much better than normal at distances of 1 foot or less, granting her a +5 competence bonus on Disable Device checks. Both lenses must be worn for the magic to be effective. 
				
Construction
				
\textbf{Requirements} Craft Wondrous Item\textit{, true seeing;}\textbf{ Cost }1,250 gp
				
\textbf{Goggles of Night}
				
\textbf{Aura} faint transmutation;\textbf{ CL }3rd
				
\textbf{Slot} eyes; \textbf{Price} 12,000 gp; \textbf{Weight }---
				
Description
				
The lenses of this item are made of dark crystal. Even though the lenses are opaque, when placed over the eyes of the wearer, they enable him to see normally and also grant him 60-foot darkvision. Both lenses must be worn for the magic to be effective. 
				
Construction
				
\textbf{Requirements} Craft Wondrous Item\textit{, darkvision;}\textbf{ Cost }6,000 gp
				
\textbf{Golem Manual}
				
\textbf{Aura} varies;\textbf{ CL }varies
				
\textbf{Slot} none; \textbf{Price} 12,000 gp (clay), 8,000 gp (flesh), 35,000 gp (iron), 22,000 gp (stone), 44,000 gp (stone guardian); \textbf{Weight} 5 lbs.
				
Description
				
A \textit{golem manual} contains information, incantations, and magical power that help a character to craft a golem. The instructions therein grant a +5 competence bonus on skill checks made to craft the golem's body. Each manual also holds the prerequisite spells needed for a specific golem (although these spells can only be used to create a golem and cannot be copied), effectively granting the builder use of the Craft Construct feat during the construction of the golem, and an increase to her caster level for the purpose of crafting a golem.
				
The spells included in a \textit{golem manual} require a spell trigger activation and can be activated only to assist in the construction of a golem. The cost of the book does not include the cost of constructing the golem's body. Once the golem is finished, the writing in the manual fades and the book is consumed in flames. When the book's ashes are sprinkled upon the golem, it becomes fully animated.
				
\textit{Clay Golem Manual}: The book contains \textit{animate objects, bless, commune, prayer, }and \textit{resurrection}. The reader may treat her caster level as two levels higher than normal for the purpose of crafting a clay golem. Moderate conjuration, divination, enchantment, and transmutation; CL 11th; Craft Construct, creator must be caster level 11th,\textit{ animate objects, commune, prayer, resurrection.}
				
\textit{Flesh Golem Manual}: The book contains \textit{animate dead, bull's strength, geas/quest,} and \textit{limited wish}. The reader may treat her caster level as one level higher than normal for the purpose of crafting a flesh golem. Moderate enchantment, necromancy \mbox{$[$}evil\mbox{$]$}, and transmutation; CL 8th; Craft Construct, creator must be caster level 8th,\textit{ animate dead, bull's strength, geas/quest, limited wish.}
				
\textit{Iron Golem Manual}: The book contains \textit{cloudkill, geas/quest, limited wish,} and \textit{polymorph any object.} The reader may treat her caster level as four levels higher than normal for the purpose of crafting an iron golem. Strong conjuration, enchantment, and transmutation; CL 16th; Craft Construct, creator must be caster level 16th,\textit{ cloudkill, geas/quest, limited wish, polymorph any object.}
				
\textit{Stone Golem Manual}: The book contains \textit{geas/quest, limited wish, polymorph any object,} and \textit{slow}. The reader may treat her caster level as three levels higher than normal for the purpose of crafting a stone golem. Strong abjuration and enchantment; CL 14th; Craft Construct, creator must be caster level 14th,\textit{ antimagic field, geas/quest, limited wish, symbol of stunning.}
				
\textit{Stone Golem Guardian Manual}: The book contains\textit{ geas/quest, imbue with spell-like ability, limited wish, polymorph any object, shield other, }and\textit{ slow. }The reader may treat her caster level as three levels higher than normal for the purpose of crafting a stone golem guardian. Strong abjuration and enchantment; CL 14th; Craft Construct, creator must be caster level 14th,\textit{ antimagic field, geas/quest, imbue with spell-like ability, limited wish, polymorph any object, shield other, slow.}
				
Construction
				
\textbf{Requirements} Craft Construct, caster must be of a specific level, additional spells\textit{; }\textbf{Cost }6,000 gp (clay), 4,000 gp (flesh), 17,500 gp (iron), 11,000 gp (stone), 22,000 gp (stone guardian)
				
\textbf{Hand of Glory}
				
\textbf{Aura} faint varied;\textbf{ CL }5th
				
\textbf{Slot} neck; \textbf{Price} 8,000 gp; \textbf{Weight} 2 lbs.
				
Description
				
This mummified human hand hangs by a leather cord around a character's neck (taking up space as a magic necklace would). If a magic ring is placed on one of the fingers of the hand, the wearer benefits from the ring as if wearing it herself, and it does not count against her two-ring limit. The hand can wear only one ring at a time. Even without a ring, the hand itself allows its wearer to use \textit{daylight} and \textit{see invisibility} each once per day. 
				
Construction
				
\textbf{Requirements} Craft Wondrous Item\textit{, animate dead, daylight, see invisibility;}\textbf{ Cost }4,000 gp
				
\textbf{Hand of the Mage}
				
\textbf{Aura} faint transmutation;\textbf{ CL }2nd
				
\textbf{Slot} neck; \textbf{Price} 900 gp; \textbf{Weight} 2 lbs.
				
Description
				
This mummified elf hand hangs by a golden chain around a character's neck (taking up space as a magic necklace would). It allows the wearer to utilize the spell \textit{mage hand} at will. 
				
Construction
				
\textbf{Requirements} Craft Wondrous Item,\textit{ mage hand;}\textbf{ Cost }450 gp
				
\textbf{Handy Haversack}
				
\textbf{Aura} moderate conjuration;\textbf{ CL }9th
				
\textbf{Slot} none; \textbf{Price} 2,000 gp; \textbf{Weight} 5 lbs.
				
Description
				
A backpack of this sort appears to be well made, well used, and quite ordinary. It is constructed of finely tanned leather, and the straps have brass hardware and buckles. It has two side pouches, each of which appears large enough to hold about a quart of material. In fact, each is like a \textit{bag of holding} and can actually hold material of as much as 2 cubic feet in volume or 20 pounds in weight. The large central portion of the pack can contain up to 8 cubic feet or 80 pounds of material. Even when so filled, the backpack always weighs only 5 pounds.
				
While such storage is useful enough, the pack has an even greater power. When the wearer reaches into it for a specific item, that item is always on top. Thus, no digging around and fumbling is ever necessary to find what a haversack contains. Retrieving any specific item from a haversack is a move action, but it does not provoke the attacks of opportunity that retrieving a stored item usually does. 
				
Construction
				
\textbf{Requirements} Craft Wondrous Item,\textit{ secret chest;}\textbf{ Cost }1,000 gp
				
\textbf{Harp of Charming}
				
\textbf{Aura} faint enchantment;\textbf{ CL }5th
				
\textbf{Slot} none; \textbf{Price} 7,500 gp; \textbf{Weight} 5 lbs.
				
Description
				
This beautiful and intricately carved harp can be held comfortably in one hand, but both hands are required to utilize its magic. When played, a \textit{harp of charming} enables the performer to work one \textit{suggestion }(as the spell, Will DC 14 negates) into the music for each 10 minutes of playing if he can succeed on a DC 14 Perform (string instruments) check. If the check fails, the audience cannot be affected by any further performances from the harpist for 24 hours. 
				
Construction
				
\textbf{Requirements} Craft Wondrous Item,\textit{ suggestion;}\textbf{ Cost }3,750 gp
				
\textbf{Hat of Disguise}
				
\textbf{Aura} faint illusion;\textbf{ CL }1st
				
\textbf{Slot} head; \textbf{Price} 1,800 gp; \textbf{Weight }---
				
Description
				
This apparently normal hat allows its wearer to alter her appearance as with a \textit{disguise self} spell. As part of the disguise, the hat can be changed to appear as a comb, ribbon, headband, cap, coif, hood, helmet, and so on. 
				
Construction
				
\textbf{Requirements} Craft Wondrous Item,\textit{ disguise self;}\textbf{ Cost }900 gp
				
\textbf{Headband of Alluring Charisma}
				
\textbf{Aura} moderate transmutation; \textbf{CL} 8th
				
\textbf{Slot }headband; \textbf{Price} 4,000 gp (+2), 16,000 gp (+4), 36,000 gp (+6); \textbf{Weight} 1 lb.
				
Description
				
This attractive silver headband is decorated with a number of small red and orange gemstones. The headband grants the wearer an enhancement bonus to Charisma of +2, +4, or +6. Treat this as a temporary ability bonus for the first 24 hours the headband is worn.
				
Construction
				
\textbf{Requirements }Craft Wondrous Item, \textit{eagle's splendor}; \textbf{Cost} 2,000 gp (+2), 8,000 gp (+4), 18,000 gp (+6)
				
\textbf{Headband of Inspired Wisdom}
				
\textbf{Aura} moderate transmutation; \textbf{CL} 8th
				
\textbf{Slot }headband; \textbf{Price} 4,000 gp (+2), 16,000 gp (+4), 36,000 gp (+6); \textbf{Weight} 1 lb.
				
Description
				
This simple bronze headband is decorated with an intricate pattern of small green gemstones. The headband grants the wearer an enhancement bonus to Wisdom of +2, +4, or +6. Treat this as a temporary ability bonus for the first 24 hours the headband is worn.
				
Construction
				
\textbf{Requirements }Craft Wondrous Item, \textit{owl's wisdom}; \textbf{Cost} 2,000 gp (+2), 8,000 gp (+4), 18,000 gp (+6)
				
\textbf{Headband of Mental Prowess}
				
\textbf{Aura} strong transmutation; \textbf{CL} 12th
				
\textbf{Slot }headband; \textbf{Price} 10,000 gp (+2), 40,000 gp (+4), 90,000 gp (+6); \textbf{Weight} 1 lb.
				
Description
				
This simple copper headband has a small yellow gem set so that when it rests upon the forehead of the wearer, the yellow gem sits perched on the wearer's brow as if it were a third eye in the middle of his forehead. Often, the headband contains additional designs to further accentuate the appearance of a third, crystal eye.
				
The headband grants the wearer an enhancement bonus to two mental ability scores (Intelligence, Wisdom, or Charisma) of +2, +4, or +6. Treat this as a temporary ability bonus for the first 24 hours the headband is worn. These bonuses are chosen when the headband is created and cannot be changed. If the headband grants a bonus to Intelligence, it also grants skill ranks as a \textit{headband of vast intelligence.}
				
Construction
				
\textbf{Requirements }Craft Wondrous Item, \textit{eagle's splendor, fox's cunning, }and/or \textit{owl's wisdom}; \textbf{Cost} 5,000 gp (+2), 20,000 gp (+4), 45,000 gp (+6)
				
\textbf{headband of Mental Superiority}
				
\textbf{Aura} strong transmutation; \textbf{CL} 16th
				
\textbf{Slot }headband; \textbf{Price} 16,000 gp (+2), 64,000 gp (+4), 144,000 gp (+6); \textbf{Weight} 1 lb.
				
Description
				
This ornate headband is decorated with numerous small white gemstones. The headband grants the wearer an enhancement bonus to all mental ability scores (Intelligence, Wisdom, and Charisma) of +2, +4, or +6. Treat this as a temporary ability bonus for the first 24 hours the headband is worn. The headband also grants skill ranks as a \textit{headband of vast intelligence.}
				
Construction
				
\textbf{Requirements} Craft Wondrous Item, \textit{eagle's splendor, fox's cunning, owl's wisdom}; \textbf{Cost} 8,000 gp (+2), 32,000 gp (+4), 77,000 gp (+6) 
				
\textbf{Headband of Vast Intelligence}
				
\textbf{Aura} moderate transmutation; \textbf{CL} 8th
				
\textbf{Slot }headband; \textbf{Price} 4,000 gp (+2), 16,000 gp (+4), 36,000 gp (+6); \textbf{Weight} 1 lb.
				
Description
				
This intricate gold headband is decorated with several small blue and deep purple gemstones. The headband grants the wearer an enhancement bonus to Intelligence of +2, +4, or +6. Treat this as a temporary ability bonus for the first 24 hours the headband is worn. A \textit{headband of vast intelligence }has one skill associated with it per +2 bonus it grants. After being worn for 24 hours, the headband grants a number of skill ranks in those skills equal to the wearer's total Hit Dice. These ranks do not stack with the ranks a creature already possesses. These skills are chosen when the headband is created. If no skill is listed, the headband is assumed to grant skill ranks in randomly determined Knowledge skills.
				
Construction
				
\textbf{Requirements }Craft Wondrous Item, \textit{fox's cunning}; \textbf{Cost} 2,000 gp (+2), 8,000 gp (+4), 18,000 gp (+6)
				
\textbf{Helm of Brilliance}
				
\textbf{Aura} strong varied;\textbf{ CL }13th
				
\textbf{Slot} head; \textbf{Price} 125,000 gp; \textbf{Weight} 3 lbs.
				
Description
				
This normal-looking helm takes its true form and manifests its powers when the user dons it and speaks the command word. Made of brilliant silver and polished steel, a newly created helm is set with large magic gems: 10 diamonds, 20 rubies, 30 fire opals, and 40 opals. When struck by bright light, the helm scintillates and sends forth reflective rays in all directions from its crown-like, gem-tipped spikes. The jewels' functions are as follows:
				\begin{itemize}\item  Diamond: Prismatic spray (save DC 20)
				\item  Ruby: \textit{Wall of fire}
				\item  Fire opal: \textit{Fireball} (10d6, Reflex DC 20 half)
				\item  Opal: \textit{Daylight}
\end{itemize}
				
The helm may be used once per round, but each gem can perform its spell-like power just once. Until all its jewels are depleted, a \textit{helm of brilliance} also has the following magical properties when activated.
				\begin{itemize}\item  It emanates a bluish light when undead are within 30 feet. This light causes 1d6 points of damage per round to all such creatures within that range.
				\item  The wearer may command any weapon he wields to become a \textit{flaming weapon}. This is in addition to whatever abilities the weapon may already have (unless the weapon already is a \textit{flaming weapon}). The command takes 1 round to take effect.
				\item  The helm provides fire resistance 30. This protection does not stack with similar protection from other sources.
\end{itemize}
				
Once all its jewels have lost their magic, the helm loses its powers and the gems turn to worthless powder. Removing a jewel destroys it.
				
If a creature wearing the helm is damaged by magical fire (after the fire protection is taken into account) and fails an additional DC 15 Will save, the remaining gems on the helm overload and detonate. Remaining diamonds become \textit{prismatic sprays} that each randomly target a creature within range (possibly the wearer), rubies become straight-line \textit{walls of fire} extending outward in a random direction from the helm wearer, and fire opals become \textit{fireballs }centered on the helm wearer. The opals and the helm itself are destroyed. 
				
Construction
				
\textbf{Requirements} Craft Wondrous Item,\textit{ detect undead, fireball, flame blade, daylight, prismatic spray, protection from energy, wall of fire;}\textbf{ Cost }62,500 gp
				
\textbf{Helm of Comprehend Languages and Read Magic}
				
\textbf{Aura} faint divination;\textbf{ CL }4th
				
\textbf{Slot} head; \textbf{Price} 5,200 gp; \textbf{Weight} 3 lbs.
				
Description
				
Appearing as a normal helmet, a \textit{helm of comprehend languages and read magic} grants its wearer the ability to understand the spoken words of any creature and to read text in any language and any magical writing. The wearer gains a +5 competence bonus on Linguistics checks to understand messages written in incomplete, archaic, or exotic forms. Note that understanding a magical text does not necessarily imply spell use. 
				
Construction
				
\textbf{Requirements} Craft Wondrous Item,\textit{ comprehend languages, read magic;}\textbf{ Cost }2,600 gp
				
\textbf{Helm of Telepathy}
				
\textbf{Aura} faint divination and enchantment;\textbf{ CL }5th
				
\textbf{Slot} head; \textbf{Price} 27,000 gp; \textbf{Weight} 3 lbs.
				
Description
				
This pale metal or ivory helm covers much of the head when worn. The wearer can use\textit{ detect thoughts }at will. Furthermore, he can send a telepathic message to anyone whose surface thoughts he is reading (allowing two-way communication). Once per day, the wearer of the helm can implant a\textit{ suggestion }(as the spell, Will DC 14 negates) along with his telepathic message. 
				
Construction
				
\textbf{Requirements} Craft Wondrous Item,\textit{ detect thoughts, suggestion;}\textbf{ Cost }13,500 gp
				
\textbf{Helm of Teleportation}
				
\textbf{Aura} moderate conjuration;\textbf{ CL }9th
				
\textbf{Slot} head; \textbf{Price} 73,500 gp; \textbf{Weight} 3 lbs.
				
Description
				
A character wearing this device may \textit{teleport} three times per day, exactly as if he had cast the spell of the same name. 
				
Construction
				
\textbf{Requirements} Craft Wondrous Item,\textit{ teleport;}\textbf{ Cost }36,750 gp
				
\textbf{Helm of Underwater Action}
				
\textbf{Aura} faint transmutation; \textbf{CL} 5th
				
\textbf{Slot }head; \textbf{Price} 24,000 gp; \textbf{Weight }3 lbs.
				
Description
				
The wearer of this helmet can see underwater. Drawing the small lenses in compartments on either side into position before the wearer's eyes activates the visual properties of the helm, allowing her to see five times farther than water and light conditions would allow for normal human vision. (Weeds, obstructions, and the like block vision in the usual manner.) If the command word is spoken, the \textit{helm of underwater action }gives the wearer a 30-foot swim speed and creates a globe of air around the wearer's head and maintains it until the command word is spoken again, enabling her to breathe freely.
				
Construction
				
\textbf{Requirements }Craft Wondrous Item, \textit{water breathing}; \textbf{Cost} 12,000 gp
				
\textbf{Horn of Blasting}
				
\textbf{Aura} moderate evocation;\textbf{ CL }7th
				
\textbf{Slot} none; \textbf{Price} 20,000 gp; \textbf{Weight} 1 lb.
				
Description
				
This horn appears to be a normal trumpet. It can be sounded as a normal horn, but if the command word is spoken and the instrument is then played, it deals 5d6 points of sonic damage to creatures within a 40-foot cone and causes them to be deafened for 2d6 rounds (a DC 16 Fortitude save reduces the damage by half and negates the deafening). Crystalline objects and creatures take 7d6 points of sonic damage, with no save unless they're held, worn, or carried by creatures (Fortitude DC 16 negates).
				
If a \textit{horn of blasting} is used magically more than once in a given day, there is a 20\% cumulative chance with each extra use that it explodes and deals 10d6 points of sonic damage to the person sounding it. 
				
Construction
				
\textbf{Requirements} Craft Wondrous Item,\textit{ shout;}\textbf{ Cost }10,000 gp
				
\textbf{Horn of Blasting, Greater}
				
\textbf{Aura} strong evocation;\textbf{ CL }16th
				
\textbf{Slot} none; \textbf{Price} 70,000 gp; \textbf{Weight} 1 lb.
				
Description
				
This horn functions as a \textit{horn of blasting,} except that it deals 10d6 points of sonic damage, stuns creatures for 1 round, and deafens them for 4d6 rounds (a DC 19 Fortitude reduces the damage by half and negates the stunning and deafening). Crystalline objects take 16d6 points of sonic damage as described for the \textit{horn of blasting.} A \textit{greater horn of blasting} also has a 20\% cumulative chance of exploding for each usage beyond the first each day. 
				
Construction
				
\textbf{Requirements} Craft Wondrous Item,\textit{ greater shout;}\textbf{ Cost }35,000 gp
				
\textbf{Horn of Fog}
				
\textbf{Aura} faint conjuration;\textbf{ CL }3rd
				
\textbf{Slot} none; \textbf{Price} 2,000 gp; \textbf{Weight} 1 lb.
				
Description
				
This small bugle allows its possessor to blow forth a thick cloud of heavy fog similar to that of an \textit{obscuring mist} spell. The fog covers a 10-foot square next to the horn blower each round that the user continues to blow the horn; a fog cloud travels 10 feet each round in a straight line from the emanation point unless blocked by something substantial such as a wall. The device makes a deep, foghorn-like noise, with the note dropping abruptly to a lower register at the end of each blast. The fog dissipates after 3 minutes. A moderate wind (11+ mph) disperses the fog in 4 rounds; a strong wind (21+ mph) disperses the fog in 1 round. 
				
Construction
				
\textbf{Requirements} Craft Wondrous Item,\textit{ obscuring mist;}\textbf{ Cost }1,000 gp
				
\textbf{Horn of Goodness/Evil}
				
\textbf{Aura} moderate abjuration;\textbf{ CL }6th
				
\textbf{Slot} none; \textbf{Price} 6,500 gp; \textbf{Weight} 1 lb.
				
Description
				
This trumpet adapts itself to its owner, so it produces either a good or an evil effect depending on the owner's alignment. If the owner is neither good nor evil, the horn has no power whatsoever. If he is good, then blowing the horn has the effect of a \textit{magic circle against evil}. If he is evil, then blowing the horn has the effect of a \textit{magic circle against good}. In either case, this ward lasts for 1 hour. The horn can be blown once per day. 
				
Construction
				
\textbf{Requirements} Craft Wondrous Item,\textit{ magic circle against good, magic circle against evil;}\textbf{ Cost }3,250 gp
				
\textbf{Horn of Valhalla}
				
\textbf{Aura} strong conjuration;\textbf{ CL }13th
				
\textbf{Slot} none; \textbf{Price} 50,000 gp; \textbf{Weight} 2 lbs.
				
Description
				
This magic instrument comes in four varieties. Each appears to be normal until someone speaks its command word and blows the horn. Then the horn summons a number of human barbarians to fight for the character who summoned them. Each horn can be blown just once every 7 days. Roll d\% and refer to the table above to see what type of horn is found. The horn's type determines what barbarians are summoned and what prerequisite is needed to use the horn. Any character who uses a \textit{horn of Valhalla} but doesn't have the prerequisite is attacked by the barbarians she herself summoned.

\begin{figure*}
\caption{Horn of Valhalla}  % TODO: don't index this table
\sffamily
\begin{tabular}{llll}
\textbf{d\%} & \textbf{Type of Horn} & \textbf{Barbarians Summoned} & \textbf{Prerequisite}                                              \\
01--40        & Silver                & 2d4+2, 2nd-level             & None                                                               \\
41--75        & Brass                 & 2d4+1, 3rd-level             & Spellcaster level 1st                                              \\
76--90        & Bronze                & 2d4, 4th-level               & Proficiency with all martial weapons or bardic performance ability \\
91--100       & Iron                  & 1d4+1, 5th-level             & Proficiency with all martial weapons or bardic performance ability
\end{tabular}
\end{figure*}
Summoned barbarians are constructs, not actual people (though they seem to be); they arrive with the starting equipment for barbarians. They attack anyone the possessor of the horn commands them to fight until they or their opponents are slain or until 1 hour has elapsed, whichever comes first. 
				
Construction
				
\textbf{Requirements} Craft Wondrous Item,\textit{ summon monster VI;}\textbf{ Cost }25,000 gp
				
\textbf{Horseshoes of Speed}
				
\textbf{Aura} faint transmutation;\textbf{ CL }3rd
				
\textbf{Slot} feet; \textbf{Price} 3,000 gp; \textbf{Weight} 12 lbs. (for four)
				
Description
				
These iron shoes come in sets of four like ordinary horseshoes. When affixed to an animal's hooves, they increase the animal's base land speed by 30 feet; this counts as an enhancement bonus. As with other effects that increase speed, jumping distances increase proportionally (see Using Skills). All four shoes must be worn by the same animal for the magic to be effective. 
				
Construction
				
\textbf{Requirements} Craft Wondrous Item\textit{, haste;}\textbf{ Cost }1,500 gp
				
\textbf{Horseshoes of a Zephyr}
				
\textbf{Aura} faint transmutation;\textbf{ CL }3rd
				
\textbf{Slot} feet; \textbf{Price} 6,000 gp; \textbf{Weight} 4 lbs. (for four).
				
Description
				
These four iron shoes are affixed like normal horseshoes. They allow a horse to travel without actually touching the ground. The horse must still run above (always around 4 inches above) a roughly horizontal surface. This means that non-solid or unstable surfaces can be crossed, and that movement is possible without leaving tracks on any sort of ground. The horse moves at its normal base land speed. All four shoes must be worn by the same animal for the magic to be effective. 
				
Construction
				
\textbf{Requirements} Craft Wondrous Item, \textit{levitate;}\textbf{ Cost }3,000 gp
				
\textbf{Incense of Meditation}
				
\textbf{Aura} moderate enchantment;\textbf{ CL }7th
				
\textbf{Slot} none; \textbf{Price} 4,900 gp; \textbf{Weight} 1 lb.
				
Description
				
This small rectangular block of sweet-smelling incense is visually indistinguishable from nonmagical incense until lit. When it is burned, the special fragrance and pearly hued smoke of this special incense are recognizable by anyone making a DC 15 Spellcraft check.
				
When a divine spellcaster lights a block of \textit{incense of meditation} and then spends 8 hours praying and meditating nearby, the incense enables him to prepare all his spells as though affected by the Maximize Spell feat. However, all the spells prepared in this way are at their normal level, not at three levels higher (as with the regular metamagic feat).
				
Each block of incense burns for 8 hours, and the effects persist for 24 hours. 
				
Construction
				
\textbf{Requirements} Craft Wondrous Item, Maximize Spell,\textit{ bless;}\textbf{ Cost }2,450 gp
				
\textbf{Instant Fortress}
				
\textbf{Aura} strong conjuration;\textbf{ CL }13th
				
\textbf{Slot} none; \textbf{Price} 55,000 gp; \textbf{Weight }1 lb.
				
Description
				
This metal cube is small, but when activated by speaking a command word it grows to form a tower 20 feet square and 30 feet high, with arrow slits on all sides and a crenellated battlement atop it. The metal walls extend 10 feet into the ground, rooting it to the spot and preventing it from being tipped over. The fortress has a small door that opens only at the command of the owner of the fortress---even \textit{knock} spells can't open the door.
				
The adamantine walls of an \textit{instant fortress} have 100 hit points and hardness 20. The fortress cannot be repaired except by a \textit{wish} or a \textit{miracle}, which restores 50 points of damage taken.
				
The fortress springs up in just 1 round, with the door facing the device's owner. The door opens and closes instantly at his command. People and creatures nearby (except the owner) must be careful not to be caught by the fortress's sudden growth. Anyone so caught takes 10d10 points of damage (Reflex DC 19 half).
				
The fortress is deactivated by speaking a command word (different from the one used to activate it). It cannot be deactivated unless it is empty. 
				
Construction
				
\textbf{Requirements} Craft Wondrous Item,\textit{ mage's magnificent mansion;}\textbf{ Cost }27,500 gp
				
\textbf{Ioun Stones}
				
\textbf{Aura} strong varied;\textbf{ CL }12th
				
\textbf{Slot} none; \textbf{Price} varies; \textbf{Weight }---
				
Description
				
These crystalline stones always float in the air and must be within 3 feet of their owner to be of any use. When a character first acquires a stone, she must hold it and then release it, whereupon it takes up a circling orbit 1d3 feet from her head. Thereafter, a stone must be grasped or netted to separate it from its owner. The owner may voluntarily seize and stow a stone (to keep it safe while she is sleeping, for example), but she loses the benefits of the stone during that time. Ioun stones have AC 24, 10 hit points, and hardness 5. The powers of each stone vary depending on its color and shape (see the table).
				

\begin{figure*}[]
\sffamily
\caption{Ioun Stones}
\begin{tabular}{llll}
\textbf{Color}     & \textbf{Shape} & \textbf{Effect}                                                              & \textbf{Market Price} \\
Clear              & Spindle        & Sustains creature without food or water                                      & 4,000 gp              \\
Dusty rose         & Prism          & +1 insight bonus to AC                                                       & 5,000 gp              \\
Deep red           & Sphere         & +2 enhancement bonus to Dexterity                                            & 8,000 gp              \\
Incandescent blue  & Sphere         & +2 enhancement bonus to Wisdom                                               & 8,000 gp              \\
Pale blue          & Rhomboid       & +2 enhancement bonus to Strength                                             & 8,000 gp              \\
Pink               & Rhomboid       & +2 enhancement bonus to Constitution                                         & 8,000 gp              \\
Pink and green     & Sphere         & +2 enhancement bonus to Charisma                                             & 8,000 gp              \\
Scarlet and blue   & Sphere         & +2 enhancement bonus to Intelligence\textsuperscript{1}                      & 8,000 gp              \\
Dark blue          & Rhomboid       & Alertness (as the feat)                                                      & 10,000 gp             \\
Iridescent         & Spindle        & Sustains creature without air                                                & 18,000 gp             \\
Pale lavender      & Ellipsoid      & Absorbs spells of 4th level or lower\textsuperscript{2}                      & 20,000 gp             \\
Pearly white       & Spindle        & Regenerate 1 point of damage per 10 minutes                                  & 20,000 gp             \\
Pale green         & Prism          & +1 competence bonus on attack rolls, saves, skill checks, and ability checks & 30,000 gp             \\
Orange             & Prism          & +1 caster level                                                              & 30,000 gp             \\
Vibrant purple     & Prism          & Stores three levels of spells, as a ring of spell storing                    & 36,000 gp             \\
Lavender and green & Ellipsoid      & Absorbs spells of 8th level or lower\textsuperscript{3}                      & 40,000 gp             \\
\end{tabular}\\
\(^{1}\) This stone has one skill associated with it, as a \emph{+2 headband of vast intelligence}\\
\(^{2}\) After absorbing 20 spell levels, the stone burns out and turns to dull gray, forever useless.\\
\(^{3}\) After absorbing 50 spell levels, the stone burns out and turns dull gray, forever useless.\\
\end{figure*}

				
Regeneration from the pearly white \textit{ioun stone} works like a \textit{ring of regeneration.} It only cures damage taken while the character is using the stone. The pale lavender and lavender-and-green stones work like a \textit{rod of absorption,} but absorbing a spell requires a readied action, and these stones cannot be used to empower spells. Stored spells in the vibrant purple stone must be placed by a spellcaster but can be used by anyone (see \textit{ring of minor spell storing}). 
				
Construction
				
\textbf{Requirements} Craft Wondrous Item, creator must be 12th level;\textbf{ Cost }half the market price
				
\textbf{Iron Bands of Binding}
				
\textbf{Aura} strong evocation;\textbf{ CL }13th
				
\textbf{Slot} none; \textbf{Price} 26,000 gp; \textbf{Weight} 1 lb.
				
Description
				
This potent item appears to be a 3-inch-diameter rusty iron sphere with bandings on the globe.
				
When the proper command word is spoken and the spherical iron device is hurled at an opponent, the bands expand and then contract to bind the target creature on a successful ranged touch attack. A single Large or smaller creature can be captured thus and held immobile (as if pinned) until the command word is spoken to bring the bands into spherical form again. The creature can break (and destroy) the bands with a DC 30 Strength check or escape them with a DC 30 combat maneuver check or Escape Artist check. \textit{Iron bands of binding} are usable once per day. 
				
Construction
				
\textbf{Requirements} Craft Wondrous Item,\textit{ grasping hand;}\textbf{ Cost }13,000 gp
				
\textbf{Iron Flask}
				
\textbf{Aura} strong conjuration;\textbf{ CL }20th
				
\textbf{Slot} none; \textbf{Price} 170,000 gp (empty); \textbf{Weight} 1 lb.
				
Description
				
These special containers are typically inlaid with runes of silver and stoppered by a brass plug bearing a seal engraved with sigils, glyphs, and special symbols. When the user speaks the command word, he can force any creature from another plane into the container, provided that creature fails a DC 19 Will save. The range of this effect is 60 feet. Only one creature at a time can be so contained. Loosing the stopper frees the captured creature. 
				
The command word can be used only once per day.
				
If the individual freeing the captured creature speaks the command word, the creature can be forced to serve for 1 hour. If freed without the command word, the creature acts according to its natural inclinations. (It usually attacks the user, unless it perceives a good reason not to.) Any attempt to force the same creature into the flask a second time provides it a +2 bonus on its saving throw and makes it hostile. A newly discovered bottle might contain any of the following:\\

\begin{tabular}{llll}
\textbf{d\%} & \textbf{Contents}     & \textbf{d\%} & \textbf{Contents}     \\
01--50        & Empty                 & 89           & Demon (glabrezu)      \\
51--54        & Large air elemental   & 90           & Demon (succubus)      \\
55--58        & Invisible stalker     & 91           & Devil (osyluth)       \\
59--62        & Large earth elemental & 92           & Devil (barbazu)       \\
63--66        & Xorn                  & 93           & Devil (erinyes)       \\
67--70        & Large fire elemental  & 94           & Devil (cornugon)      \\
71--74        & Salamander            & 95           & Agathion (avoral)     \\
75--78        & Large water elemental & 96           & Azata (ghaele)        \\
79--82        & Xill                  & 97           & Archon (trumpet)      \\
83--85        & Yeth hound            & 98           & Rakshasa              \\
86           & Demon (shadow)         & 99           & Demon (balor)         \\
87           & Demon (vrock)          & 100          & Devil (pit fiend)    \\
88           & Demon (hezrou)        \\
\end{tabular} 

Construction
				
\textbf{Requirements} Craft Wondrous Item,\textit{ trap the soul;}\textbf{ Cost }85,000 gp
				
\textbf{Lantern of Revealing}
				
\textbf{Aura} faint evocation;\textbf{ CL }5th
				
\textbf{Slot} none; \textbf{Price} 30,000 gp; \textbf{Weight} 2 lbs.
				
Description
				
This lantern operates as a normal hooded lantern. While it is lit, it also reveals all invisible creatures and objects within 25 feet of it, just like the spell \textit{invisibility purge}. 
				
Construction
				
\textbf{Requirements} Craft Wondrous Item,\textit{ invisibility purge;}\textbf{ Cost }15,000 gp
				
\textbf{Lens of Detection}
				
\textbf{Aura} moderate divination;\textbf{ CL }9th
				
\textbf{Slot} eyes; \textbf{Price} 3,500 gp; \textbf{Weight} 1 lb.
				
Description
				
This circular prism lets its user detect minute details, granting a +5 competence bonus on Perception checks. It also aids in tracking, adding a +5 competence bonus on Survival checks when tracking. The lens is about 6 inches in diameter and set in a frame with a handle. 
				
Construction
				
\textbf{Requirements} Craft Wondrous Item,\textit{ true seeing;}\textbf{ Cost }1,750 gp
				
\textbf{Lyre of Building}
				
\textbf{Aura} moderate transmutation;\textbf{ CL }6th
				
\textbf{Slot} none; \textbf{Price} 13,000 gp; \textbf{Weight} 5 lbs.
				
Description
				
This magical instrument is usually made of gold and inlaid with numerous gems. If the proper chords are struck, a single use of this lyre negates any attacks made against inanimate construction (walls, roof, floor, and so on) within 300 feet. This includes the effects of a \textit{horn of blasting}, a \textit{disintegrate} spell, or an attack from a ram or similar siege weapon. The lyre can be used in this way once per day, with the protection lasting for 30 minutes.
				
The lyre is also useful with respect to building. Once a week, its strings can be strummed so as to produce chords that magically construct buildings, mines, tunnels, ditches, etc. The effect produced in 30 minutes of playing is equal to the work of 100 humans laboring for 3 days. Each hour after the first, a character playing the lyre must make a DC 18 Perform (string instruments) check. If it fails, she must stop and cannot play the lyre again for this purpose until a week has passed. 
				
Construction
				
\textbf{Requirements} Craft Wondrous Item,\textit{ fabricate;}\textbf{ Cost }6,500 gp
				
\textbf{Mantle of Faith}
				
\textbf{Aura} strong abjuration \mbox{$[$}good\mbox{$]$};\textbf{ CL }20th
				
\textbf{Slot} chest; \textbf{Price} 76,000 gp; \textbf{Weight }---
				
Description
				
This holy garment, worn over normal clothing, grants damage reduction 5/evil to the character wearing it. 
				
Construction
				
\textbf{Requirements} Craft Wondrous Item,\textit{ stoneskin;}\textbf{ Cost }38,000 gp
				
\textbf{Mantle of Spell Resistance}
				
\textbf{Aura} moderate abjuration;\textbf{ CL }9th
				
\textbf{Slot} chest; \textbf{Price} 90,000 gp; \textbf{Weight }---
				
Description
				
This garment, worn over normal clothing or armor, grants the wearer spell resistance 21. 
				
Construction
				
\textbf{Requirements} Craft Wondrous Item,\textit{ spell resistance;}\textbf{ Cost }45,000 gp
				
\textbf{Manual of Bodily Health}
				
\textbf{Aura} strong evocation (if \textit{miracle} is used);\textbf{ CL }17th
				
\textbf{Slot} none; \textbf{Price} 27,500 gp (+1), 55,000 gp (+2), 82,500 gp (+3), 110,000 gp (+4), 137,500 gp (+5); \textbf{Weight} 5 lbs.
				
Description
				
This thick tome contains tips on health and fitness, but entwined within the words is a powerful magical effect. If anyone reads this book, which takes a total of 48 hours over a minimum of 6 days, he gains an inherent bonus from +1 to +5 (depending on the type of manual) to his Constitution score. Once the book is read, the magic disappears from the pages and it becomes a normal book. 
				
Construction
				
\textbf{Requirements} Craft Wondrous Item,\textit{ wish} or\textit{ miracle; }\textbf{Cost }26,250 gp (+1), 52,500 gp (+2), 78,750 gp (+3), 105,000 gp (+4), 131,250 gp (+5)
				
\textbf{Manual of Gainful Exercise}
				
\textbf{Aura} strong evocation (if \textit{miracle} is used);\textbf{ CL }17th
				
\textbf{Slot} none; \textbf{Price} 27,500 gp (+1), 55,000 gp (+2), 82,500 gp (+3), 110,000 gp (+4), 137,500 gp (+5); \textbf{Weight} 5 lbs.
				
Description
				
This thick tome contains exercise descriptions and diet suggestions, but entwined within the words is a powerful magical effect. If anyone reads this book, which takes a total of 48 hours over a minimum of 6 days, she gains an inherent bonus from +1 to +5 (depending on the type of manual) to her Strength score. Once the book is read, the magic disappears from the pages and it becomes a normal book. 
				
Construction
				
\textbf{Requirements} Craft Wondrous Item,\textit{ wish }or\textit{ miracle; }\textbf{Cost }26,250 gp (+1), 52,500 gp (+2), 78,750 gp (+3), 105,000 gp (+4), 131,250 gp (+5)
				
\textbf{Manual of Quickness of Action}
				
\textbf{Aura} strong evocation (if \textit{miracle} is used);\textbf{ CL }17th
				
\textbf{Slot} none; \textbf{Price} 27,500 gp (+1), 55,000 gp (+2), 82,500 gp (+3), 110,000 gp (+4), 137,500 gp (+5); \textbf{Weight} 5 lbs.
				
Description
				
This thick tome contains tips on coordination exercises and balance, but entwined within the words is a powerful magical effect. If anyone reads this book, which takes a total of 48 hours over a minimum of 6 days, he gains an inherent bonus from +1 to +5 (depending on the type of manual) to his Dexterity score. Once the book is read, the magic disappears from the pages and it becomes a normal book. 
				
Construction
				
\textbf{Requirements} Craft Wondrous Item,\textit{ wish }o\textit{r miracle; }\textbf{Cost }26,250 gp (+1), 52,500 gp (+2), 78,750 gp (+3), 105,000 gp (+4), 131,250 gp (+5)
				
\textbf{Marvelous Pigments}
				
\textbf{Aura} strong conjuration;\textbf{ CL }15th
				
\textbf{Slot} none; \textbf{Price} 4,000 gp; \textbf{Weight }---
				
Description
				
These pigments enable their possessor to create actual, permanent objects simply by depicting their form in two dimensions. The pigments are applied by a stick tipped with bristles, hair, or fur. The emulsion flows from the application to form the desired object as the artist concentrates on the image. One pot of \textit{marvelous pigments} is sufficient to create a 1,000-cubic-foot object by depicting it two-dimensionally over a 100-square-foot surface. 
				
Only normal, inanimate objects can be created. Creatures can't be created. The pigments must be applied to a surface. It takes 10 minutes and a DC 15 Craft (painting) check to depict an object with the pigments. \textit{Marvelous pigments} cannot create magic items. Objects of value depicted by the pigments---precious metals, gems, jewelry, ivory, and so on---appear to be valuable but are really made of tin, lead, glass, brass, bone, and other such inexpensive materials. The user can create normal weapons, armor, and any other mundane item (including foodstuffs) whose value does not exceed 2,000 gp. The effect is instantaneous. 
				
Construction
				
\textbf{Requirements} Craft Wondrous Item,\textit{ major creation;}\textbf{ Cost }2,000 gp
				
\textbf{Mask of the Skull}
				
\textbf{Aura} strong necromancy and transmutation;\textbf{ CL }13th
				
\textbf{Slot} head; \textbf{Price} 22,000 gp; \textbf{Weight} 3 lbs.
				
Description
				
This fearsome-looking mask of ivory, beaten copper, or pale wood is typically fashioned into the likeness of a human skull with a missing lower jaw, allowing the bottom half of the wearer's face to remain visible when the mask is worn.
				
Once per day, after it has been worn for at least 1 hour, the mask can be loosed to fly from the wearer's face. It travels up to 50 feet away from the wearer and attacks a target assigned to it. The grinning skull mask makes a touch attack against the target based on the wearer's base attack bonus. If the attack succeeds, the target must make a DC 20 Fortitude save or take 130 points of damage, as if affected by a \textit{finger of death} spell. If the target succeeds on his saving throw, he nevertheless takes 3d6+13 points of damage. After attacking (whether successful or not), the mask flies back to its user. The mask has AC 16, 10 hit points, and hardness 6. 
				
Construction
				
\textbf{Requirements} Craft Wondrous Item\textit{, animate objects, finger of death, fly;}\textbf{ Cost }11,000 gp
				
\textbf{Mattock of the Titans}
				
\textbf{Aura} strong transmutation; \textbf{CL} 16th
				
\textbf{Slot} none; \textbf{Price} 23,348 gp; \textbf{Weight} 120 lbs.
				
Description
				
This digging tool is 10 feet long. Any creature of at least Huge size can use it to loosen or tumble earth or earthen ramparts (a 10-foot cube every 10 minutes). It also smashes rock (a 10-foot cube per hour). If used as a weapon, it is the equivalent of a Gargantuan\textit{ +3 adamantine warhammer,} dealing 4d6 points of base damage. 
				
Construction
				
\textbf{Requirements} Craft Wondrous Item, Craft Magic Arms and Armor,\textit{ move earth; }\textbf{Cost }13,348 gp
				
\textbf{Maul of the Titans}
				
\textbf{Aura} strong evocation;\textbf{ CL }15th
				
\textbf{Slot} none; \textbf{Price} 25,305 gp; \textbf{Weight} 160 lbs.
				
Description
				
This mallet is 8 feet long. If used as a weapon, it is the equivalent of a \textit{+3 greatclub} and deals triple damage against inanimate objects. The wielder must have a Strength of at least 18 to wield it properly. Otherwise, she takes a --4 penalty on attack rolls. 
				
Construction
				
\textbf{Requirements} Craft Wondrous Item, Craft Magic Arms and Armor\textit{, clenched fist; }\textbf{Cost }12,805 gp
				
\textbf{Medallion of Thoughts}
				
\textbf{Aura} faint divination;\textbf{ CL }5th
				
\textbf{Slot} neck; \textbf{Price} 12,000 gp; \textbf{Weight }---
				
Description
				
This appears to be a normal pendant disk hung from a neck chain. Usually fashioned from bronze, copper, or silver, the medallion allows the wearer to read the thoughts of others, as with the spell \textit{detect thoughts}. 
				
Construction
				
\textbf{Requirements} Craft Wondrous Item,\textit{ detect thoughts;}\textbf{ Cost }6,000 gp
				
\textbf{Mirror of Life Trapping}
				
\textbf{Aura} strong abjuration;\textbf{ CL }17th
				
\textbf{Slot} none; \textbf{Price} 200,000 gp; \textbf{Weight} 50 lbs.
				
Description
				
This crystal device is usually about 4 feet square and framed in metal or wood. The frame typically depicts dragons, demons, devils, genies, coiling nagas, or other powerful creatures that are well known for their magical powers. It can be hung or placed on a surface and then activated by giving a command word. The same command word deactivates the mirror. A \textit{mirror of life trapping} has 15 extradimensional compartments within it. Any creature coming within 30 feet of the device and looking at its own reflection must make a DC 23 Will save or be trapped within the mirror in one of the cells. A creature not aware of the nature of the device always sees its own reflection. The probability of a creature seeing its reflection, and thus needing to make the saving throw, drops to 50\% if the creature is aware that the mirror traps life and seeks to avoid looking at it (treat as a gaze attack).
				
When a creature is trapped, it is taken bodily into the mirror. Size is not a factor, but constructs and undead are not trapped, nor are inanimate objects and other nonliving matter. A victim's equipment (including clothing and anything being carried) remains behind. If the mirror's owner knows the right command word, he can call the reflection of any creature trapped within to its surface and engage his powerless prisoner in conversation. Another command word frees the trapped creature. Each pair of command words is specific to each prisoner.
				
If the mirror's capacity is exceeded, one victim (determined randomly) is set free in order to accommodate the latest one. If the mirror is destroyed (Hardness 1, 5 hit points), all victims currently trapped in it are freed. 
				
Construction
				
\textbf{Requirements} Craft Wondrous Item,\textit{ imprisonment;}\textbf{ Cost }100,000 gp
				
\textbf{Mirror of Opposition}
				
\textbf{Aura} strong necromancy;\textbf{ CL }15th
				
\textbf{Slot} none; \textbf{Price} 92,000 gp; \textbf{Weight} 45 lbs.
				
Description
				
This item resembles a normal mirror about 4 feet long and 3 feet wide. It can be hung or placed on a surface and then activated by speaking a command word. The same command word deactivates the mirror. If a creature sees its reflection in the mirror's surface, an exact duplicate of that creature comes into being. This opposite immediately attacks the original. The duplicate has all the possessions and powers of its original (including magic). Upon the defeat or destruction of either the duplicate or the original, the duplicate and its items disappear completely. The mirror functions up to four times per day. Destroying the mirror (Hardness 1, 5 hit points) causes all of the duplicates to immediately vanish.
				
Construction
				
\textbf{Requirements} Craft Wondrous Item,\textit{ clone;}\textbf{ Cost }46,000 gp
				
\textbf{Necklace of Adaptation}
				
\textbf{Aura} moderate transmutation; \textbf{CL} 7th
				
\textbf{Slot }neck; \textbf{Price} 9,000 gp; \textbf{Weight} 1 lb.
				
Description
				
This necklace is a heavy chain with a platinum medallion. The magic of the necklace wraps the wearer in a shell of fresh air, making him immune to all harmful vapors and gases (such as \textit{cloudkill }and \textit{stinking cloud }effects, as well as inhaled poisons) and allowing him to breathe, even underwater or in a vacuum.
				
Construction
				
\textbf{Requirements }Craft Wondrous Item, \textit{alter self}; \textbf{Cost} 4,500 gp
				
\textbf{Necklace of Fireballs}
				
\textbf{Aura} moderate evocation;\textbf{ CL }10th
				
\textbf{Slot} neck (does not take up slot); \textbf{Price} 1,650 gp (type I), 2,700 gp (type II), 4,350 gp (type III), 5,400 gp (type IV), 5,850 gp (type V), 8,100 gp (type VI), 8,700 gp (type VII); \textbf{Weight }1 lb.
				
Description
				
This item appears to be a string of beads, sometimes with the ends tied together to form a necklace. (It does not count as an item worn around the neck for the purpose of determining which of a character's worn magic items is effective.) If a character holds it, however, all can see the strand as it really is---a golden chain from which hang a number of golden spheres. The spheres are detachable by the wearer (and only by the wearer), who can easily hurl one of them up to 70 feet. When a sphere arrives at the end of its trajectory, it detonates as a\textit{ fireball} spell (Reflex DC 14 half).
				
Spheres come in different strengths, ranging from those that deal 2d6 points of fire damage to those that deal 10d6. The market price of a sphere is 150 gp for each die of damage it deals.
\begin{figure*}[]
\sffamily
\textbf{Necklace of Fireballs}\\
\setlength{\tabcolsep}{10pt}
\begin{tabular}{lllllllllll}
\textbf{Necklace} & \textbf{10d6} & \textbf{9d6} & \textbf{8d6} & \textbf{7d6} & \textbf{6d6} & \textbf{5d6} & \textbf{4d6} & \textbf{3d6} & \textbf{2d6} & \textbf{Market Price} \\
Type I            & ---             & ---            & ---            & ---            & ---            & 1            & ---            & 2            & ---            & 1,650 gp              \\
Type II           & ---             & ---            & ---            & ---            & 1            & ---            & 2            & ---            & 2            & 2,700 gp              \\
Type III          & ---             & ---            & ---            & 1            & ---            & 2            & ---            & 4            & ---            & 4,350 gp              \\
Type IV           & ---             & ---            & 1            & ---            & 2            & ---            & 2            & ---            & 4            & 5,400 gp              \\
Type V            & ---             & 1            & ---            & 2            & ---            & 2            & ---            & 2            & ---            & 5,850 gp              \\
Type VI           & 1             & ---            & 2            & ---            & 2            & ---            & 4            & ---            & ---            & 8,100 gp              \\
Type VII          & 1             & 2            & ---            & 2            & ---            & 2            & ---            & 2            & ---            & 8,700 gp             
\end{tabular}
\end{figure*}				
Each \textit{necklace of fireballs} contains a combination of spheres of various strengths. Some traditional combinations, designated types I through VII, are detailed above.
				
If the necklace is being worn or carried by a character who fails her saving throw against a magical fire attack, the item must make a saving throw as well (with a save bonus of +7). If the necklace fails to save, all its remaining spheres detonate simultaneously, often with regrettable consequences for the wearer. 
				
Construction
				
\textbf{Requirements} Craft Wondrous Item,\textit{ fireball;}\textbf{ Cost }825 gp (type I), 1,350 gp (type II), 2,175 gp (type III), 2,700 gp (type IV), 2,925 gp (type V), 4,050 gp (type VI), 4,350 gp (type VII)
				
\textbf{Orb of Storms}
				
\textbf{Aura} strong varied;\textbf{ CL }18th
				
\textbf{Slot} none; \textbf{Price} 48,000 gp; \textbf{Weight} 6 lbs.
				
Description
				
This glass sphere is 8 inches in diameter. The possessor can call forth all manner of weather, even supernaturally destructive storms. Once per day, she can call upon the orb to use a \textit{control weather} spell. Once per month, she can conjure a \textit{storm of vengeance. }The possessor of the orb is continually protected by an \textit{endure elements} effect. 
				
Construction
				
\textbf{Requirements} Craft Wondrous Item,\textit{ control weather, endure elements, storm of vengeance;}\textbf{ Cost }24,000 gp
				
\textbf{Pearl of Power}
				
\textbf{Aura} strong transmutation;\textbf{ CL }17th
				
\textbf{Slot} none; \textbf{Price} 1,000 gp (1st), 4,000 gp (2nd), 9,000 gp (3rd), 16,000 gp (4th), 25,000 gp (5th), 36,000 gp (6th), 49,000 gp (7th), 64,000 gp (8th), 81,000 gp (9th), 70,000 gp (two spells); \textbf{Weight }---
				
Description
				
This seemingly normal pearl of average size and luster is a potent aid to all spellcasters who prepare spells (clerics, druids, rangers, paladins, and wizards). Once per day on command, a \textit{pearl of power} enables the possessor to recall any one spell that she had prepared and then cast that day. The spell is then prepared again, just as if it had not been cast. The spell must be of a particular level, depending on the pearl. Different pearls exist for recalling one spell per day of each level from 1st through 9th and for the recall of two spells per day (each of a different level, 6th or lower). 
				
Construction
				
\textbf{Requirements} Craft Wondrous Item, creator must be able to cast spells of the spell level to be recalled;\textbf{ Cost }500 gp (1st), 2,000 gp (2nd), 4,500 gp (3rd), 8,000 gp (4th), 12,500 gp (5th), 18,000 gp (6th), 24,500 gp (7th), 32,000 gp (8th), 40,500 gp (9th), 35,000 gp (two spells)
				
\textbf{Pearl of the Sirines}
				
\textbf{Aura} moderate abjuration and transmutation;\textbf{ CL }8th
				
\textbf{Slot} none; \textbf{Price} 15,300 gp; \textbf{Weight }---
				
Description
				
This pearl is worth at least 1,000 gp for its beauty alone, yet if it is clasped firmly in hand or held to the breast while the possessor attempts actions related to the pearl's powers, she understands and is able to employ the item.
				
The pearl enables its possessor to breathe in water as if she were in clean, fresh air. Her swim speed is 60 feet, and she can cast spells and act underwater without hindrance. 
				
Construction
				
\textbf{Requirements} Craft Wondrous Item,\textit{ freedom of movement, water breathing;}\textbf{ Cost }8,150 gp
				
\textbf{Periapt of Health}
				
\textbf{Aura} faint conjuration;\textbf{ CL }5th
				
\textbf{Slot} neck; \textbf{Price} 7,500 gp; \textbf{Weight }---
				
Description
				
The wearer of this blue gem on a silver chain (worn on the neck) is immune to disease, including supernatural diseases. 
				
Construction
				
\textbf{Requirements} Craft Wondrous Item,\textit{ remove disease;}\textbf{ Cost }3,750 gp
				
\textbf{Periapt of Proof against Poison}
				
\textbf{Aura} faint conjuration;\textbf{ CL }5th
				
\textbf{Slot} neck; \textbf{Price} 27,000 gp; \textbf{Weight }---
				
Description
				
This item is a brilliant-cut black gem on a delicate silver chain meant to be worn about the neck. The wearer is immune to poison, although poisons active when the periapt is first donned still run their course. 
				
Construction
				
\textbf{Requirements} Craft Wondrous Item,\textit{ neutralize poison;}\textbf{ Cost }13,500 gp
				
\textbf{Periapt of Wound Closure}
				
\textbf{Aura} moderate conjuration;\textbf{ CL }10th
				
\textbf{Slot} neck; \textbf{Price} 15,000 gp; \textbf{Weight }---
				
Description
				
This stone is bright red and dangles on a gold chain meant to be worn on the neck. The wearer of this periapt automatically becomes stable if his hit points drop below 0 (but not if the damage is enough to kill the wearer). The periapt doubles the wearer's normal rate of healing or allows normal healing of wounds that would not do so normally. Hit point damage caused by bleeding is negated for the wearer of the periapt, but he is still susceptible to damage from bleeding that causes ability damage or drain. 
				
Construction
				
\textbf{Requirements} Craft Wondrous Item,\textit{ heal;}\textbf{ Cost }7,500 gp
				
\textbf{Phylactery of Faithfulness}
				
\textbf{Aura} faint divination;\textbf{ CL }1st
				
\textbf{Slot} headband; \textbf{Price} 1,000 gp; \textbf{Weight }---
				
Description
				
This item is a tiny box containing religious scripture. The box is affixed to a leather cord and tied around the forehead, worn so that the box sits upon the wearer's brow. There is no mundane way to determine what function this religious item performs until it is worn. The wearer of a \textit{phylactery of faithfulness} is aware of any action or item that could adversely affect his alignment and his standing with his deity, including magical effects. He acquires this information prior to performing such an action or becoming associated with such an item if he takes a moment to contemplate the act. 
				
Construction
				
\textbf{Requirements} Craft Wondrous Item, \textit{detect chaos, detect evil, detect good, detect law;}\textbf{ Cost }500 gp
				
\textbf{Phylactery of Negative Channeling}
				
\textbf{Aura} moderate necromancy \mbox{$[$}evil\mbox{$]$}; \textbf{CL} 10th
				
\textbf{Slot }headband; \textbf{Price} 11,000 gp; \textbf{Weight} ---
				
Description
				
This item is a boon to any character able to channel negative energy, increasing the amount of damage dealt to living creatures by +2d6. This also increases the amount of damage healed by undead creatures.
				
Construction
				
\textbf{Requirements }Craft Wondrous Item, creator must be a 10th-level cleric; \textbf{Cost} 5,500 gp
				
\textbf{Phylactery of Positive Channeling}
				
\textbf{Aura} moderate necromancy \mbox{$[$}good\mbox{$]$}; \textbf{CL} 10th
				
\textbf{Slot }headband; \textbf{Price} 11,000 gp; \textbf{Weight} ---
				
Description
				
This item allows channelers of positive energy to increase the amount of damage dealt to undead creatures by +2d6. This also increases the amount of damage healed by living creatures.
				
Construction
				
\textbf{Requirements }Craft Wondrous Item, creator must be a 10th-level cleric; \textbf{Cost} 5,500 gp
				
\textbf{Pipes of Haunting}
				
\textbf{Aura} faint necromancy;\textbf{ CL }4th
				
\textbf{Slot} none; \textbf{Price} 6,000 gp; \textbf{Weight} 3 lbs.
				
Description
				
This magic item appears to be a small set of pan pipes. When played by a person who succeeds on a DC 15 Perform (wind instruments) check, the pipes create an eerie, spellbinding tune. Those within 30 feet who hear the tune must succeed on a DC 13 Will save or become frightened for 4 rounds. Creatures with 6 or more Hit Dice are unaffected. \textit{Pipes of haunting} can be sounded twice a day. 
				
Construction
				
\textbf{Requirements} Craft Wondrous Item,\textit{ scare;}\textbf{ Cost }3,000 gp
				
\textbf{Pipes of the Sewers}
				
\textbf{Aura} faint conjuration;\textbf{ CL }2nd
				
\textbf{Slot} none; \textbf{Price} 1,150 gp; \textbf{Weight} 3 lbs.
				
Description
				
If the possessor learns the proper tune, he can use these pipes to attract 1d3 rat swarms if rats are within 400 feet. For each 50-foot distance the rats have to travel, there is a 1-round delay. The piper must continue playing until the rats appear, and when they do so, the piper must make a DC 10 Perform (wind instruments) check. Success means that they obey the piper's telepathic commands so long as he continues to play. Failure indicates that they turn on the piper. If for any reason the piper ceases playing, the rats leave immediately. The Perform DC increases by +5 for each time the rats have been successfully called in a 24-hour period.
				
If the rats are under the control of another creature, add the HD of the controller to the Perform check DC. Once control is assumed, another check is required each round to maintain it if the other creature is actively seeking to reassert its control. 
				
Construction
				
\textbf{Requirements} Craft Wondrous Item,\textit{ charm animal, summon nature's ally I, }wild empathy ability\textit{;}\textbf{ Cost }575 gp
				
\textbf{Pipes of Sounding}
				
\textbf{Aura} faint illusion;\textbf{ CL }2nd
				
\textbf{Slot} none; \textbf{Price} 1,800 gp; \textbf{Weight} 3 lbs.
				
Description
				
When played by a character who has the Perform (wind instruments) skill, these shiny metallic pan pipes create a variety of sounds. The figment sounds are the equivalent of \textit{ghost sound}. 
				
Construction
				
\textbf{Requirements} Craft Wondrous Item,\textit{ ghost sound;}\textbf{ Cost }900 gp
				
\textbf{Portable Hole}
				
\textbf{Aura} strong conjuration;\textbf{ CL }12th
				
\textbf{Slot} none; \textbf{Price} 20,000 gp; \textbf{Weight }---
				
Description
				
A \textit{portable hole} is a circle of cloth spun from the webs of a phase spider interwoven with strands of ether and beams of starlight, resulting in a portable extradimensional space. When opened fully, a \textit{portable hole} is 6 feet in diameter, but it can be folded up to be as small as a pocket handkerchief. When spread upon any surface, it causes an extradimensional space 10 feet deep to come into being. This hole can be picked up from inside or out by simply taking hold of the edges of the cloth and folding it up. Either way, the entrance disappears, but anything inside the hole remains, traveling with the item.
				
The only air in the hole is that which enters when the hole is opened. It contains enough air to supply one Medium creature or two Small creatures for 10 minutes. The cloth does not accumulate weight even if its hole is filled. Each \textit{portable hole} opens on its own particular nondimensional space. If a \textit{bag of holding} is placed within a \textit{portable hole}, a rift to the Astral Plane is torn in that place. Both the bag and the cloth are sucked into the void and forever lost. If a \textit{portable hole} is placed within a \textit{bag of holding}, it opens a gate to the Astral Plane. The hole, the bag, and any creatures within a 10-foot radius are drawn there, the \textit{portable hole} and \textit{bag of holding} being destroyed in the process. 
				
Construction
				
\textbf{Requirements} Craft Wondrous Item,\textit{ plane shift;}\textbf{ Cost }10,000 gp
				
\textbf{Restorative Ointment}
				
\textbf{Aura} faint conjuration;\textbf{ CL }5th
				
\textbf{Slot} none; \textbf{Price} 4,000 gp; \textbf{Weight} 1/2 lb.
				
Description
				
A jar of this unguent is 3 inches in diameter and 1 inch deep, and contains five applications. Placed upon a poisoned wound or swallowed, the ointment detoxifies any poison (as \textit{neutralize poison} with a +5 bonus on the check). Applied to a diseased area, it removes disease (as \textit{remove disease }with a +5 bonus on the check). Rubbed on a wound, the ointment cures 1d8+5 points of damage (as \textit{cure light wounds}). 
				
Construction
				
\textbf{Requirements} Craft Wondrous Item,\textit{ cure light wounds, neutralize poison, remove disease;}\textbf{ Cost }2,000 gp
				
\textbf{Ring Gates}
				
\textbf{Aura} strong conjuration;\textbf{ CL }17th
				
\textbf{Slot} none; \textbf{Price} 40,000 gp; \textbf{Weight} 1 lb. each.
				
Description
				
These always come in pairs---two iron rings, each about 18 inches in diameter. The rings must be on the same plane of existence and within 100 miles of each other to function. Whatever is put through one ring comes out the other, and up to 100 pounds of material can be transferred each day. (Objects only partially pushed through and then retracted do not count.) This useful device allows for instantaneous transport of items or messages, and even attacks. A character can reach through to grab things near the other ring, or even stab a weapon through if so desired. Alternatively, a character could stick his head through to look around. A spellcaster could even cast a spell through a \textit{ring gate.} A Small character can make a DC 13 Escape Artist check to slip through. Creatures of Tiny, Diminutive, or Fine size can pass through easily. Each ring has an \texttt{{}"{}}entry side\texttt{{}"{}} and an \texttt{{}"{}}exit side,\texttt{{}"{}} both marked with appropriate symbols. 
				
Construction
				
\textbf{Requirements} Craft Wondrous Item,\textit{ gate;}\textbf{ Cost }20,000 gp
				
\textbf{Robe of the Archmagi}
				
\textbf{Aura} strong varied;\textbf{ CL }14th
				
\textbf{Slot} body; \textbf{Price} 75,000 gp; \textbf{Weight} 1 lb.
				
Description
				
This normal-appearing garment can be white (01--45 on d\%, good alignment), gray (46--75, neither good nor evil alignment), or black (76--100, evil alignment). To most wearers, the robe offers no powers or has no effects unless the wearer's alignment doesn't match that of the robe (see below). Only an arcane spellcaster can fully realize this potent magic item's powers once the robe is donned. These powers are as follows.
				\begin{itemize}\item  +5 armor bonus to AC.
				\item  Spell resistance 18.
				\item  +4 resistance bonus on all saving throws.
				\item  +2 enhancement bonus on caster level checks made to overcome spell resistance.
\end{itemize}
				
As mentioned above, all \textit{robes of the archmagi} are attuned to a specific alignment. If a white robe is donned by an evil character, she immediately gains three permanent negative levels. The same is true with respect to a black robe donned by a good character. An evil or good character who puts on a gray robe, or a neutral character who dons either a white or black robe, gains two permanent negative levels. While these negative levels remain as long as the garment is worn and cannot be overcome in any way (including \textit{restoration} spells), they are immediately removed if the robe is removed. 
				
Construction
				
\textbf{Requirements} Craft Wondrous Item, \textit{antimagic field, mage armor }or\textit{ shield of faith, }creator must be of same alignment as robe;\textbf{ Cost }37,500 gp
				
\textbf{Robe of Blending}
				
\textbf{Aura} moderate transmutation;\textbf{ CL }10th
				
\textbf{Slot} body; \textbf{Price} 8,400 gp; \textbf{Weight} 1 lb.
				
Description
				
Once per day this simple wool robe allows you to assume the form of another humanoid creature, as if using \textit{alter self}. This change lasts for 1 hour, although you can end it prematurely as a free action. While in this form, you also gain the ability to speak and understand the basic racial languages of your chosen form. For example, if you take the form of an orc, you can speak and understand Orc. 
				
Construction
				
\textbf{Requirements} Craft Wondrous Item,\textit{ alter self, tongues;}\textbf{ Cost }4,200 gp
				
\textbf{Robe of Bones}
				
\textbf{Aura} moderate necromancy \mbox{$[$}evil\mbox{$]$};\textbf{ CL }6th
				
\textbf{Slot} body; \textbf{Price} 2,400 gp; \textbf{Weight} 1 lb.
				
Description
				
This sinister item functions much like a \textit{robe of useful items} for the serious necromancer. It appears to be an unremarkable robe, but a character who dons it notes that it is adorned with small embroidered figures representing undead creatures. Only the wearer of the robe can see the embroidery, recognize them for the creatures they become, and detach them. One figure can be detached each round. Detaching a figure causes it to become an actual undead creature (see the list below). The skeleton or zombie is not under the control of the wearer of the robe, but may be subsequently commanded, rebuked, turned, or destroyed. A newly created \textit{robe of bones }always has two embroidered figures of each of the following undead:
				\begin{itemize}\item  Human skeleton
				\item  Wolf skeleton
				\item  Heavy horse skeleton
				\item  Fast goblin zombie
				\item  Tough human zombie
				\item  Plague ogre zombie 
\end{itemize}
				
Construction
				
\textbf{Requirements} Craft Wondrous Item, \textit{animate dead;}\textbf{ Cost }1,200 gp
				
\textbf{Robe of Eyes}
				
\textbf{Aura} moderate divination;\textbf{ CL }11th
				
\textbf{Slot} body; \textbf{Price} 120,000 gp; \textbf{Weight} 1 lb.
				
Description
				
This valuable garment appears to be a normal robe until it is put on. Its wearer is able to see in all directions at the same moment due to scores of visible, magical eye-like patterns that adorn the robe. She also gains 120-foot darkvision.
				
The \textit{robe of eyes} sees all forms of invisible or ethereal creatures or objects within 120 feet.
				
The wearer of a \textit{robe of eyes} gains a +10 competence bonus on Perception checks. She retains her Dexterity bonus to AC even when flat-footed, and can't be flanked. She is not able to avert or close her eyes when confronted by a creature with a gaze attack.
				
A \textit{light} or \textit{continual flame} spell cast directly on a \textit{robe of eyes} causes it to be blinded for 1d3 minutes. A \textit{daylight} spell blinds it for 2d4 minutes. 
				
Construction
				
\textbf{Requirements} Craft Wondrous Item,\textit{ true seeing;}\textbf{ Cost }60,000 gp
				
\textbf{Robe, Monk's}
				
\textbf{Aura} moderate transmutation; \textbf{CL} 10th
				
\textbf{Slot }body; \textbf{Price} 13,000 gp; \textbf{Weight} 1 lb.
				
Description
				
This simple brown robe, when worn, confers great ability in unarmed combat. If the wearer has levels in monk, her AC and unarmed damage is treated as a monk of five levels higher. If donned by a character with the Stunning Fist feat, the robe lets her make one additional stunning attack per day. If the character is not a monk, she gains the AC and unarmed damage of a 5th-level monk (although she does not add her Wisdom bonus to her AC). This AC bonus functions just like the monk's AC bonus.
				
Construction
				
\textbf{Requirements }Craft Wondrous Item, \textit{righteous might }or \textit{transformation}; \textbf{Cost} 6,500 gp
				
\textbf{Robe of Scintillating Colors}
				
\textbf{Aura} moderate illusion;\textbf{ CL }11th
				
\textbf{Slot} body; \textbf{Price} 27,000 gp; \textbf{Weight} 1 lb.
				
Description
				
The wearer of this robe can cause the garment to display a shifting pattern of incredible hues, color after color cascading from the upper part of the robe to the hem in sparkling rainbows of dazzling light. The colors daze those near the wearer, conceal the wearer, and illuminate the surroundings. It takes 1 full round after the wearer speaks the command word for the colors to start flowing on the robe. The colors create the equivalent of a gaze attack with a 30-foot range. Those who look at the wearer are dazed for 1d4+1 rounds (Will DC 16 negates). This is a mind-affecting pattern effect.
				
Every round of continuous scintillation of the robe gives the wearer better concealment. The miss chance on attacks against the wearer starts at 10\% and increases by 10\% each round until it reaches 50\% (total concealment).
				
The robe illuminates a 30-foot radius continuously.
				
The effect can be used no more than a total of 10 rounds per day. 
				
Construction
				
\textbf{Requirements} Craft Wondrous Item,\textit{ blur, rainbow pattern;}\textbf{ Cost }13,500 gp
				
\textbf{Robe of Stars}
				
\textbf{Aura} strong varied;\textbf{ CL }15th
				
\textbf{Slot} body; \textbf{Price} 58,000 gp; \textbf{Weight} 1 lb.
				
Description
				
This garment is typically black or dark blue and embroidered with small white or silver stars. The robe has three magical powers.
				\begin{itemize}\item  It enables its wearer to travel physically to the Astral Plane, along with all that she is wearing or carrying.
				\item  It gives its wearer a +1 luck bonus on all saving throws.
				\item  Its wearer can use up to six of the embroidered stars on the chest portion of the robe as \textit{+5 shuriken.} The robe grants its wearer proficiency with such weapons. Each shuriken disappears after it is used. The stars are replenished once per month.
\end{itemize}
				
Construction
				
\textbf{Requirements} Craft Wondrous Item,\textit{ magic missile, astral projection }or\textit{ plane shift;}\textbf{ Cost }29,000 gp
				
\textbf{Robe of Useful Items}
				
\textbf{Aura} moderate transmutation;\textbf{ CL }9th
				
\textbf{Slot} body; \textbf{Price} 7,000 gp; \textbf{Weight} 1 lb.
				
Description
				
This appears to be an unremarkable robe, but a character who dons it notes that it is adorned with small cloth patches of various shapes. Only the wearer of the robe can see these patches, recognize them for what items they become, and detach them. One patch can be detached each round. Detaching a patch causes it to become an actual item, as indicated below. A newly created \textit{robe of useful items} always has two each of the following patches:
				\begin{itemize}\item  Dagger
				\item  Bullseye lantern (full and lit)
				\item  Mirror (a highly polished 2-foot-by-4-foot steel mirror)
				\item  Pole (10-foot length)
				\item  Hempen rope (50-foot coil)
				\item  Sack
\end{itemize}
				
In addition, the robe has several other patches. Roll 4d4 for the number of other patches and then roll for each patch on the table below to determine its nature.
\setlength{\tabcolsep}{1pt}
\begin{tabularx}{\linewidth}{lX}
\textbf{d\%} & \textbf{Result}                                                                                                          \\
01--08        & Bag of 100 gold pieces                                                                                                   \\
09--15        & Coffer, silver (6 in. by 6 in. by 1 ft.), 500 gp value                                                                   \\
16--22        & Door, iron (up to 10 ft. wide and 10 ft. high and barred on one side---must be placed upright, attaches and hinges itself) \\
23--30        & Gems, 10 (100 gp value each)                                                                                             \\
31--44        & Ladder, wooden (24 ft. long)                                                                                             \\
45--51        & Mule (with saddle bags)                                                                                                  \\
52--59        & Pit, open (10 ft. by 10 ft. by 10 ft.)                                                                                   \\
60--68        & Potion of cure serious wounds                                                                                            \\
69--75        & Rowboat (12 ft. long)                                                                                                    \\
76--83        & Minor scroll of one randomly determined spell                                                                            \\
84--90        & War dogs, pair (treat as riding dogs)                                                                                    \\
91--96        & Window (2 ft. by 4 ft., up to 2 ft. deep)                                                                                \\
97--100       & Portable ram                                                                                                            
\end{tabularx}
Multiple items of the same kind are permissible. Once removed, a patch cannot be replaced. 
				
Construction
				
\textbf{Requirements} Craft Wondrous Item,\textit{ fabricate;}\textbf{ Cost }3,500 gp
				
\textbf{Rope of Climbing}
				
\textbf{Aura} faint transmutation;\textbf{ CL }3rd
				
\textbf{Slot} none; \textbf{Price} 3,000 gp; \textbf{Weight} 3 lbs.
				
Description
				
A 60-foot-long \textit{rope of climbing} is no thicker than a wand, but it is strong enough to support 3,000 pounds. Upon command, the rope snakes forward, upward, downward, or in any other direction at 10 feet per round, attaching itself securely wherever its owner desires. It can unfasten itself and return in the same manner.
				
A \textit{rope of climbing} can be commanded to knot or unknot itself. This causes large knots to appear at 1-foot intervals along the rope. Knotting shortens the rope to a 50-foot length until the knots are untied, but lowers the DC of Climb checks while using it by 10. A creature must hold one end of the rope when its magic is invoked. 
				
Construction
				
\textbf{Requirements} Craft Wondrous Item,\textit{ animate rope;}\textbf{ Cost }1,500 gp
				
\textbf{Rope of Entanglement}
				
\textbf{Aura} strong transmutation;\textbf{ CL }12th
				
\textbf{Slot} none; \textbf{Price} 21,000 gp; \textbf{Weight} 5 lbs.
				
Description
				
A \textit{rope of entanglement} looks just like any other hempen rope about 30 feet long. Upon command, the rope lashes forward 20 feet or upward 10 feet to entangle a victim. An entangled creature can break free with a DC 20 Strength check or a DC 20 Escape Artist check.
				
A \textit{rope of entanglement} has AC 22, 12 hit points, hardness 10, and damage reduction 5/slashing. The rope repairs damage to itself at a rate of 1 point per 5 minutes, but if a \textit{rope of entanglement} is severed (all 12 hit points lost to damage), it is destroyed. 
				
Construction
				
\textbf{Requirements} Craft Wondrous Item,\textit{ animate objects, animate rope or entangle;}\textbf{ Cost }10,500 gp
				
\textbf{Salve of Slipperiness}
				
\textbf{Aura} moderate conjuration;\textbf{ CL }6th
				
\textbf{Slot} none; \textbf{Price} 1,000 gp; \textbf{Weight }---
				
Description
				
This substance provides a +20 competence bonus on all Escape Artist checks and combat maneuver checks made to escape from a grapple. The salve also grants a +10 competence bonus to the wearer's Combat Maneuver Defense for the purpose of avoiding grapple attempts. In addition, such obstructions as webs (magical or otherwise) do not affect an anointed individual. Magic ropes and the like do not avail against this salve. If it is smeared on a floor or on steps, the area should be treated as a long-lasting \textit{grease} spell. The salve requires 8 hours to wear off normally, or it can be wiped off with an alcohol solution (even wine).
				
\textit{Salve of slipperiness} is needed to coat the inside of a container meant to hold \textit{sovereign glue}. 
				
Construction
				
\textbf{Requirements} Craft Wondrous Item,\textit{ grease;}\textbf{ Cost }500 gp
				
\textbf{Scabbard of Keen Edges}
				
\textbf{Aura} faint transmutation;\textbf{ CL }5th
				
\textbf{Slot} none; \textbf{Price} 16,000 gp; \textbf{Weight} 1 lb.
				
Description
				
This scabbard can shrink or enlarge to accommodate any knife, dagger, sword, or similar weapon up to and including a greatsword. Up to three times per day on command, the scabbard casts \textit{keen edge }on any blade placed within it. 
				
Construction
				
\textbf{Requirements} Craft Wondrous Item,\textit{ keen edge;}\textbf{ Cost }8,000 gp
				
\textbf{Scarab of Protection}
				
\textbf{Aura} strong abjuration and necromancy;\textbf{ CL }18th
				
\textbf{Slot} neck; \textbf{Price} 38,000 gp; \textbf{Weight }---
				
Description
				
This device appears to be a silver medallion in the shape of a beetle. If it is held for 1 round, an inscription appears on its surface letting the holder know that it is a protective device.
				
The scarab's possessor gains spell resistance 20. The scarab can also absorb energy-draining attacks, death effects, and negative energy effects. Upon absorbing 12 such attacks, the scarab turns to powder and is destroyed. 
				
Construction
				
\textbf{Requirements} Craft Wondrous Item, \textit{death ward, spell resistance;}\textbf{ Cost }19,000 gp
				
\textbf{Scarab, Golembane}
				
\textbf{Aura} moderate divination;\textbf{ CL }8th
				
\textbf{Slot} neck; \textbf{Price} 2,500 gp; \textbf{Weight }---
				
Description
				
This beetle-shaped pin enables its wearer to detect any golem within 60 feet, although he must concentrate (a standard action) in order for the detection to take place. A scarab enables its possessor to combat golems with weapons, unarmed attacks, or natural weapons as if those golems had no damage reduction. 
				
Construction
				
\textbf{Requirements} Craft Wondrous Item,\textit{ detect magic, }creator must be at least 10th level;\textbf{ Cost }1,250 gp
				
\textbf{Silversheen}
				
\textbf{Aura} faint transmutation;\textbf{ CL }5th
				
\textbf{Slot} none; \textbf{Price} 250 gp; \textbf{Weight }---
				
Description
				
This shimmering paste-like substance can be applied to a weapon as a standard action. It gives the weapon the properties of alchemical silver for 1 hour, replacing the properties of any other special material it might have. One vial coats a single melee weapon or 20 units of ammunition. 
				
Construction
				
\textbf{Requirements} Craft Wondrous Item;\textbf{ Cost }125 gp
				
\textbf{Slippers of Spider Climbing}
				
\textbf{Aura} faint transmutation;\textbf{ CL }4th
				
\textbf{Slot} feet; \textbf{Price} 4,800 gp; \textbf{Weight} 1/2 lb.
				
Description
				
When worn, a pair of these slippers enables movement on vertical surfaces or even upside down along ceilings, leaving the wearer's hands free. Her climb speed is 20 feet. Severely slippery surfaces---icy, oiled, or greased surfaces---make these slippers useless. The slippers can be used for 10 minutes per day, split up as the wearer chooses (minimum 1 minute per use). 
				
Construction
				
\textbf{Requirements} Craft Wondrous Item,\textit{ spider climb;}\textbf{ Cost }2,400 gp
				
\textbf{Sovereign Glue}
				
\textbf{Aura} strong transmutation;\textbf{ CL }20th
				
\textbf{Slot} none; \textbf{Price} 2,400 gp (per ounce); \textbf{Weight }---
				
Description
				
This pale amber substance is thick and viscous. Because of its particular powers, it can be contained only in a flask whose inside has been coated with 1 ounce of \textit{salve of slipperiness,} and each time any of the bonding agent is poured from the flask, a new application of the \textit{salve of slipperiness} must be put in the flask within 1 round to prevent the remaining glue from adhering to the side of the container. A flask of \textit{sovereign glue,} when found, holds anywhere from 1 to 7 ounces of the stuff (1d8--1, minimum 1), with the other ounce of the flask's capacity taken up by the \textit{salve of slipperiness.} One ounce of this adhesive covers 1 square foot of surface, bonding virtually any two substances together in a permanent union. The glue takes 1 round to set. If the objects are pulled apart (a move action) before that time has elapsed, that application of the glue loses its stickiness and is worthless. If the glue is allowed to set, then attempting to separate the two bonded objects has no effect, except when \textit{universal solvent} is applied to the bond. \textit{Sovereign glue }is dissolved by \textit{universal solvent}.
				
Construction
				
\textbf{Requirements} Craft Wondrous Item,\textit{ make whole;}\textbf{ Cost }1,200 gp
				
\textbf{Stone of Alarm}
				
\textbf{Aura} faint abjuration;\textbf{ CL }3rd
				
\textbf{Slot} none; \textbf{Price} 2,700 gp; \textbf{Weight} 2 lbs.
				
Description
				
This stone cube, when given the command word, affixes itself to any object. If that object is touched thereafter by anyone who does not first speak that same command word, the stone emits a piercing screech for 1 hour that can be heard up to a quarter-mile away (assuming no intervening barriers). 
				
Construction
				
\textbf{Requirements} Craft Wondrous Item,\textit{ alarm;}\textbf{ Cost }1,350 gp
				
\textbf{Stone of Good Luck (Luckstone)}
				
\textbf{Aura} faint evocation;\textbf{ CL }5th
				
\textbf{Slot} none; \textbf{Price} 20,000 gp; \textbf{Weight }---
				
Description
				
This small bit of agate grants its possessor a +1 luck bonus on saving throws, ability checks, and skill checks. 
				
Construction
				
\textbf{Requirements} Craft Wondrous Item,\textit{ divine favor;}\textbf{ Cost }10,000 gp
				
\textbf{Stone Salve}
				
\textbf{Aura} strong abjuration and transmutation;\textbf{ CL }13th
				
\textbf{Slot} none; \textbf{Price} 4,000 gp per ounce; \textbf{Weight }---
				
Description
				
This ointment has two uses. If an ounce of it is applied to the flesh of a petrified creature, it returns the creature to flesh as the \textit{stone to flesh} spell. If an ounce of it is applied to the flesh of a nonpetrified creature, it protects the creature as a \textit{stoneskin} spell. 
				
Construction
				
\textbf{Requirements} Craft Wondrous Item,\textit{ stone to flesh, stoneskin;}\textbf{ Cost }2,000 gp
				
\textbf{Strand of Prayer Beads}
				
\textbf{Aura} faint, moderate or strong (many schools);\textbf{ CL }1st (\textit{blessing}), 5th (\textit{healing}), 7th (\textit{smiting}), 9th (\textit{karma}), 11th (\textit{wind walking}), 17th (\textit{summons})
				
\textbf{Slot} none; \textbf{Price} 9,600 gp (lesser), 45,800 gp (standard), 95,800 gp (greater); \textbf{Weight }1/2 lb.
				
Description
				
This item appears to be nothing more than a string of prayer beads until the owner casts a divine spell while the beads are carried. Once that occurs, the owner instantly knows the powers of the prayer beads and understands how to activate the strand's special magical beads. Each strand includes two or more special beads, each with a different magic power selected from the following list.
% <thead class="stat-block-breaker">

Special Bead TypeSpecial Bead Ability
% </thead class="stat-block-breaker">

\textit{Bead of blessing}Wearer can cast \textit{bless.}
\textit{Bead of healing}Wearer can cast his choice of \textit{cure serious wounds, remove blindness/deafness, }or \textit{remove disease.}
\textit{Bead of karma}Wearer casts his spells at +4 caster level. Effect lasts 10 minutes.
\textit{Bead of smiting}Wearer can cast \textit{chaos hammer, holy smite, order's wrath, }or \textit{unholy blight }(Will DC 17 partial).
\textit{Bead of summons}Summons a powerful creature of appropriate alignment from the Outer Planes (an angel, devil, etc.) to aid the wearer for 1 day. (If the wearer uses the \textit{bead of summons }to summon a deity's emissary frivolously, the deity takes that character's items and places a \textit{geas }upon him as punishment at the very least.)
\textit{Bead of wind walking}Wearer can cast \textit{wind walk.}
				
A \textit{lesser strand of prayer beads} has a \textit{bead of blessing} and a \textit{bead of healing}. A \textit{strand of prayer beads} has a \textit{bead of healing}, a \textit{bead of karma,} and a \textit{bead of smiting}. A \textit{greater strand of prayer beads} has a \textit{bead of healing}, a \textit{bead of karma}, a \textit{bead of summons}, and a \textit{bead of wind walking}.
				
Each special bead can be used once per day, except for the \textit{bead of summons}, which works only once and then becomes nonmagical. The \textit{beads of blessing, smiting,} and \textit{wind walking} function as spell trigger items; the \textit{beads of karma} and \textit{summons }can be activated by any character capable of casting divine spells. The owner need not hold or wear the \textit{strand of prayer beads} in any specific location, as long as he carries it somewhere on his person.
				
The power of a special bead is lost if it is removed from the strand. Reduce the price of a strand of prayer beads that is missing one or more beads by the following amounts: \textit{bead of blessing} --600 gp, \textit{bead of healing} --9,000 gp, \textit{bead of karma} --20,000 gp, \textit{bead of smiting} --16,800 gp, \textit{bead of summons} --20,000 gp, \textit{bead of wind walking }--46,800 gp. 
				
Construction
				
\textbf{Requirements} Craft Wondrous Item and one of the following spells per bead, as appropriate:\textit{ bless (blessing);} \textit{cure serious wounds, remove blindness/ deafness, }or\textit{ remove disease (healing); righteous might (karma); gate (summons); chaos hammer, holy smite, order's wrath,} or\textit{ unholy blight (smiting), wind walk (wind walking);}\textbf{ Cost }4,800 gp (lesser), 22,900 gp (standard), 47,900 gp (greater)
				
\textbf{Sustaining Spoon}
				
\textbf{Aura} faint conjuration;\textbf{ CL }5th
				
\textbf{Slot} none; \textbf{Price} 5,400 gp; \textbf{Weight }---
				
Description
				
If this unremarkable appearing utensil is placed in an empty container, the vessel fills with a thick, pasty gruel. Although the gruel tastes like warm, wet cardboard, it is highly nourishing and contains everything necessary to sustain any herbivorous, omnivorous, or carnivorous creature. The spoon can produce sufficient gruel each day to feed up to four humans. 
				
Construction
				
\textbf{Requirements} Craft Wondrous Item, \textit{create food and water;}\textbf{ Cost }2,700 gp
				
\textbf{Tome of Clear Thought}
				
\textbf{Aura} strong evocation (if \textit{miracle} is used);\textbf{ CL }17th
				
\textbf{Slot} none; \textbf{Price} 27,500 gp (+1), 55,000 gp (+2), 82,500 gp (+3), 110,000 gp (+4), 137,500 gp (+5); \textbf{Weight} 5 lbs.
				
Description
				
This heavy book contains instruction on improving memory and logic, but entwined within the words is a powerful magical effect. If anyone reads this book, which takes a total of 48 hours over a minimum of 6 days, she gains an inherent bonus from +1 to +5 (depending on the type of tome) to her Intelligence score. Once the book is read, the magic disappears from the pages and it becomes a normal book. 
				
Construction
				
\textbf{Requirements} Craft Wondrous Item,\textit{ miracle }or\textit{ wish; }\textbf{Cost }26,250 gp (+1), 52,500 gp (+2), 78,750 gp (+3), 105,000 gp (+4), 131,250 gp (+5)
				
\textbf{Tome of Leadership and Influence}
				
\textbf{Aura} strong evocation (if \textit{miracle} is used);\textbf{ CL }17th
				
\textbf{Slot} none; \textbf{Price} 27,500 gp (+1), 55,000 gp (+2), 82,500 gp (+3), 110,000 gp (+4), 137,500 gp (+5); \textbf{Weight} 5 lbs.
				
Description
				
This ponderous book details suggestions for persuading and inspiring others, but entwined within the words is a powerful magical effect. If anyone reads this book, which takes a total of 48 hours over a minimum of 6 days, he gains an inherent bonus from +1 to +5 (depending on the type of tome) to his Charisma score. Once the book is read, the magic disappears from the pages and it becomes a normal book. 
				
Construction
				
\textbf{Requirements} Craft Wondrous Item,\textit{ miracle} or\textit{ wish; }\textbf{Cost }26,250 gp (+1), 52,500 gp (+2), 78,750 gp (+3), 105,000 gp (+4), 131,250 gp (+5)
				
\textbf{Tome of Understanding}
				
\textbf{Aura} strong evocation (if \textit{miracle }is used);\textbf{ CL }17th
				
\textbf{Slot} none; \textbf{Price} 27,500 gp (+1), 55,000 gp (+2), 82,500 gp (+3), 110,000 gp (+4), 137,500 gp (+5); \textbf{Weight} 5 lbs.
				
Description
				
This thick book contains tips for improving instinct and perception, but entwined within the words is a powerful magical effect. If anyone reads this book, which takes a total of 48 hours over a minimum of 6 days, she gains an inherent bonus from +1 to +5 (depending on the type of tome) to her Wisdom score. Once the book is read, the magic disappears from the pages and it becomes a normal book. 
				
Construction
				
\textbf{Requirements} Craft Wondrous Item, \textit{miracle or wish; }\textbf{Cost }26,250 gp (+1), 52,500 gp (+2), 78,750 gp (+3), 105,000 gp (+4), 131,250 gp (+5)
				
\textbf{Unguent of Timelessness}
				
\textbf{Aura} faint transmutation;\textbf{ CL }3rd
				
\textbf{Slot} none; \textbf{Price} 150 gp; \textbf{Weight }---
				
Description
				
When applied to any matter that was once alive, such as wood, paper, or a dead body, this ointment allows that substance to resist the passage of time. Each year of actual time affects the substance as if only a day had passed. The coated object gains a +1 resistance bonus on all saving throws. The unguent never wears off, although it can be magically removed (by dispelling the effect, for instance). One flask contains enough material to coat eight Medium or smaller objects. A Large object counts as two Medium objects, and a Huge object counts as four Medium objects. 
				
Construction
				
\textbf{Requirements} Craft Wondrous Item, \textit{gentle repose};\textbf{ Cost }75 gp
				
\textbf{Universal Solvent}
				
\textbf{Aura} faint transmutation;\textbf{ CL }3rd
				
\textbf{Slot} none; \textbf{Price} 50 gp; \textbf{Weight }---
				
Description
				
This substance has the unique property of being able to dissolve \textit{sovereign glue}, tanglefoot bags, and all other adhesives. Applying the solvent is a standard action. 
				
Construction
				
\textbf{Requirements} Craft Wondrous Item,\textit{ acid arrow;}\textbf{ Cost }25 gp
				
\textbf{Vest of Escape}
				
\textbf{Aura} faint conjuration and transmutation;\textbf{ CL }4th
				
\textbf{Slot} chest; \textbf{Price} 5,200 gp; \textbf{Weight }---
				
Description
				
Hidden within secret pockets of this simple silk vest are magic lockpicks that provide a +4 competence bonus on Disable Device checks. The vest also grants its wearer a +6 competence bonus on Escape Artist checks. 
				
Construction
				
\textbf{Requirements} Craft Wondrous Item,\textit{ knock, grease;}\textbf{ Cost }2,600 gp
				
\textbf{Vestment, Druid's}
				
\textbf{Aura} moderate transmutation;\textbf{ CL }10th
				
\textbf{Slot} body; \textbf{Price} 3,750 gp; \textbf{Weight }---
				
Description
				
This light garment is worn over normal clothing or armor. Most such vestments are green, embroidered with plant or animal motifs. When this item is worn by a character with the wild shape ability, the character can use that ability one additional time each day. 
				
Construction
				
\textbf{Requirements} Craft Wondrous Item,\textit{ polymorph }or wild shape ability;\textbf{ Cost }1,375 gp
				
\textbf{Well of Many Worlds}
				
\textbf{Aura} strong conjuration;\textbf{ CL }17th
				
\textbf{Slot} none; \textbf{Price} 82,000 gp; \textbf{Weight }---
				
Description
				
This strange, interdimensional device looks just like a \textit{portable hole}. Anything placed within it is immediately cast to another world---a parallel world, another planet, or a different plane (chosen randomly). If the well is moved, it opens to a new plane (also randomly determined). It can be picked up, folded, or rolled, just as a \textit{portable hole} can be. Objects from the world the well touches can come through the opening just as easily---it is a two-way portal.
				
Construction
				
\textbf{Requirements} Craft Wondrous Item,\textit{ gate;}\textbf{ Cost }41,000 gp
				
\textbf{Wind Fan}
				
\textbf{Aura} faint evocation;\textbf{ CL }5th
				
\textbf{Slot} none; \textbf{Price} 5,500 gp; \textbf{Weight }---
				
Description
				
A \textit{wind fan} appears to be nothing more than a wood and papyrus or cloth instrument with which to create a cooling breeze. By uttering the command word, its possessor causes the fan to duplicate a \textit{gust of wind} spell. The fan can be used once per day with no risk. If it is used more frequently, there is a 20\% cumulative chance per usage during that day that the device tears into useless, nonmagical tatters. 
				
Construction
				
\textbf{Requirements} Craft Wondrous Item,\textit{ gust of wind;}\textbf{ Cost }2,750 gp
				
\textbf{Wings of Flying}
				
\textbf{Aura} moderate transmutation;\textbf{ CL }10th
				
\textbf{Slot} shoulders; \textbf{Price} 54,000 gp; \textbf{Weight} 2 lbs.
				
Description
				
A pair of these wings might appear to be nothing more than a plain cloak of old, black cloth, or they could be as elegant as a long cape of blue feathers. When the wearer speaks the command word, the cloak turns into a pair of bat or bird wings that empower her to fly with a speed of 60 feet (average maneuverability), also granting a +5 competence bonus on Fly skill checks. 
				
Construction
				
\textbf{Requirements} Craft Wondrous Item,\textit{ fly;}\textbf{ Cost }27,000 gp
        	

\section{Intelligent Items}

\label{f0}
\begin{table}[]
\sffamily
\caption{Table: Intelligent Item Alignment}
\begin{tabular}{ll}
\textbf{d\%} & \textbf{Alignment of Item}\\
01-10 & Chaotic good \\
 11-20 & Chaotic neutral* \\
 21-35 & Chaotic evil \\
 36-45 & Neutral evil* \\
 46-55 & Lawful evil \\
 56-70 & Lawful good \\
 71-80 & Lawful neutral* \\
 81-90 & Neutral good* \\
 91-100 & Neutral\\
\end{tabular}
* The item can also be used by any character whose alignment corresponds to the non-neutral portion of the item's alignment.\\
\end{table}

\begin{table}[]
\sffamily
\caption{Table: Intelligent Item Ability Scores}
\begin{tabular}{lll}
\textbf{Score} & \textbf{Base Price Modifier} & \textbf{Ego Modifier}\\
10 &  --- &  --- \\
 11 &  +200 gp &  --- \\
 12 &  +500 gp &  +1 \\
 13 &  +700 gp &  +1 \\
 14 &  +1,000 gp &  +2 \\
 15 &  +1,400 gp &  +2 \\
 16 &  +2,000 gp &  +3 \\
 17 &  +2,800 gp &  +3 \\
 18 &  +4,000 gp &  +4 \\
 19 &  +5,200 gp &  +4 \\
 20 &  +8,000 gp &  +5\\
\end{tabular}
\end{table}

\begin{table}[]
\sffamily
\caption{Table: Intelligent Item Senses and Communication}
\begin{tabular}{lll}
\textbf{Ability} & \textbf{Base Price Modifier} & \textbf{Ego Modifier}\\
Empathy & --- & --- \\
Speech & +500 gp & --- \\
Telepathy & +1,000 gp & +1 \\
Senses (30 ft.) & --- & --- \\
Senses (60 ft.) & +500 gp & --- \\
Senses (120 ft.) & +1,000 gp & --- \\
Darkvision & +500 gp & --- \\
Blindsense & +5,000 gp & +1 \\
Read languages & +1,000 gp & +1\\
Read magic  & +2,000 gp & +1\\
\end{tabular}
\end{table}
% <
\begin{table}[]
\sffamily
\caption{Table: Intelligent Item Powers}
\begin{tabular}{llll}
\textbf{d\%} & \textbf{Item Power} & \textbf{Base Price Modifier} & \textbf{Ego Modifier}\\
01-10 & Item can cast a 0-level spell at will & +1,000 gp & +1 \\
 11-20 & Item can cast a 1st-level spell 3/day & +1,200 gp & +1 \\
 21-25 & Item can use \textit{magic aura} & +2,000 gp & +1 \\
 26-35 & Item can cast a 2nd-level spell 1/day & +2,400 gp & +1 \\
 36-45 & Item has 5 ranks in one skill* & +2,500 gp & +1 \\
 46-50 & Item can sprout limbs and move with a speed of 10 feet & +5,000 gp & +1 \\
 51-55 & Item can cast a 3rd-level spell 1/day & +6,000 gp & +1 \\
 56-60 & Item can cast a 2nd-level spell 3/day & +7,200 gp & +1 \\
 61-70 & Item has 10 ranks in one skill* & +10,000 gp & +2 \\
 71-75 & Item can change shape into one other form of the same size & +10,000 gp & +2 \\
 76-80 & Item can \textit{fly, }as per the spell, at a speed of 30 feet & +10,000 gp & +2 \\
 81-85 & Item can cast a 4th-level spell 1/day & +11,200 gp & +2 \\
 86-90 & Item can \textit{teleport} itself 1/day & +15,000 gp & +2 \\
 91-95 & Item can cast a 3rd-level spell 3/day & +18,000 gp & +2 \\
 96-100 & Item can cast a 4th-level spell 3/day & +33,600 gp & +2\\
\end{tabular}
* Intelligent items can only possess Intelligence-, Wisdom-, or Charisma-based skills, unless they also possess some form of ability to move.
\end{table}
\begin{table}[]
\sffamily
\caption{Table: Intelligent Item Purpose}
\begin{tabular}{lll}
\textbf{d\%} & \textbf{Purpose} & \textbf{Ego Modifier}\\
01-20 & Defeat/slay diametrically opposed alignment* & +2 \\
 21-30 & Defeat/slay arcane spellcasters (including spellcasting monsters and those that use spell-like abilities) & +2 \\
 31-40 & Defeat/slay divine spellcasters (including divine entities and servitors) & +2 \\
 41-50 & Defeat/slay non-spellcasters & +2 \\
 51-55 & Defeat/slay a particular creature type (see the \textit{bane} special ability for choices)  & +2 \\
 56-60 & Defeat/slay a particular race or kind of creature & +2 \\
 61-70 & Defend a particular race or kind of creature & +2 \\
 71-80 & Defeat/slay the servants of a specific deity & +2 \\
 81-90 & Defend the servants and interests of a specific deity & +2 \\
 91-95 & Defeat/slay all (other than the item and the wielder) & +2 \\
 96-100 & Choose one & +2\\

\end{tabular}
* The purpose of the neutral (N) version of this item is to preserve the balance by defeating/slaying powerful beings of the extreme alignments (LG, LE, CG, CE).\\
\end{table}
% <
\begin{table}[]
\sffamily
\caption{Table: Special Purpose Item Dedicated Powers}
\begin{tabular}{llll}
\textbf{d\%} & \textbf{Dedicated Power} & \textbf{Base Price Modifier} & \textbf{Ego Modifier}\\
01-20 & Item can detect any special purpose foes within 60 feet & +10,000 gp & +1 \\
 21-35 & Item can use a 4th-level spell at will & +56,000 gp & +2 \\
 36-50 & Wielder gets +2 luck bonus on attacks, saves, and checks & +80,000 gp & +2 \\
 51-65 & Item can use a 5th-level spell at will & +90,000 gp & +2 \\
 66-80 & Item can use a 6th-level spell at will & +132,000 gp & +2 \\
 81-95 & Item can use a 7th-level spell at will & +182,000 gp & +2 \\
 96-100 & Item can use \textit{true resurrection }on wielder, once per month  & +200,000 gp & +2\\
\end{tabular}
\end{table}

				
Magic items sometimes have intelligence of their own. Magically imbued with sentience, these items think and feel the same way characters do and should be treated as NPCs. Intelligent items have extra abilities and sometimes extraordinary powers and special purposes. Only permanent magic items (as opposed to single-use items or those with charges) can be intelligent. (This means that potions, scrolls, and wands, among other items, are never intelligent.) In general, less than 1\% of magic items have intelligence.
				
Intelligent items can actually be considered creatures because they have Intelligence, Wisdom, and Charisma scores. Treat them as constructs. Intelligent items often have the ability to illuminate their surroundings at will (as magic weapons do); many cannot see otherwise.
				
Unlike most magic items, intelligent items can activate their own powers without waiting for a command word from their owner. Intelligent items act during their owner's turn in the initiative order.
				
\subsection{Designing an Intelligent Item}

				
Creating a magic item with intelligence follows these simple guidelines. Intelligent items must have an alignment, mental ability scores, languages, senses, and at least one other special ability. These statistics and abilities can be improved during creation, increasing the item's overall cost. Many of these abilities add to an item's Ego score. Intelligent items with high Ego scores are difficult to control and can sometimes take control of their owner, making them dangerous to possess.
				
An intelligent magic item has a base price increase of 500 gp. When determining the total value of an intelligent item, add this value to the sum of the prices of all of its additional abilities gained through being intelligent, before adding them to the magic item's base price.
				
\subsection{Intelligent Item Alignment}

				
Any item with intelligence has an alignment (see Table: Intelligent Item Alignment). Note that intelligent weapons already have alignments, either stated or by implication. If you're generating a random intelligent weapon, that weapon's alignment must fit with any alignment-oriented special abilities it has.
				
Any character whose alignment does not correspond to that of the item (except as noted by the asterisks on the table) gains one negative level if he or she so much as picks up the item. Although this negative level never results in actual level loss, it remains as long as the item is in hand and cannot be overcome in any way (including by \textit{restoration }spells). This negative level is cumulative with any other penalties the item might place on inappropriate wielders. Items with Ego scores (see below) of 20 to 29 bestow two negative levels. Items with Ego scores of 30 or higher bestow three negative levels.
				
\subsection{Intelligent Item Ability Scores}

				
Intelligent magic items possess all three mental ability scores: Intelligence, Wisdom, and Charisma. Each one of these ability scores begins at a value of 10, but can be increased to as high as 20. Table: Intelligent Item Ability Scores shows the cost to increase one of the item's ability scores. This cost must be paid for each ability score raised above 10. For example, an intelligent magic item with a 15 Intelligence, 12 Wisdom, and 10 Charisma would cost at least 2,400 gp more than the base item (including the 500 gp for being an intelligent item). 
				
\subsection{Languages Spoken by Item}

				
Like a character, an intelligent item understands Common plus one additional language per point of Intelligence bonus. Choose appropriate languages, taking into account the item's origin and purposes. If the item does not possess speech, it can still read and understand the languages it knows.
				
\subsection{Senses and Communication}

				
Every intelligent magic item begins with the ability to see and hear within 30 feet, as well as the ability to communicate empathically with its owner. Empathy only allows the item to encourage or discourage certain actions through urges and emotions. Additional forms of communication and better senses increase the item's cost and Ego score, as noted on Table: Intelligent Item Senses and Communication.
				
\textbf{Empathy (Su)}: Empathy allows the item to encourage or discourage certain actions by communicating emotions and urges. It does not allow for verbal communication.
				
\textbf{Speech (Su)}: An intelligent item with the capability for speech can talk using any of the languages it knows.
				
\textbf{Telepathy (Su)}: Telepathy allows an intelligent item to communicate with its wielder telepathically, regardless of its known languages. The wielder must be touching the item to communicate in this way.
				
\textbf{Senses}: Senses allow an intelligent magic item to see and hear out to the listed distance. Adding darkvision or blindsense allows the item to use those senses out to the same range as the item's base senses.
				
\textbf{Read Languages (Ex)}: The item can read script in any language, regardless of its known languages.
				
\textbf{\textit{Read Magic} (Sp)}: An intelligent magic item with this ability can read magical writings and scrolls as if through \textit{read magic. }This ability does not allow the magic item to activate scrolls or other items. An intelligent magic item can suppress and resume this ability as a free action. 
				
\subsection{Intelligent Item Powers}

				
Each intelligent item should possess at least one power, although more powerful items might possess a host of powers. To find the item's specific powers, choose or roll on Table: Intelligent Item Powers. All powers function at the direction of the item, although intelligent items generally follow the wishes of their owner. Activating a power or concentrating on an active one is a standard action the item takes. The caster level for these effects is equal to the item's caster level. Save DCs are based off the item's highest mental ability score.
				
\subsection{Special Purpose Items}

				
Some intelligent items have special purposes that guide their actions. Intelligent magic items with a special purpose gain a +2 Ego bonus. An item's purpose must suit the type and alignment of the item and should always be treated reasonably. A purpose of \texttt{{}"{}}defeat/slay arcane spellcasters\texttt{{}"{}} doesn't mean that the sword forces the wielder to kill every wizard she sees. Nor does it mean that the sword believes it is possible to kill every wizard, sorcerer, and bard in the world. It does mean that the item hates arcane spellcasters and wants to bring the local wizards' cabal to ruin, as well as end the rule of a sorcerer-queen in a nearby land. Likewise, a purpose of \texttt{{}"{}}defend elves\texttt{{}"{}} doesn't mean that if the wielder is an elf, he only wants to help the wielder. It means that the item wants to be used in furthering the cause of elves, stamping out their enemies and aiding their leaders. A purpose of \texttt{{}"{}}defeat/slay all\texttt{{}"{}} isn't just a matter of self-preservation. It means that the item won't rest (or let its wielder rest) until it places itself above all others. 
				
Table: Intelligent Item Purpose has a number of sample purposes that a magic item might possess. If the wielder specifically ignores or goes against an intelligent item's special purpose, the item gains a +4 bonus to its Ego until the wielder cooperates. This is in addition to the +2 Ego bonus gained by items with a special purpose.
				
\subsection{Dedicated Powers}

				
A dedicated power operates only when an intelligent item is in pursuit of its special purpose. This determination is always made by the item. It should always be easy and straightforward to see how the ends justify the means. Unlike its other powers, an intelligent item can refuse to use its dedicated powers even if the owner is dominant (see Items Against Characters). The caster level for these effects is equal to the item's caster level. Save DCs are based on the item's highest mental ability score. See Table: Special Purpose Item Dedicated Powers for a list of dedicated powers.
				
\subsection{Item Ego}

				
Ego is a measure of the total power and force of personality that an item possesses. An item's Ego score is the sum of all of its Ego modifiers plus an additional bonus for the cost of the base magic item (excluding the cost of all of the intelligent item enhancements). An item's Ego score helps determine whether the item or the character is dominant in their relationship, as detailed below.


\begin{tabular}{ll}
\textbf{Base Magic Item Value} & \textbf{Ego Modifier} \\
Up to 1,000 gp                 & ---                     \\
1,001 gp to 5,000 gp           & +1                    \\
5,001 gp to 10,000 gp          & +2                    \\
10,001 gp to 20,000 gp         & +3                    \\
20,001 gp to 50,000 gp         & +4                    \\
50,001 gp to 100,000 gp        & +6                    \\
100,001 gp to 200,000 gp       & +8                    \\
200,001 gp and higher          & +12                  
\end{tabular}

				
\subsection{Items against Characters}

				
When an item has an Ego of its own, it has a will of its own. The item is absolutely true to its alignment. If the character who possesses the item is not true to that alignment's goals or the item's special purpose, personality conflict---item against character---results. Similarly, any item with an Ego score of 20 or higher always considers itself superior to any character, and a personality conflict results if the possessor does not always agree with the item.
				
When a personality conflict occurs, the possessor must make a Will saving throw (DC = item's Ego). If the possessor succeeds, she is dominant. If she fails, the item is dominant. Dominance lasts for 1 day or until a critical situation occurs (such as a major battle, a serious threat to either the item or the character, and so on). Should an item gain dominance, it resists the character's desires and demands concessions such as any of the following:
				\begin{itemize}\item  Removal of associates or items whose alignment or personality is distasteful to the item.
				\item  The character divesting herself of all other magic items or items of a certain type.
				\item  Obedience from the character so the item can direct where they go for its own purposes.
				\item  Immediate seeking out and slaying of creatures hateful to the item.
				\item  Magical protections and devices to safeguard the item from molestation when it is not in use.
				\item   That the character carry the item with her on all occasions.
				\item  That the character relinquish the item to a more suitable possessor due to alignment differences or conduct.
\end{itemize}
				
In extreme circumstances, the item can resort to even harsher measures, such as the following:
				\begin{itemize}\item  Force its possessor into combat.
				\item  Refuse to strike opponents.
				\item  Strike at its wielder or her associates.
				\item  Force its possessor to surrender to an opponent.
				\item  Cause itself to drop from the character's grasp.
\end{itemize}
				
Naturally, such actions are unlikely when harmony reigns between the character's and item's alignments or when their purposes and personalities are well matched. Even so, an item might wish to have a lesser character possess it in order to easily establish and maintain dominance over him, or a higher-level possessor so as to better accomplish its goals.
				
All magic items with personalities desire to play an important role in whatever activity is under way, particularly combat. Such items are natural rivals, even with others of the same alignment. No intelligent item wants to share its wielder with others. An intelligent item is aware of the presence of any other intelligent item within 60 feet, and most intelligent items try their best to mislead or distract their host so that she ignores or destroys the rival. Of course, alignment might change this sort of behavior. 
				
Items with personalities are never totally controlled or silenced by the characters that possess them, even though they may never successfully control their possessors. They may be powerless to force their demands, but most remain undaunted and continue to air their wishes and demands.
        	

\section{Cursed Items}

\label{f0}

\begin{table}[]
\sffamily
\caption{Table: Common Item Curses}
\begin{tabular}{lllllll}
\textbf{d\%} & \textbf{Curse}\\
01-15 & Delusion \\
 16-35 & Opposite effect or target \\
 36-45 & Intermittent functioning \\
 46-60 & Requirement \\
 61-75 & Drawback \\
 76-90 & Completely different effect \\
 91-100 & Substitute specific cursed item on Table: Specific Cursed Items\\
\end{tabular}
\end{table}
		
Cursed items are magic items with some sort of potentially negative impact. Occasionally they mix bad with good, forcing characters to make difficult choices. Cursed items are almost never made intentionally. Instead they are the result of rushed work, inexperienced crafters, or a lack of proper components. While many of these items still have functions, they either do not work as intended or come with serious drawbacks. When a magic item creation skill check fails by 5 or more, roll on Table: Common Item Curses to determine the type of curse possessed by the item.
				
\textbf{Identifying Cursed Items}: Cursed items are identified like any other magic item with one exception: unless the check made to identify the item exceeds the DC by 10 or more, the curse is not detected. If the check is not made by 10 or more, but still succeeds, all that is revealed is the magic item's original intent. If the item is known to be cursed, the nature of the curse can be determined using the standard DC to identify the item.
				
\textbf{Removing Cursed Items}: While some cursed items can be simply discarded, others force a compulsion upon the user to keep the item, no matter the costs. Others reappear even if discarded or are impossible to throw away. These items can only be discarded after the character or item is targeted by a \textit{remove curse} or similar magic. The DC of the caster level check to undo the curse is equal to 10 + the item's caster level. If the spell is successful, the item can be discarded on the following round, but the curse reasserts itself if the item is used again.
				
\subsection{Common Cursed Item Effects}

				
The following are some of the most common cursed item effects. GMs should feel free to invent new cursed item effects to fit specific items. 
				
\textbf{Delusion}: The user believes the item is what it appears to be, yet it actually has no magical power other than to deceive. The user is mentally fooled into thinking the item is functioning and cannot be convinced otherwise without the casting of \textit{remove curse}.
				
\textbf{Opposite Effect or Target}: These cursed items malfunction, so that either they do the opposite of what the creator intended, or they target the user instead of someone else. The interesting point to keep in mind here is that these items aren't always bad to have. Opposite-effect items include weapons that impose penalties on attack and damage rolls rather than bonuses. Just as a character shouldn't necessarily immediately know what the enhancement bonus of a noncursed magic item is, she shouldn't immediately know that a weapon is cursed. Once she knows, however, the item can be discarded unless some sort of compulsion is placed upon it that compels the wielder to keep and use it. In such cases, a \textit{remove curse }spell is generally needed to get rid of the item.
				
\textbf{Intermittent Functioning}: The three varieties of intermittent functioning items all function perfectly as intended---at least some of the time. The three varieties are unreliable, dependent, and uncontrolled items.
				
\textit{Unreliable}: Each time the item is activated, there is a 5\% chance (01--05 on d\%) that it does not function.
				
\textit{Dependent}: The item only functions in certain situations. To determine the situation, select or roll on the following table.
\begin{tabular}{ll}
\textbf{d\%} & \textbf{Situation}                                    \\
01-03        & Temperature below freezing                            \\
04-05        & Temperature above freezing                            \\
06-10        & During the day                                        \\
11-15        & During the night                                      \\
16-20        & In direct sunlight                                    \\
21-25        & Out of direct sunlight                                \\
26-34        & Underwater                                            \\
35-37        & Out of water                                          \\
38-45        & Underground                                           \\
46-55        & Aboveground                                           \\
56-60        & Within 10 feet of a random creature type              \\
61-64        & Within 10 feet of a random race or kind of creature   \\
65-72        & Within 10 feet of an arcane spellcaster               \\
73-80        & Within 10 feet of a divine spellcaster                \\
81-85        & In the hands of a nonspellcaster                      \\
86-90        & In the hands of a spellcaster                         \\
91-95        & In the hands of a creature of a particular alignment  \\
96           & In the hands of a creature of a particular gender     \\
97-99        & On holy days or during particular astrological events \\
100            & More than 100 miles from a particular site           
\end{tabular}

\textit{Uncontrolled}: An uncontrolled item occasionally activates at random times. Roll d\% every day. On a result of 01--05 the item activates at some random point during that day. 
				
\textbf{Requirement}: Some items have stringent requirements that must be met for them to be usable. To keep an item with this kind of curse functioning, one or more of the following conditions must be met.
				\begin{itemize}\item  Character must eat twice as much as normal.
				\item  Character must sleep twice as much as normal.
				\item  Character must undergo a specific quest (one time only, and the item functions normally thereafter).
				\item  Character must sacrifice (destroy) 100 gp in valuables per day.
				\item  Character must sacrifice (destroy) 2,000 gp worth of magic items each week.
				\item  Character must swear fealty to a particular noble or to his entire family.
				\item  Character must discard all other magic items.
				\item  Character must worship a particular deity.
				\item  Character must change her name to a specific name. The item only works for characters of that name.
				\item  Character must add a specific class at the next opportunity if not of that class already.
				\item  Character must have a minimum number of ranks in a particular skill.
				\item  Character must sacrifice some part of her life energy (2 points of Constitution) one time. If the character gets the Constitution points back (such as from a \textit{restoration }spell), the item ceases functioning. (The item does not cease functioning if the character receives a Constitution increase caused by level gain, a \textit{wish, }or the use of a magic item.)
				\item  Item must be cleansed with holy water each day.
				\item  Item must be used to kill a living creature each day.
				\item  Item must be bathed in volcanic lava once per month.
				\item  Item must be used at least once a day, or it won't function again for its current possessor.
				\item  Item must draw blood when wielded (weapons only). It can't be put away or exchanged for another weapon until it has scored a hit.
				\item  Item must have a particular spell cast upon it each day (such as \textit{bless, atonement, }or \textit{animate objects}).
\end{itemize}
				
Requirements are so dependent upon suitability to the item that they should never be determined randomly. An intelligent item with a requirement often imposes its requirement through its personality. If the requirement is not met, the item ceases to function. If it is met, usually the item functions for one day before the requirement must be met again (although some requirements are one time only, others monthly, and still others continuous).
				
\textbf{Drawback}: Items with drawbacks are usually still beneficial to the possessor but carry some negative aspect. Although sometimes drawbacks occur only when the item is used (or held, in the case of some weapons), usually the drawback remains with the character for as long as she has the item.
				
Unless otherwise indicated, drawbacks remain in effect as long as the item is possessed. The DC to save against any of these effects is usually equal to 10 + the item's caster level.

\begin{tabular}{ll}
\textbf{d\%} & \textbf{Drawback}                                                                                       \\
01-04        & Character's hair grows 1 inch longer every hour.                                                        \\
05-09        & Character either shrinks 6 inches (01-50 on d\%) or grows that much taller (51-100). Only happens once. \\
10-13        & Temperature around item is 10° F cooler than normal.                                                    \\
14-17        & Temperature around item is 10° F warmer than normal.                                                    \\
18-21        & Character's hair color changes.                                                                         \\
22-25        & Character's skin color changes.                                                                         \\
26-29        & Character now bears some identifying mark (tattoo, weird glow, or the like).                            \\
30-32        & Character's gender changes.                                                                             \\
33-34        & Character's race or kind changes.                                                                       \\
35           & Character is afflicted with a random disease that cannot be cured.                                      \\
36-39        & Item continually emits a disturbing sound (moaning, weeping, screaming, cursing, insults).              \\
40           & Item looks ridiculous (garishly colored, silly shape, glows bright pink).                               \\
41-45        & Character becomes selfishly possessive.                                                                 \\
46-49        & Character becomes paranoid about losing the item and afraid of damage occurring to it.                  \\
50-51        & Character's alignment changes.                                                                          \\
52-54        & Character must attack nearest creature (5\% chance {[}01-05 on d\%{]} each day).                        \\
55-57        & Character is stunned for 1d4 rounds once item function is finished (or randomly, 1/day).                \\
58-60        & Character's vision is blurry (-2 penalty on attack rolls, saves, and skill checks requiring vision).    \\
61-64        & Character gains one negative level.                                                                     \\
65           & Character gains two negative levels.                                                                    \\
66-70        & Character must make a Will save each day or take 1 point of Intelligence damage.                        \\
71-75        & Character must make a Will save each day or take 1 point of Wisdom damage.                              \\
76-80        & Character must make a Will save each day or take 1 point of Charisma damage.                            \\
81-85        & Character must make a Fortitude save each day or take 1 point of Constitution damage.                   \\
86-90        & Character must make a Fortitude save each day or take 1 point of Strength damage.                       \\
91-95        & Character must make a Fortitude save each day or take 1 point of Dexterity damage.                      \\
96           & Character is polymorphed into a specific creature (5\% chance {[}01-05 on d\%{]} each day).             \\
97           & Character cannot cast arcane spells.                                                                    \\
98           & Character cannot cast divine spells.                                                                    \\
99           & Character cannot cast any spells.                                                                       \\
100          & Either pick one of the above that's appropriate or create a drawback specifically for that item.       
\end{tabular}
\section{Specific Cursed Items}

\label{f0}
\begin{table}[]
\sffamily
\caption{Table: Specific Cursed Items}
\begin{tabular}{ll}
\textbf{d\%} & \textbf{Item}\\
01--05 & Incense of obsession \\
 06--15 & Ring of clumsiness \\
 16--20 & Amulet of inescapable location \\
 21--25 & Stone of weight \\
 26--30 & Bracers of defenselessness \\
 31--35 & Gauntlets of fumbling \\
 36--40 & --2 sword, cursed \\
 41--43 & Armor of rage \\
 44--46 & Medallion of thought projection \\
 47--52 & Flask of curses \\
 53--54 & Dust of sneezing and choking \\
 55 & Helm of opposite alignment \\
 56--60 & Potion of poison \\
 61 & Broom of animated attack \\
 62--63 & Robe of powerlessness \\
 64 & Vacuous grimoire \\
 65--68 & Spear, cursed backbiter \\
 69--70 & Armor of arrow attraction \\
 71--72 & Net of snaring \\
 73--75 & Bag of devouring \\
 76--80 & Mace of blood \\
 81--85 & Robe of vermin \\
 86--88 & Periapt of foul rotting \\
 89--92 & Sword, berserking \\
 93--96 & Boots of dancing \\
 97 & Crystal hypnosis ball \\
 98 & Necklace of strangulation \\
 99 & Poisonous cloak \\
 100 & Scarab of death\\
\end{tabular}
\end{table}

				
Perhaps the most dangerous and insidious of all cursed items are those whose intended functions are completely replaced by a curse. Yet even these items can have their uses, particularly as traps or weapons. The following are provided as specific examples of cursed items. Instead of prerequisites, each cursed item is associated with one or more ordinary magic items whose creation might result in the cursed item. Cursed items can be sold, if the curse is not known to the buyer, as if they were the item they appear to be.
				
Cursed suits of armor and weapons can come in many forms, and the examples listed here are merely the most common. For example, a \textit{cursed --2 sword}, might appear as a \textit{+3 shortsword }or a \textit{+1 dagger}, with a similar negative instead of the listed --2.
				
\textbf{Amulet of Inescapable Location}
				
\textbf{Aura} moderate abjuration; \textbf{CL} 10th
				
\textbf{Slot} neck; \textbf{Weight }1/2 lb.
				
Description
				
This device appears to prevent location, scrying and detection, or influence by \textit{detect thoughts }or telepathy, as per an \textit{amulet of proof against detection and location}. Actually, the amulet gives the wearer a --10 penalty on all saves against divination spells. 
				
Creation
				
\textbf{Magic Items}\textit{ amulet of proof against detection and location}
				
\textbf{Armor of Arrow Attraction}
				
\textbf{Aura} strong abjuration; \textbf{CL} 16th
				
\textbf{Slot} armor; \textbf{Weight }50 lbs.
				
Description
				
Magical analysis indicates that this armor is a normal suit of \textit{+3 full plate. }The armor works normally with regard to melee attacks but actually attracts ranged weapons. The wearer takes a --15 penalty to AC against ranged weapons. The true nature of the armor does not reveal itself until the character is fired upon in earnest. 
				
Creation
				
\textbf{Magic Items}\textit{ +3 full plate}
				
\textbf{Armor of Rage}
				
\textbf{Aura} strong necromancy; \textbf{CL} 16th
				
\textbf{Slot} body; \textbf{Weight }50 lbs.
				
Description
				
This armor is similar in appearance to \textit{breastplate of command} and functions as a suit of \textit{+1 breastplate}. However, when it is worn, the armor causes the character to take a --4 penalty to Charisma. All unfriendly characters within 300 feet have a +1 morale bonus on attack rolls against her. The effect is not noticeable to the wearer or those affected. In other words, the wearer does not immediately notice that donning the armor is the cause of her problems, nor do foes understand the reason for the depth of their enmity.
				
Creation
				
\textbf{Magic Items}\textit{ breastplate of command, +1 breastplate}
				
\textbf{Bag of Devouring}
				
\textbf{Aura} strong conjuration; \textbf{CL} 17th
				
\textbf{Slot} none; \textbf{Weight }15 lbs.
				
Description
				
This bag appears to be an ordinary sack. Detection for magical properties makes it seem as if it were a \textit{bag of holding}. The sack is, however, something entirely different and more insidious. It is---in fact, one of the feeding orifices of an extradimensional creature.
				
Any substance of animal or vegetable nature is subject to \texttt{{}"{}}swallowing'' if thrust within the bag. The \textit{bag of devouring }is 90\% likely to ignore any initial intrusion, but anytime thereafter that it senses living flesh within (such as if someone reaches into the bag to pull something out), it is 60\% likely to close around the offending member and attempt to draw the whole victim in. The bag has a +8 bonus on combat maneuver checks made to grapple. If it pins a creature, it pulls them inside as a free action. The bag has CMD of 18 for those attempting to break free.
				
The bag can hold up to 30 cubic feet of matter. It acts as a \textit{bag of holding type I}, but each hour it has a 5\% cumulative chance of swallowing the contents and then spitting the stuff out in some nonspace or on some other plane. Creatures drawn within are consumed in 1 round. The bag destroys the victim's body and prevents any form of raising or resurrection that requires part of the corpse. There is a 50\% chance that a \textit{wish}, \textit{miracle}, or \textit{true resurrection} spell can restore a devoured victim to life. Check once for each destroyed creature. If the check fails, the creature cannot be brought back to life by mortal magic. 
				
Creation
				
\textbf{Magic Items}\textit{ bag of holding }(any type)
				
\textbf{Boots of Dancing}
				
\textbf{Aura} strong enchantment; \textbf{CL} 16th
				
\textbf{Slot} feet; \textbf{Weight }1 lb.
				
Description
				
These boots appear and function as one of the other kinds of magic boots. When the wearer is in (or fleeing from) melee combat, \textit{boots of dancing }impede movement, making him behave as if \textit{irresistible dance }had been cast upon him. Only a \textit{remove curse }spell enables the wearer to be rid of the boots once their true nature is revealed. 
				
Creation
				
\textbf{Magic Items}\textit{ boots of elvenkind, boots of levitation, boots of speed, boots of striding and springing, boots of teleportation, boots of the winterlands, winged boots}
				
\textbf{Bracers of Defenselessness}
				
\textbf{Aura} strong conjuration; \textbf{CL} 16th
				
\textbf{Slot} arms; \textbf{Weight }1 lb.
				
Description
				
These bejeweled and shining bracers initially appear to be \textit{bracers of armor +5 }and actually serve as such until the wearer is attacked in anger by an enemy with a Challenge Rating equal to or greater than her level. At that moment and thereafter, the bracers cause a --5 penalty to AC. Once their curse is activated, \textit{bracers of defenselessness }can be removed only by means of a \textit{remove curse }spell. 
				
Creation
				
\textbf{Magic Items}\textit{ bracers of armor +5}
				
\textbf{Broom of Animated Attack}
				
\textbf{Aura} moderate transmutation; \textbf{CL} 10th
				
\textbf{Slot} none; \textbf{Weight }3 lbs.
				
Description
				
This item is indistinguishable in appearance from a normal broom. It is identical to a \textit{broom of flying }by all tests short of attempted use.
				
If a creature attempts to fly using the broom, the broom does a loop-the-loop with its hopeful rider, dumping him on his head from 1d4+5 feet off the ground (no falling damage, since the fall is less than 10 feet). The broom then attacks the victim, swatting the victim's face with the straw or twig end and beating him with the handle end. The broom gets two attacks per round with each end (two swats with the straw and two with the handle, for a total of four attacks per round). It attacks with a +5 bonus on each attack roll. The straw end causes a victim to be blinded for 1 round when it hits. The handle deals 1d6 points of damage when it hits. The broom has AC 13, CMD 17, 18 hit points, and hardness 4. 
				
Creation
				
\textbf{Magic Items}\textit{ broom of flying}
				
\textbf{Crystal Hypnosis Ball}
				
\textbf{Aura} strong divination; \textbf{CL} 17th
				
\textbf{Slot} none; \textbf{Weight }7 lbs.
				
Description
				
This cursed scrying device is indistinguishable, at first glance, from a normal \textit{crystal ball. }However, anyone attempting to use the scrying device becomes fascinated for 1d6 minutes, and a telepathic \textit{suggestion }is implanted in his mind (Will DC 19 negates).
				
The user of the device believes that the desired creature or scene was viewed, but actually he came under the influence of a powerful wizard, lich, or even some power or being from another plane. Each further use brings the \textit{crystal hypnosis ball} gazer deeper under the influence of the controller, either as a servant or a tool. Note that throughout this time, the user remains unaware of his subjugation. 
				
Creation
				
\textbf{Magic Items}\textit{ crystal ball}
				
\textbf{Dust of Sneezing and Choking}
				
\textbf{Aura} moderate conjuration; \textbf{CL} 7th
				
\textbf{Slot} none; \textbf{Weight }---
				
Description
				
This fine dust appears to be \textit{dust of appearance. }If cast into the air, it causes those within a 20-foot spread to fall into fits of sneezing and coughing. Those failing a DC 15 Fortitude save take 3d6 points of Constitution damage immediately. Those who succeed on this saving throw are nonetheless disabled by choking (treat as stunned) for 5d4 rounds. 
				
Creation
				
\textbf{Magic Items}\textit{ dust of appearance, dust of tracelessness}
				
\textbf{Flask of Curses}
				
\textbf{Aura} moderate conjuration; \textbf{CL} 7th
				
\textbf{Slot} none; \textbf{Weight }2 lbs.
				
Description
				
This item looks like an ordinary beaker, bottle, container, decanter, flask, or jug. It may contain a liquid, or it may emit smoke. When the flask is first unstoppered, all within 30 feet must make a DC 17 Will save or be cursed, taking a --2 penalty on attack rolls, saving throws, and skill checks until a \textit{remove curse }spell is cast upon them. 
				
Creation
				
\textbf{Magic Items}\textit{ decanter of endless water, efreeti bottle, eversmoking bottle, iron flask}
				
\textbf{Gauntlets of Fumbling}
				
\textbf{Aura} moderate transmutation; \textbf{CL} 7th
				
\textbf{Slot} hands; \textbf{Weight }2 lbs.
				
Description
				
These gauntlets perform according to their appearance until the wearer finds herself under attack or in a life-and-death situation. At that time, the curse is activated. The wearer becomes fumble-fingered, with a 50\% chance each round of dropping anything held in either hand. The gauntlets also lower Dexterity by 2 points. Once the curse is activated, the gloves can be removed only by means of a \textit{remove curse }spell, a \textit{wish, }or a \textit{miracle.} 
				
Creation
				
\textbf{Magic Items}\textit{ gauntlet of rust, gloves of arrow snatching, glove of storing, gloves of swimming and climbing}
				
\textbf{Helm of Opposite Alignment}
				
\textbf{Aura} strong transmutation; \textbf{CL} 12th
				
\textbf{Slot} head; \textbf{Weight} 3 lbs.
				
Description
				
When placed upon the head, this item's curse immediately takes effect (Will DC 15 negates). On a failed save, the alignment of the wearer is radically altered to an alignment as different as possible from the former alignment---good to evil, chaotic to lawful, neutral to some extreme commitment (LE, LG, CE, or CG). Alteration in alignment is mental as well as moral, and the individual changed by the magic thoroughly enjoys his new outlook. A character who succeeds on his save can continue to wear the helmet without suffering the effect of the curse, but if he takes it off and later puts it on again, another save is required. 
				
Only a \textit{wish} or a \textit{miracle} can restore a character's former alignment, and the affected individual does not make any attempt to return to the former alignment. In fact, he views the prospect with horror and avoids it in any way possible. If a character of a class with an alignment requirement is affected, an \textit{atonement} spell is needed as well if the curse is to be obliterated. When a \textit{helm of opposite alignment} has functioned once, it loses its magical properties. 
				
Creation
				
\textbf{Magic Items}\textit{ hat of disguise, helm of comprehend languages and read magic, helm of telepathy}
				
\textbf{Incense of Obsession}
				
\textbf{Aura} moderate enchantment; \textbf{CL} 6th
				
\textbf{Slot} none; \textbf{Weight }---
				
Description
				
These blocks of incense appear to be \textit{incense of meditation. }If meditation is conducted while \textit{incense of obsession }is burning, the user becomes totally confident that her spell ability is superior due to the magic incense. She uses her spells at every opportunity, even when not needed or useless. The user remains obsessed with her abilities and spells until all have been used or cast, or until 24 hours have elapsed. 
				
Creation
				
\textbf{Magic Items}\textit{ incense of meditation}
				
\textbf{Mace of Blood}
				
\textbf{Aura} moderate abjuration; \textbf{CL} 8th
				
\textbf{Slot} none; \textbf{Weight }8 lbs.
				
Description
				
This \textit{+3 heavy mace }must be coated in blood every day, or else its bonus fades away until the mace is coated again. The character using this mace must make a DC 13 Will save every day it is within his possession or become chaotic evil. 
				
Creation
				
\textbf{Magic Items}\textit{ +3 heavy mace}
				
\textbf{Medallion of Thought Projection}
				
\textbf{Aura} moderate divination; \textbf{CL} 7th
				
\textbf{Slot} neck; \textbf{Weight }---
				
Description
				
This device seems like a \textit{medallion of thoughts}, even down to the range at which it functions, except that the thoughts overheard are muffled and distorted, requiring a DC 15 Will save to sort them out. However, while the user thinks she is picking up the thoughts of others, all she is really hearing are figments created by the medallion itself. These illusory thoughts always seem plausible and thus can seriously mislead any who rely upon them. What's worse, unknown to her, the cursed medallion actually broadcasts her thoughts to creatures in the path of the beam, thus alerting them to her presence. 
				
Creation
				
\textbf{Magic Items}\textit{ medallion of thoughts}
				
\textbf{Necklace of Strangulation}
				
\textbf{Aura} strong conjuration; \textbf{CL} 18th
				
\textbf{Slot} neck; \textbf{Weight }---
				
Description
				
A \textit{necklace of strangulation }appears to be a wondrous piece of magical jewelry. When placed on the neck, the necklace immediately tightens, dealing 6 points of damage per round. It cannot be removed by any means short of a \textit{limited wish, wish, }or \textit{miracle }and remains clasped around the victim's throat even after his death. Only when he has decayed to a dry skeleton (after approximately 1 month) does the necklace loosen, ready for another victim. 
				
Creation
				
\textbf{Magic Items}\textit{ necklace of adaptation, necklace of fireballs, periapt of health, periapt of proof against poison, periapt of wound closure}
				
\textbf{Net of Snaring}
				
\textbf{Aura} moderate evocation; \textbf{CL} 8th
				
\textbf{Slot} none; \textbf{Weight }6 lbs.
				
Description
				
This net provides a +3 bonus on attack rolls but can only be used underwater. Underwater, it can be commanded to shoot forth up to 30 feet to trap a creature. If thrown on land, it changes course to target the creature that threw it.
				
Creation
				
\textbf{Magic Items}\textit{ +3 net}
				
\textbf{Periapt of Foul Rotting}
				
\textbf{Aura} moderate abjuration; \textbf{CL} 10th
				
\textbf{Slot} neck; \textbf{Weight }---
				
Description
				
This engraved gem appears to be of little value. If any character keeps the periapt in her possession for more than 24 hours, she contracts a terrible rotting affliction that permanently drains 1 point of Dexterity, Constitution, and Charisma every week. The periapt (and the affliction) can be removed only by application of a \textit{remove curse }spell followed by a \textit{cure disease }and then a \textit{heal, miracle, limited wish}, or \textit{wish }spell. The rotting can also be countered by crushing a \textit{periapt of health }and sprinkling its dust upon the afflicted character (a full-round action), whereupon the \textit{periapt of foul rotting} likewise crumbles to dust. 
				
Creation
				
\textbf{Magic Items}\textit{ periapt of health, periapt of proof against poison, periapt of wound closure}
				
\textbf{Poisonous Cloak}
				
\textbf{Aura} strong abjuration; \textbf{CL} 15th
				
\textbf{Slot} shoulders; \textbf{Weight }1 lb.
				
Description
				
This cloak is usually made of a wool, although it can be made of leather. A \textit{detect poison }spell can reveal the presence of poison in the cloak's fabric. The garment can be handled without harm, but as soon as it is actually donned, the wearer takes 4d6 points of Constitution damage unless she succeeds on a DC 28 Fortitude save. 
				
Once donned, a \textit{poisonous cloak} can be removed only with a \textit{remove curse} spell; doing this destroys the magical property of the cloak. If a \textit{neutralize poison} spell is then used, it is possible to revive a dead victim with a \textit{raise dead} or \textit{resurrection spell}. 
				
Creation
				
\textbf{Magic Items}\textit{ cloak of arachnida, cloak of the bat, cloak of etherealness, cloak of resistance +5, major cloak of displacement}
				
\textbf{Potion of Poison}
				
\textbf{Aura} strong conjuration; \textbf{CL} 12th
				
\textbf{Slot} none; \textbf{Weight }---
				
Description
				
This potion has lost its beneficial abilities and become a potent poison. This poison deals 1d3 Constitution damage per round for 6 rounds. A poisoned creature can make a DC 14 Fortitude save each round to negate the damage and end the affliction. 
				
Creation
				
\textbf{Magic Items} any potion
				
\textbf{Robe of Powerlessness}
				
\textbf{Aura} strong transmutation; \textbf{CL} 13th
				
\textbf{Slot} body; \textbf{Weight }1 lb.
				
Description
				
A \textit{robe of powerlessness }appears to be a magic robe of another sort. As soon as a character dons this garment, she takes a --10 penalty to Strength, as well as to Intelligence, Wisdom, or Charisma, forgetting spells and magic knowledge accordingly. If the character is a spellcaster, the robe targets the character's primary spellcasting score, otherwise it targets Intelligence. The robe can be removed easily, but in order to restore mind and body, the character must receive a \textit{remove curse }spell followed by \textit{heal}. 
				
Creation
				
\textbf{Magic Items}\textit{ robe of the archmagi, robe of blending, robe of bones, robe of eyes, robe of scintillating colors, robe of stars, robe of useful items}
				
\textbf{Robe of Vermin}
				
\textbf{Aura} strong abjuration; \textbf{CL} 13th
				
\textbf{Slot} body; \textbf{Weight }1 lb.
				
Description
				
The wearer notices nothing unusual when the robe is donned, and it functions normally. However, as soon as he is in a situation requiring concentration and action against hostile opponents, the true nature of the garment is revealed: the wearer immediately suffers a multitude of bites from the insects that magically infest the garment. He must cease all other activities in order to scratch, shift the robe, and generally show signs of the extreme discomfort caused by the bites and movement of these pests.
				
The wearer takes a --5 penalty on initiative checks and a --2 penalty on all attack rolls, saves, and skill checks. If he tries to cast a spell, he must make a concentration check (DC 20 + spell level) or lose the spell. 
				
Creation
				
\textbf{Magic Items}\textit{ robe of the archmagi, robe of blending, robe of bones, robe of eyes, robe of scintillating colors, robe of stars, robe of useful items}
				
\textbf{Ring of Clumsiness}
				
\textbf{Aura} strong transmutation; \textbf{CL} 15th
				
\textbf{Slot} ring; \textbf{Weight }---
				
Description
				
This ring operates exactly like a \textit{ring of feather falling}. However, it also makes the wearer clumsy. She takes a --4 penalty to Dexterity and has a 20\% chance of spell failure when trying to cast any arcane spell that has a somatic component. (This chance of spell failure stacks with other arcane spell failure chances.) 
				
Creation
				
\textbf{Magic Items}\textit{ ring of feather falling}
				
\textbf{Scarab of Death}
				
\textbf{Aura} strong abjuration; \textbf{CL} 19th
				
\textbf{Slot} neck; \textbf{Weight }---
				
Description
				
If this small scarab brooch is held for more than 1 round or carried in a living creature's possessions for 1 minute, it changes into a horrible burrowing beetle-like creature. The thing tears through any leather or cloth, burrows into flesh, and reaches the victim's heart in 1 round, causing death. A DC 25 Reflex save allows the wearer to tear the scarab away before it burrows out of sight, but he still takes 3d6 points of damage. The beetle then returns to its scarab form. Placing the scarab in a container of wood, ceramic, bone, ivory, or metal prevents it from coming to life and allows for long-term storage of the item. 
				
Creation
				
\textbf{Magic Items}\textit{ amulet of mighty fists, amulet of natural armor, amulet of the planes, amulet of proof against detection and location, brooch of shielding, golembane scarab, scarab of protection}
				
\textbf{Spear, Cursed Backbiter}
				
\textbf{Aura} moderate evocation; \textbf{CL} 10th
				
\textbf{Slot} none; \textbf{Weight }3 lbs.
				
Description
				
This is a \textit{+2 shortspear, }but each time it is used in melee against a foe and the attack roll is a natural 1, it damages its wielder instead of her intended target. When the curse takes effect, the spear curls around to strike its wielder in the back, automatically dealing the damage to the wielder. The curse even functions when the spear is hurled, and in such a case the damage to the hurler is doubled. 
				
Creation
				
\textbf{Magic Items}\textit{ +2 shortspear}, any magic weapon
				
\textbf{Stone of Weight (Loadstone)}
				
\textbf{Aura} faint transmutation; \textbf{CL} 5th
				
\textbf{Slot} none; \textbf{Weight }1 lb.
				
Description
				
This dark, polished stone reduces the possessor's base land speed to half of normal. Once picked up, the stone cannot be disposed of by any nonmagical means---if it is thrown away or smashed, it reappears somewhere upon the possessor's person. If a \textit{remove curse }spell is cast upon a \textit{loadstone, }the item may be discarded normally and no longer haunts the individual. 
				
Creation
				
\textbf{Magic Items}\textit{ ioun stone, stone of alarm, stone of controlling earth elementals, stone of good luck}
				
\textbf{Sword, --2 Cursed}
				
\textbf{Aura} strong evocation; \textbf{CL} 15th
				
\textbf{Slot} none; \textbf{Weight }4 lbs.
				
Description
				
This longsword performs well against targets in practice, but when used in combat its wielder takes a --2 penalty on attack rolls.
				
All damage dealt is also reduced by 2 points, but never below a minimum of 1 point of damage on any successful hit. The sword always forces that character to employ it rather than another weapon. The sword's owner automatically draws it and fights with it even when she meant to draw or ready some other weapon.
				
Creation
				
\textbf{Magic Items}\textit{ +2 longsword}, any magic weapon
				
\textbf{Sword, Berserking}
				
\textbf{Aura} moderate evocation; \textbf{CL} 8th
				
\textbf{Slot} none; \textbf{Weight }12 lbs.
				
Description
				
This sword appears to be a \textit{+2 greatsword. }However, whenever it is used in battle, its wielder goes berserk (gaining all the benefits and drawbacks of the barbarian's rage ability). He attacks the nearest creature and continues to fight until unconscious or dead or until no living thing remains within 30 feet. Although many see this sword as a cursed object, others see it as a boon. 
				
Creation
				
\textbf{Magic Items}\textit{ +2 greatsword}, any magic weapon
				
\textbf{Vacuous Grimoire}
				
\textbf{Aura} strong enchantment; \textbf{CL} 20th
				
\textbf{Slot} none; \textbf{Weight }2 lbs.
				
Description
				
A book of this sort looks like a normal one on some mildly interesting topic. Any character who opens the work and reads so much as a single word therein must make two DC 15 Will saves. The first is to determine if the reader takes 1 point of permanent Intelligence and Charisma drain. The second is to find out if the reader takes 2 points of permanent Wisdom drain. To destroy the book, it must be burned while \textit{remove curse} is being cast. If the grimoire is placed with other books, its appearance instantly alters to conform to the look of those other works. 
				
Creation
				
\textbf{Magic Items}\textit{ blessed book, manual of bodily health, manual of gainful exercise, manual of quickness of action, tome of clear thoughts, tome of leadership and influence, tome of understanding}
        	

\section{Artifacts}

\label{f0}				
Artifacts are extremely powerful. Rather than merely another form of magical equipment, they are the sorts of legendary relics that whole campaigns can be based on. Each could be the center of a whole set of adventures---a quest to recover it, a fight against an opponent wielding it, a mission to cause its destruction, and so on. 
				
Unlike normal magic items, artifacts are not easily destroyed. Instead of construction information, each artifact includes one possible means by which it might be destroyed.
				
Artifacts can never be purchased, nor are they found as part of a random treasure hoard. When placing an artifact in your game, be sure to consider its impact and role. Remember that artifacts are fickle objects, and if they become too much of a nuisance, they can easily disappear or become lost once again.
				
\subsection{Minor Artifacts}

				
Minor artifacts are not necessarily unique items. Even so, they are magic items that no longer can be created, at least by common mortal means.
				
\textbf{Book of Infinite Spells}
				
\textbf{Aura} strong (all schools); \textbf{CL} 18th
				
\textbf{Slot} none; \textbf{Weight} 3 lbs.
				
Description
				

% <span class="stat-description-char">
This work bestows upon any character of any class the ability to use the spells within its pages. However, any character not already able to use spells gains one 
% </span class="stat-description-char">
negative level 
% <span class="stat-description-char">
for as long as the book is in her possession or while she uses its power. A 
% </span class="stat-description-char">
\textit{book of infinite spells}
% <span class="stat-description-char">
 contains 1d8+22 pages. The nature of each page is determined by a d\% roll: 01--50, arcane spell; 51--100, divine spell. 
% </span class="stat-description-char">

				
Determine the exact spell randomly.
				
Once a page is turned, it can never be flipped back---paging through a \textit{book of infinite spells }is a one-way trip. If the book is closed, it always opens again to the page it was on before the book was closed. When the last page is turned, the book vanishes.
				
Once per day the owner of the book can cast the spell to which the book is opened. If that spell happens to be one that is on the character's class spell list, she can cast it up to four times per day. The pages cannot be ripped out without destroying the book. Similarly, the spells cannot be cast as scroll spells, nor can they be copied into a spellbook---their magic is bound up permanently within the book itself.
				
The owner of the book need not have the book on her person in order to use its power. The book can be stored in a place of safety while the owner is adventuring and still allow its owner to cast spells by means of its power.
				
Each time a spell is cast, there is a chance that the energy connected with its use causes the page to magically turn despite all precautions. The chance of a page turning depends on the spell the page contains and what sort of spellcaster the owner is.
% <thead class="stat-description-char">

\begin{tabular}{ll}
\textbf{Condition} & \textbf{Chance of Page Turning} \\
Caster employing a spell usable by own class and level & 10\% \\
Caster employing a spell not usable by own class and level & 20\% \\
Nonspellcaster employing divine spell & 25\% \\
Nonspellcaster employing arcane spell & 30\% \\
\end{tabular}

				
Treat each spell use as if a scroll were being employed, for purposes of determining casting time, spell failure, and so on. 
				
Destruction
				
The \textit{book of infinite spells }can be destroyed when the current page contains the \textit{erase }spell, by casting the spell on the book itself.
				
\textbf{Deck of Many Things}
				
\textbf{Aura} strong (all schools); \textbf{CL} 20th
				
\textbf{Slot} none; \textbf{Weight }---
				
Description
				
A \textit{deck of many things }(both beneficial and malign) is usually found in a box or leather pouch. Each deck contains a number of cards or plaques made of ivory or vellum. Each is engraved with glyphs, characters, and sigils. As soon as one of these cards is drawn from the pack, its magic is bestowed upon the person who drew it, for better or worse.
				
The character with a \textit{deck of many things }who wishes to draw a card must announce how many cards she will draw before she begins. Cards must be drawn within 1 hour of each other, and a character can never draw from this deck any more cards than she has announced. If the character does not willingly draw her allotted number (or if she is somehow prevented from doing so), the cards flip out of the deck on their own. If the Jester is drawn, the possessor of the deck may elect to draw two additional cards.
				
Each time a card is taken from the deck, it is replaced (making it possible to draw the same card twice) unless the draw is the Jester or the Fool, in which case the card is discarded from the pack. A \textit{deck of many things }contains 22 cards. To simulate the magic cards, you may want to use tarot cards, as indicated in the second column of the accompanying table. If no tarot deck is available, substitute ordinary playing cards instead, as indicated in the third column. The effects of each card, summarized on the table, are fully described below.
				
\begin{table}[]
\begin{tabular}{llll}
\textbf{Plaque} & \textbf{Tarot Card} & \textbf{Playing Card}     & \textbf{Summary of Effect}                                 \\
Balance         & XI. Justice         & Two of spades             & Change alignment instantly.                                \\
Comet           & Two of swords       & Two of diamonds           & Defeat the next monster you meet to gain one level.        \\
Donjon          & Four of swords      & Ace of spades             & You are imprisoned.                                        \\
Euryale         & Ten of swords       & Queen of spades           & –1 penalty on all saving throws henceforth.                \\
The Fates       & Three of cups       & Ace of hearts             & Avoid any situation you choose, once.                      \\
Flames          & XV. The Devil       & Queen of clubs            & Enmity between you and an outsider.                        \\
Fool            & 0. The Fool         & Joker (with trademark)    & Lose 10,000 experience points and you must draw again.     \\
Gem             & Seven of cups       & Two of hearts             & Gain your choice of 25 pieces of jewelry or 50 gems.       \\
Idiot           & Two of pentacles    & Two of clubs              & Lose 1d4+1 Intelligence. You may draw again.               \\
Jester          & XII. The Hanged Man & Joker (without trademark) & Gain 10,000 XP or two more draws from the deck.            \\
Key             & V. The Hierophant   & Queen of hearts           & Gain a major magic weapon.                                 \\
Knight          & Page of swords      & Jack of hearts            & Gain the service of a 4th-level fighter.                   \\
Moon            & XVIII. The Moon     & Queen of diamonds         & You are granted 1d4 wishes.                                \\
Rogue           & Five of swords      & Jack of spades            & One of your friends turns against you.                     \\
Ruin            & XVI. The Tower      & King of spades            & Immediately lose all wealth and property.                  \\
Skull           & XIII. Death         & Jack of clubs             & Defeat dread wraith or be forever destroyed.               \\
Star            & XVII. The Star      & Jack of diamonds          & Immediately gain a +2 inherent bonus to one ability score. \\
Sun             & XIX. The Sun        & King of diamonds          & Gain beneficial medium wondrous item and 50,000 XP.        \\
Talons          & Queen of pentacles  & Ace of clubs              & All magic items you possess disappear permanently.         \\
Throne          & Four of wands       & King of hearts            & Gain a +6 bonus on Diplomacy checks plus a small castle.   \\
Vizier          & IX. The Hermit      & Ace of diamonds           & Know the answer to your next dilemma.                      \\
The Void        & Eight of swords     & King of clubs             & Body functions, but soul is trapped elsewhere.            
\end{tabular}
\end{table}

\textit{Balance}: The character must change to a radically different alignment. If the character fails to act according to the new alignment, she gains a negative level.
				
\textit{Comet}: The character must single-handedly defeat the next hostile monster or monsters encountered, or the benefit is lost. If successful, the character gains enough XP to attain the next experience level.
				
\textit{Donjon}: This card signifies imprisonment---either by the \textit{imprisonment }spell or by some powerful being. All gear and spells are stripped from the victim in any case. Draw no more cards.
				
\textit{Euryale}: The medusa-like visage of this card brings a curse that only the Fates card or a deity can remove. The --1 penalty on all saving throws is otherwise permanent.
				
\textit{The Fates}: This card enables the character to avoid even an instantaneous occurrence if so desired, for the fabric of reality is unraveled and respun. Note that it does not enable something to happen. It can only stop something from happening or reverse a past occurrence. The reversal is only for the character who drew the card; other party members may have to endure the situation.
				
\textit{Flames}: Hot anger, jealousy, and envy are but a few of the possible motivational forces for the enmity. The enmity of the outsider can't be ended until one of the parties has been slain. Determine the outsider randomly, and assume that it attacks the character (or plagues her life in some way) within 1d20 days.
				
\textit{Fool}: The payment of XP and the redraw are mandatory. This card is always discarded when drawn, unlike all others except the Jester.
				
\textit{Gem}: This card indicates wealth. The jewelry is all gold set with gems, each piece worth 2,000 gp, and the gems are worth 1,000 gp each. 
				
\textit{Idiot}: This card causes the drain of 1d4+1 points of Intelligence immediately. The additional draw is optional.
				
\textit{Jester}: This card is always discarded when drawn, unlike all others except the Fool. The redraws are optional.
				
\textit{Key}: The magic weapon granted must be one usable by the character. It suddenly appears out of nowhere in the character's hand.
				
\textit{Knight}: The fighter appears out of nowhere and serves loyally until death. He or she is of the same race (or kind) and gender as the character. This fighter can be taken as a cohort by a character with the Leadership feat.
				
\textit{Moon}: This card bears the image of a moonstone gem with the appropriate number of \textit{wishes }shown as gleams therein; sometimes it depicts a moon with its phase indicating the number of \textit{wishes }(full = four; gibbous = three; half = two; quarter = one). These \textit{wishes }are the same as those granted by the 9th-level wizard spell and must be used within a number of minutes equal to the number received.
				
\textit{Rogue}: When this card is drawn, one of the character's NPC friends (preferably a cohort) is totally alienated and made forever hostile. If the character has no cohorts, the enmity of some powerful personage (or community, or religious order) can be substituted. The hatred is secret until the time is ripe for it to be revealed with devastating effect.
				
\textit{Ruin}: As implied by its name, when this card is drawn, all nonmagical possessions of the drawer are lost.
				
\textit{Skull}: A dread wraith appears. The character must fight it alone---if others help, they get dread wraiths to fight as well. If the character is slain, she is slain forever and cannot be revived, even with a \textit{wish }or a \textit{miracle}.
				
\textit{Star}: The 2 points are added to any ability the character chooses. They cannot be divided among two abilities. 
				
\textit{Sun}: Roll for a medium wondrous item until a useful item is indicated.
				
\textit{Talons}: When this card is drawn, every magic item owned or possessed by the character is instantly and irrevocably lost, except for the deck.
				
\textit{Throne}: The character becomes a true leader in people's eyes. The castle gained appears in any open area she wishes (but the decision where to place it must be made within 1 hour).
				
\textit{Vizier}: This card empowers the character drawing it with the one-time ability to call upon a source of wisdom to solve any single problem or answer fully any question upon her request. The query or request must be made within 1 year. Whether the information gained can be successfully acted upon is another matter entirely.
				
\textit{The Void}: This black card spells instant disaster. The character's body continues to function, as though comatose, but her psyche is trapped in a prison somewhere---in an object on a far plane or planet, possibly in the possession of an outsider. A \textit{wish }or a \textit{miracle }does not bring the character back, instead merely revealing the plane of entrapment. Draw no more cards. 
				
Destruction
				
The \textit{deck of many things} can be destroyed by losing it in a wager with a deity of law. The deity must be unaware of the nature of the deck.
				
\textbf{Philosopher's Stone}
				
\textbf{Aura} strong transmutation; \textbf{CL} 20th
				
\textbf{Slot} none; \textbf{Weight} 3 lbs.
				
Description
				
This rare substance appears to be an ordinary, sooty piece of blackish rock. If the stone is broken open (break DC 20), a cavity is revealed at the stone's heart. This cavity is lined with a magical type of quicksilver that enables any character with at least 10 ranks in Craft (alchemy) to transmute base metals (iron and lead) into silver and gold. A single \textit{philosopher's stone }can turn up to 5,000 pounds of iron into silver (worth 25,000 gp), or up to 1,000 pounds of lead into gold (worth 50,000 gp). However, the quicksilver becomes unstable once the stone is opened and loses its potency within 24 hours, so all transmutations must take place within that period.
				
The quicksilver found in the center of the stone may also be put to another use. If mixed with any cure potion while the substance is still potent, it creates a special oil of life that acts as a \textit{true resurrection} spell for any dead body it is sprinkled upon. 
				
Destruction
				
The philosopher's stone can be destroyed by being placed in the heel of a titan's boot for at least 1 entire week.
				
\textbf{Sphere of Annihilation}
				
\textbf{Aura} strong transmutation; \textbf{CL} 20th
				
\textbf{Slot} none; \textbf{Weight }---
				
Description
				
A \textit{sphere of annihilation }is a globe of absolute blackness 2 feet in diameter. Any matter that comes in contact with a sphere is instantly sucked into the void and utterly destroyed. Only the direct intervention of a deity can restore an annihilated character.
				
A \textit{sphere of annihilation} is static, resting in some spot as if it were a normal hole. It can be caused to move, however, by mental effort (think of this as a mundane form of \textit{telekinesis}, too weak to move actual objects but a force to which the sphere, being weightless, is sensitive). A character's ability to gain control of a \textit{sphere of annihilation} (or to keep controlling one) is based on the result of a control check against DC 30 (a move action). A control check is 1d20 + character level + character Int modifier. If the check succeeds, the character can move the sphere (perhaps to bring it into contact with an enemy) as a free action.
				
Control of a sphere can be established from as far away as 40 feet (the character need not approach too closely). Once control is established, it must be maintained by continuing to make control checks (all DC 30) each round. For as long as a character maintains control (does not fail a check) in subsequent rounds, he can control the sphere from a distance of 40 feet + 10 feet per character level. The sphere's speed in a round is 10 feet + 5 feet for every 5 points by which the character's control check result in that round exceeded 30.
				
If a control check fails, the sphere slides 10 feet in the direction of the character attempting to move it. If two or more creatures vie for control of a \textit{sphere of annihilation}, the rolls are opposed. If none are successful, the sphere slips toward the one who rolled lowest.
				
See also \textit{talisman of the sphere}. 
				
Destruction
				
Should a \textit{gate} spell be cast upon a \textit{sphere of annihilation}, there is a 50\% chance (01--50 on d\%) that the spell destroys it, a 35\% chance (51--85) that the spell does nothing, and a 15\% chance (86--100) that a gap is torn in the spatial fabric, catapulting everything within a 180-foot radius into another plane. If a \textit{rod of cancellation} touches a \textit{sphere of annihilation}, they negate each other in a tremendous explosion. Everything within a 60-foot radius takes 2d6 \mbox{$\times$} 10 points of damage. \textit{Dispel magic} and \textit{mage's disjunction} have no effect on a sphere.
				
\textbf{Staff of the Magi}
				
\textbf{Aura} strong (all schools); \textbf{CL} 20th
				
\textbf{Slot} none; \textbf{Weight} 5 lbs.
				
Description
				
A long wooden staff, shod in iron and inscribed with sigils and runes of all types, this potent artifact contains many spell powers and other functions. Unlike a normal staff, a \textit{staff of the magi} holds 50 charges and cannot be recharged normally. Some of its powers use charges, while others don't. A \textit{staff of the magi} does not lose its powers if it runs out of charges. The following powers do not use charges:
				\begin{itemize}\item  \textit{Detect magic}
				\item  \textit{Enlarge person} (Fortitude DC 15 negates)
				\item  \textit{Hold portal}
				\item  \textit{Light}
				\item  \textit{Mage armor}
				\item  \textit{Mage hand}
\end{itemize}
				
The following powers drain 1 charge per usage:
				\begin{itemize}\item  \textit{Dispel magic}
				\item  \textit{Fireball} (10d6 damage, Reflex DC 17 half)
				\item  \textit{Ice storm}
				\item  \textit{Invisibility}
				\item  \textit{Knock}
				\item  \textit{Lightning bolt} (10d6 damage, Reflex DC 17 half)
				\item  \textit{Passwall}
				\item  \textit{Pyrotechnics} (Will or Fortitude DC 16 negates)
				\item  \textit{Wall of fire}
				\item  \textit{Web}
\end{itemize}
				
These powers drain 2 charges per usage:
				\begin{itemize}\item  \textit{Monster summoning IX}
				\item  \textit{Plane shift} (Will DC 21 negates)
				\item  \textit{Telekinesis} (400 lbs. maximum weight; Will DC 19 negates)
\end{itemize}
				
A \textit{staff of the magi} gives the wielder spell resistance 23. If this is willingly lowered, however, the staff can also be used to absorb arcane spell energy directed at its wielder, as a \textit{rod of absorption} does. Unlike the rod, this staff converts spell levels into charges rather than retaining them as spell energy usable by a spellcaster. If the staff absorbs enough spell levels to exceed its limit of 50 charges, it explodes as if a retributive strike had been performed (see below). The wielder has no idea how many spell levels are cast at her, for the staff does not communicate this knowledge as a \textit{rod of absorption} does. (Thus, absorbing spells can be risky.)
				
Destruction
				
A \textit{staff of the magi} can be broken for a retributive strike. Such an act must be purposeful and declared by the wielder. All charges in the staff are released in a 30-foot spread. All within 10 feet of the broken staff take hit points of damage equal to 8 times the number of charges in the staff, those between 11 feet and 20 feet away take points equal to 6 times the number of charges, and those 21 feet to 30 feet distant take 4 times the number of charges. A DC 23 Reflex save reduces damage by half. 
				
The character breaking the staff has a 50\% chance (01--50 on d\%) of traveling to another plane of existence, but if she does not (51--100), the explosive release of spell energy destroys her (no saving throw). 
				
\textbf{Talisman of Pure Good}
				
\textbf{Aura} strong evocation \mbox{$[$}good\mbox{$]$}; \textbf{CL} 18th
				
\textbf{Slot} none; \textbf{Weight }---
				
Description
				
A good divine spellcaster who possesses this item can cause a flaming crack to open at the feet of an evil divine spellcaster who is up to 100 feet away. The intended victim is swallowed up forever and sent hurtling to the center of the earth. The wielder of the talisman must be good, and if he is not exceptionally pure in thought and deed, the evil character gains a DC 19 Reflex saving throw to leap away from the crack. Obviously, the target must be standing on solid ground for this item to function. 
				
A \textit{talisman of pure good} has 6 charges. If a neutral (LN, N, CN) divine spellcaster touches one of these stones, he takes 6d6 points of damage per round of contact. If an evil divine spellcaster touches one, he takes 8d6 points of damage per round of contact. All other characters are unaffected by the device. 
				
Destruction
				
The \textit{talisman of pure good} can be destroyed by placing it in the mouth of a holy man who died while committing a truly heinous act of his own free will.
				
\textbf{Talisman of the Sphere}
				
\textbf{Aura} strong transmutation; \textbf{CL} 16th
				
\textbf{Slot} none; \textbf{Weight} 1 lb.
				
Description
				
This small adamantine loop and handle is typically fitted with a fine adamantine chain so that it can be worn about as a necklace. A \textit{talisman of the sphere} is worse than useless to those unable to cast arcane spells. Characters who cannot cast arcane spells take 5d6 points of damage merely from picking up and holding a talisman of this sort. However, when held by an arcane spellcaster who is concentrating on control of a \textit{sphere of annihilation, }a \textit{talisman of the sphere }doubles the character's modifier on his control check (doubling both his Intelligence bonus and his character level for this purpose).
				
If the wielder of a talisman establishes control, he need check for maintaining control only every other round thereafter. If control is not established, the sphere moves toward him. Note that while many spells and effects of cancellation have no effect upon a \textit{sphere of annihilation}, the talisman's power of control can be suppressed or canceled. 
				
Destruction
				
A \textit{talisman of the sphere} can only be destroyed by throwing the item into a \textit{sphere of annihilation}.
				
\textbf{Talisman of Ultimate Evil}
				
\textbf{Aura} strong evocation \mbox{$[$}evil\mbox{$]$}; \textbf{CL} 18th
				
\textbf{Slot} none; \textbf{Weight }---
				
Description
				
An evil divine spellcaster who possesses this item can cause a flaming crack to open at the feet of a good divine spellcaster who is up to 100 feet away. The intended victim is swallowed up forever and sent hurtling to the center of the earth. The wielder of the talisman must be evil, and if she is not exceptionally foul and perverse in the sights of her evil deity, the good character gains a DC 19 Reflex save to leap away from the crack. Obviously, the target must be standing on solid ground for this item to function. 
				
A \textit{talisman of ultimate evil} has 6 charges. If a neutral (LN, N, CN) divine spellcaster touches one of these stones, she takes 6d6 points of damage per round of contact. If a good divine spellcaster touches one, she takes 8d6 points of damage per round of contact. All other characters are unaffected by the device. 
				
Destruction
				
If a \textit{talisman of ultimate evil} is given to the newborn child of a redeemed villain, it instantly crumbles to dust.
				
\subsection{Major Artifacts}

				
Major artifacts are unique items---only one of each such item exists. These are the most potent of magic items, capable of altering the balance of a campaign. Unlike all other magic items, major artifacts are not easily destroyed. Each should have only a single, specific means of destruction.
				
\textbf{Axe of the Dwarvish Lords}
				
\textbf{Aura} strong conjuration and transmutation; \textbf{CL} 20th
				
\textbf{Slot} none; \textbf{Weight} 12 lbs.
				
Description
				
This is a +\textit{6 keen throwing goblinoid bane dwarven waraxe}. Any dwarf who holds it doubles the range of his or her darkvision. Any nondwarf who grasps the \textit{Axe }takes 4 points of temporary Charisma damage; these points cannot be healed or restored in any way while the \textit{Axe }is held. The current owner of the \textit{Axe }gains a +10 bonus on Craft (armor, jewelry, stonemasonry, traps, and weapons) checks. The wielder of the \textit{Axe }can summon an elder earth elemental (as \textit{summon monster IX; }duration 20 rounds) once per week. 
				
Destruction
				
The \textit{Axe of the Dwarvish Lords }rusts away to nothing if it is ever used by a goblin to behead a dwarven king.
				
\textbf{Codex of the Infinite Planes}
				
\textbf{Aura} overwhelming transmutation; \textbf{CL} 30th
				
\textbf{Slot} none; \textbf{Weight} 300 lbs.
				
Description
				
The \textit{Codex} is enormous---supposedly, it requires two strong men to lift it. No matter how many pages are turned, another always remains. Anyone opening the \textit{Codex }for the first time is utterly annihilated, as with a \textit{destruction }spell (Fortitude DC 30). Those who survive can peruse its pages and learn its powers, though not without risk. Each day spent studying the \textit{Codex }allows the reader to make a Spellcraft check (DC 50) to learn one of its powers (choose the power learned randomly; add a +1 circumstance bonus on the check per additional day spent reading until a power is learned). However, each day of study also forces the reader to make a Will save (DC 30 + 1 per day of study) to avoid being driven insane (as the \textit{insanity }spell). The powers of the \textit{Codex of the Infinite Planes }are as follows: \textit{astral projection, banishment, elemental swarm, gate, greater planar ally, greater planar binding, plane shift, }and \textit{soul bind. }Each of these spell-like abilities are usable at will by the owner of the \textit{Codex }(assuming that he or she has learned how to access the power). The \textit{Codex of the Infinite Planes }has a caster level of 30th for the purposes of all powers and catastrophes, and all saving throw DCs are 20 + spell level. Activating any power requires a Spellcraft check (DC 40 + twice the spell level of the power; the character can't take 10 on this check). Any failure on either check indicates that a catastrophe befalls the user (roll on the table below for the effect). A character can only incur one catastrophe per power use. 
% <thead href="../skills/spellcraft.html#spellcraft">

\begin{tabular}{ll}
\textbf{d\%} & \textbf{Catastrophe}                                                                                                                                                                               \\
01–25        & \textbf{Natural Fury}: An \textit{earthquake }spell centered on the reader strikes every round for 1 minute, and an intensified \textit{storm of vengeance }spell is centered and targeted on the reader. \\
26–50        & \textbf{Fiendish Vengeance}: A \textit{gate }opens and 1d3+1 balors, pit fiends, or similar evil outsiders step through and attempt to destroy the owner of the \textit{Codex. }\\
51–75        & \textbf{Ultimate Imprisonment}: Reader's soul is captured (as \textit{trap the soul; }no save allowed) in a random gem somewhere on the plane while his or her body is entombed beneath the earth (as \textit{imprisonment}). \\
76–100       & \textbf{Death}: The reader utters a \textit{wail of the banshee }and then is subject to a \textit{destruction }spell. This repeats every round for 10 rounds until the reader is dead.\\
\end{tabular}
				
The \textit{Codex of the Infinite Planes }is destroyed if one page is torn out and left on each plane in existence. Note that tearing out a page immediately triggers a catastrophe.
				
\textbf{The Orbs of Dragonkind}
				
\textbf{Aura} strong enchantment; \textbf{CL} 20th
				
\textbf{Slot} none; \textbf{Weight} 5 lbs.
				
Description
				
Each of these fabled \textit{Orbs} contains the essence and personality of an ancient dragon of a different variety (one for each of the major ten different chromatic and metallic dragons). The bearer of an \textit{Orb }can, as a standard action, dominate dragons of its particular variety within 500 feet (as \textit{dominate monster}), the dragon being forced to make a DC 25 Will save to resist. Spell resistance is not useful against this effect. Each \textit{Orb of Dragonkind }bestows upon the wielder the AC and saving throw bonuses of the dragon within. These values replace whatever values the character would otherwise have, whether they are better or worse. These values cannot be modified by any means short of ridding the character of the \textit{Orb. }A character possessing an \textit{Orb of Dragonkind }is immune to the breath weapon---but only the breath weapon---of the dragon variety keyed to the \textit{Orb. }Finally, a character possessing an \textit{Orb }can herself use the breath weapon of the dragon in the \textit{Orb }three times per day.
				
All \textit{Orbs of Dragonkind} can be used to communicate verbally and visually with the possessors of the other \textit{Orbs}. The owner of an \textit{Orb} knows if there are dragons within 10 miles at all times. For dragons of the \textit{Orb's} particular variety, the range is 100 miles. If within 1 mile of a dragon of the \textit{Orb's} variety, the wielder can determine the dragon's exact location and age. The bearer of one of these \textit{Orbs} earns the enmity of dragonkind forever for profiting by draconic enslavement, even if she later loses the item. Each \textit{Orb} also has an individual power that can be invoked once per round at caster level 10th.
				\begin{itemize}\item  \textit{Black Dragon Orb}: \textit{Fly}.
				\item  \textit{Blue Dragon Orb}: \textit{Haste}.
				\item  \textit{Brass Dragon Orb}: \textit{Teleport}.
				\item  \textit{Bronze Dragon Orb}: \textit{Scrying} (Will DC 18 negates).
				\item  \textit{Copper Dragon Orb}: \textit{Suggestion} (Will DC 17 negates).
				\item  \textit{Gold Dragon Orb}: The owner of the gold \textit{Orb} can call upon any power possessed by one of the other \textit{Orbs}---including the dominate and breath weapon abilities but not AC, save bonuses, or breath weapon immunity---but can only use an individual power once per day. She can dominate any other possessor of an \textit{Orb} within 1 mile (Will DC 23 negates).
				\item  \textit{Green Dragon Orb}: \textit{Spectral hand}.
				\item  \textit{Red Dragon Orb}: \textit{Wall of fire}.
				\item  \textit{Silver Dragon Orb}: \textit{Cure critical wounds} (Will DC 18 half).
				\item  \textit{White Dragon Orb}: \textit{Protection from energy (cold)} (Fortitude DC 17 negates)
\end{itemize}
				
Destruction
				
An \textit{orb of dragonkind} immediately shatters if it is caught in the breath weapon of a dragon who is a blood relative of the dragon trapped within. This causes everyone within 90 feet to be struck by the breath weapon of that dragon, released as the orb explodes.
				
\textbf{The Shadowstaff}
				
\textbf{Aura} strong conjuration; \textbf{CL} 20th.
				
\textbf{Slot} none; \textbf{Weight} 1 lb.
				
Description
				
This artifact was crafted ages ago, weaving together wispy strands of shadow into a twisted black staff. The \textit{Shadowstaff }makes the wielder slightly shadowy and incorporeal, granting him a +4 bonus to AC and on Reflex saves (which stacks with any other bonuses). However, in bright light (such as that of the sun, but not a torch) or in absolute darkness, the wielder takes a --2 penalty on all attack rolls, saves, and checks. The \textit{Shadowstaff} also has these powers.
				\begin{itemize}\item  \textit{Summon Shadows}: Three times per day the staff may summon 2d4 shadows. Immune to turning, they serve the wielder as if called by a \textit{summon monster V }spell cast at 20th level.
				\item  \textit{Summon Nightshade}: Once per month, the staff can summon an advanced shadow demon that serves the wielder as if called by a \textit{summon monster IX} spell cast at 20th level.
				\item  \textit{Shadow Form}: Three times per day the wielder can become a living shadow, with all the movement powers granted by \textit{gaseous form}. 
				\item  \textit{Shadow Bolt}: Three times per day the staff can project a ray attack that deals 10d6 points of cold damage to a single target. The shadow bolt has a range of 100 feet. 
\end{itemize}
				
Destruction
				
The \textit{Shadowstaff} fades away to nothingness if it is exposed to true sunlight for a continuous 24 hour period.
        	

\section{Magic Item Creation}

\label{f0}				
To create magic items, spellcasters use special feats which allow them to invest time and money in an item's creation. At the end of this process, the spellcaster must make a single skill check (usually Spellcraft, but sometimes another skill) to finish the item. If an item type has multiple possible skills, you choose which skill to make the check with. The DC to create a magic item is 5 + the caster level for the item. Failing this check means that the item does not function and the materials and time are wasted. Failing this check by 5 or more results in a cursed item (see Cursed Items for more information).
				
Note that all items have prerequisites in their descriptions. These prerequisites must be met for the item to be created. Most of the time, they take the form of spells that must be known by the item's creator (although access through another magic item or spellcaster is allowed). The DC to create a magic item increases by +5 for each prerequisite the caster does not meet. The only exception to this is the requisite item creation feat, which is mandatory. In addition, you cannot create potions, spell-trigger, or spell-completion magic items without meeting their spell prerequisites.
				
While item creation costs are handled in detail below, note that normally the two primary factors are the caster level of the creator and the level of the spell or spells put into the item. A creator can create an item at a lower caster level than her own, but never lower than the minimum level needed to cast the needed spell. Using metamagic feats, a caster can place spells in items at a higher level than normal.
				
Magic supplies for items are always half of the base price in gp. For many items, the market price equals the base price. Armor, shields, weapons, and items with value independent of their magically enhanced properties add their item cost to the market price. The item cost does not influence the base price (which determines the cost of magic supplies), but it does increase the final market price.
				
In addition, some items cast or replicate spells with costly material components. For these items, the market price equals the base price plus an extra price for the spell component costs. The cost to create these items is the magic supplies cost plus the costs for the components. Descriptions of these items include an entry that gives the total cost of creating the item.
				
The creator also needs a fairly quiet, comfortable, and well-lit place in which to work. Any place suitable for preparing spells is suitable for making items. Creating an item requires 8 hours of work per 1,000 gp in the item's base price (or fraction thereof), with a minimum of at least 8 hours. Potions and scrolls are an exception to this rule; they can take as little as 2 hours to create (if their base price is 250 gp or less). Scrolls and potions whose base price is more than 250 gp, but less than 1,000 gp, take 8 hours to create, just like any other magic item. The character must spend the gold at the beginning of the construction process. Regardless of the time needed for construction, a caster can create no more than one magic item per day. This process can be accelerated to 4 hours of work per 1,000 gp in the item's base price (or fraction thereof) by increasing the DC to create the item by +5.
				
The caster can work for up to 8 hours each day. He cannot rush the process by working longer each day, but the days need not be consecutive, and the caster can use the rest of his time as he sees fit. If the caster is out adventuring, he can devote 4 hours each day to item creation, although he nets only 2 hours' worth of work. This time is not spent in one continuous period, but rather during lunch, morning preparation, and during watches at night. If time is dedicated to creation, it must be spent in uninterrupted 4-hour blocks. This work is generally done in a controlled environment, where distractions are at a minimum, such as a laboratory or shrine. Work that is performed in a distracting or dangerous environment nets only half the amount of progress (just as with the adventuring caster).
				
A character can work on only one item at a time. If a character starts work on a new item, all materials used on the under-construction item are wasted.
				
\subsection{Magic Item Gold Piece Values}

				
Many factors must be considered when determining the price of new magic items. The easiest way to come up with a price is to compare the new item to an item that is already priced, using that price as a guide. Otherwise, use the guidelines summarized on Table: Estimating Magic Item Gold Piece Values.
\begin{table*}[]
\sffamily
\caption{Table: Estimating Magic Item Gold Piece Values}
\begin{tabular}{lll}
\textbf{Effect} & \textbf{Base Price} & \textbf{Example}\\
Ability bonus (enhancement) & Bonus squared $\times$ 1,000 gp & Belt of incredible dexterity \\
 Armor bonus (enhancement) & Bonus squared $\times$ 1,000 gp & +1 chainmail \\
 Bonus spell & Spell level squared $\times$ 1,000 gp &   \\
 AC bonus (deflection) & Bonus squared $\times$ 2,000 gp & Ring of protection \\
 AC bonus (other)\(^{1}\) & Bonus squared $\times$ 2,500 gp & Ioun stone \\
 Natural armor bonus (enhancement) & Bonus squared $\times$ 2,000 gp & Amulet of natural armor \\
 Save bonus (resistance) & Bonus squared $\times$ 1,000 gp & Cloak of resistance \\
 Save bonus (other)\(^{1}\) & Bonus squared $\times$ 2,000 gp & Stone of good luck \\
 Skill bonus (competence) & Bonus squared $\times$ 100 gp & Cloak of elvenkind \\
 Spell resistance & 10,000 gp per point over SR 12; SR 13 minimum & Mantle of spell resistance \\
 Weapon bonus (enhancement) & Bonus squared $\times$ 2,000 gp & +1 longsword\\
\textbf{Spell Effect} &  \textbf{Base Price} & \textbf{Example}\\
Single use, spell completion & Spell level $\times$ caster level $\times$ 25 gp & Scroll of haste \\
 Single use, use-activated & Spell level $\times$ caster level $\times$ 50 gp & Potion of cure light wounds \\
 50 charges, spell trigger & Spell level $\times$ caster level $\times$ 750 gp & Wand of fireball \\
 Command word              & Spell level $\times$ caster level $\times$ 1,800 gp & Cape of the mountebank \\
 Use-activated or continuous & Spell level $\times$ caster level $\times$ 2,000 gp\(^{2}\) & Lantern of revealing\\
\textbf{Special} & \textbf{Base Price}, \textbf{Adjustment Example}\\
Charges per day & Divide by (5 divided by charges per day) & Boots of teleportation \\
 No space limitation\(^{3}\) & Multiply entire cost by 2 & Ioun stone \\
 Multiple different abilities & Multiply lower item cost by 1.5  & Helm of brilliance \\
 Charged (50 charges) & 1/2 unlimited use base price & Ring of the ram\\
Component & Extra Cost & Example\\
Armor, shield, or weapon & Add cost of masterwork item & +1 composite longbow \\
 Spell has material component cost & Add directly into price of item per charge\(^{4}\) & Wand of stoneskin\\
 \end{tabular}\\
Spell Level: A 0-level spell is half the value of a 1st-level spell for determining price.\\
\(^{1}\) Such as a luck, insight, sacred, or profane bonus.\\
\(^{2}\) If a continuous item has an effect based on a spell with a duration measured in rounds, multiply the cost by 4. If the duration of the spell is 1 minute/level, multiply the cost by 2, and if the duration is 10 minutes/level, multiply the cost by 1.5. If the spell has a 24-hour duration or greater, divide the cost in half.\\
\(^{3}\) An item that does not take up one of the spaces on a body costs double.\\
\(^{4}\) If item is continuous or unlimited, not charged, determine cost as if it had 100 charges. If it has some daily limit, determine as if it had 50 charges.\\
\end{table*}
		
\textbf{Multiple Similar Abilities}: For items with multiple similar abilities that don't take up space on a character's body, use the following formula: Calculate the price of the single most costly ability, then add 75\% of the value of the next most costly ability, plus 1/2 the value of any other abilities.
				
\textbf{Multiple Different Abilities}: Abilities such as an attack roll bonus or saving throw bonus and a spell-like function are not similar, and their values are simply added together to determine the cost. For items that take up a space on a character's body, each additional power not only has no discount but instead has a 50\% increase in price.
				
\textbf{0-Level Spells}: When multiplying spell levels to determine value, 0-level spells should be treated as 1/2 level.
				
\textbf{Other Considerations}: Once you have a cost figure, reduce that number if either of the following conditions applies:
				
\textit{Item Requires Skill to Use}: Some items require a specific skill to get them to function. This factor should reduce the cost about 10\%.
				
\textit{Item Requires Specific Class or Alignment to Use}: Even more restrictive than requiring a skill, this limitation cuts the price by 30\%.
				
Prices presented in the magic item descriptions (the gold piece value following the item's slot) are the market value, which is generally twice what it costs the creator to make the item.
				
Since different classes get access to certain spells at different levels, the prices for two characters to make the same item might actually be different. An item is only worth two times what the caster of the lowest possible level can make it for. Calculate the market price based on the lowest possible level caster, no matter who makes the item.
				
Not all items adhere to these formulas. First and foremost, these few formulas aren't enough to truly gauge the exact differences between items. The price of a magic item may be modified based on its actual worth. The formulas only provide a starting point. The pricing of scrolls assumes that, whenever possible, a wizard or cleric created it. Potions and wands follow the formulas exactly. Staves follow the formulas closely, and other items require at least some judgment calls.
				
\subsection{Creating Magic Armor}

				
To create magic armor, a character needs a heat source and some iron, wood, or leatherworking tools. He also needs a supply of materials, the most obvious being the armor or the pieces of the armor to be assembled. Armor to be made into magic armor must be masterwork armor, and the masterwork cost is added to the base price to determine final market value. Additional magic supply costs for the materials are subsumed in the cost for creating the magic armor---half the base price of the item.
				
Creating magic armor has a special prerequisite: The creator's caster level must be at least three times the enhancement bonus of the armor. If an item has both an enhancement bonus and a special ability, the higher of the two caster level requirements must be met. Magic armor or a magic shield must have at least a +1 enhancement bonus to have any armor or shield special abilities.
				
If spells are involved in the prerequisites for making the armor, the creator must have prepared the spells to be cast (or must know the spells, in the case of a sorcerer or bard) and must provide any material components or focuses the spells require. The act of working on the armor triggers the prepared spells, making them unavailable for casting during each day of the armor's creation. (That is, those spell slots are expended from the caster's currently prepared spells, just as if they had been cast.)
				
Creating some armor may entail other prerequisites beyond or other than spellcasting. See the individual descriptions for details.
				
Crafting magic armor requires one day for each 1,000 gp value of the base price.
				
\textbf{Item Creation Feat Required}: Craft Magic Arms and Armor.
				
\textbf{Skill Used in Creation}: Spellcraft or Craft (armor).
				
\subsection{Creating Magic Weapons}

				
To create a magic weapon, a character needs a heat source and some iron, wood, or leatherworking tools. She also needs a supply of materials, the most obvious being the weapon or the pieces of the weapon to be assembled. Only a masterwork weapon can become a magic weapon, and the masterwork cost is added to the total cost to determine final market value. Additional magic supplies costs for the materials are subsumed in the cost for creating the magic weapon---half the base price of the item based upon the item's total effective bonus.
				
Creating a magic weapon has a special prerequisite: The creator's caster level must be at least three times the enhancement bonus of the weapon. If an item has both an enhancement bonus and a special ability, the higher of the two caster level requirements must be met. A magic weapon must have at least a +1 enhancement bonus to have any melee or ranged special weapon abilities.
				
If spells are involved in the prerequisites for making the weapon, the creator must have prepared the spells to be cast (or must know the spells, in the case of a sorcerer or bard) but need not provide any material components or focuses the spells require. The act of working on the weapon triggers the prepared spells, making them unavailable for casting during each day of the weapon's creation. (That is, those spell slots are expended from the caster's currently prepared spells, just as if they had been cast.)
				
At the time of creation, the creator must decide if the weapon glows or not as a side-effect of the magic imbued within it. This decision does not affect the price or the creation time, but once the item is finished, the decision is binding.
				
Creating magic double-headed weapons is treated as creating two weapons when determining cost, time, and special abilities.
				
Creating some weapons may entail other prerequisites beyond or other than spellcasting. See the individual descriptions for details.
				
Crafting a magic weapon requires 1 day for each 1,000 gp value of the base price.
				
\textbf{Item Creation Feat Required}: Craft Magic Arms and Armor.
				
\textbf{Skill Used in Creation}: Spellcraft, Craft (bows) (for magic bows and arrows), or Craft (weapons) (for all other weapons).
				
\subsection{Creating Potions}


\begin{table}[]
\sffamily
\caption{Potion Base Costs (By Brewer's Class)}
\begin{tabular}{lllll}
               & \textbf{Cleric} & \\
\textbf{Spell} & \textbf{Druid}  &                   &               & \textbf{Paladin} \\
\textbf{Level} & \textbf{Wizard} & \textbf{Sorcerer} & \textbf{Bard} & \textbf{Ranger} \\
0                    & 25 gp                          & 25 gp             & 25 gp         & ---                         \\
1st                  & 50 gp                          & 50 gp             & 50 gp         & 50 gp                     \\
2nd                  & 300 gp                         & 400 gp            & 400 gp        & 400 gp                    \\
3rd                  & 750 gp                         & 900 gp            & 1,050 gp      & 1,050 gp                 
\end{tabular}\\
* Caster level is equal to class level --3.
\end{table}

Prices assume that the potion was made at the minimum caster level. The cost to create a potion is half the base price.

				
The creator of a potion needs a level working surface and at least a few containers in which to mix liquids, as well as a source of heat to boil the brew. In addition, he needs ingredients. The costs for materials and ingredients are subsumed in the cost for brewing the potion: 25 gp \mbox{$\times$} the level of the spell \mbox{$\times$} the level of the caster.
				
All ingredients and materials used to brew a potion must be fresh and unused. The character must pay the full cost for brewing each potion. (Economies of scale do not apply.)
				
The imbiber of the potion is both the caster and the target. Spells with a range of personal cannot be made into potions.
				
The creator must have prepared the spell to be placed in the potion (or must know the spell, in the case of a sorcerer or bard) and must provide any material component or focus the spell requires.
				
Material components are consumed when he begins working, but a focus is not. (A focus used in brewing a potion can be reused.) The act of brewing triggers the prepared spell, making it unavailable for casting until the character has rested and regained spells. (That is, that spell slot is expended from the caster's currently prepared spells, just as if it had been cast.) Brewing a potion requires 1 day.
				
\textbf{Item Creation Feat Required}: Brew Potion.
				
\textbf{Skill Used in Creation}: Spellcraft or Craft (alchemy)
				
\subsection{Creating Rings}

				
To create a magic ring, a character needs a heat source. He also needs a supply of materials, the most obvious being a ring or the pieces of the ring to be assembled. The cost for the materials is subsumed in the cost for creating the ring. Ring costs are difficult to determine. Refer to Table: Estimating Magic Item Gold Piece Values and use the ring prices in the ring descriptions as a guideline. Creating a ring generally costs half the ring's market price.
				
Rings that duplicate spells with costly material components add in the value of 50 \mbox{$\times$} the spell's component cost. Having a spell with a costly component as a prerequisite does not automatically incur this cost. The act of working on the ring triggers the prepared spells, making them unavailable for casting during each day of the ring's creation. (That is, those spell slots are expended from the caster's currently prepared spells, just as if they had been cast.)
				
Creating some rings may entail other prerequisites beyond or other than spellcasting. See the individual descriptions for details.
				
Forging a ring requires 1 day for each 1,000 gp of the base price.
				
\textbf{Item Creation Feat Required}: Forge Ring.
				
\textbf{Skill Used in Creation}: Spellcraft or Craft (jewelry).
				
\subsection{Creating Rods}

				
To create a magic rod, a character needs a supply of materials, the most obvious being a rod or the pieces of the rod to be assembled. The cost for the materials is subsumed in the cost for creating the rod. Rod costs are difficult to determine. Refer to Table: Estimating Magic Item Gold Piece Values and use the rod prices in the rod descriptions as a guideline. Creating a rod costs half the market value listed.
				
If spells are involved in the prerequisites for making the rod, the creator must have prepared the spells to be cast (or must know the spells, in the case of a sorcerer or bard) but need not provide any material components or focuses the spells require. The act of working on the rod triggers the prepared spells, making them unavailable for casting during each day of the rod's creation. (That is, those spell slots are expended from the caster's currently prepared spells, just as if they had been cast.)
				
Creating some rods may entail other prerequisites beyond or other than spellcasting. See the individual descriptions for details.
				
Crafting a rod requires 1 day for each 1,000 gp of the base price.
				
\textbf{Item Creation Feat Required}: Craft Rod.
				
\textbf{Skill Used in Creation}: Spellcraft, Craft (jewelry), Craft (sculptures), or Craft (weapons).
				
\subsection{Creating Scrolls}


\begin{table}[]
\sffamily
\caption{Scroll Base Costs (By Scriber's Class)}
\begin{tabular}{lllll}
               & \textbf{Cleric} & \\
\textbf{Spell} & \textbf{Druid}  &                   &               & \textbf{Paladin} \\
\textbf{Level} & \textbf{Wizard} & \textbf{Sorcerer} & \textbf{Bard} & \textbf{Ranger} \\
0           & 12 gp 5 sp            & 12 gp 5 sp & 12 gp 5 sp & ---                \\
1st         & 25 gp                 & 25 gp      & 25 gp      & 25 gp            \\
2nd         & 150 gp                & 200 gp     & 200 gp     & 200 gp           \\
3rd         & 375 gp                & 450 gp     & 525 gp     & 525 gp           \\
4th         & 700 gp                & 800 gp     & 1,000 gp   & 1,000 gp         \\
5th         & 1,125 gp              & 1,250 gp   & 1,625 gp   & ---                \\
6th         & 1,650 gp              & 1,800 gp   & 2,400 gp   & ---                \\
7th         & 2,275 gp              & 2,450 gp   & ---          & ---                \\
8th         & 3,000 gp              & 3,200 gp   & ---          & ---                \\
9th         & 3,825 gp              & 4,050 gp   & ---          & ---               
\end{tabular}\\
* Caster level is equal to class level --3.
\end{table}
Prices assume that the scroll was made at the minimum caster level. The cost to create a scroll is half the base price.
				
To create a scroll, a character needs a supply of choice writing materials, the cost of which is subsumed in the cost for scribing the scroll: 12.5 gp \mbox{$\times$} the level of the spell \mbox{$\times$} the level of the caster.
				
All writing implements and materials used to scribe a scroll must be fresh and unused. A character must pay the full cost for scribing each spell scroll no matter how many times she previously has scribed the same spell.
				
The creator must have prepared the spell to be scribed (or must know the spell, in the case of a sorcerer or bard) and must provide any material component or focus the spell requires. A material component is consumed when she begins writing, but a focus is not. (A focus used in scribing a scroll can be reused.) The act of writing triggers the prepared spell, making it unavailable for casting until the character has rested and regained spells. (That is, that spell slot is expended from the caster's currently prepared spells, just as if it had been cast.)
				
Scribing a scroll requires 1 day per 1,000 gp of the base price. Although an individual scroll might contain more than one spell, each spell must be scribed as a separate effort, meaning that no more than 1 spell can be scribed in a day.
				
\textbf{Item Creation Feat Required}: Scribe Scroll.
				
\textbf{Skill Used in Creation}: Spellcraft, Craft (calligraphy), or Profession (scribe).
				
\subsection{Creating Staves}

				
To create a magic staff, a character needs a supply of materials, the most obvious being a staff or the pieces of the staff to be assembled.
				
The materials cost is subsumed in the cost of creation: 400 gp \mbox{$\times$} the level of the highest-level spell \mbox{$\times$} the level of the caster, plus 75\% of the value of the next most costly ability (300 gp \mbox{$\times$} the level of the spell \mbox{$\times$} the level of the caster), plus 1/2 the value of any other abilities (200 gp \mbox{$\times$} the level of the spell \mbox{$\times$} the level of the caster). Staves are always fully charged (10 charges) when created.
				
If desired, a spell can be placed into the staff at less than the normal cost, but then activating that particular spell drains additional charges from the staff. Divide the cost of the spell by the number of charges it consumes to determine its final price. Note that this does not change the order in which the spells are priced (the highest level spell is still priced first, even if it requires more than one charge to activate). The caster level of all spells in a staff must be the same, and no staff can have a caster level of less than 8th, even if all the spells in the staff are low-level spells.
				
The creator must have prepared the spells to be stored (or must know the spells, in the case of a sorcerer or bard) and must provide any focus the spells require as well as material component costs sufficient to activate the spell 50 times (divide this amount by the number of charges one use of the spell expends). Material components are consumed when he begins working, but focuses are not. (A focus used in creating a staff can be reused.) The act of working on the staff triggers the prepared spells, making them unavailable for casting during each day of the staff 's creation. (That is, those spell slots are expended from the caster's currently prepared spells, just as if they had been cast.)
				
Creating a few staves may entail other prerequisites beyond spellcasting. See the individual descriptions for details.
				
Crafting a staff requires 1 day for each 1,000 gp of the base price.
				
\textbf{Item Creation Feat Required}: Craft Staff.
				
\textbf{Skill Used in Creation}: Spellcraft, Craft (jewelry), Craft (sculptures), or Profession (woodcutter).
				
\subsection{Creating Wands}

\begin{table}[]
\sffamily
\caption{Wand Base Costs (By Crafter's Class)}
\begin{tabular}{lllll}
               & \textbf{Cleric} & \\
\textbf{Spell} & \textbf{Druid}  &                   &               & \textbf{Paladin} \\
\textbf{Level} & \textbf{Wizard} & \textbf{Sorcerer} & \textbf{Bard} & \textbf{Ranger} \\
0                    & 375 gp                         & 375 gp            & 375 gp        & ---                         \\
1st                  & 750 gp                         & 750 gp            & 750 gp        & 750 gp                    \\
2nd                  & 4,500 gp                       & 6,000 gp          & 6,000 gp      & 6,000 gp                  \\
3rd                  & 11,250 gp                      & 13,500 gp         & 15,750 gp     & 15,750 gp                 \\
4th                  & 21,000 gp                      & 24,000 gp         & 30,000 gp     & 30,000 gp                
\end{tabular}\\
* Caster level is equal to class level --3.
\end{table}
Prices assume that the wand was made at the minimum caster level. The cost to create a wand is half the base price.
				
To create a magic wand, a character needs a small supply of materials, the most obvious being a baton or the pieces of the wand to be assembled. The cost for the materials is subsumed in the cost for creating the wand: 375 gp \mbox{$\times$} the level of the spell \mbox{$\times$} the level of the caster. Wands are always fully charged (50 charges) when created.
				
The creator must have prepared the spell to be stored (or must know the spell, in the case of a sorcerer or bard) and must provide any focuses the spell requires. Fifty of each needed material component are required (one for each charge). Material components are consumed when work begins, but focuses are not. A focus used in creating a wand can be reused. The act of working on the wand triggers the prepared spell, making it unavailable for casting during each day devoted to the wand's creation. (That is, that spell slot is expended from the caster's currently prepared spells, just as if it had been cast.)
				
Crafting a wand requires 1 day per each 1,000 gp of the base price.
				
\textbf{Item Creation Feat Required}: Craft Wand.
				
\textbf{Skill Used in Creation}: Spellcraft, Craft (jewelry), Craft (sculptures), or Profession (woodcutter).
				
\subsection{Creating Wondrous Items}

				
To create a wondrous item, a character usually needs some sort of equipment or tools to work on the item. She also needs a supply of materials, the most obvious being the item itself or the pieces of the item to be assembled. The cost for the materials is subsumed in the cost for creating the item. Wondrous item costs are difficult to determine. Refer to Table: Estimating Magic Item Gold Piece Values and use the item prices in the item descriptions as a guideline. Creating an item costs half the market value listed.
				
If spells are involved in the prerequisites for making the item, the creator must have prepared the spells to be cast (or must know the spells, in the case of a sorcerer or bard) but need not provide any material components or focuses the spells require. The act of working on the item triggers the prepared spells, making them unavailable for casting during each day of the item's creation. (That is, those spell slots are expended from the caster's currently prepared spells, just as if they had been cast.)
				
Creating some items may entail other prerequisites beyond or other than spellcasting. See the individual descriptions for details.
				
Crafting a wondrous item requires 1 day for each 1,000 gp of the base price.
				
\textbf{Item Creation Feat Required}: Craft Wondrous Item.
				
\textbf{Skill Used In Creation}: Spellcraft or an applicable Craft or Profession skill check.
				
\subsection{Adding New Abilities}

				
Sometimes, lack of funds or time make it impossible for a magic item crafter to create the desired item from scratch. Fortunately, it is possible to enhance or build upon an existing magic item. Only time, gold, and the various prerequisites required of the new ability to be added to the magic item restrict the type of additional powers one can place.
				
The cost to add additional abilities to an item is the same as if the item was not magical, less the value of the original item. Thus, a \textit{+1 longsword} can be made into a \textit{+2 vorpal longsword,} with the cost to create it being equal to that of a \textit{+2 vorpal sword} minus the cost of a \textit{+1 longsword}.
				
If the item is one that occupies a specific place on a character's body, the cost of adding any additional ability to that item increases by 50\%. For example, if a character adds the power to confer invisibility to her \textit{ring of protection +2,} the cost of adding this ability is the same as for creating a \textit{ring of invisibility} multiplied by 1.5. 

